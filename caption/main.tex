% main.tex
% 定义使用xelatex编译。
% 下面这个注释是会被vscode的latex workshop插件解释的
%!TEX program=xelatex

\documentclass{ctexart} % 使用文章类,定义小四号为基本字号
\usepackage{setspace} % 改变行距的包
\usepackage{ulem} % \uline{更优雅的下划线}
%%%%%%% 页面布局 %%%%%%%%
% 定义写论文时通用的页边距
\usepackage[a4paper, top=2cm, bottom=2cm, left=3cm, right=3cm]{geometry}

%%%%%%%%%% space %%%%%%%%%%
% 页码从1开始
\setcounter{page}{1}
\pagestyle{plain}  % 页脚居中显示页码
% 其他间距
\renewcommand{\baselinestretch}{1.2}
\setlength{\parindent}{2em}
\setlength{\floatsep}{3pt plus 3pt minus 2pt}      % 图形之间或图形与正文之间的距离
\setlength{\abovecaptionskip}{10pt plus 1pt minus 1pt} % 图形中的图与标题之间的距离
\setlength{\belowcaptionskip}{3pt plus 1pt minus 2pt} % 表格中的表与标题之间的距

%%%%%%%% 字体样式定义 %%%%%%%%
\usepackage{fontspec}
% 英文全部使用新罗马字体
\setmainfont[Mapping=tex-text]{Times New Roman}
\setsansfont[Mapping=tex-text]{Times New Roman}
\setmonofont{Times New Roman}

\setCJKmainfont[BoldFont=方正黑体_GBK, ItalicFont=方正仿宋_GBK]{方正书宋_GBK} % 一般字体 
\setCJKsansfont{方正黑体_GBK} % 方正黑
\setCJKmonofont{方正楷体_GBK} % 方正楷

% punctstyle设置标点符号挤压
% 默认是弹性调整符号的空隙
% 有全角、半角、开明、行末半角、plain
\punctstyle{kaiming}

%%%%%%%%%%% 标题样式定义 %%%%%%%%%%%%
\usepackage{titlesec}
%  \titleformat{command}[shape]  %定义标题类型和标题样式,字体
%  {format}  %定义标题格式:字号(大小),加粗,斜体
%  {label}  %定义标题的标签,即标题的标号等
%  {sep}  %定义标题和标号之间的水平距离
%  {before-code}  %定义标题前的内容
%  [after-code]  %定义标题后的内容

% 节 节后有一个空隙(但是默认就带有2个换行)
\titleformat{\section}[block]
{\fontsize{14pt}{0}\bfseries}
{\thesection .}{1em}{}[\vspace{-1ex}]

% 小节 节后有一个空隙(但是默认就带有2个换行)
\titleformat{\subsection}[block]
{\fontsize{12pt}{0}\bfseries}
{\thesubsection}{1em}{}[\vspace{-1ex}]

% 小小节 节后有一个空隙(但是默认就带有2个换行)
\titleformat{\subsubsection}[block]
{\fontsize{12pt}{0}}
{\thesubsubsection}{1em}{}[\vspace{-1ex}]

%%%%%%%%%% 列表格式 %%%%%%%%%%
\usepackage{enumitem}
\setlist{noitemsep}
\setlist[1,2]{labelindent=\parindent}
\setlist[enumerate,1]{label=\arabic*、}
\setlist[enumerate,2]{label=(\arabic*)}
\setlist{
    topsep=0pt,
    itemsep=0pt,
    partopsep=0pt,
    parsep=\parskip,
}

%%%%%%%%%%%% 脚注格式 %%%%%%%%%%%%%%%%
%\renewcommand{\thefootnote}{*}

%%%%%%%%%%%% 引用跳转 %%%%%%%%%%%%%%%%%
% 该包可实现引用自动跳转
% 如需跳转图文 添加代码 ~\ref{标签名}
\usepackage{hyperref}
\hypersetup{hidelinks} % 去除超链接难看的框框
\bibliographystyle{gbt7714-2005-unsrt}

% natbib支持自然引用和数字引用,而且可以压缩序号
\usepackage[numbers,sort&compress]{natbib}
\setlength{\bibsep}{0.3ex} % 缩小参考文献的垂直间距

%%%%%%%%% 图形导入 %%%%%%%%%
\usepackage{graphicx} % 可以导入图形的包
\graphicspath{{images/}} % 引入该目录,写文件时不再用写路径

%%%%%%%%% 浮动体标题设置 %%%%%%%%%
\usepackage[labelsep=space]{caption}
\renewcommand{\captionfont}{\small \centering} % 标题字号变小、居中
\renewcommand{\figurename}{图} %重定义编号前缀词
\renewcommand{\tablename}{表} %重定义编号前缀词

% 表格
\usepackage{booktabs} % 科技排版常用四横线表格
\usepackage{tabularx} % X列格式平均分配列宽解决方案
\usepackage{multirow} % 合并表格行
\usepackage{array}

% 引用不仅仅是数字还包括前缀
% 使用该包后标签必须以fig:balabala 这样的形式开头才能识别
% 引用使用prettyref而不是ref
\usepackage{prettyref}
% 重定义引用前缀,并且添加超链接
\newrefformat{fig}{\hyperref[#1]{图\ref*{#1}}}
\newrefformat{eq}{\hyperref[#1]{式\ref*{#1}}}
\newrefformat{tab}{\hyperref[#1]{表\ref*{#1}}}
\newrefformat{sec}{\hyperref[#1]{ \ref*{#1} 节}} % 考虑到引用节的前面一般都是中文。在本文中全是,所以直接统一加上空格。

% 数学公式配置
\usepackage{amsmath} % 数学公式包
\usepackage{amssymb} % 数学符号包
% 一些漂亮的数学公式字体,只需要use包就好,在这里mark一下:
% mathptmx字体、fourier、Eulervm
% 不使用默认的字体是因为默认字体相对新罗马来说变形较厉害
% 但是不存在times new roman字体的整套数学字体,因为它不全
% 所以要设置字体为times风格:使用字体包 newtxmath
\usepackage{newtxmath}
\usepackage{bm} % 可使用mathbf加粗公式中的字母

%%%%%%%% 符号输入 %%%%%%%
\usepackage{textcomp} % 使用\textcelsius 输入摄氏度
\usepackage{pdfpages} % 用于导入现成的pdf页面。底层是graphics,graphis的相关参数可以使用。

% 文档开始
\begin{document}
哈佛大学积极心理学讲义

\section{介绍}
\subsection{什么是积极心理学}
早上好。很高兴回到这里。很高兴在这里见到你们。我之所以教授这门课,是因为在我像你们现在一样还是本科生时,很希望有人能给我上这样一门课。但这这并不意味着它也是你想上的课,或者说适合你的课。我希望几堂课后,告诉你们这门课讲的是什么。这样你们就能确定它是否适合你。

1992年我来到哈佛求学。一开始我学的是计算机集中器。然后大二期间突然顿悟了。我意识到我身处在一个人杰地灵的地方,周围都是出色的同学,优秀的教师。我成绩优异。擅长体育运动。我当时在校队打快速壁球(垒球按飞行速度分为快速、中速、慢速、超慢速)。社交也游刃有余。一切都很顺利,除了不快乐。而且我也不明白为什么。也就是在那时,我决定要找出原因,变得快乐。于是我将研究方向从计算机科学转向了哲学及心理系。并思考这样一个问题:“如何才能更幸福?”

时光荏苒,我的确变得更幸福了。而我幸福的秘诀就是……我接触的新领域,那时并未正式命名,但本质上属于积极心理学范畴。研究积极心理学,并把其理念应用到生活中让我变幸福了。并且这种幸福一直持续下去。我意识到发生在我身上的这些改变,于是我决定与更多的人分享我的收获。于是我决定成为一名教师教授这门学科。这就是积极心理学。1504号心理学课程。我们将一起探索这一相对新兴且引人入胜的的领域。希望我们能从中悟出课程以外的东西,即探索我们自己。

我第一次教这门课程是在2002年。当时以探讨班的形式。只有8名学生,中途有2名退出了。只剩我和其余6个人。一年后,学生稍微多了一点,有300多人参加。到了第三年,也就是上一次开课,有850人参加。是当时哈佛人数最多的课程。这引起了媒体的关注,因为他们想知道原因。他们想弄清楚这个现象。“竟然有比经济学导论更热门的课程。怎么可能呢?”于是我被请去参加各类媒体采访,包括报纸、广播、电视。

采访过程中,我注意到一种有趣的模式。我先参加采访。结束后,制片人或主持人会送我出来,说些诸如:“Tal,多谢你抽空参加采访。不过你跟我想象的不太一样。”的话。我并不在乎,不过总得回应人家。因此我漫不经心地问:“有何不同?”他们会说:“这个嘛,我们以为你很外向”。下一次采访结束时仍是如此。“多谢接受采访。不过Tal,你跟我想象得不太一样”。我又随口一问:“有何不一样?”她说:“这个嘛,我们没想到你会这么内向”。下一次采访仍是如此。“有何不同?”“这个嘛,更开朗,更外向”。下一次采访是:“这个嘛,你太害羞了”。因为采访中我容易紧张。

差不多有几十次采访。每次都是“更活泼,更开朗”。“别内向,外向点”。诸如此类。最绝的一次是波士顿一家地方台。我去参加采访,聊了很多。我认为确实是一次不错的采访。主持人是个热情开朗的男生。采访结束,他送我出门,揽着我的肩说。“多谢接受我们采访”。然后又是那句。“不过,Tal,你跟我想象得不太一样”。我问:“有何不同?”你要知道,那时候我已经被打击惨了。不过我还是漫不经心地问:“有何不同?”。他看着我说:“我也说不上,Tal,我以为你会更高些”。更高些?什么?是1米72不够格传授快乐吗。

关于这种对话模式我考虑了很久,仔细从头到尾思考了整件事。我似乎明白为什么他们期望我是另一个样子了。因为他们要先说服自己,才能说服观众。“这门课怎么会比经济学导论更热门”?唯一的解释就是。授课者非常活泼外向,魅力四射。当然还身材高大。可惜我的名字少了一个L。要真是他们想的那样就好了,真可惜。问题在于,他们找解释找错了地方。换句话说,他们不该关注信息传达者。而应该关注信息本身。

我怎么知道的呢。因为我旁听过其他大学积极心理学课,遍及全国乃至全球。美国有超过200所大学开设了本课程。而且几乎其中所有院校的这门课都是最受欢迎的课程。这就是知识的力量。越来越多的机构组织开设这门课。包括咨询公司,其中一些甚至是全球知名咨询公司。越来越多的中学开始引入积极心理学。小学也是。各国政府都对这一新领域表现出兴趣。为什么?因为它有效。它的的确确有效。

你们看,直到最近,自助运动(一种自我完善运动,一般通过派发流行的自助书籍传播)都还是殷盛人生、快乐、幸福感的主导精神。心理自助运动带来了什么。我们可以看到生动有趣通俗易懂的书籍,热情外向、魅力四射且身材高大的宣讲者。吸引大众参与他们的讲座和课堂。但是,这个“但是”很重要。其中许多书籍讨论都缺少实质内容。往往都是语言上的巨人,行动上的矮子。比如,《获得快乐你需要知道的五件事》《成功领袖的三要素》《成功、幸福、完美爱情的秘诀》。夸大其词但效果甚微。

再来说说学术界。学术界给我们带来了什么。大量精确的实质内容。数据被一而再,再而三地反复分析。行之有效的好东西。但是,又有一个大大的转折。很少有人会阅读专业学术期刊。想想看,课堂之外有多少人读过最近12期《人格与社会心理学》(The Journal of Personality and Social Psychology)杂志?很多人都没听过这本杂志。我的博士课题带头人曾经估算过:学术期刊上的一篇论文平均只有7人阅读。其中还包括作者的母亲。我虽然是半开玩笑,但其实很可悲。因为作为学者,我觉得很可悲。因为这些论文都非常精彩,非常重要,能大有作为。但是对大众来说晦涩难懂。所以我们需要积极心理学,需要这门课程。

\uline{积极心理学及本课程的宗旨是要在象牙塔及大众间构建桥梁。}换句话说:就是要把精确、事实、经验基础,这些源于学术界的知识结晶与自助运动和新时代运动(新时代运动是一个非宗教的西方精神运动)相结合。充分发挥两者所长。这也是积极心理学大受欢迎的原因:它是一门实用科学。

本课程将分为两个部分。第一部分会和其他心理学。或者其他任何课程一样。我会在这里给你们上课,引导你们进行研究,向你们介绍严谨的学术工作。需要你们撰写报告,学术论文,和其他课程一样参加考试。而教学内容的另一部分。你们读每一篇文章,写每一篇文章时。都需要思考,如何把这些理念运用到生活中,运用到交往中,运用到社交中。这两个部分就是学术与应用。无论是阅读材料还是课上提到的,我不会因为某个理论有趣而去介绍它。而是因为这个理论既严谨又能被应用。

再说几句题外话。有几个问题课前就有人问我了。不幸的是这学期将是我最后一次在哈佛开设积极心理学或其他课程。但愿两年内,明年可能性不大。但是两年内,学校会再次开设积极心理学课程。但我无法保证。

1. 关于反馈与提问。如果你有任何问题或不明白的地方。如果你同意或反对什么观点。请给我或助教写邮件。我们一定会回复的。如果某个问题问的人数较多,我们会公开回答。当然是匿名的,除非你特别注明可以提及你的名字。有时候讲座过程中,突然有紧急情况或有什么非问不可的问题。无法等待。如果那样的话请直接举手。因为就跟你要去厕所一样。憋不住了,无法等待。想去就去。

2. 紧急情况下…… 我们会为此中断课程,稍作休息。所以大可以打断我 我会回答任何问题。

3. 所有幻灯片以及课程视频都会放在网上。课后几天就能下载。幻灯片其实课前就能下载。这样你上课时就能用到。可惜视频不能提前提供,我们试过了。但效果不好。所以会在课后一两天内放到网上。这样做的原因。首先,我更希望你们出席课程。能出现在课堂里和大家一起在课堂的气氛中学习,而不是仅仅面对着电脑。我之所以把资料放在网上是为了让你们可以重温或者学习错过的课程。另一个重要的原因,之所以提前提供幻灯片。是因为我希望你们能充分理解材料并参与课堂讨论。而不是忙于记录我说的每一个字。记住每一个词,背诵每一句话。

4. 我不希望你们被动地记录幻灯片上的内容或者我说的话。而是要主动记录,也就是要充分理解材料。比如,如果你们听到某个理论觉得挺有趣的,那就标上星号写下来。或者觉得也许我可以应用这点。那就写下来。或者跟我妈妈讲讲。或者跟我室友队友讲讲。那就写下来。

主动笔记与被动笔记有两方面不同。首先正如我刚才说的。这门课是关于如何改变生活。我不会仅为了学术之美而教授此课。虽然这一领域的确有许多学术之美。所以发现可以实际运用的就写下来。第二个原因是主动参与会使你记住更多更牢,更好地理解材料。

5. 贯穿整个课程。从下周开始。我们将进行我所说的“练习时间"”而不是"休息时间"。其实类似"休息时间"。这段时间会我们停止课程,进行内省。这是真正意义上的课堂上的安静时刻。我会停一两分钟,你们可以盯着我或者周围人发呆。或者思考一下之前讨论的内容。或者思考如何解答我提出的引入问题。

这是今年新提出的,上学期并没有。因为上次课程结束后到现在的两年,我做了大量关于安静的研究。读了很多关于安静时刻的重要性的研究。无论是课堂里,讲座里,还是家中。无论是对公司领导,爱情关系,甚至学龄前儿童。你们中很多人在经历"练习时间"时可能会疑惑。“我一年付四万美金就是为了这个?坐在教室里发呆”?首先,每次只有一两分钟。一堂课最多不超过两次。其次,这可能是你从本课程中学到的最重要的东西,即享受安静这一理念。

我来读一段麻省理工两位教授的研究。我所提到名字不会出现在幻灯片上,不必背诵或记录。只是为了启发你们。David Foster和Matthew Wilson教授都来自麻省理工大学。他们研究证实了练习时间或内省时间的重要性。他们做的工作是,老鼠处在迷宫中及脱离迷宫后分别对它们进行脑扫描。以下是他们的发现。试验结果表明,当某种经历正在进行时。即老鼠进行迷宫时。真正的学习阶段。是当你尝试分辨什么才是重要的、什么该舍、什么该留。这些是在经历之后进行安静的自省时发生的。

他们的试验表明。接连不断反复进入迷宫的老鼠比进行一次迷宫后稍事放松甚至来点小酒的老鼠学到的少得多。这很能说明问题,对人类也一样。不仅仅是实验小白鼠,所有人类都是。他们认为“重现可能形成一种学习记忆机制”。包括学习、理解、记忆、保留。当我们思考时,我们回想重现素材时。更容易记住之前的经历。所以休息时间的重要性再怎么强调都不为过。

Parker Palmer在他的教学著作《教学的勇气》一书中提到了以下一段话:“语言不是教学的唯一媒介。安静同样可以进行教育。安静让我们有机会反省我们所说所闻。在真正的教育中。安静为学生进行内省提供可靠环境。是一种最深层次的学习媒介。”而安静恰恰是我们文化所缺失的。很多人可能读过《万里任禅游》,作者Robert M Pirsig。还写过另一本书没那么知名,叫《寻找莱拉》。这本书是对美洲原住民的人类学研究。将他们的文化与美国传承的欧洲文化进行对比。两种文化显著区别之一是。印第安人崇尚安静。他发现和印第安人坐在一起,围坐在篝火边两三个小时。一句话也没说。只是坐在那儿。看着对方。微笑享受美好时光内省。就这样几个小时。他指出在我们文化中沉默让人不适。我们试图打破沉默。这是一项重要的文化差异。我们为缺乏安静付出了代价。当我们谈到恋爱,美德,道德以及快乐和幸福时,我们会重点讨论这一代价。

下面介绍一下积极心理学的背景。它是如何诞生的以及这门课是如何诞生的。从很多方面来说积极心理学是人本主义心理学的产物和衍生。人本主义心理学本质上是对当时各种心理学派系的一种反应。人本主义心理学的创始人称其为心理学上的“第三势力”。为什么是“第三势力”?

因为第一势力是行为主义。代表人物有斯金纳、华生、桑代克。这是第一势力。第二势力是精神分析学。创建者包括弗洛伊德、荣格以及阿德勒。这是第二势力。第三势力,人本主义心理学作为对其的异议出现。首先是对行为主义的异议。行为主义认为人的主体性是一个行为集合。就像一只被强化、惩罚、奖励驱动被击打而四处滚动的台球。而人本主义心理学认为我们不只是被击打的台球。我们有精神有灵魂。我们有重要的认知与思想。不能只靠行为理解和改善人生。然后是第二势力精神分析学。精神分析学主要通过潜意识分析你的理解方式。即你的理解方式决定如何改善生活。还有防卫机制、生物本能、神经症。如果你理解这些黑暗势力,就能更好地处理、了解并改善生活质量。

人本主义心理学认为人类不止如此。不仅仅是生理本能。不仅仅是神经症。不仅仅是牛顿学说世界里的台球。我们要重视人的本质。给予人更多的尊严和自由。但有一个问题,人本主义心理学缺少严谨的方法论。虽然它引入了许多精彩的理念。比如对于幸福感的研究。乐观主义的研究。善良、道德、美德、爱、两性关系、巅峰体验、自我实现、移情。所有这些精彩的概念都会在本学期探讨。

但人本主义心理学的认识论并不严谨。如何形成理念,如何学习?所以在很多方面,大部分成为了自助运动。有趣、有益、重要的理念。意图当然是好的。但某种程度上,缺乏学术严谨性。所以它在学术上影响很小。所以很少有大学开设人本主义心理学。几乎没有。这就是为什么许多都演变成了新时代运动。但我们很快会了解到是人本主义心理学孕育了积极心理学。所以我们先见见祖父祖母。

比如Rollo May(美国人本主义心理学家,也是存在心理治疗的代表之一)和Carl Rogers(人本主义心理学的理论家和发起者)。还有最著名的Abraham Maslow(马斯洛,美国著名心理学家,第三代心理学的开创者)。曾是美国心理学会主席,是布兰迪斯大学教授。他于1954年提出了人本主义心理学。他写了一章《走进积极心理学》。1954年他在其中写到我们需要研究善良、美德、快乐与乐观。可以说他走在了时代的前列。

如果说马斯洛是祖父。那Karen Horney(卡伦霍妮,德裔美国心理学家和精神病学家,新佛洛依德主义的主要代表人物)就是祖母了。她最初是精神分析学者,学习弗洛伊德的理论。意识到其过于注重消极面、神经症、精神病。她认为,还必须关注那些在人体内起作用的东西。我们需要研究培养那些好的品质。因为它们也是我们的一部分。从而拓展了人本心理学。并由此产生了积极心理学。

还有Aaron Antonovsky(美国以色列社会活动家及学者),第三位祖父级人物,提出了关注健康的理念。他提出了或者说引进了一个新概念:健康本源学(他个人创造的新词)。由两部分组成,saluto健康,geneis起源。健康的起源。这是病理学常规模型的替代模型。也就是说除了研究病理学,无论是生理健康 还是心理健康,还需要研究健康的起源。这也是预防医学所关注的。这在1970年代 是一个全新的理念。我们会仔细讨论Aaron Antonovsky。现在转到父辈。

Martin Seligman,被认为是积极心理学之父,与一群相关学者于1998年确立了这一领域。和马斯洛一样也是美国心理协会会长。他任职期间的首要任务是实现两个目标。第一,让学院式心理学变得通俗。也就是说连接象牙塔与普罗大众。这是他任职期间的第一目标。第二,是引进一个积极的心理学。需要着眼于有用的东西。不仅仅是研究抑郁、焦虑、精神分裂、神经症。还需要关注爱、两性关系、自尊、动机、恢复以及幸福感。他提出了这些理念从那时起便蓬勃发展起来。这些都发生在1998年。

在Martin Seligman之前,1998年前,Ellen Langer教授就已经研究了这些领域。将人本主义精神与学术科学严谨性结合。我们对她的讨论会比对其他人多。还有一位哈佛的教授。积极心理学的另一位父辈Philip Stone。两年前的昨天去世了。两位都是我的心理学导师,带我进入了积极心理学领域的研究。

1998年我第一次参加积极心理学峰会。Stone教授带我同去。我那时是他的研究生。1999年,他首次在哈佛开设了本课程,在全球范围内也是首批。我是他的助教。两年后,他又重新开设了课程,我仍担任助教。后来我毕业了。他提议我接手他的课程。一直到现在,这就是1504号心理学课程。

我再来讲讲…… 接下来的半小时。我会向你们介绍这门课的内容。首先 这门课不光是传授信息。而且明确地指出要变形。这是什么意思?如今大多数教育都只是传达信息。什么是信息。比如我们有一个容器,也就是我们的思想。信息就是关于接收数据,接收科学,接收信息,并将其储存到容器里(填鸭式教学)。这就是信息。等容器填满了,我们就毕业了。信息数据是越多越好,但这还不够。因为信息本身无法决定我们的幸福感。我们的成功,自尊,动机水平,恋爱关系及其质量才是其关键。除了信息以外还有其他的因素。所谓“变形”则是把容器的形状改变。

trans即改变form即形状。改变形状,这就是变形。我第一次听到这个概念,是在上学时Robert Kegan教授提出的。只有信息还不够,举个例子。你去参加运动会,目标是进入前三,获得奖牌。但是只获得第八名。你会如何分析,你会如何解读?“太糟糕了,我彻底失败了”。你灰心丧气,感到无力。但从另一角度看。同样的比赛你期望获得前三,但只得到第八。你可以解读为“我从中学到了什么?我还需要更努力地训练”。你会变得更有动力。从经验中学习。也就是说,同样的客观信息。“我是第八名,我目标前三”。同样的信息。截然不同的解读。一个认为是灾难。另一个则当成机遇。一个让人失去动力,另一个增加动力。

还有一个很普遍的例子。世上有很多人似乎拥有了一切。人生顺利,生活富庶,但仍旧不快乐。而另一些人拥有的不多,但从未停止享受人生,热爱生活。还有另外一种情况,拥有一切的人充满感恩享受生活。生活窘迫的人觉得自己是受害者。也就是说,重要的不仅仅是获得了什么信息。还有是何形状。如何解读,如何理解,关注的焦点,都取决于容器的形状。这是我在读大学时所认识到的。表面上看,我拥有了一切。体育,学习,社交都很成功。但是我对生命的理解,关注和解读并不正面。我不快乐。我们会看到解读通常比信息更重要。

有一句话我会在课程中经常引用。快乐取决于我们的心态,而不是地位高低或存款多少。因此,我们需要转变心态。这对建立幸福感来说很重要。所以我们在实际操作时,不会传达过多信息。而是挖掘更多东西(不是联想意义上的,而是学术意义上的)。也就是说,我们要挖掘自身潜能。这种潜能一直存在,只是我们没有发现,或者被其他东西掩盖了。我们要发现和利用它,以便关注它理解它。

我讲个故事来说明吧。比如米开朗基罗。曾经有个记者问他:“您是如何创造出《大卫》这件巨作的”?米开朗基罗回答:“很简单,我去了趟采石场。看见一块巨大的大理石。我在它身上看到了大卫。我只要凿去多余的石头。只留下有用的。凿去多余的石头之后,大卫就诞生了”。虽然说的比做的容易。但是这个故事道出了这门课程精髓。即凿除多余石块。也就是摆脱限制和阻碍,或者对失败的恐惧。这些东西并不是与生俱来的,但如今的文化和环境下却出现在了大多数人身上。要凿除完美主义,它使我们虚弱,甚至伤害我们。凿除成功的能力,因为我们可能害怕成功。也许我们可能对生活中所拥有的东西感到内疚。这些都会反过来限制我们。也许甚至还要凿除恋爱关系中的限制。明白我们在恋爱中失利的原因。这些就是这门课的主要内容。 

就像梭罗说的那样。“减法比加法更能使灵魂成长”。减法包括除去那些阻碍我们发挥潜能的限制。因为我们天生有潜能。我们会关注人类本性。它是与生俱来的,无论是上帝赐予的,还是进化产生的。我们有许多潜力,但是渐渐地被外部声音与文化中的某部分所限制与禁锢。老子说过:“为学日益,为道日损”。学即信息。道即变形。 

我最近参加一次大型学术交流会之前,接受了一家相关期刊的采访。采访者问我:“能给读者传授些积极心理学方法和建议吗”?于是我谈了时下一些热门话题。感恩的重要性,体育锻炼的重要性,花时间经营爱情的重要性。谈到了休息、简化等等。我正滔滔不绝时,她打断了我,说:“谢谢,Tal,这些都不错。那些事情的重要性我明白。但这些我们的读者都已经知道了。我想要的是轰动的因素,能让我意外的东西。能否告诉我们的读者?”我想了一会这个问题。然后意识到根本没有什么轰动的因素。我跟她这么说了:“所谓的轰动之处就是没有轰动之处”。就是这样:没有经过转变,快速见效的说法都是皇帝的新衣。是不存在的。是语言上的巨人,行动上的矮子。

一个美好的令人满意的丰富的生活包含了起起落落。包含痛苦和再次振作。包括了失败和东山再起。包括成功和庆功。盛衰荣辱起起落落。我们下个星期会讲到。没有什么秘诀能让人过上幸福生活。而你们将在本课程中学到的许多东西都是你们之前听说过的。也许对你来说没有什么新鲜的内容。你们早已知道。你会说这都是常识。是的,很多都是常识。然而伏尔泰曾经说过:“常识往往不平常”。特别是在实际运用中。所以本课程的目标是让常识更为人所知。尤其是应用到实际中。

如果你打算上这门课的话,在本课程结束之时,在本学期结束时,我不想你跑来告诉我:“Tal,感谢你教给了这么多新东西”。那不是我所期待的,我觉得那也不会发生。我希望的是你过来跟我说:“感谢你提醒我注意一些我本知道的事情”。这就是本课程要做的。我会经常性地提醒,一周两次。经常提醒你们记起你们已知的东西,你们内心深处的东西,你们“心中的大卫”。本课程希望做到的是帮助你们凿掉一些束缚,不论是阻碍你认识已熟知事物的认知束缚,还是阻碍你从已熟知事物中获取益处的情绪束缚,抑或是行为束缚。

ABC准则、影响、行为和认知。我们在谈到改变时会讲到这些。我要让常识更平常。信息本身还不够,因此要在信息高速公路上增加变形高速公路。阳关大道或是羊肠小道来跟上快速增长的步伐。因为就像我们下次课会说到,抑郁率呈上升趋势,焦虑率呈上升趋势。不只是在美国,是全球化的现象。简直就是全球传染病。要想解决它,有再多地信息也没用,并不够。 

下面是Archibald MacLeish的话。他生前是一位诗人,是哈佛的教授。他说:“错的不是科学中的重大发现。有信息总比无知强,不管是何种信息,也不管是何种无知。而是错在信息背后的信念,认为信息将改变世界的信念。但他不会”。仅仅往我们的容器里填入越来越多的东西,越来越多的信息,越来越多的数据是不够的。我们需要更多。本课程将采用人性的授课方法。我来给你们读一小段Abraham Maslow对这种方式的看法:“如果有人学了一门心理学的课,或者看了一本心理学方面的书。大部分内容在我看来都是偏离主题的。即偏离人性。大部分内容都把学习当作联想的形成。技能和能力的获得这些外在表现。这些并不是人性的本质。也不是人和人格的本质。”。外在是指信息吗,内在是指转变。形态的改变。当我们谈到转变时。实际上是非常字面的说法。形态的改变,大脑的改变。我们一会就谈到,比如我们会谈到冥想。 

我们现在知道,自1998年起,各种研究的核磁共振成像显示大脑是可以被改变的。一个新的概念叫做神经形成或者神经可塑性。即我们的大脑实际上是在变化和改变的。在我们的一生中其形态都在改变。所以我不只是在比喻,我说的也是字面意义。

继续Abraham Maslow的话:“人文哲学衍生出关于学习,教学和教育的新概念。简单地说这个概念认为教育的功能,教育的目标,人类的目标,人性的目标,那些所有被人类关注的目标,最终都是人的自我实现:成为一个完整的人。达到人类或者特定的某个人所能达到的极限。通俗点说就是帮助一个人成为你能成为的最好的自己”。这比军队的征兵广告出现得还早。做最好的自己,这就是本课程所要讲的。这就是人性的方法。关于开发我们的潜能,解除那些束缚。这对许多人来说也许很天真很理想主义。天真谈不上,但倒很理想主义。我们也会谈到和讨论理想主义和保持理想主义的重要性。如果我们要介绍个人的改变,人与人之间的改变,或群体与社会的改变的话。

下一点,这门课不是提供关于美好生活和幸福的答案的。而是关于辨识正确的问题。“问过之后便会有收获”,圣经如是说。这门课涵盖了我所理解的教育宗旨。也就是对信息和转变的探索,必须由一个问题开始。探索(quest),问题(question),这两者在词源上有所关联并非巧合。在本课程中,你们会提问和被问许多问题。你们将会看到是这些问题创造了现实,我们下次会讲到关于提问的重要性,无论是问自己,问你的搭档,学生,父母,未来的雇员和同事等等。提问十分重要。

Peter Drucker(彼得德鲁克,现代管理之父)说:“在管理决策中最常见的毛病是只强调找到正确答案而非正确的问题”。Peter Drucker是20世纪最有影响力的管理学大师,最近刚刚去世。他说最大的错误是没有问对问题。正如我们下节课会讲到的,这也是在研究中潜在的最大的错误,也是在实践中最大的错误。没有问对问题。不论是管理组织个人的生活规划这点都适用。好了,当我说问题重要而答案没那么重要时,并不是从相对论的观点来说的,我不是相对论者。我认为掌握有些问题的确切答案是很重要的。但是,我要表达的意思其实是,在教育上求知与提问是同样重要的。

教育家Neil Postman(尼尔波兹曼,世界著名的媒体文化研究者和批评者)曾经说过。"孩子进校时像问号,而毕业时却成了句号"。我的希望是本课培养出更多的问号而非句号。再次重申,这门课主要是帮你凿掉多余的石头。因为在孩童时期,我们总在问问题。对什么都很好奇。我来放一段我最喜爱的心理学家之一的录影。其实他是喜剧演员宋飞(美国喜剧《宋飞正传》男主角,他在剧中扮演自己)。他讲述的是……我们这个学期会看很多录像。他讲述的是我们孩童时候的样子。

Video:
头两年我自己做道具服,必定特烂啊。  
装成鬼啊,流浪汉啊什么的。  
然后第三年我终于从父母那儿苦求得了件万圣节超人装。  
这也没啥。  
纸板箱,自制的上衣还有面具。  
记不记得面具后面的皮筋,那玩意儿质量不错对吧。  
我激动了大概10秒钟左右。  
完了皮筋就从固定破图书订里弹出来了。  
你到了第一家。  
“不给糖就……”啪皮筋断了。  
我都疯了。  
“等上我啊,我弄好它。”  
“喂,等上等上(Wait up)。”  
小孩就那么说话。  
他们不说等等我。  
他们说:“Wait up”。  
因为在你小时候,整个人生都是朝上的。  
未来在上方,想要的东西也都在上方。  
"Wait up, hold up, shut up."  
"Mama, clean up. Let me stay up."  
父母总是相反的,总是下啊下啊。  
"Just calm down. Slow down."  
"Come down here. Sit down. Put that down."  

所以孩子们的这种好奇心。这种向上看的态度。这种开放的心态是与封闭相反的。我希望在本课程中能产生这种观念。教育的真正目的,在于提供一个让人可以连续发问的环境。

所以下面讲讲John Carter所作的纵向研究。John Carter河对面那边商学院的领导管理学教授。他于1972年来到哈佛做老师,并开始追踪哈佛1973届的MBA班的学生,一追追20年。他想尽可能收集这个班级的所有信息。20年后,也就是90年代初。当这个研究结束时。他发现这些学生都极其成功。坐拥丰厚的财富,手握深远的。不论是在组织上,还是社会上。他们都表现出色。但是在这群极其成功的哈佛MBA里,他发现有一小部分人是格外成功的。比其它人要更成功。无论是在收入上,还是在影响力上,抑或是在总体生活质量上。格外的成功。他想弄明白其中的原因。这一小部分人和其他人有何区别。 

他只找出了两点。无关乎智商,智商对他们持久成功没有影响。和念MBA之前的出身也没关系。他们的成功和这一点关系都没有。只有两件事决定他们。是否处于这个异常卓越的小群体内。第一,那部分人完全相信自己。他们深信自己能做好。他们鞭策自己。他们鼓励自己。以后讲到自证预言的时候我们还会展开。他们坚信:“我会做到。我会成功”。这是第一点,自信。

第二点是,这部分人一直在问问题。一直问问题。最初是问他们的老板。后来是问他们的雇员。他们的搭档,孩子,父母,朋友。他们一直在问问题。他们保持好奇心。一直向上看,抱着开放的心态,想要更加了解世界。他们不会说:“我现在有了MBA学位就行了,我知道的够多了”。他们是活到老学到老。他们一直在问问题。这两个区别性的特质。导致了格外成功和成功的人之间的差别。

有个问题给予了我写书的灵感,促使我开设这门课程,对我的个人生活产生了影响。被我称为“问题的问题”。就是“我们如何能帮助自己和他人,个人,群体和社会变得更幸福?”注意不是帮助我们自己和他人变得幸福,而是变得更幸福。为什么这么说。因为很多人问我。“那么Tal 你幸福吗?”我真的没法回答那个问题。我不知道那是什么意思。我如何判断自己是否幸福?是和别人比较?是不是存在一个点超过之后就变幸福了?幸福不是二进制,非此即彼,0或1。要么幸福要么不幸福。幸福是连续性的。所以针对这个问题“我今天的确比我15年前刚开始关注如何追求幸福时更幸福”。

我当然希望15年后的我比今天更幸福。但幸福需要我们终身去追求。希望本课程能成为追求过程的一部分。但只是一部分。你们不会在课结束时感到更幸福。当然我希望你们能比现在幸福。因为很多人坐在这听讲座时。比如,有关自尊的讲座,或者当我们畅谈幸福时。他们说:“等等,我的自尊心强吗”?他们会反思自己:“我的自尊是较强的还是较弱的”?毫不相关,没有意义,无法回答。问题是:“我如何能提高我的自尊”?健康的自尊,不包括自恋。“我怎样才能变得幸福”?那才是你应该问的,也就是问题的问题。

本课程不是一个积极心理学的概论。如果你想了解关于积极心理学的概论。我可以推荐一些很不错的课本,Lopez或是Peterson的书。还有一本积极心理学手册,一本很厚的书。里面有这个领域里大多数你想知道的内容。你也可以用它自卫,非常管用。但是本很棒的书,写得非常好。讲积极心理学的奥义写得浅显易懂。但我们课上并不讲那些。这门课不是积极心理学概论。而是对“问题中的问题”进行选择性探索。从那个角度讲,这门课是兼收并蓄的(大杂烩)。我的背景是心理学和哲学。研究过组织行为学。做过几年商业顾问,至今还做一些这方面的工作。我在教育领域做过很多工作。 

我把我的这些阅历,不只是积极心理学方面,还包括临床心理学方面的知识结合到课程里来。甚至还涵盖了认知心理学,社会心理学等等内容。这是一个兼收并蓄的课程。因为我引导问题是:什么有助于幸福?如果我认为精神病理学的内容对幸福有贡献,我会兼收并用。而如果一些有关组织行为学的咨询领域的东西有用,那也会成为本课程的一部分。只要能在一个学期内讲完就行。所以本课程兼收并蓄,但它并不涉及文化差异。当然,我会引用一些东方思想。我曾经在亚洲住过几年,在那工作。持续研究过东方哲学和心理学。但我的专业还是在西方心理学上。而课程的重点将会是西方心理学。但是那并不意味着。积极心理学不适用于世界上其他地方的人。

最近有一场高资历科学家与心理学家之间的会议。像Paul Ekman和Richard Davidson等一系列在心理学领域内重要人物出席。和达赖喇嘛还有他的一些僧侣们展开会谈。他们谈论了心理学的未来,心理学研究。还有如何进行冥想研究等等。而他们谈论较多的话题之一就是文化差异。当提到这个问题时,达赖喇嘛突然好像很不舒服。当记录整个会议的Daniel Goleman问他怎么啦。会议是在印度开的,他说对文化差异的过度强调令他深感不适。对达赖喇嘛你可以有很多评价。但你无法否认他在文化方面敏感。他可以说是还在世的最敏感的人之一。他居然都说,我们太过专注文化差异了。随后他补充道,并非是文化差异不存在,但文化共同性要比差异多。而我们不应该无视这些共同性。

Daniel Goleman如此评价达赖喇嘛。“达赖喇嘛对于文化差异的直面抵触使我们稍感惊讶”。所以我其实是乐意介绍这些观点的。但是首先,这不是我的专业方向。在研究文化差异方面的人会比我做得更好。其次,因为我希望所研究的是普遍现象。不同文化中普遍的规律。所以我们会将研究重心放在这个范畴,但不限于此。我们的讨论比心理学这一领域更详尽。我们将会深入研究你们自身。为什么呢?当我开这门课程时,我没想过:“我需要介绍些什么东西,以便能取悦课程的参与者”? 

我没想过这些。我所考虑的是:如果我是个本科生,会想上什么课?如果我坐在台下,怎样才能让我更幸福?换言之,我是从我非常个人的观点出发来思考的。在课上,我会鼓励你们。当然你们要阅读别人的研究报告。浏览大量的样本。但最主要的,我会鼓励你们审视自己的内心。去研究自己。无论是为应付两周后的开始的每周一次的课后论文。还是你们最后的期末作业,要求以陈述报告的形式完成。你们不用做演讲展示,但要以书面形式呈递,以你们最感兴趣或最重要的话题。以段落的形式阐述:我是如何能把这些理念应用于我的生活的。

“练习时间”是用来思索如何接受和利用这些观点,是用来研究我们自己。因为正如Carl Rogers所说:越是个人的东西越普遍。正如Maslow所补充:“我们必须记住,认识自己内心深处,同时也是认识人类的内心深处”。当我们更了解自己,当我们更认同自己,我们就更能好得认同他人。事实上,这从某种程度上说是移情的来源,健康的移情。基于此的研究表明了解自己,研究自己,反省自己的人。较少对别人做出过分的,不道德的行为。比如种族歧视的行为。在一定程度上这是反直觉的:难道不要首先研究别人,才能对别人更感同身受吗?是的,那也需要,不过还不够。了解自己重要。因为当我们看到自己的深层本性时,我们看到的也是共有人性的一部分。不管我们来自哪里,人与人之间都有相似性。 

而这就是达赖喇嘛所说的:不是阻止跨文化交流,交流很重要,但同时不要忽视自身。不要忽视我们每个人身上的普遍性。C.S.Lewis说:“整个宇宙中有且只有一件事是不需要依靠对外界的观察习得的,那就是我们自己。我们有某种内在的信息,我们天生知晓”。当然我们研究自己时同样存在偏见。这就是为什么只研究自己是不够的。我们更要审视它丰富它学习他人。这就是为什么在研究和学习的同时,也要研究我们的内心。两者同样重要。我们不能因为那些可能会犯的错误和偏见而因噎废食,停止探究自己。所以我们会审视自身,比其他任何课程都多。 

最后,你们可能头一次听说,但这不是10A号英语课和55号数学课。就是说你不用像英语课或者历史课那样做大量的教材。这门课也不像55号数学课一样难。这点你们大可放心。

我相信这有些人上过这门课。本课程严谨而有趣。有趣是因为研究我们自己很有意思。有时会让我们痛苦,有时我们看到一些我们不愿意看到的东西。但总的来说还是有趣的,有意思的。于此同时,它的研究也是严谨的。这门课中,你将遇到很多观点是简单的通俗的常识。但它们是简单不是过度简单。这两者是有区别的。Oliver Wendell Holmes,这个报告厅就是他捐建的,但不能完全确定是他所有的……说过:“我不会认为与复杂性一样的简单性是微不足道的,但我会用一生用来研究与复杂性相对的简单性”。

Homles这里的意思是他关心的不是直接的简单,容易的,即兴的,胡思乱想的观点。他所关心的是经过提炼得出的简单。经过咀嚼,经过消化,经过简单的思考。那些经过提炼的思想。如果我们能从复杂的事物中提炼出简单的原理。那就很好,那才是他感兴趣的。并且也是我们这学期将要介绍的那些积极心理学研究者所感兴趣的话题。从复杂中提炼出简单。这两种简单之间有很大的区别。尽管乍看起来也许是相似的。

本课程需要的是一种非常不同的努力。与别的课程非常不同的努力。它不需要你像55号课和10A号英语课那样的努力。它所需要的努力是一种应用的努力。努力把它应用到你的生活里。努力生活中行为的改变。在我开始介绍本课程的一些细节要求之前。我想先讲一个关于Peter Drucker的故事。

我之前引用过Peter Drucker的话,他是现代管理学研究之父。Peter Drucker一直活到94岁。在他生命的最后,他头脑还百分百清醒。但它无法自如地活动。无法去公司机构。因此他邀请那些想咨询他的人,想向他学习的人来他家。包括国家总统,首相。世界500强公司的CEO和他一起过周末。而在星期五,每一次聚会都是这样开始的。对每一个世界级领袖,无论是商界的,非盈利机构的,还是政界的,他都会对他们说如下的一番话:“我不希望你在星期一叫醒我时告诉我这多么神奇”。意思是这个周末有多神奇。“星期一我想听到你们跟我说,你们做了些什么改变”。

在这学期末或课程结束后。如果你喜欢它,请务必告诉我你喜欢。它很有趣。但更重要的是,你们所做的改变是什么。这对你们的生活有怎样的影响,那需要我们付出努力。我们将用一周的时间只讲改变,这和积极心理学一点直接的关系也没有。只是讲改变。因为改变是那么难。因为我们知道大多数组织的变革都失败了。因为我们知道大部分人的个人改变都失败了。除非我们在改变行为上的同时,改变我们的认知和情感。情感和认知还不够,还要有行为改变。

改变行为模式需要勇气。一些你们要交上来的课后论文都不会被打分 。只会有及格和不及格。你们必须交论文,然后你们就及格了。但有些也许是你所写过的最困难的论文。有一些会是最简单的。很自然地就能写出来。那是关于试图改变,关于反省,关于付出时间,关于提取精华。而且这只能通过这种方式完成。所以如果你真的想通过这门课改变生活,一切取决于你。我将会给你们介绍材料。我将会给你们介绍积极心理学这个精彩新领域。而由你们决定是否接受并应用它。

我想谈一下教学大纲及要求。我会给你们一点时间去问问题。但在这之前。我也想对。我知道你们中的有些人正在家里看这堂课。对进修学校的同学们表示欢迎。很高兴你们能听我的课。有空过来看看。而你们将显然参与到我们的课程中来。但也可以找Deb Levy。他是进修学校的助教。 

Sean Achor是文理学院的助教。我想请他来说几句话。向你们做个自我介绍。而你们会被介绍给我们的其他助教。我们今年有支很棒的团队。这就是Sean。嘿 上午好。能听到我吗?现在能听得到吗? 很好。能回来再次为积极心理学做助教非常荣幸。Tal太谦虚了。很荣幸他能给我们上课。他还把家搬来了。把家从以色列搬到这一整个学期。包括他的妻子和他的两个孩子 只是为了能和我们一起教这门课。对我们来说能和他们一起共事是个极好的机会。我真的很激动。上一次我们教这门课时。我们做了一个调查 看看坐在这上课的都是什么类型的人。他们就像你们一样。看看为什么你们会上这门课。因为我们得到的关于这门课的评价常常是。为什么哈佛的学生会不幸福?他们有什么可不快乐的?他们认为每个上这门课的人会上这门课。是因为他们已经很幸福了。他们想研究自己到底有多棒。他们想学一些能明白地告诉室友他们得意的东西。但是结果是去年有超过三分之一的人。选这门课是因为他们感觉抑郁。他们想学习积极心理学。还有三分之一是因为他们想学会乐观。另外三分之一的原因则完全不同。 

我想 今年还有三分之一的人选这门课是因为。Tal上了Jon Stewart的Daily Show 我很喜欢这门课。Tal要告诉你们的教学大纲是……。其实我们了解你们的一些其他事。你知不知道选了积极心理学的人中。有75%的人是俱乐部的干部。35%是俱乐部的高层干部。这意味着 你认为哈佛有大约2000个俱乐部。你在一个只有3个人的俱乐部里。又恰好是主席。我们了解到你们不快乐的其他原因。我们了解到上积极心理学课的人。在4年里恋爱次数的平均数在0到1之间。0到1之间。不!我不信。不要走。- 但那是在上积极心理学课之前。- 是在上课之前。我们之后会调查一下你们。性伴侣数目的平均数是在0到0.5之间。我实在不知道0.5个性伴侣是什么。这门课程将会很棒。它过去就很棒 而且我们有一个很出色的教学团队。很大的教学团队。 

Tal实际上已经给我们布置了作业。所以老师们不仅要学你们学的那些材料。和你们讨论专业的内容。还要学习如何成为更好的老师。他给了我们要读的书。他事实上给我们布置了任务。从那个意义上说这是个很特别的课程。至于教学大纲 网上会有。Tal说现在我们决定要环保。所以这门课程中不能使用树木制品 除了这个木制的大礼堂。我们下周会在网上分组。所以我们星期三会知道有多少分组。下个周末我们会做分组任务。给你们的周转时间就非常少了。如果需要的话 你们可以在周一换组。虽然我不希望你们改。分组会在下一个星期开始。非常谢谢 如果还有什么问题 给我发邮件。 

好的 那么……。我想我不需要跟们解释。为什么这学期有一节课不是我教而是Sean教的。而那节是讲幽默的。你们自己想想为什么。我确实认为如果你想教什么你就必须擅长。教学大纲。本课程 是向你介绍我关于这门课程的想法的。是我所谓的。整体的课程。就是指纵向的整体和横向的整体。纵向意味着每一节课都和下一节课有关系。和整个学期内的每一节课都有关系 呈一个螺旋状。所以我们下周所讲的内容。我们将在第7课 第17课和第24课里。再回顾一遍。所有内容在一个更高的水平上呈螺旋状互相联系。换言之 会带来深刻的理解和我所希望的对材料的吸收。所以是纵向相互联系的。同样也是横向相互联系的。就是说课程中的每一部分都强调并影响另一部分。 

上课还不够 你们在分组作业里面将收获更多。分组作业是强制的。你们要在你们的分组里做大量。和你的课后论文相关的工作。许多学生说这是本课程最重要的一部分。因为这样你们才有了练习时间。这是你真正开始凿掉的时候。和期末的大作业有很大的关系。期末作业是一个演示稿。你们不用真的做演示 但要向几个朋友演示以获得反馈。那部分不计分。期末作业是以期末论文计分的。但只是你们交上来的。你们提交的那份。为什么写期末论文 是因为像演示那样的最好的方式。因为最好的学习方式是教。你们将把这些材料教给别人。任何在你的陈述中你所感兴趣的内容。那么阅读。非常有关系 会把你们带到理解和吸收的另一个水平高度。在我讲完之前有问题吗? 你们有什么要问的吗?好了 那就让我说 最后一秒钟结束语。对于能回到这里我感到非常激动。我的家人对于回到这里也很激动。期望能和你们共度一个有意义的 愉快的。令人享受的和更加幸福的学期。谢谢。 

\subsection{为什么要学习积极心理学} 

早上好。很高兴你们都来了。我还担心在这种天气没人会来。很高兴你们都很好。首先做几个声明。我通过邮件收到的几个问题。在开始课程之前我要说的几件事情。首先 有人问我"这门课程是为谁设置的?"。"这是积极心理学"。"它是为非常不开心的人设置的吗?"。"它只是为抑郁的人设置的吗?"。"这门课程是为谁设置的?"。这门课程的授课对象是。对积极心理学有兴趣的人及想更开心的人。如果你非常开心 你还可以更开心。如果你非常不开心 你也可以更开心。所以这门课程适合任何。对它感兴趣的人。 

我会邀请有兴趣投入。我们上次谈到的那种努力的人上来。你不需要投入那种。为了明白某个概念而大为头痛的努力。而是努力把这些观念融入你的生活。也许你之所以上这门课程。是因为你对积极心理学的理论感兴趣。这也没问题。你将会得到很多……。我们每节课都会谈到非常多的研究。从下节课开始。我们将接二连三地谈到那些研究。所以你也可以得到理论方面的知识。但是 如果你修积极心理学。是为了个人利益 那么你就需要努力。对此今天我会作进一步的阐述。所以这门课程任何人都可以修。包括那些非常开心而且想更开心的人。及那些不开心但是想更开心的人。 

考试成绩 这不是个问题。这门课程的宗旨以你们的利益为上。所以我建议如果你想及格。那么就要努力。因此是否及格的决定因素之一。就是所有的回应报告必须交上来。回应报告不会评分。它们基本上就是心得体会。也就是你会思考能够融入。你的生活的想法和事情吗?它们是必需要交的。你交上来 你就及格。不交就不及格。除此之外。如果你想参加考试 那也没问题。第三件事。在这门课程你将会接触很多理论和观点。不是所有理论观点都会引起你的共鸣。2008年出版了一本很棒的书。就在两周前 书名叫《快乐之道》。它由Sonja Lyubomirsky所著。在书中她谈到了一个概念……。她是加州大学河滨分校的教授。哈佛大学的毕业生 斯坦福大学的研究生。她谈到了找到"适合"的重要性。也就是某种方法。某种工具 某种观点是否适合你。并不是你听到的每种观点。每种研究。以及你运用的每种介入方法……。你在生活中将会运用介入方法。不管是作善举。还是表达感激。还是体育锻炼 还是写日记。这些事情你在整个学期里都要做。然而并不是一切都适合你。你将会接触它。尝试它。然后你将作出决定。"对 我想把它融入到生活中"。或者"不 它对我没什么意义"。所以记住这一点很重要。我所说的一切都有研究论证。但是研究没有说它适合所有人。它说这对于大多数人或者对于很多人。都有作用或者正在起作用。 

所以你要积极参与到课程中。而非某种学说的被动接受者。鉴别哪些东西适合你……不是一切东西。我敢说并非一切东西都适合你。但是有很多东西会适合你。关于我之前提到的回应报告。它们在下午五点钟就要交 很抱歉。从下周二开始你将在每周二下午五点收到。然后在周日下午五点前交到你的任务组。回应报告对大多数人来说。都是有趣而好玩的活动。它不会评分 它让你去思考。并通过它们去成长。要写论文的人。你们中有多少人要写论文?请举手让我看看。好的 很抱歉 我没有在开玩笑。要写论文的人 你们不用参加期中考试。我知道那段时间里非常紧张……。我自己也经历过那段时期。所以你们不用参加期中考试。你的期末考占更多分数。除非你想参加期中考试。当然我们非常欢迎你参加。我们不会把你扔到教室外面。写论文的人可以不用参加。网上将会有很多公告。请经常查看网站。我们会发布很多东西 而不是向你发邮件。我们会发布公告。需要经常查看一下 大概每天六七次。这只是开玩笑。一两天一次就已经足够了。 

在开始课程之前……。今天的课程非常令人兴奋。我们会邀请进修学院任务组组长Deb Levy。哈佛进修学院的同学们。我知道该学院的一些同学在这里。还有一些在家里……。有请 Deb Levy。好的。我大概有二三十分钟的发言时间。所以我会从我小时候讲起……这只是开玩笑。我是进修学院的任务组组长 我们很激动。进修学院有296名学生在选修这门课程。也就是说他们会收看视频。进修学院的学生们。你们会参加电话会议。人们会打电话进来。所以我们将会有电话会议。这是个很好的机会。我们跟新西兰的人打声招呼。还有法国的 肯塔基州的 莱克星顿的人。真是难以置信。另一件事就是 我想拍一张照片。教室里不准拍照。也给你拍一张 Tal。我不会授课。但是Tal和我有着相似的系统知识。所以如果Tal因某种原因缺席的话。我会代你授课。很好。另一件事就是 进修学院的学生们。希望你们要耐心点。我们会尽快把信息发出去。大概下周你们就会分组。能来这里真是太激动了。好的 谢谢你 Deb。 

我想以一个故事开始今天的课程。那是两年前发生在我身上的事。恰好两年。那是我上次教积极心理学的最后一节课。那学期开始的时候。同时那是压力很大的一个时期 我的导师。我要把这节课程。以及将来的积极心理学课程献给。在课程开始前一天去世的Philip Stone。那时候我压力非常大。我生病了 病得非常严重。然而我还是上完了课。我教授周四的课程 它从两年前开始。课程在周二上。后来改为周四 我服了很多药。然而我还是上完了课。我回家后无法入睡。我感到非常痛 所以我去看医生了。那是周五下午 我说。"我必须要看医生 我服的药不起作用"。我去看了医生 验了血。几天无法入睡之后。由于痛苦我终于睡着了。那是周五晚上。午夜的时候电话响了。我没有去接 我很快睡着了。我的妻子Tommy拿起了话筒。那是医生打来的。医生对Tommy说。结果出来了。Tal要立刻到医院来。她对医生说。他刚睡着 他几天没睡了。能不能等到明天再去?医生说 不行 他要到Beth Israel医院。那里有最好的实验室可以给他提供治疗。她没有再说。Tommy叫醒了我 告诉我发生的事。我起来了。她不能带我去医院 因为David。当时他才一岁 他正在睡觉。我们不想叫醒他。我们叫了一辆计程车带我到Beth Israel医院。路上……这是我不做。Leverett的常驻导师资格的第二年。我们驶经查尔斯河。然后我们经过哈佛。我当时看着哈佛 看着美丽的河。四周静悄悄。周五午夜之后没有多少车辆。我情不自禁地想。如果情况真的很严重呢?他们为什么午夜叫我去医院。还指定是Beth Israel医院?一定是出了很严重的问题。我开始思考。我说。如果我只剩下一年时间呢? 我在这一年里会做什么?我变得非常伤心 因为我看不到David长大。我以后再也不会有孩子。在上面要小心谨慎。我变得十分忧郁悲伤。然后我问自己 好的 在职业上。我在最后一年想做什么?我知道在生活在我要做什么。我会用所有时间和家人一起度过。但是在职业上在这一年我想完成什么?我的第一反应就是。我要留下一个系列的课程。一个向人们介绍积极心理学的系列课程。我到达医院之后再进行了一些检查。结果是没什么严重情况。他们让我服抗生素。几天之内我就开始复原了。 

但是今天我要和你们探讨。为什么对我最重要的事情……。那时和现在都是如此。就是留下一系列。关于积极心理学的课程。向你们介绍这个奇妙的领域?为什么是积极心理学?为什么是这个领域?为什么不只是研究。社会心理学或临床心理学的幸福?为什么世界上这么多学者。都在研究积极心理学这个概念?这是我今天要说的。 

2000年 David Myers做了一项研究。霍普学院的David Myers 你们当中学过。社会心理学的可能读过他的书。他做了心理学概念方面的研究。他研究的是。"消极研究"和"积极研究"之间的比率。这是他的研究结果。从1967年到2000年。这是积极心理学的形成期。他发现在这33年里有5000篇文章研究愤怒。5000篇文章研究愤怒。超过41000篇文章研究焦虑。还有超过54000篇研究抑郁。然后他看看积极的概念 积极的研究。他寻找关于快乐的研究。他仅仅找到了415篇研究。情况会有所改善。他寻找关于幸福的研究。他发现在33年里有接近2000篇文章研究幸福。生活满意度研究最多 超过2500篇研究文章。但是 如果你看看。消极研究和积极研究的对比。你得到的比率是21比1。每份研究某个积极方面。积极因素的文章。不管是研究健康 满意程度 快乐还是幸福。都有21份消极方面的文章 包括抑郁 焦虑。精神分裂 神经衰弱等等。21比1的比率。我得承认这是非常令人郁闷的比率。实际上它让我非常生气和忧虑。那些研究主要集中于不起作用的东西上。大部分都是这样 而且这也不是新现象。 

这是Abraham Maslow写的一些东西。我们上次提到过他。他在1954年谈到过这种现象。"心理学这一学科对于消极方面的研究"。"远比对于积极方面的研究成功"。"它向我们展示了人类的短处"。"他的缺点 他的过失"。"但很少谈到他的潜能 他的长处"。"他的实际愿望或精神高度"。"好像心理学自愿固步自封"。"让自己仅限于研究黑暗低劣的一半"。心理学研究什么?学心理学的人。可能会猜。我们大量地研究偏见。我们大量地研究抑郁和焦虑。我们大量地研究相似性。我们大量地研究判断错误和过失。大概就是专门研究这方面而很少……。21比1的比率表明很少研究积极方面。看到这个比率时 我想。那是1954年 到现在也没多大改变。当我思考它调查它的时候。我觉得心理学需要帮助。我真的这样觉得。你可以从个人的角度想想它。如果一个人一天里有21小时感到消沉。而只有一个小时感觉良好会怎样?或者有一天感觉良好 便有21天觉得焦虑。和消沉会怎样?你会说这个人需要帮助。我觉得这个领域需要帮助。但问题是 这是正确的比喻吗?我们应该这样去看待它吗?因为21比1的比率是不健康的。从个人的角度来看肯定也是这样。但是它从多个角度反映了现实。 

因为我们今天所看到的。是越来越多的研究表明。世界上出现了越来越多的抑郁症。还有越来越多的焦虑 东南西北。世界各地。有些人会争辩道……。那些主张继续研究这片领域。而少点研究积极心理的人认为。我们应该让这个比率大于21比1。因为我们想缓和人们受到的。焦虑和抑郁。 

今天的抑郁症病例比1960年高十倍。部分的原因是人们的意识程度高了。我们对抑郁症的诊断更准确。但这并非全部原因。还因为抑郁症在客观上增加了。其中的一种迹象就是 最客观的"诊断"。不幸的是 就是自杀。自杀人数在世界范围内明显上升。不仅是在美国。不管是在中国 澳大利亚还是这里。现在抑郁症人群的平均年龄小于15岁。孩子们在很小的时候就接触了。信息高速公路。通常来说他们都没有准备好。他们无法有效地应对它。所以当我们看到这个数据时。我们认为我们需要更多的研究帮助人们。克服抑郁症 克服焦虑。这应引起足够的重视。这非常重要 极度的重要。我主张的。是天平的偏移。这样它将不会再是21比1。要有更多的研究关于积极心理学。不是要只研究它 根本不是。而是让天平偏移一点。这里的情况如何?我们的象牙塔里的情况如何? 

有一篇文章……我无法找到更新的研究。它是在2004年发表的。《哈佛克里姆森》。这篇文章说 经过他们6个月的研究。哈佛大学八成的学生在过去一年。都经历过抑郁期。我们说的不是大多数人一天里。经常出现的情绪起伏 我也是如此。我们说的是抑郁。它会持续一段时间。根据这份研究 47%的哈佛学生……。这不是学术研究。但是我待会就和你们分享学术研究。它发表在高级期刊上。但是克里姆森里的研究表明。47%的哈佛学生在过去一年。经历过无法正常生活学习的抑郁期。他们无法出门。他们要痛苦地度日。47%的人。人们看到这个数据的时候 他们说。"当然了 我们要努力研究精神病理学"。"21比1还不足够 30比1才像样"。 

这样的情况各个校园都有。而不仅仅是哈佛。绝对不是哈佛才有的现象。Richard Kadison。他是这里的心理健康服务中心的领导。他最近在医学领域的领先杂志。《新英格兰医学杂志》上。发表了一篇文章。他谈到了一项调查。调查对象是全国13500名大学生。他们来自不同类别的学院。大学 公立学校 私立学校。他们在这项重要的研究里发现。全国45%的大学生在过去一年。经历过无法正常生活学习的抑郁期。 

《哈佛克里姆森》的数据是47%。在全国范围的数据是45% 基本上是一样。两者之间的差异并不大。这是一个全国性的现象。这份研究表明。全国94%的大学生感到压抑。因为他们必须要做的事而不堪负重。这本应该是我们一生中最美好的四年。事情不对劲。不仅是美国才有这种现象。我最近刚从一次游历回来。我到了欧洲 包括英国 法国。我在中国逗留了很久 也去了澳洲。所有这些地方。政府都非常担忧……。大学校长们都非常担忧。越来越多的抑郁现象。焦虑 精神紊乱。自杀率的上升 上面提到的所有国家都是如此。所以这是全球性流行病。我们再看看 21比1的比率是好事吗?它重要吗?它不是应该扩大吗?我们怎么可以去研究幸福。健康 爱和快乐?我们不是应该首先处理。真正迫切的问题吗? 比如抑郁。焦虑 神经衰弱等等。 

这样说确实有些道理。但是在这个课程里我要主张的是。我们也要……。不是专门 也不用主要。我们也要集中研究积极心理学。我要说三个为什么应该这样做的理由。首先 集中研究有用的东西至关重要。因为有用的或者集中研究的会成为现实。如果我们集中研究起作用的。它就会在世界起更大的作用。在我们身上以及关系上起更大作用。其次 为什么积极心理学。作为一门独立领域的研究学科。是重要的 这是因为快乐。并非只是对痛苦的否定。摆脱我正在经历的。抑郁或者焦虑并不意味着。我同时就变得快乐了。情况并非是这样 问题不会这样被解决。最后 预防在今天来说非常重要。预防困苦最有效的方法……。不管是抑郁还是焦虑。实际上是通过专注于培养积极心态。我会跟你们分享一些关于这点的研究。这三个原因表明我们需要积极心理学。 

我从专注于有效方法的重要性开始。在《积极心理学手册》的序言里。我们上次提到的Martin Seligman……。他被认为是积极心理学之父。影响了很多学者 他说。"积极心理学的目标是促成一种变化"。"让心理学从只关注补救生活中最糟糕的事"。"到同时建立生活中最美好的事"。请留意他说的是同时。他没有说专门 甚至没有说主要。专注于研究有效的东西至关重要。不管是对我们的关系 对我们自身。对其他人 对哈佛。对美国还是对世界而言。这是为了获得更多有效的东西。问题是"如何专注于研究有效的东西?"。这个问题的答案。就在我们提出的问题的本身。我通过一个案例研究来阐明这点。心������学家在20世纪40年代末开始。研究受危人群。越来越多的钱 政府资金 大学资金。慈善基金都投入到研究城市或地区里。所谓的受危儿童的身上。他们更容易退学。以后更容易犯罪 少女妊娠等等。他们为这项研究投入了大量金钱。大量的精力。心理学家们问的问题是。"为什么这些人会失败"。"为什么受危人群中"。"有如此高比例的学生退学?"。"怀孕?犯罪?"。我们下周再谈统计数据。"为什么他们当中有这么多人会失败?"。他们提出了很重要的问题。心理学家们动机良好 头脑灵活 投入了很多钱。投入了很多资源。然而 变化却非常少。这些研究带来的变化微乎其微。很多地区的情况继续恶化。情况没有任何好转 虽然动机良好。虽然资源充足 虽然人们头脑灵活。它们都被投入去研究这个问题。得到的答案却非常有意思。我们需要更好的教育 更好的建筑。更多的资源。但是实际上发生的变化很少。 

然后出现了一种范式。20世纪80年代出现了范式转移。通过Antonovsky的努力 我上次提到过他。今天我要再次提到他 通过Antonovsky。及Ellen Langer和Alice Isen等其他人的努力。心理学家们提出了不同的问题。他们不去问为什么这些人会失败。积极心理学家开始问是什么让某些人。成功了?即便面对的是不理想的环境?也许是有很多人失败了 但并非所有人。有些人成功了 而且非常成功。心理学家们此时便开始问 为什么。为什么他们如此成功?用Frost的话来说 这才是至关重要的。心理学家们开始识别这些人的各种因素。以及各个方面并深入分析他们。他们开始研究这些成功的人。并识别出各种因素。然后他们通过研究想出介入方法。 

突然间结果出来了 真正的结果。切实的结果 打破了数十年零结果的局面。这仅仅是基于一个简单的问题。那时候出现的主要概念。通过这些心理学家的研究。正是他们开始问积极的问题。开始专注于研究成功的孩子们。那时出现的概念是心理弹性。现在大家都在谈心理弹性。我们在学校谈心理弹性。在工作上谈 在卧室里谈。在哪里都谈心理弹性。然而 在20世纪80年代。很少人谈及甚至很少人知道它的意思。研究心理弹性的时候 它被定义如下。它的定义关系重大。心理弹性。一种现象 特征是积极适应的模式。即使是面对非常不利或危险的环境。这些成功的孩子最终成功了。这是通过纵向研究得出的结论。五年后 十年后 三十年后。这些成功的孩子都具有适应力。一开始 当他们研究这些孩子时。他们认为这些孩子一定很优秀。百里挑一甚至千里挑一。因此是不可以模仿的。然而。当他们继续研究。这些成功的人时 继续研究虽然环境不理想。然而仍然成功的人时。他们发现这些并非是非常优秀的孩子。实际上这些是普通的孩子 他们性格普通。但是成就非凡。例如 我告诉你们几件事。这些孩子都很乐观。不是那种不切实际的盲目乐观。而是乐观地相信。事情会被圆满地解决。 

我们会经常把乐观主义当作诠释谈及。也就是Martin Seligman和karen Reivich的乐观主义。他们的乐观在于认为。"好吧 也许这次不会成功"。"它以后会成功的"。"我从刚发生的事学到了东西"。他们对生活有信心 认为它有意义。有时它是宗教信仰 但却不总是这样。这种信仰是做他们相信的事。他们中很多人都是理想主义者。我们在这门课程里谈及的话题之一。就是理想主义就是现实主义。因为我们的内心需要理想主义。所以这些孩子有"意义感"。不管是个人成功方面的意义。以及他们的行事方式的意义。或者是服务社区的意义。还有事情的目标意义。顺便说一下 我说的这一切。它们对所有人都一样重要。我们之前提到的心理弹性。不但对受危人群很重要。对哈佛大学里的人也一样重要。也不仅是在考试期间重要。把心理弹性与幸福联系起来非常重要。思考这些特点 你就会展示它们。我在这里提到的这些特点。你可以学到它们。有趣的一点就是它们都可以习得。从多个方面来说 这门课程说的就是这个。当心理学家们识别出这些特点之后。他们开始教授它们 人们也开始学了。情况便发生了变化。除了信心以及"意义感"。利社会行为 帮助他人。从无助转变到有益 它们都非常重要。我们要讨论的事情之一就是。帮助他人的意义。它不但帮助他人。它还会帮助我们自己。我们便进入了自助和助他人的上升式螺旋里。因为我们帮助他人时也在帮助自己。 

当我们帮助自己时在帮助他人。循环往复。所以他们都是利社会者。他们都在帮助别人。他们的眼光集中在自己的长处上。而不是缺点。他们不忽视自己的缺点 但他们会问。我真正擅长的是什么?重申一次 这门课程的目标之一。就是你会认清自己的长处。不管是通过线上测试。通过回应报告。通过小组学习。还是通过阅读 你都会思考它。认清长处。他们擅长的是什么?他们为自己设立目标。他们面向未来。他们不但思考事情现在有多糟糕。同时也在思考。五年后或十年后的目标。我们会用三节课谈论设立目标。这是心理弹性的重要组成部分。他们有一个榜样。他们会说。"我想像她那样 我想像他那样"。榜样可以是老师 可以是亲戚朋友。有时候。它是一个历史人物或小说的人物。一个他们可以效仿的人。榜样给予他们力量 给他们方向感。最后 最重要的就是他们不单干。他们有社会的支持 他们不会说。"我够能干 可以自己做"。他们说"我够能干 我可以请求帮助"。因为那确实需要某种能力。还有承认弱点的勇气。承认有某种需要的勇气。想想。你在哈佛 在生活中你有这些东西吗?如果没有 你可以培养这一切。不管是社会支持……。不用是每天。与你交谈的一百人。它可以是一两个好朋友 父母 室友。这至关重要。它是心理弹性最重要的因素。社会支持重要的一点。就是找到适合的人。那些你向他们请求帮助。便会给予回应的人。现在我向你们展示一个并不太好的榜样。不太好的社会支持。这是Grace和Karen之间的一段交流。 

视频片断来自剧集《Will and Grace》。《Will and Grace》(第3季第9集)。 

这台游戏机是份不错的生日礼物。你觉得你的继子会喜欢吗? 

我不知道 亲爱的。他可以吃它或者摩擦它吗? 

我们换一种方式了解。他的兴趣是什么? 

火腿。还是你自己到处看看吧。 

好的 谢谢。 

你可以处理一下。这些小孩吗? 

夫人 这是玩具店。 

好的 处理一下。 

别挡道 矮个子。 

Karen。今天我不能和你买东西了 我得走了。 

怎么回事?发生什么事了? 

为什么要装害羞? 

没什么。我在皮肤科医生那里做了点手术。看起来有点恐怖。现在我不想任何人看到它。 

Grace。我敢说它并没有你想的那么糟糕。也许不值一提 一个小点 让妈妈看看。 

啊 我的天啊。 

嘘 Karen 冷静下来。 

你叫我怎么冷静下来?我看到它的心脏在我面前跳动。 

嘘。 

好的。好的 非常抱歉。你患疱疹多久了? 

这不是疱疹。它跟疱疹挨不着边。 

是吗?它可以参加疱疹的家庭野餐了。 

这是一个奇怪的雀斑。医生想要冻结它。然后他告诉我要十天才能复原。他怎么会认为我可以丑陋十天?因此 我要拿着你的尼龙包擦它。 

你在看什么 大肚婆?以前没见过疱疹吗?天啊 胖子真是麻木不仁。 

我的天啊 我的天啊。从这块小玩具镜照出来的样子看来。它变大了 就像芭比梦想中的青春痘。幸亏我取消了和Mark的约会。 

没错 因为它看起来。要交通柱和警察拦护带才能遮住。 

来吧 我们离开这里。 

Karen也许不是社会支持的好选择。但是有很多人却是好的选择。想想一个问题的力量。想想一个问题带来的结果。数十年间。很多可以得到帮助的人没有得到。因为没有提出正确的问题。只有在积极问题……。专注于积极的问题被提出来后。心理学家们突然能够看到。数十年来一直都在眼前的东西。它就在那里 显而易见 等待着被发现。但是他们完全忽略了。他们聪明 动机良好 资金充足。但是仍然没有提出正确的问题。问题制造现实 它们制造可能性。一个问题引起一场探索。他们在积极心理学方面的成就……。其中很大部分都是通过Antonovsky的努力。就是让我们从致病模型转到有益健康模型。健康本源学 健康的基础 起源。据Antonovsky所说。研究疾病很重要。不管是精神疾病还是身体疾病。但是研究健康的人也同样重要。看看他们为什么会如此健康。不管是否是哈佛大学里的人。哈佛大学是一个压力非常大的环境。但是有些人却能走完它。表现优秀 茁壮成长。他们也经历起起伏伏。我们都是这样 但是总体上。他们的经历更积极更快乐。他们是如何做到的?为什么? 

Antonovsky说我们要研究这点。这样我们就能了解健康的本源。他研究了这些东西。而且它在该领域产生了很大影响。Antonovsky说"那些熟悉科学史的人"。"都知道重要的进步"。"伴随着新问题的形成而来"。"问题才是突破点"。"答案来之不易"。"但重要的是新问题"。"有益健康问题"。"也就是我向你传播的"(正在发生)。"是一个极为新颖的问题"。"它推动新范式的形成"。"以帮助我们了解健康和疾病"。"它对研究人员和医生有着重要意义"。"生物学家和社会科学家也是如此"。正是那个问题缔造了。积极心理学和积极社会学。他是训练有素的社会学家和其他领域的专家。问题导致改变。问题缔造现实。 

现在我想和你们做一个练习。我们在课堂上会做一些练习。这是第一个。为了阐明问题的重要性。我要让你数几何图形的个数。你在屏幕上看到多少个……。不 不是这个屏幕。是下一个屏幕。你们是哈佛大学的学生 没问题的。下一个屏幕。我要让你们数几何图形的个数。你在屏幕上看到的 这是个非常难的问题。我把它给世界各地的人看过。我让数学家和视觉艺术家看过。这里的挑战是。你只有30秒去看它。30秒之后告诉我你在屏幕上你看到了。多少个几何图形。准备好了吗?30秒。你看到屏幕上有多少个几何图形?开始。好的 31秒过去了。应该没有问题了。如果你没有做过……。我知道你们有些人做过这个练习。但是如果你以前没做过这个练习。应该是大多数人。我希望你们能参与其中。你数出了屏幕上的多少个图形?把数字说出来。6 8 48 58 44 36 110 38。多少?上面呢?你们的视角很好。8 有多于110个的吗?好的 多少?300?200 有多于200个或少于6个的吗? 

好的 很大的范围。但是我承认这是个很难的问题。顺便说一下 如果你下载的PPT上。有这幅图……。后来我取消下载了。但是如果你下载了它。请现在不要看。数到的个数在6个到200个之间。这是一个很难的问题。实际上。它难到我根本不知道有多少个几何图形。在屏幕上。但是我有另一个问题要问你们。如果你知道这个问题的答案。就把手举起来。如果你以前没做过这个练习。就把手举起来 别说出来。如果你知道答案 把手举起来。时钟上显示的是几点钟?如果你知道答案就把手举起来。如果你认为你知道答案 举一半。也许知道 举四分之一。剩下的可以离开了。有几个举了一半 所以在整个教室里。这里有多少……这里大概有六七百个学生。在七百个哈佛学生里。五个半人能看到时钟上的时间。但是我知道 我们今天都有电子表。这很困难。 

我来问你们另一个较容易的问题。如果你知道答案就把手举起来。没错。你看到巴士上有多少个小孩?如果你知道答案 把手举起来。如果你觉得你知道 举一半。也许知道 举四分之一。大多数人都在想"什么巴士?"。"什么小孩?"。就在那里。整个教室里的七百个哈佛学生。大概有11.75人数出来了。但那没关系。这不是55号数学课程。我明白。另一个问题。它比较容易。屏幕上最左边的几何图形。它的主要颜色是什么?不是较大的那个图形。而是屏幕上最左边的几何图形。主要颜色是什么?如果你知道答案 把手举起来。如果觉得自己知道 举一半。好的 大概是12.25人。根据我的估计研究。这个教室里大概有五到七人。是色盲的。真的 根据统计资料。其余的人则没有借口说看不清。我们来看看。时钟上显示的时间?有谁觉得很难看清时间吗?十点十分。这是有点难。可以看见车上有多少个小孩?汽车在这里 有些人还是看不见。五个 还有颜色。最左边几何图形的主要颜色?黄色。怎么回事? 

这是说 这些并不是非常难的问题。就算是积极心理学的期末考。都要比它难。真的 难很多。为什么?原因是 我问了你们某个特定的问题。那个问题让你注意现实的某一部分。这是件好事。因为如果我们总是专注于一切事情。那就不是件好事。每种噪音都会让我们分心。所有物体运动都会让我们分心。所以我们能集中注意力是件好事。但是 我们也要记得。这种集中精神的能力带来的后果。却不总是好事或者有益的。因为在你们看来。巴士上没有小孩。在你们看来。上面只有几何图形。换句话说 我的问题给你们大多数人。制造了一种特定的现实。一种只有几何图形。巴士上没有小孩的现实。 

这有着非常重要的含义。想想以下的问题 蜜月期过后。夫妻间问的最多的问题是什么?他们度过蜜月期之后。不管是一个月 一年还是两年。那段时期后他们会开始问什么问题?"怎么回事?有什么问题?"。"我们怎样去改善关系?'。这是一个非常重要的问题。非常重要。但如果这是我们所问的。唯一一些问题。那么我们看到的也只有这件事。我们能看到的只是我们的不足。出了差错的事情。需要去改善的事情。我伴侣的缺点。我们关系的缺点。如果我只是问"什么出了差错"。"怎么回事 我要改善什么" 重申一次。我们不需要去掉这些问题。它们很重要 但如果它们是唯一的问题。而通常它们确实是。被提出或主要提出的问题。那么在这对夫妇看来 不是客观上。在这对夫妇看来。他们的关系里没有好的事物。就像在你们看来。巴士上没有小孩。虽然他们就在你们面前。盯着你。但是他们不存在。或者从个人的角度思考它。这非常重要。我们问得最多的问题……。大多数美国人 澳大利亚人 中国人。以色列人 欧洲人 非洲人。大多数人最经常问自己的是什么问题?我之所以要提到那些地方。是因为这是跨文化的研究。大多数人都会问自己。我的弱点是什么?我要改善什么?通常排除了。我的长处是什么?我擅长于什么?如果我们问自己的。唯一问题是。我的缺点 我的不足是什么?那么我们能看到的。只有自己的缺点和不足。在我们看来。好的东西 我们的长处 热情 美德。我们身上的美好事物并不存在。就像在你们看来。巴士上的小孩并不存在。 

现在我问你们。一个主要关注缺点的人。看不到。也不欣赏他们的长处 热情。他们的美德的人。一个这样的人会有很高的自尊心。自信心和很多的快乐吗?我们还奇怪 为什么这么多婚姻会失败?我们还奇怪 为什么抑郁 焦虑。以及自尊心低落的情况增加了这么多?动机是有的。它们都是好的动机 我们问的是。"我们如何改善?怎样才能做得更好?"。但如果我们不问积极的问题。在我们看来那部分现实并不存在。就像数十年来对那些心理学家一样。他们问题的答案 问题的解决方案。并不存在。即使它就在那里。在他们的眼前 在那些成功孩子的身上。在他们的心理弹性里。问题缔造现实。我们所问的问题通常决定了。我们追求的东西。我们会走的道路。我们会过的生活。不管是从个人方面来说。从人际关系方面来说。从组织方面来说都是这样。什么问题……。我知道你们很多会要当心理辅导师。心理辅导师最经常问的问题是什么?不管他们是否说出来。他们首次见到客户问的问题是什么?"有什么问题?有什么需要改善的?"。"有什么缺点需要改正?"。再说一次 它们是很重要的问题。但如果你只问这些问题。那么你就忽视了这个机构的。优势以及美德。你所做的只是在让他们失去活力。你在让公司慢慢地变弱。欣赏起作用的东西。也同样重要 如果不是更重要的话。不管是在机构里 人际关系里或个人方面。欣赏好的事物非常重要。 

看看"Appreciate"这个词。它有两个意思 其一就是为某事感激。而不是认为某事理所当然。这样做很好。我们不应该把长处 成功看作理所当然。我们不应把别人看作理所当然。这是很好的事情。但是"Appreciate"还有一个意思就是增值。钱在银行里会增值。经济会增长。当我们感激好的东西。好的东西便会增值 它会增长。不幸的是 反过来也是成立的。当我们不感激好的东西。把它看作理所当然 好的东西会贬值。那就是大多数婚姻。蜜月期后发生的情况。那就是发生在多数人身上的情况。特别是对于那些有动力去提高。想变得更好的人。那是件好事。如果那让你感到开心的话。同时。欣赏自己的优点也同样重要。我的长处 优点是什么。我们在课堂中将经常练习这点。但是不会做到自我陶醉的程度。我们会谈到自我陶醉。就在本学期倒数第二节课。自我陶醉不是自信 不是自尊。它正好相反。我们说的是有根据的自信。有根据的大方有益的快乐。为了过上那种生活。我们还要欣赏起作用的东西。用比喻的意义上说 还要留意巴士上的小孩。Stavros和Torres在一本关于婚姻的好书里说。"我们会看到要寻找的东西"。"错失不去寻找的东西 虽然它在那里"。"我们的阅历"。"被我们的关注点深深地影响了"。 

问题经常会缔造现实。理解什么是问题的首要事情。就是我们要理解那些问题。在这点我同意一位重要哲学家。一位20世纪和21世纪的哲学家。他阐明了理解问题的重要性。荷马辛普森。能不能把声音调低一点。因为它非常大声 谢谢。辛普森一家。(第810集)。现在我们要做几项测试。这是一台简单的测谎仪。我会问你几个问题 你回答是或否。你要老实回答。明白吗?明白。很精彩。所以第一步是真正理解问题。但是我们理解问题之后。知道我们要问什么问题也很重要。我之前提到过培养心理弹性的要事之一。就是要有一个榜样。现在我要告诉你们我的榜样。实际上 正是因为她。我才决定成为一名教师。 

她叫Marva Collins。Marva Collins20世纪30年代生于阿拉巴马州。她父亲是非洲裔美国人 母亲是印地安人。她出生的时候。种族岐视风行。幸运的是 对于Marva。她父亲对她非常有信心。从小他就对她说。"Marva 你将会有所作为"。"你可以成为一名秘书"。他之说以说秘书是因为。那就是无形的障碍。或者说是有形的障碍。鉴于她的种族背景。鉴于她的性别。Marva Collins努力工作。她很聪明 她成功了。她成为了一名秘书。做了几年秘书之后 虽然干得不错。她觉得这不适合她 不是她的使命。她真正的热情在于教书。她想成为学校里的一名教师。于是她上夜校。几年之后。她得到了教师证书。她结婚了并和丈夫搬到了芝加哥。她在那加入了芝加哥内城的公立学校。她在那里看到的现实。是大量犯罪。毒品泛滥 最重要的是没有希望。老师们的希望是让孩子们。在学校里尽可能呆久点。为什么?因为这样他们就不会12岁加入街头帮派。这样他们就不会接触毒品 不会犯罪。"我们怎样让学生呆在学校里?"。老师们问。Marva Collins面对这种现实说。"事情将有所改观"。 

在上课的第一天……。她教一年级到四年级。在上课的第一天。她对学生们说。"我们要练习自信"。她不断地重复地说着这则信息。整个学期犹如不断重播的唱片。最后坚持了整年 几年。"我相信你 你能做好 你能成功"。"承担生活的责任"。"停止抱怨 停止抱怨政府"。"停止抱怨老师 停止抱怨父母"。"成功与否全在你自己"。她继续不断地重复着这则信息。她对学生们充满期望。把眼光放在他们的长处。和优点上 并而加以培养。奇迹开始发生了。这些被老师认为是"不可教"的学生。到了四年级便可以读。欧里庇得斯 爱默生和莎翁的作品。这些"不可教"的学生在十岁时。便可以做高中的数学。关于Marva Collins的谣言开始传播。她怎能让这些学生在教室里呆这么长时间。而其他学生却想着离开学校?她一定是在强迫他们。 

Marva Collins受够了谣言的中伤。她离开公共学校成立了自己的学校。就在她的厨房里 开始只有四个学生。其中两个是她的孩子。逐渐地 越来越多的学生进入了Marva Collins学校。她称之为西岸小学。一开始进入这间学校的学生。都是从公共学校退学的。Marva Collins是他们成为街头混混前最后的希望。奇迹继续发生。逐渐地 越来越多的学生进入。她不得不搬出去。他们在芝加哥租了一间小室。冬天寒冷无比 夏天酷热难当。然而学生们被他们的热情所驱动。他们继续学习 奇迹继续发生在他们身上。Marva Collins所有的学生。都从小学毕业了。所有学生都上了高中并毕业了。她所有的学生最后都上了大学。并从大学毕业。没错 那些"不可教"的学生。Marva Collins数十年生活困苦。然而她努力平衡收支。毕竟 她的学生都无法付学费。但是 月继一月 她熬过去了。 

1979年 情况一夜发生了变化。CBS《60分钟》的制片人。得知了Marva Collins的故事 并制作了15分钟的节目。她一夜成名。1980年11月。新当选总统罗纳德里根打电话给Marva Collins。请她做他的教育部长。我想她父亲是对的。Marva Collins拒绝了邀请 她说。"我太喜欢教书了 我属于教室"。恰好八年过去了。新当选总统老布什。再次打电话给Marva Collins。请她做他的教育部长。再一次 她说"我太喜欢教书了"。"我属于教室"。1995年 一位富裕的慈善家捐赠了。数千万美元给Marva Collins。现在全国都有Marva Collins学校。数千名学生在里面学习。来自世界各地的数百名教师。前来目睹Marva Collins的奇迹。今天Marva Collins的学生里有政治家。商人 律师 医生。而最多的 就是老师。因为他们知道他们老师的功劳。 

我要展示这位杰出女性的一个片段。请把声音调大一点。它声音不大。Marva Collins"积极学习法"。我想我做得很好。我想我很聪明 我很特别。我会教每个孩子这样想。当他们不守规矩的时候。他们的惩罚就是要写100个原因。说明为什么他们棒到要做那样的事。而且它们要按字母顺序写。我很可爱 我很漂亮 我很勇敢。我给他们例子直到他们明白。我令人快乐 我很兴奋 我很厉害。我很棒 我是榜样 我无与伦比。我很热情 我很可爱 我很重要。我从不懒惰。一直写到最后一个字母 如果他们再犯。他们必须要用另一个同义词。他们不能再用可爱了。 

现在孩子们会对新学生说。我知道你为什么不再不守规矩了。因为我厌倦告诉Ms. Collins我有多棒了。她非常棒。这是她的书。对于那些……我知道。你们中有些人对教书有兴趣。如果有一本你要读的书 就是这本。对于那些对做领导有兴趣的人。如果有一本你要读的书 就是这本。对于那些已经是或以后想做父母的人。如果有一本你要读的书。就是这本。对于剩下的人。如果有一本你要读的书……。 

她传递的信息是什么?首先。她本身就是榜样。她和孩子们学习榜样。他们读小说 历史书。他们读关于英雄的书 谈论英雄的事。他们都认清了谁是榜样。他们从社区里挑出榜样。在家庭里挑出榜样 不断地这样做。这就是你培养心理弹性要做的事。首先 她就是榜样。她有着很高的期望。我们要大量练习自信。我们要表现优秀 我们要成功。她有很多期待 她能看到潜力。她欣赏每个人的潜力。停止抱怨别人 承担自己生活的责任。Marva Collins并不是软弱的人。教室里的她要求很严格。同时 她尊重每个人。她不是不切实际的"感觉良好"。我们不惜一切代价让他们感觉良好。根本不是。她相信他们 她尊重他们。而且她要求很严格。这是领导的重要因素。这就是为什么我之前提到。它是一本好的领导书籍。有很多非常和善的前总裁。他们的主要目标就是和善和受欢迎。完成工作的关键。完成事情的关键。同时也要尊重他人。她有乐观精神。你可以做好。你会做得很好。帮助他们为自己和社区设立目标。最后。从专注缺点到专注优点。来自教育学校的Howard Gardner。谈到了多元智力 他说我们要停止问。一个学生是否聪明。我们要问这个学生有什么优点。认清这个学生的优点。长处之后 我们便欣赏它。当我们欣赏它的时候。优点和那整个人都会增值。假如有一粒种子……种子是有潜力的。它会长成花草树木……假如有一粒种子。如果它没有受到灌溉 没有阳光照耀会怎样?它会枯萎死去。人类的潜力也是这样。如果我们不灌溉它 如果我们不照耀它。它会枯萎死去。 

人际关系也是这样。如果我们……我们会经常谈到关系。如何培养健康长期的关系。如果你不灌溉它 不照耀它。如果你不欣赏它的好处 好处就会贬值。本质上。Marva Collins所做的。也就是心理弹性的作用就是。缔造一个与传统智慧不同的榜样。范式转移实质上是从被动接受者……。也就是政府没有投入足够资金。这是一个问题 应该重视。用外部条件 比如资源。影响内在性格很重要。但那并不足够。从一个不积极 被动的受害者。成为一个主动的人 她改变了看法。你并非"不可教"。你可以发展 你可以做好 你可以成功。而她所做的实质上。就是把他们带到连续体的极点。因为每件事每个人都处于连续体的某处。我是说 不同的地方和不同的事物。想想你自己的生活。你是哪种人?各种情况的被动受害者还是主动者? 

例如 假如说我女朋友离开了我。当我还是这里的学生的时候。我主要想的是 我女朋友要离开我了。是的。那点我们以后再谈。当我们熟悉彼此之后。现在我有点害羞。但是假如我女朋友离开了我。如果我是被动受害者 那我只会自怜。为自己感到难过 思考。思考这种情况以及它有多糟糕。我从一个被动受害者变成抱怨者。她很糟糕。都是她的错 我怨她。我抱怨我的父母 他们养育不当。我抱怨她的父母 我抱怨布什总统。抱怨之后 我变得沮丧和愤怒。对她生气 对我父母生气 对她父母生气。布什总统 希拉里 总之我很愤怒。结果呢 很少结果。因为我沉迷于反思和自怜中。 

我们再反观积极主动者。首先 从定义来看 我会行动。我承担责任。经历痛苦之后我会到处走走。它很痛苦 下节课我们会谈谈。经历痛苦的重要性。谈谈允许自己人性化。但是我经历痛苦之后 我会行动。我去能认识他人的地方。我去匹诺曹(哈佛的比萨店)。或者另一个约会地点。我想时世变迁了。现在的哈佛和我毕业时的不同了。我承担责任。这样做的结果是……。我们会谈谈自我知觉理论。它是Daryl Bem的研究。我们会详细讨论它 你不用现在把它写下来。我会行动 行动会增加我们的自信。结果是更多的希望和乐观。就像在自我应验预言课程中所说的。希望和乐观会变成自我应验预言。我更容易找到伴侣。我更可能变得开心。重申一次 作为积极主动者。并不意味着不给我们自己时间。空间去让自己感受痛苦的情感。以及摆脱这种情感。没错 我们一定会摆脱它 然而。我们要在适合的时间……。它可能是现在 可能是一两天后。去行动 去承担责任 去做事情 这样。我们对希望和乐观的自信就会增加。我要说几件关于。做一个积极主动者及责任的事。它可以融入你的哈佛生活 这取决于你。让你的哈佛时光充满意义。是你的责任。从这门课程学尽量多的东西是你的责任。我们作为教师。肯定会为它创造条件。我们会以各种方式支持你们。 

然而 最终。小组讨论是你们的责任。作出行动是你们的责任。下周你们在小组里。首先要问的问题之一就是。"你如何让它成为一个优秀的小组?"。"你能做什么? 你能有什么贡献?"。"你能给小组带来什么优势"。"让它成为一个优秀的小组?"。而不是抱怨其他学生 组长 布什 克林顿。你要为它承担责任。 

关于责任 有一个很好的故事。Nathaniel Branden的书里提到了。当我们谈到自尊时。我们要读点Nathaniel Branden的东西。Nathaniel Branden谈到了六根支柱。自尊的六根重要支柱。其中一根就是自我负责。自尊心强的人会承担责任。想培养高自尊心。培养自信的人会承担责任。承担生活的责任等等。在他的研讨会上 他在书中说到的。主要一件事就是 明白你必须。为生活承担责任就是理解。没人会来。没人会来……穿着闪亮铠甲的骑士。不会来带你到幸福乐园。没人会来让你的生活更美好。没人会来。你为自己的生活负责 获得自信。自尊 幸福。没人会来。他在研讨会上谈到了这个。那是一个为期三天的研讨会。这已经是第三天了。研讨会进展顺利。参与者都学到了很多。他说 他告诉他们没人会来。其中一名参与者举手说。"Branden博士 事实并非这样"。Nathaniel Branden问他"为什么?" 他说。"Branden博士 你来了" Branden回应道。"没错 我来了 但我是来告诉你没人会来"。 

没人会来。能否最大限度地从这次经历中获益全在于你。1504号课程 你的小组 你的哈佛经历等等。我们作为教师。我们迫不及待想为之创造适合条件。星期四再见。 

第3课-幸福是一种随机现象吗? 

早上好 同学们。今天是本学期正式开学的第一天。很高兴在这看到你们。我只想说几句话。还有向进修学院的同学们打声招呼。上次我们向新西兰的同学问好。今天轮到爱尔兰的同学了。现在先说这里的本科生和研究生。分组问题。你们明天会收到Sean Achor的邮件。你们填好各自的想加入的小组。我们本周会分好小组。选择你最想加入的组非常重要。下周我们就开始。 

还记得上节课的问题吗。最主要的问题是"为什么要学习积极心理学?"。我提到了三个原因。为何我们把它作为独立的研究 而不简单地。"研究一下幸福和爱情"。不像一贯人们所做的那样?我们需要积极心理学是为了改变。现在的21比1这个比率。每有一项关于抑郁或焦虑的研究。就有21项 对不起。应该是每有一项关于快乐或幸福的研究。就有21项关于抑郁或焦虑的研究。我们想稍微改变这个比率。我提过三个原因。关于为什么要改变这个比率。尽管世界上越来越多人抑郁。而焦虑就像是全球性流行病。遍布美国 中国 澳大利亚 英国所有大学。尽管如此 我还是认为我们该改变那比率。做更多"积极研究" 或者说。集中研究可行的事。原因在于。我们之前讨论完的第一个原因是。我们提出的问题很重要。无论它是不是我们的研究问题。或我们为自己所问的。或为我们的伴侣问的。如果我们只问同一类问题。或只是问"有哪些不如意的事?出什么问题了?为什么那么多孩子因为成长环境而堕落?"。如果我们只问了这些问题。就会错过现实中很重要的部分。像你们在练习中没看到公车上的孩子一样。你们大多数人都没看到。如果我们也问积极问题。那就会出现新的可能性 新的探索。就像他们对研究学者所问的一样。不再问"为什么有这么多人堕落?"。而问"那些人是怎样取得成功的?为什么有些人能成功。即使成长环境恶劣?"。 

我们会以Marva Collins的故事为例。她的例子很典型 能说明很多观点。这些观点会贯穿我们整节课。Marva Collins所做的。就是帮助她的学生从消极的受害者-。你是成长环境的受害者。是你的家庭教育 邻里关系 和社会环境的受害者。诸如此类的 从消极的受害者变成积极的作用者。是的 那很困难 艰辛 不公平。但这是你自己的责任 没人会来帮你。能否改变自己的人生 由你自己决定。但她改变了千千万万人的人生。现在还在这么做。换句话说 如果我们看看芝加哥学院体制。Marva Collins就在那体制下执教。还是那个老问题。"怎样才能让学生们在学校。留得久一点?如何让他们在10或12岁后还留在学校。让他们远离街头帮派?让他们远离毒品和犯罪?"。"如何使他们安全地留在学校?"。这是个很重要的问题 但还不只如此。Marva Collins重新定义这个问题。她提出的问题是。"我们如何培养学生 激发他们的潜能?"。这问题改变了一切。因为她看到了每个学生的潜能。她看到每个学生身上的优点和品德。潜能 优点 品德 能力。这些是其他老师没看到的。因为他们没提出这个问题。打个比方 他们只会问。"你在屏幕上看到多少个几何图案?"。他们完全忽略了车上的孩子。完全忽略了学生的潜能。如果我们看不见学生的潜能。不去栽培它。它就会枯萎而死。 

遗憾的是 大部分人类潜能都因此而被磨灭。无论在哪 这都是大部分人类潜能的遭遇。人与人之间 很多团体中。很多大学里或人们自己 这些潜能都因被忽视而磨灭。疑问引起改变。因此提问是非常重要的。提出积极心理学的问题很重要。也就是"健康创成"的问题。"健康源于什么?成功源于什么?幸福源于什么?"。这是第一个原因 而第二个原因是。在谈第二个原因时。如果Marva Collins今天在场。那她会问我们这个问题:。"我们如何在于自身和家人中。栽培伟大的种子。如何在在社区和团体中 栽培伟大的种子。在国家和世界中呢"。当我们提出这个极为重要的问题。就会看到之前没看到的潜能。第二个研究积极心理学的原因。研究可行的事和。研究快乐 人际关系 幸福的原因。是因为幸福不会因为驱除了忧愁。而自动增长。幸福与忧愁 快乐与紧张。精神病与抑郁 都有很大的内在联系。要快乐并不容易。如果我们感觉很压抑或焦虑。就是说他们必然是有关联的 然而。单单摆脱焦虑与压抑并不会。使我们快乐。这一点算是传统的常识。是很多从业的心理学家。会了解的一点常识。"摆脱抑郁。一切都会变好的"。事实并非如此。 

解释这一点的一个例子是。享受美食的能力。美味大餐。如果我们消化不良 那很难享受美食。因此我们得先治好消化不良。然而 治好消化不良。不保证我们就能享受美食。要享受美味大餐 就得出去吃。只是治愈消化不良远远不够。还要做好下一步的事。可以从多方面来看我们的经验。在心理连续谱上看有效心理经验。有时候会跌倒0以下。不愉快或痛苦的经历。积极或愉快的经历。分别在0与正值之间。紧张 愤怒 焦虑 压抑 精神病。这些消极和痛苦的方面。幸福 满足 愉悦 兴奋 快乐。属于另一方面。也就是积极心理学研究的方面。再强调一次 摆脱消极。不能保证我们变得积极。因此早在19世纪40年代。梭罗。认为 大多数人都活在沉默的压抑中。这是没错的。但平克弗洛伊德说过。"人们在舒适地麻木" 舒适地麻木。还不止如此 我们要怎样摆脱这种"舒适地麻木"?要怎样摆脱"沉默的压抑"?得到振奋 快乐 幸福?为了达到这目的 我们需用心经营这些情感因素。再说一次 幸福不会自动出现。并不是没有痛苦就能感到幸福。因此我们才需要积极心理学。积极心理学本质上就是健康模型。健康本源学。健康 生理 心理 情感之源是什么?我们如何让人们从智力上 情感上。心理上 人际关系上 人格上全面发展?我们如何让他们全面发展 而不仅仅是帮他们摆脱。生活中不如意的事? 

根据该模型 我们在许多层面上都走了极端。第一个层次:。我们是否重视缺点。也就是疾病模型所说的我们要摆脱的缺点。或者说 我们是否专注于优点?当你问别人这个问题。盖洛普组织做了这项研究。一项全球民意调查显示。无论是在日本 中国 美国或欧洲。大多数人认为 如果要成功。注重自己的缺点比优点更重要。大错特错。那些同样注重自己优点的人。改变那个比率的人 更注重优点的人。不仅更开心 从长远来看他们也。更成功。这同样适用于领导能力。积极心理学认为 我们至少。注重优点的程度要和注重缺点一样多。无论是组织还是个人自身。我们最注重的是克服缺点。还是培养才能?我们擅长什么 并是否取得了进步?个人或组织的自然倾向是什么。我们是否更注重自己的自然倾向?注重健康的同时也要注重幸福。如果我们是较为积极的话。我们如何活出自己的精彩?如何摆脱痛苦的经历。或积极寻找快乐?如何摆脱悲伤?还是遵循宣言(独立宣言)行事。追求幸福? 

举个例 下面的两种人看起来非常相似。某人可能每周工作80小时。他习惯逃避 逃避家事。逃避处理私事。但他看起来却和下面这种人一样。她每周工作80小时 对工作。充满热情。表面看来他们一样 但他们的内心感觉很不同。一种是疾病模型。让我们逃避不如意的事。另一种是健康模型。让我们追求激情 追求喜欢做的事。对疾病模型而言 最优水平是零。我们都平平安安 别受伤就好了。再说一次 免受伤害是非常重要的。摆脱抑郁非常重要。但依健康模式来看 这是不够的。让我们超越那种程度。让我们兴奋起来 快乐起来。因为理想状态并不只是放松。它是创造性的紧张状态。我们会谈到它 会读到相关的文章。当我们"沉迷"之时 那就是我们感到兴奋。为所做之事着迷的状态。那样大大超越了"舒适地麻木"。事实上 在那种状态下会有点不舒适。那是在我们舒适区之外的。可算是我们的延伸地带。不是会让人受伤的恐慌地带 而是延伸地带。其中有兴奋 还有点紧张。还可能会有成长。所以 你的目的是什么?你想去哪?你要追求什么?你想逃避愉悦吗?想逃避痛苦吗?你想逃避不快乐吗?还是你想追求幸福和快乐?你更注重你的不足之处。或缺点吗?还是更注重优点?什么是最有利的选择?什么是理想状态?有无形的障碍 零点吗?还是可以继续。寻找到更多的刺激 享受 激情?有一些关于健康模型的可怕说法。因为没有界线 如今它的使用程度。也远比疾病模型少。积极心理学。健康心理学的研究还处于初级阶段。而且有更多的研究是关于。如何摆脱抑郁。如何发展自己的优点。但幸运的是 这就是为什么要学习积极心理学。 

大量的学者致力于。研究这些观点 它意义非凡。因为今天你们会看到整个学期的概览。有很多方法可以让我们的生活质量。达到正值。这还不是全部。我说过有三个原因。关键在于我们侧重点 侧重点创造现实。快乐并不只是对不快乐的否定。积极心理学之所以重要的第三个原因是。积极心理学领域。还有与积极心理学。相关的领域。不仅让我们从零点上升到正值。积极心理学还帮助我们处理消极情绪。帮助我们处理焦虑不安。处理抑郁和痛苦的经历 及情感问题。当我们培养出积极性。我们本质上是自我防御的。我解释一下。过去10年多时间的研究发现。处理我们不断增加的抑郁病例。个体的抑郁或焦虑。最有效方法是。不要直接专注于抑郁和焦虑。这点也很重要。事实证明 处理这种现象。最有效的方法是。培养积极性和个人优点。培育激情。提出这种问题"我人生中什么是有意义的?我的目的是什么?我为什么在这里?毕业后我真正想做的是什么?"。提出这些问题。并认真思考这些问题的人。所追求的。不是避免痛苦。而是。追求更多快乐。使生活更有意义 更成功。更重要的是 我将"更成功"。称之为"终极货币"。正如快乐货币和幸福货币一样。 

究其原因 是因为有两种不同的方法。对付疾病 一是积极心理学方法。它指的是 疾病相当于不健康。相对地 健康等于没疾病。我们想一下疾病模式。疾病模式是 "我们感到不适是因为生病了"。听到吗?我们病了是因为我们在生病。真的很深奥。你们可能要好好想一下这句话。可能要给点时间大家消化这话。我重新说一遍。如果治好病 就会变得健康。这是个传统的模型。把病治好 就会变得健康 这很好。积极心理学模型略有不同。它说的是。"你生病了 是因为你的生活还不够健康。因为你不去追求那些让你健康的东西"。什么使你健康?这我之前已谈过 追求有意义的生活 人生的目的。培养健康的人际关系。如果不具备这些 疾病就会趁虚而入。两种模型。健康模型和疾病模型的差异。不仅仅是语义上的。这是Abraham Maslow所说的神经官能症。"神经官能症描述的是人能变成怎样。或是他该变成怎样。从生理上来说就是是。一个人是否能顺利地成长和发展。人类和个人的潜能已经消失。世界逐渐缩小 意识也是。能力也被抑制"。我来解释一下这话的意思。他是说 我们正在生病 是因为我们没有。尽力做好我们本该做到的事。我们没做到自我实现。没做到该做的。我们削弱自身力量 因此才会生病。才会不适。这和疾病模型所说的有很大不同。"好 你不舒服 那就治病去吧"。他说的是:"不 你不舒服。要注重健康 加强你的健康。因为你病了。因为你没有注重健康"。他将这称为。"个人成长的失败"。 

如果我们感到焦虑。如果我们不注重培养自己的能力。不注重培养自己的人际关系 就会失败。积极心理学研究的是关于。培育自身成长和积极性。我们研究积极性 它处于这一边。大家可以在这上面看到。在积极的一方 当我们注重培养这些情感因素。就会助于我们更好地处理。消极因素。我想引用Martin Seligman的话。他说过这样一句话。"在过去的十年。心理学家关心的都是预防的问题。怎样才能预防像抑郁症这样的心理问题。或年轻人中常出现的药物滥用或精神分裂症。谁天生敏感脆弱 谁又是。制造这些问题的人?如何防止校园暴力。这些暴力事件通常是由掌握武器 缺乏父母监管。或受到教唆的孩子引起的。他问了这个问题 根据疾病模型。我们需要帮助他们直接。处理抑郁症。还有他们的焦虑和忧愁。才能防止所有这些社会弊病。无论是暴力还是不快乐。他是这么说的。"五十多年的经验告诉我们 疾病模型。不能让我们更有力地预防。这些严重问题。事实上 在预防方面取得的重大进展。基本上来自一个观点。也就是注重培养能力 而非改正错误"。换句话说 健康模型倡导。培养能力。增强自身长处 改善人际关系。帮助人们找到对自己人生有意义的事。和他们的热情所在。那就是我们将要致力做到的。同样对克服消极因素 有所帮助。健康模型与疾病模型不同。疾病模型提倡直接处理疾病的。Seligman不是说"排除"疾病模型。而是说也可以用健康模型。他还说。 

我们已发现有些人类的力量。能抵抗心理疾病。勇气 未来规划 乐观 人际技巧。信念 职业道德 希望 诚实 毅力。专注及洞察力 诸如此类。研究证实 乐观可帮助儿童和成人。预防抑郁和焦虑。使未来两年的发病率降低一半。同样 如果我们要防止。青少年受不良的社会环境影响。而滥用药物的情况。但预防的效果不佳。相反 它认定并放大了这个事实。就是青少年们已滥用药物"。Marva Collins所做的就是。注重健康并用心栽培它。浇水 给予阳光并时刻关注它。对这些观点的探讨将贯穿整个课程。健康模式能做的就是-。这是我们这堂课着重讲的内容。培养能力。培养我们处理。消极因素的能力。无论是人际关系上的消极和痛苦经历。还是我们自身的。那意味着什么 

我为大家做两个类比。培养能力就是创造一个强大的。心理免疫系统。这是Nathaniel Branden的观点。心理学免疫系统。如果我们有强大的生理免疫系统会怎样?这是否意味着我们不会生病?当然不是 还是会生病。但这意味着我们不会常生病 就算病了。也能恢复得更快。这就是培养的力量。乐观 追求 意义 规划。正是这些因素的作用。他们会放大并改变。我们看待和体验世界的方式。让我们突破框架 引发潜能。使我们能更好地应对不可避免的困难。不可避免的困难的确存在。没有谁的人生是一帆风顺的。它可以增强我们的免疫系统。另一个类比是关于引擎的。如果我们用的引擎很小。想把车开上一个陡峭崎岖的山坡。引擎很容易损毁 爆炸。但如果我们的引擎够大。要上那山就容易得多。不费吹灰之力就能做到。所以我们在做的是 培养积极性。按那个比喻来说 我们正在增强。心理"引擎"。我们能够更好将负值转化为零。更别说还能变得更快乐。因为快乐不会自发而来。不会因为消除了忧愁 就变得快乐。 

我想说说我们这里的事。记得上次。我提出的《哈佛克里姆森》杂志里的文章。很可惜我找不到。更近期一点的 那是2004年的?但今天的情况非常相似。那篇文章提到的一点是。我们需要为哈佛学生的心理健康投入更多。这点很重要的 我非常赞同。这只是其中一点。投入到其他一些地方。因为只是投入资源。帮助我们应对抑郁 焦虑和忧愁。是远远不够的。同样重要的是 投入资源。帮助我们培养自身能力。应对困难和艰辛的能力。这些困难一直存在并会继续出现的。换句话说 必须在以下方面投入更多资源。例如帮助学生发掘。自己的兴趣。帮助学生发现人生的意义。使他们明确自己。真正想做的事。帮助他们克服那种诱惑。身边常出现的诱惑。会让他们偏离自己的道路。还有凿掉那些束缚。摆脱杂念 帮助他们找到自己的真实个性。还有帮助学生了解自己的强项。并在哈佛实现一切。所有这些能力。所有这些技能 大多数都。我不只是说哈佛。而是全球范围内 没有一所学校会教这些技能。所以我们需要教他们。这并不是说哈佛现在。正投入大量资源的方面不重要。这些方面都非常重要的。举个小例子。学习咨询处。我不知道你们有多少人利用过该资源。我念本科时已开始接触它。至今还在与他们一起做某些工作。去年我与他们合作过。他们真的很棒。 

同时 我们还需要培养积极性。探究如何让我们从零点升到正直。该如何做。这就是积极心理学的作用。我希望1504号课程在一定程度上会有帮助。总之。"积极心理学运动提醒我们。这个领域已与过去不同。心理学不仅是对疾病 弱点和伤害进行研究。它也研究优点和美德。治疗不只是修复错误。同时也要塑造正确的东西。心理学关注的不仅仅是疾病或健康状况。还包括工作 教育 洞察力 爱 成长和玩乐。在追求理想状态时。积极心理学不会盲目主观。或自欺欺人 或敷衍了事。相反 它尝试用最科学的方法来探究。极其复杂的人类行为所表现出来的。特殊问题"。还涉及到如何将学校与社会连接起来。这个我认为最重要的领域。对我们来说都有意义。好 我继续说下去。我想先说说下一节关于选择的课程。两节或两节半课。我将会谈到。本课程的基本前提。正如我刚才所说。本课程并不是积极心理学的调研课。这是关于选择"问题中的问题"的。有什么可以帮助我们个人。帮助我们的社会更快乐 不是快乐 是更快乐。三个月后 期末时。你会变得更快乐。到学期末 你很可能比现在的你。更快乐。越来越快乐。基本前提是什么?备课时我想和大家说些什么呢?我想为大家介绍。五个基本前提。介绍这些前提 要从。它的对立面说起。下面这些问题非常明确。我们从哪里来 我从哪里来。教学人员从哪里来。了解这些后我们就能建立本课程的基础。 

记得一开始我就跟你们说过。这课程就像个螺旋。内部的一切都是相关的。第一节课讲到的内容。与今天要谈的课题有关。也和第19课有关。从许多方面来看 这个前提。跟我们开头的两节课有关。会建立这个螺旋的基础。那其他一切就更容易建起来。这里有五个基本前提。我会先简短地介绍一下 再逐个阐述。在接下来的几节课里 会分别对它们。进行学习 研究 和应用。正如我开始时提到的 这课程是关于搭建桥梁的。搭建学科之间的桥梁 广集思源。并搭建学术界与社会之间的桥梁。专业化常在学术界占主导地位。本课程的做法。正好相反。如果我认为没有改变的可能。就不会教这节课。心理学的研究有很多。很多证据显示 改变是非常困难的事。我认为改变是可能的。无论是个人变化。还是组织变革。我们将探讨为什么是可能的。这只是整个螺旋的最基本层次。我们本周只讨论"改变"。我们会详细讨论。技巧 方法 工具。第三 这个前提与。决定幸福的内部因素有关。这是我要讨论的。与决定幸福的。外部环境无关。 

并不是说 外部环境不重要。我们同样应该注重改善外部环境。无论是我们自己 或对整个社会。然而 幸福主要。在于我们如何看待世界。社会习俗规条 我们理性和感性上的认知。"只能顺从人性"。在各个方面。这酿成了历史上许多冲突。无论是政治上 还是宗教上。无论是哲学 还是心理学。我们如何看待人性?是人性有缺陷 因此需要进一步完善吗?还是尽管我们不喜欢人性中的缺陷。也必须接受它?我会谈到后者。就是说我们必须顺从于人性。接受它所有的缺陷和优点。而不是试图从心理层面上。完善它。我们下节课再说这话题。颇具争议的是。精神健康和幸福的重要基础。最后。我想说的是 幸福不应该只是。我们追求的最终目标 还是一种道德追求。而不是一种普通的追求。但也有些追求。是更高级 更重要 更高尚的。这一点也颇具争议。我也会涉及疾病。和大家也许关心过的忧愁。同样 这留待下节课再讲。现在我们先从搭建"桥梁"说起。说回第一节课开始时。谈到的问题。也就是 搭起学校与社会之间的桥梁的想法。学术界内外 都有很多人。将世界一分为二。他们将外面的世界看成是现实。是肮脏的 不洁 被亵渎的 而学术界。则是崇高的 有理想的 神圣的。神圣与亵渎。这种区分带来伤害。它同时伤害了学术界及非学术界的人。哲学家Alfred North Whitehead。"大学小心翼翼地将我们。与外界活动隔开。这样最容易打击学习兴趣 阻碍学习进步。大学不适合独身生活。必须要与身边的人互动。这点对大学生活非常重要。 

他当时提到另一种心理学。他说另一种心理学就是。抛开实验室的研究。实验室研究意义重大。但不只是注重这种研究。而是走到外面的世界。与外界互动 接触那些"肮脏"事物。在外面工作 并吸取经验。从"肮脏经验"中学习。将得到的经验带回实验室。诸如此类 就像一个上升的螺旋。他称之为重要的另一种心理学。这也是Alfred North Whitehead所提倡的。现在你们来到这里。且大多数人将来都不会进入学术界。你可能会想。"好 学术界要结合实践做研究。那与我有何关系?我们的联系在哪?"。它不仅与你有联系 它与你息息相关。下面就是原因。世界最需要的。不是其他什么 就是务实的理想主义者。 

我做过6年莱弗里特宿舍的住校导师。我研究生在读时 就开始任教。在与学生的交谈中 让我印象最深的是。无论是在莱弗里特宿舍或其他地方。学生们的使命感 你们的使命感。你们对成功的渴望。让世界变得更美好的愿望。据我了解 很多学生毕业后。无论是与我一起工作。还是我辅导的学生。当我了解他们的前路 并不是些空话。这些学生勇往直前 做了非常棒的事。无论是校外校内的。无论他们是否刚建立起自身。但可以确定的是 他们潜藏的思想。很多时候会涌现出来。"我怎样才能使这个世界变得更美好?"。充满热情 理想化。美好 极其美好。渴望改变世界。是所有学生都有的想法。很多人认为这一代人。以自我为中心。这一代人只关心。"赚更多的钱"。"买大一点的房子"。"更成功 取得更高荣誉。变得更有威望"。这是一种错误的刻板印象。是的 荣誉 威望 金钱是重要的。这对大多数人来说都很重要。甚至是最重要的。但有这些刻板印象的人。他们只看到这些是最重要的吗?他们并不渴望改变世界。 

每年哈佛有。约1800名学生是。PBHA(菲利普斯布鲁克斯楼协会)的成员。不仅如此。还有一些没加入PBHA的学生甘当志愿者。几乎所有人都是。看一下统计数据。你们每个人。很快。离开哈佛后 都将加入公司。无论它是否你的主要工作。这个组织是个社会企业。非盈利机构 只为贡献社会而存在的。你会加入这样一个组织。你会捐钱给这些组织。哈佛毕业生一向慷慨 无论是对时间。还是金钱和努力。无论是商学院 法学院。医学院 教育学院。你对社会贡献甚多 是因为你在乎。无论是要贡献金钱还是时间 通常两者都要。都是成见。关于美国人也有些成见。凭经验来说 美国人。或只是说哈佛学生。根据发展趋势和统计。美国人是世界上最慷慨的人。不仅仅是因为他们更富有。的确 美国人更富有。他们常捐钱。无论是在扶贫或医疗救助方面。有研究证实 美国人也花最多时间。比其他任何人所花的时间都多。平均每周有四个小时做志愿者的工作。做与自身工作无关的善事。他们比世界上其他人更具社会目标。对于这个美好国家的成见。这里很美好。能来到这里是件美好的事。无论是在哈佛 还是在美国 真的很荣幸。你们许多人。不久将会身居要职。你们能够做善事吗。在一个非营利机构里行善。在你母校的董事会。贡献金钱和时间。 

然而。这是个原因。我与许多研究生相处过。或当导师时的学生。学生曾跟我谈及他们的困扰。他们曾对我说 "你知道。我曾有世界上最良好的意愿。我有最善良的心 想为世界作贡献。我奉献出时间和金钱。但还是觉得若有所失。我觉得我能改变世界的能力。越来越弱了"。为什么呢?因为美好的意愿和理想主义。虽然必要 但却是不够的。远远不够。因为很多时候 即使有很好的意愿。我们的能力还是有限。或在某些情况下 我们带来的伤害多于帮助。有一些研究表明。即使到了今天仍有。想做好事的人却好心干坏事。数十年来的心理学家都有很好的意愿。想帮助高危人群。大量的金钱-数百万美元投资在研究上。但收效甚微。为什么呢?因为他们没有提出这样的问题。这特定情况下 他们应该问这个问题。这个健康本源的问题 为什么有些人。成长环境恶劣 依然能成功?即使不提出这个问题 他们也是心怀善意的。充满理想主义。这是不够的。很多时候 这些干预会产生。消极的受害者心态。而不是积极作用者心态。这是Marva Collins提出的观点。是根据Karen Reivich和Martin Seligman的研究所提的。这正是心理学起作用的地方。因为我们可以展开研究。并应用到现实中。又回到了那个问题 为什么。搭起学校与社会的桥梁如此重要。大多数这类研究并没运用于现实。例如。有多少教师在课室努力工作。日复一日地研究皮格马利翁效应?我们下周或再下周会探讨。皮格马利翁效应。说明为什么教师的期望就是自我实现的预言。 

如果我们有很高的期望。如果我们在学生身上看到伟大的潜能。那么这种潜能就更可能被激发。或者 如果我们看不到这潜能。它就会像刚萌芽的幼苗一样被扼杀。有多少教师了解这些研究?他们怎样通过自己的信念。创造出自我实现的预言 来影响学生?有多少教师了解Marva Collins?每个教师在第一次接受教师培训时。需要了解Marva Collins和皮格马利翁效应。但事实并非如此。接下来谈。自我尊重。你如何提升自尊?如果我在这里做一项民意调查 保证大多数人。会说通过赞美来提升。赞美别人 赞美孩子。这将增强他们的自尊。说对了一部分。如果把这当作全部的真理。则是有害的。因为有很多研究显示。当我们不加区别地赞扬别人。我们实际上是在长期地。默默地伤害他们 而非帮助他们。无论是对他们的幸福感 还是成功。但关于提升自尊有个常见的说法。要经常赞美别人和孩子。这点很重要。同样重要的是 要懂得如何赞美。有多少人了解。斯坦福大学心理学家Carol Dweck的作品。你们在这几周内会了解到。很多好心的理想主义者都不了解这些。他们依然认为。提升自尊就要不加区别地赞扬。这最终造成的伤害大于帮助。有多少心理学家或干涉主义者。怀有好的意愿 同时又了解。Albert Bandura关于自我效能感论的研究?要怎样培养?还不够。很多时候结果都是弊大于利。有多少心理学家了解。这个心理和生理的新领域?有多少人知道下面这个让人欣慰的案例。瑜伽练习。比任何试过的干预措施都有效。犯人在监狱里练瑜伽。会降低他们的二次犯罪率。在狱中练过瑜伽的犯人。获释后较少会再进监狱。虽奇怪 但却是事实。有多少人知道-。 

这点与我提出的第一个观点有关-。多少人知道 冥想能转换我们的大脑思维。使我们更容易感染到积极情绪。在面对痛苦时更坚强。多少人知道 每周三次。体育锻炼 每次三十分钟 与最强劲的精神药物。有同样的效果?每周三次 每次30分钟。多少心理学家或精神科医生规定病人。"每周跑步锻炼三次 早上来见我"?还不够。这是实际与理想主义相结合。在解决冲突时。大多数人。好心想要解决冲突。让人们坐下来。好好谈谈。然后就能过上幸福快乐的日子。我们有1954年的研究。那些想从社会心理方面解决冲突的人 Muzafer Sherif。显示接触假说。也就是让人们彼此交谈的方法。并不起作用。很多时候反而让情况更糟。很多时候会使冲突进一步恶化。因为人们只是聚在一起进行商谈。接触还不够。你需要的是。引用Muzafer警长的话。后来由Elliot Aronson详细阐述-。你需要的是一个非常明确的目标。一个得集体实现的目标。你自己一人做不来的。由一个内部有冲突的小组来实现。这样随着时间的推移 就能解决冲突。不仅是让人们聚在一起而已。可以想到 这离我们的意愿不远了。因为在阿拉伯 以色列冲突中。双方都有很多人想结束战争。美国也有很多人想结束它。他们做了什么呢?我们把他们关到一个房间里吧。无论是在戴维营 奥斯陆或埃及。让他们交谈 解决他们的冲突。和他们的问题 然后我们就能过上幸福生活。 

那样会发生什么事呢?局势会恶化。现在我们都知道这个道理了。Muzafer警长在1954年就证明了。接触假设 只让人们一起交谈。最可能的结果是引起更多冲突。有很多人都试图解决冲突。不只在中东 世界各地都有这种情况。人们都心存好意。但往往在不经意间使事情变得更糟。理想主义和善意是不够的。我们需要将理论结合实践。需要你们这样做。认真对待。但当我认真对待这事。就出现了一个问题。有时研究所得并不一定是好消息。如果仅仅是将。以色列人和阿拉伯人聚在一起 冲突就结束。那事情就简单得多 容易得多。这将让事情变得更简单。非常容易。如果我们要培养孩子自尊。仅仅通过多一些赞美。告诉他们 他们有多棒。这很容易做到 不是吗?这样感觉很好 他们感觉良好 我们也是。但从长远来看 仅仅这样做并没有帮助。只是容易而已。研究往往带来坏消息。只是接触还不够。只是赞扬还不够。然后人们下意识地选择。不自觉地忽视研究所得的坏结果。虽然跟随自己的感觉是重要。但更重要的是同时注重感觉和思维。 

试想一下 如果一个航空工程师。在早上醒来 说:。"万有引力。真的带给我很多麻烦。很麻烦。如果没有万有引力 事情就易办多了。航空设备也不用这么麻烦"。然后他决定不考虑万有引力设计飞机。他会设计出怎样一架飞机呢?肯定会失败。航空工程师必定要考虑到现实。现实就是万有引力的存在。我们得接受现实。同样 研究显示的结果就是现实。有哪些现象 哪些可行 哪些不可行。我们需要遵循现实。考虑现实。能否搭起学校与社会的桥梁。取决于我们自己。在哈佛你们会有32种专业。各自的课程有不同的主题。你要将这些知识一一吸收并应用。无论是心理学还是经济学。当然如果是工程学或计算机科学。就会比社会科学和人文科学。更容易理解。重点是 你要为自己负责。因为没人会来帮你。没人会帮你。第二前提。作为一个实际的理想主义者。它的基础必须是。相信改变是可能的。因为如果从个人层面或社会层面来说。改变都是不可能的 那我在这里干什么?为什么我要花时间在这里?倒不如做个享乐主义者。尽情享受生活。你们可能会说。"好吧 改变是有可能的。我相信这一点 但为什么我们需要把它。作为课程的基本前提。而不说'改变是虚幻的'?"。 

在心理学上。改变是有可能的 这点十分重要。下面我用一项研究来说明。明尼苏达州的双胞胎研究。一个在心理学领域非常有名的研究。是两个杰出的心理学家Lykken和Tellegen做的。这个研究是关于。基因的重要性。与教育相比 它对性格的影响多大?要如何测试呢?同卵双胞胎。他们拥有相同的基因谱。研究那些被分开抚养长大的同卵双胞胎。因为如果他们由相同的父母抚养 人们会说。"他们这么相似 是因为有同样的成长环境。他们外表一模一样 上一样的学校。有同样的父母 等等"。如果你能找到 出生时就分开了的。同卵双胞胎。在完全不同的环境里长大。现实中能找到这种人。学者发现有很多这样的人。有时甚至在不同的洲长大。他们对这些双胞胎进行了研究。发现了一个明显的现象。这些双胞胎极其相似。有时相似之处非常惊人。像有一组双胞胎。他们娶了名字相同的妻子。他们在不同的国家长大。直到37岁才知道对方的存在。娶了相似的妻子 喝同一种啤酒。为自己的孩子取相同的名字。有些相似之处令人难以置信。有例外。但也只有很少例外。让心理学家更感兴趣的是。他们的个性非常相似。让积极心理学学者 和研究幸福和快乐的人。更感兴趣的是。这些双胞胎的幸福和快乐水平非常相似。即使他们在完全不同的环境中长大。Lykken和Tellegen发表了一篇文章。一篇在80年代非常有影响力的论文。论文题为"幸福是一种随机现象"。论文用以下这句话作结尾。"想变得更快乐。和想变高一样徒劳。只会适得其反"。这句话有两方面让我觉得不快。这是句非常有影响力的名言。曾被登上《纽约时报》。他们也在电视上说过这话。受访时也说过。这就有问题了。 

那我们在这里做什么?如果这是那次研究的结果。一项严谨的研究。而不是马马虎虎的轻松的研究。我们能做些什么呢?我认为。简单来说 改变是可能的。别盲目相信我的话。下面我来详细阐述这话。有证据显示。人们的确是会改变的。我们有研究显示 经常接受治疗的人。会容易改变。斯坦福大学的心理学家Albert Bandura的研究。表明 人们常因遇到一个特别的句子。读到或听到它。这句话就会改变他们的生活。在书上读到 或是有某种经历-。积极的或痛苦的经历-。有这样一个概念叫做 创伤后应激障碍(PTSD)。还有一种概念叫做 创伤后成长。人们的快乐水平会上下变化。这是他们的经验之谈。有可靠证据显示 并非每一个人的。幸福水平都取决于基因。事实上 有研究。显示基因很重要 影响甚多。探讨改变的时候我们再谈这问题。还有其他因素和基因一样重要。像Lykken和Tellegen那样 人们常错误地。概括出一个结论 "变化是不可能的"。我称之为"大众错误"。平均而言。当研究这40或50组被分开抚养的双胞胎。大体看来。他们是一样的。然而 没有单独来看。因为 虽然他们大多数是非常相似的。但并不是全部都一样。 

这让我想起一个统计学笑话。在一个平均水深为10英寸的游泳池里 谁会被淹死。只知道平均值。并不知道游泳池的水深。由于该池平均10英寸深。或许它某些区域有20英尺深。如果它是一个大水池的话。同样 当你平均地看这些双胞胎。大体来说 他们都极其相似。但也有少数的异常值。很多时候 这个异常值。这些差异最有趣。因为他们。不只延伸了我们的想象 还让我们能理解。怎样的时间和地点 有可能产生变化。当我们看到了例外-。无论是在Lykken和Tellegen的研究。还是别的出现了例外的研究-。当发现不是所有的双胞胎都一样。主要的问题。已不再是 改变是否有可能。而是怎样实现这种改变。例外证实了那个规律。认为"改变是不可能"的。研究是有害的。设想一个8岁的女孩。她很忧伤。然后在杂志上读到那个研究。研究告诉她 基本上你的基因决定了一切。你与生俱来的一切 将伴随你终生。她会更忧伤。她只有8岁 但会觉得非常焦虑与痛苦。她就会想。"我的人生就是这样 我生来不幸"。那很可能会成为一个自我实现的预言。她始终不快乐。有时这研究让她比以前更不快乐。因为现在她简直是绝望了。 

改变是可能的。你知道我经常说。我是教积极心理学的最佳人选。为什么?因为我不是生来就有"快乐基因"。从遗传学上讲 我天生就。比较容易焦虑。容易得抑郁症。这是经过测试证实的。在后面的课程上我们会谈到。我之所以研究积极心理学。探索心理学领域。正如我第一节课时提到的。是因为我不快乐。随着时间的推移 出现了很多这类研究。由于提出了正确的问题。我变得更快乐。个人层面上 是有可能变得更快乐。现在的我比15年前的我更快乐。我希望15年后的我。比今天的我更快乐。这是个终身的过程。但这是有可能的 很多人证实了这是有可能的。而那些认为这不可能的人。他们用科学来争论。很多时候 只有阻碍作用而毫无帮助。最近Lykken和Tellegen。受到《时代》杂志的采访 谈"快乐"问题。下面是他们说的几句话。2005年 Lykken说。"我作了一个愚蠢的声明。显然 我们能改变快乐水平 变得更高或更低"。于是他们背弃了原来的声明。这个声明轰动一时。改变当然是可能的。我们怎样做更负责任的研究。怎样杜绝研究产生坏影响? 

同时 研究必须是真实的。我们并不是要"发明"研究。研究的目的在于查明事实。现实中真正发生了的事实。首先 我认为在健康研究方面。主要的问题是"让我们注重可行的方法"。这是我们过去探讨过的问题。第二 我认为。除了要研究可行的方法 还要研究最优秀的个体。为什么说"最优秀的个体"呢?我们不但要研究人们因何快乐。不要只是研究快乐的人。不要只是研究快乐良好的爱情。让我们研究的最快乐的人。让我们研究最成功的爱情。从中学习借鉴。这是个与大众研究。完全不同的研究方法。我想说的是 不要研究大众。研究快乐指数最高的5%的人。这样就可以更好地理解这一现象。这就是Abraham Maslow。提出的"成长尖端统计学"。以下这话引自他的书。"设计出这类调研是为了。改变我们对统计的观念。尤其是对抽样调查的看法。我非常支持。'成长尖端统计学'。我的头衔来自以下的研究发现。最优秀的遗传基因会在。幼苗的成长尖端出现"。他的意思是。"让我们学习圣人。非凡的人"。这样能帮助我们了解。人类的潜能。我再引述他另一句更长的话。这话非常重要 是我作这个研究的原因。"如果想知道人类最快能跑多快。那从好的样本中测出平均数是。没用的。而应该收集田径类奥运冠军的信息。看他们能跑多快。如果想知道心灵成长的可能性。人类价值增长或道德发展的可能性。那我认为 我们可以研究。最道德或圣洁的人。整体来说 我认为人类历史。是人性逐渐缺失的过程的。一个记录。人性的潜在极限。实际上一直被低估了。显然 我们平常称之为'正常'的东西。在心理学上 实际是种大众精神病理。看上去平平无奇却又非常广泛。我们甚至没有注意到它"。 

你们明白。他话里的含义吗?基本上 他的意思是。"不要只研究为什么大多数人失败。让我们也研究为什么有些人。但为什么有的人尽管条件恶劣也能成功"。不要只研究整体。整体看来的话 人是不能改变的。让我们来研究改变了的人。真的改变了自身生活。和身边其他人生活的人。这是个激进的研究方法。对研究我们自身来说。这也是个激进的方法。因为很多时候 如果我们只研究平均水平。那就只会看到平均的结果 只看到几何形状。完全忽略了汽车上的孩子。很多时候。最紧迫的问题的答案。出人意料。就在汽车的孩子身上。除了我还有其他人。我承认我属于这一类。但有谁在听到我说的这些话时。感到不舒服吗?说真的 当我谈到。"让我们集中精力研究最优秀的个体 圣人 非凡的人"。有人感到不舒服吗?我觉得不大舒服。你们有些人应该也有这感觉。因为这难道不是精英主义吗?我们不是应该研究普罗大众吗?因为我们关心的不只是精英。我们关心的是普罗大众。以下的问题有两个原因。我为什么还要继续说?我必须承认。每次教到或想到这个统计。我就觉得有点不舒服。研究最优秀的个体 为什么如此重要。为什么成长尖端统计学。如此重要 我要建议。大家使用。 

首先。因为它不是排除了普罗大众。就像积极心理学不是说 排除不可行的事。排除病理研究"。同样 成长尖端统计学不是提倡。"我们别研究大众了"。它提倡也去研究最优秀的个体。因此在这问题上无须计较研究对象是不是精英。第二点 更为重要的是。研究最优秀的个体 能让每个人从中受益。而所谓的"大众"。在这种研究中。"大众"比"精英"受益更多。为什么呢?例如 对适应力的研究。我们本可以继续研究几百年来的。平均受危人群。应该会取得一点点进展。事实证明进展甚微。只有当我们开始研究。这些"最优秀"的个体 成功的孩子。那些"超级"孩子。只有当我们开始研究这些。才能了解如何帮助到整体。只要将心理韧性研究应用到生活中。每个人受益良多。这就是个成长尖端统计学的研究实例。那关于冥想的研究呢?我想研究冥想。是不是可以在哈佛大二学生里随机抽样。来研究冥想?还是去西藏的山顶。研究冥想了几十年的人?当然我该去研究他们。的确有心理学家这样做过。他们研究了冥想者的大脑。谈到冥想时再谈这问题。通过研究说明他们的大脑。如何通过冥想来转换。像心理学家on Kabat Zinn。Richard Davidson和Herbert Benson。能从这些最优秀的个体中吸取经验。并应用到普通人的生活中。如果我每天冥想15或20分钟。就能从中受益。千千万万的"普罗大众"。因对最优秀的个体的研究。而受惠。那爱情呢?你能想象。对整个人类历史上的恋爱关系进行研究。并求平均值吗?人类史上的普通恋爱关系是指什么?人类史上的普通恋爱关系。其一是女人是被压制的一方。这就是人类史上的普通恋爱关系。如果我们只研究这种恋爱关系呢?是否有帮助? 

不 John Stuart Mill这样的人。研究了他自身的恋爱关系。这在当时是极不寻常的。研究过后他发现恋爱关系能改变。他这本书是以女性主导地位为题的。这本书是19世纪最重要的书之一。是女权运动和平等运动的。导火线。如果他只研究普通的恋爱关系呢?那样对人际关系有帮助吗?完全没有。那教学呢?要了解如何教学 你想怎样做?研究平均水平的教师还是研究Marva Collins。然后将Marva Collins的经验应用到所有老师身上?当着眼于研究最优秀的人体时 每个人都得益。这就是为什么Maslow说人性。和人的潜能会被削弱。如果我们只研究平均水平的话。同样涉及到研究我们个人的最大长处。而不仅是研究其他人的最大长处。无论是不是我们的最佳经验。因为研究我们的最开心。最成功。恋爱关系最佳时的状态。那我们也可以从中学习并应用到将来。当我们研究平均水平 我们只是描述日常生活。研究我们当中最优秀的个体时。我们潜意识里在学习。Maslow又说:"虽然他们人数少。但我们可以学到很多有用的东西。从对这些最优秀。最成熟的 心理最健康的个体的研究当中。和对大众的峰值的研究当中可以知道。他们达到了短暂的自我实现"。 

如果我们自己吸取这些经验。那么问题就不再是 是否有可能。越来越多地体验它。而是"怎样去体验它"。好。我们讨论过改变自己。为什么成长尖端统计学。会是积极心理学研究中的。第二个重要理念。第一个重要理念是"研究可行的事"。第二个重要理念是"研究最优秀的个体"。但这是关乎个体的改变。那社会变革呢?最后我想介绍以下这项研究。心理学最有名的研究。"剑桥-萨默维尔青年学习心理学"。这项研究从上世纪30年代开始。由哈佛大学与麻省理工学院共同进行。这些人拥有最高的思维。心理学思维 哲学思维 心理医生。他们聚在一起讨论。"让我们做一个劳斯莱斯计划吧"。"这是能想到的最佳干预方案"。他们没有金钱方面的限制。他们的一切要求都得到了满足。他们从受危人群中选了250名孩子。干预不是快速修复。不是能一夜改变的 他们实行五年干预。他们所做的是。工作人员每月两次探访受试者。帮助他们处理家庭中的冲突。以及处理日常生活中的问题。他们中有一半人得到了学院的辅导。只要有需要 就能得到学术帮助。精神科医生也是。谁有需要 医生都随时候命。你可尽情利用这些资源。无论你需要什么 这些精英都能满足你。他们加入了童子军 基督教青年会。或其他的青年运动 获益良多。 

从这些经验可看出来。他们得到了一切。这简直是梦寐以求的治疗。不仅在20世纪30年代 即使在今天也是。心理学家做梦也会梦到这些。这项研究就是这样进行的。调查结果非常重要。还有一个随机分派项目。有另外250个孩子 他们什么也没得到。也列为了研究对象。就像受到"五年干预"的孩子一样。250名对照组的孩子。40年的追踪调查。这调查不只是今天 明天或五年内的事。而是研究了他们的大半生。这是一项非常严谨的研究。这是严谨的干预。其结果令人震惊。即使所有参与这项研究的人-。无论是心理健康学者。哲学家 心理学家。教授还是精神科医生-。都说这是最佳 最高效的研究方案。但当他们看到原始的样本数据。调研结果让他们震惊。 

少年犯罪。对照组与干预组没有区别。超过三分之一有案底。超过20%犯轻罪 无案底。在少年犯罪方面两组并无差异。成人后的犯罪率。同样没有区别。20%以上的犯罪率。无论是对财产或对人所犯的罪。在这两组之间。样本规模相同的两组间。并无差异。其他测量数据显示。身体健康和心理健康 两组没有区别。酗酒方面两组有显著差异。酗酒人数有差异。工作地位有差异。成为白领的人数。两组有差异。所以至少还是有成果的。至少发现了几点显著的结果。当发现这几个差异时。那很棒吧。其实不尽然。这些研究结果被扭曲了。这项研究实际表明 干预组酒鬼数量。比对照组更多。对照组更多人。在工作上表现出色。相较干预组而言。换而言之 干预弊大于利。理想主义 良好意愿 钱财 都是不切实际的。很多关注这项研究的人。这是一个开创性的研究 心理学史上。很少如此重要的研究。他们认为"社会变革应该是不可能的"。再给我一分钟 快讲完了。他们认为 社会变革是不可能的。是吗?首先 例外是存在的 我们中也有例外。这证实了规则-有一个真正可行的方案。无论是Karen Reivich的研究。还是宾夕法尼亚大学Martin Seligman的"适应力计划"。Marva Collins肯定是个例外。她显示了干预是如何起作用的。想想以下这个有趣的问题。Marva所做的事与那项研究的区别。她没有给学生权利意识。她赞扬他们 给他们严厉的爱。而不是免费的午餐。她没有按需要来将他们分类。那项研究大概是根据需要来划分孩子的。有许多差异。但研究这个例外的关键是。大家坐在一切。说"什么是可行的?我们研究最好的吧。研究可行的事 将它应用到生活中"。我们将这思想传播出去。 

我们根据Maslow20世纪50年代所说的来行事。他表明"曼哈顿计划"反驳了。当时最大的问题。这不仅是对心理学 甚至对所有人类都。带来了历史紧迫感"。曼哈顿计划中 他们创造了原子弹。无论你是否同意。这计划。很快就说完。积极地说 他们汇集了最优秀的人才。奥本海默 基拉特 费米费曼 波尔。联合他们 共同拯救自由世界。无论你是否同意这计划。积极来说 它都是最伟大的计划。聚集起了最优秀的人才。这就是Maslow建议心理学家该做的事。同时也是积极心理学的目标。让全世界的人们思考这些问题。实际的理想主义者。研究可行的方法。研究最好的个体 他们会带来改变。 

下周再见。 

第4课-积极的环境能改变人 

大家好。我们是"哈佛召回"组合。想向教员和同学们。传达一份特殊的情人节讯息。 

再也找不到比你更甜蜜的爱人。 

比你更甜蜜。 

再也找不到比你更珍贵的爱人。 

比你更珍贵 因为你我 亲如母女。 

你我 亲如父子。 

你我 亲如姐妹。 

你我 亲如兄弟。 

你是唯一 我的一切。 

我会永远为你唱这首歌。 

我祈祷有人如你 感谢上帝。 

我终于找到你。 

我祈祷 你也爱我 祈祷。 

你也爱我。 

情人节快乐。早上好。请他们献歌时。本来想选另一首歌。但是…算了吧。我们确实爱你们。今天课程的内容。是上节课的延续。是这门课的基本前提。"我们来自哪里 我们将去哪里"。从各个方面展开论述。螺旋的基础。我们将在本学期一起创建它。上次我们讲到改变有多么困难。我们谈到"双胞胎研究" 举例说明了。Lykken和Tellegen提出的。也许改变我们幸福水平。和试图改变身高一样困难和徒劳无功。然后谈到这些研究学者么犯的。一般性的失误和错误。误解改变的本质。因为如果一个人在改变。问题已不再是"是否可能改变?"。而是"怎样才可能改变"。还谈到剑桥-萨默维尔研究。证明劳斯莱斯干预彻底失败。五年来 剑桥 哈佛和麻省理工的。顶尖科学家 研究人员。精神病专家和心理学家。沥尽心血 带着美好的意图。实施改变 但最终失败。不仅没有实现正面的改变。实际上是带去了负面的改变。 

还记得吗? 干预组的酗酒比例。和对照组相比是增加的。未参与干预的对照组。更有可能在二三十年后获得升职。改变是困难的。但我们又说。"Marva Collins实现了改变。所以改变是可能的"。Martin Seligmen和Karen Reivich。及大量学者都成功地实现改变。困难在于。如果我们想成为实践理想主义者。就要理解是什么带来改变然后去做。传播理念。传播研究的理念。即使研究并非总是传达好消息。它传达的是行之有效的方法。可为的方式而不是空洞的梦想。渴望 希望 愿望。那远远不够。好的意愿 理想主义 好的意图是不够的。我们需要扎根于研究。这正是Maslow的想法。当他谈及类似的曼哈顿计划时。科学家 积极心理学家 当时的心理学家。社学科学工作者聚在一起。在流行学术领域中挑选出。几种观念。几个有效的项目 再复制它们。研究最好的 正如Mariam同学。课后找到我时说的。"流行学术其实是要将杰出大众化"。我喜欢这个说法。 

将杰出大众化。研究最好的再应用在其他人身上。我们有了这样一个伟大的计划。有了Maslow创造类似曼哈顿计划的伟大想法。但是如果我不想参与计划?不想成为学者。只想做自己的事 我能否实现改变?答案是 绝对能够。人若想在世间有所为。真正实现改变。面对的最显著障碍之一。是他们低估自己实现改变的能力。心理学界有很多研究。爱默生和莫斯科维奇是先驱。他们和其他学者都证明少数人。经常是一个人 如何实现重大改变。能实现显著的改变。爱默生说。"人类历史是少数派。和一个人的少数派的权力记录。很多社会科学研究支持这个观点。人类学家Margaret Mead说。"永远不要怀疑一小群。有思想 坚定的市民可以改变世界。事实上 正是这群人改变着世界"。所有改变从一个人或一小众人的思想开始。然后不断扩大。问题是"它如何扩大"。以及为什么我们难以理解。我们能够做出改变这个事实。并接受 被同化以及据此生活。如果我们能了解我们需要理解的。是改变如何发生。改变以指数级发生。我们与其他人的联系及他们与更多人的联系。形成了一个指数函数。可以用你们熟习的。"蝴蝶效应"为例加以解释。一只蝴蝶在新加坡拍动翅膀。理论上能在佛罗里达引起龙卷风。原因在于粒子的连续碰撞。 

它也解释了六度分隔理论。我们都是关联和相互关联的。在一个潜在善的网络里。为了说明人类网络的指数本质。我们来以笑为例。研究证明笑有传染性。别人笑会引起你发笑。你发笑会引起别人发笑。以此类推。即使路人与你擦身而过时 你没笑。表面上你没有笑。但你面部的细微肌肉。会收缩 让你感觉更好。笑是传染的。如果你的笑。感染了三个人。这三个人。每个人又引起另外三人发笑。那九个人。每个人再用笑容感染三个人。只需要20度的分隔。从你用笑容感染三个人开始。全世界就会笑起来。社会网络的指数本质。让别人感觉良好也有感染力。恭维别人。如果你能让三个人。甚至四个人度过美妙的一天。他们会推展 让四人有美好的一天。以此类推。只需要很短的时间。整个世界都会感觉更加美好。这是指数函数的本质。笑和笑声是传染的。为了说明这一点 我会插播一段录像。(西班牙语)。很难不笑 不是吗?我要再看一遍。 

报歉。今天讲座的主题是婴儿。一会还有另外一段。让我们先理解指数函数的本质。以便理解"一"的力量。我想找个人和我作个交易。交易是这样的 我要做的是要求你们。纯属自愿。可以不做 交易是这样的。我要求你们每天给我…。还是我给你们吧。我给每个和我作交易的人。每天一千美元 连给30天。你们给我的是 第一天。一分钱。第二天 二分钱 第三天 四分钱。每天给我前一天的两倍钱。从一分钱开始。谁愿意和我作这笔交易?我会在未来30天每天给你一千美元。你们要做的是在未来30天给我一分钱。第二天二分 然后是四分 八分 以此类推。 

任何人?谁想做这笔交易?一个人。还有吗?我少了三万美元。还有吗? 很好。好 是这样。结果将是这样。在第30天 我再给你一千美元。你将从我这里一共得到三万美元。在第30天 我将从你那里一共得到。包括前29天。不 在第30天。我将从你那里得到5368709.12美元。再翻倍 乘以二。这将是我在一个月赚到的钱。第一天一分钱 第二天二分钱…。很多人会觉得不可思议。因为他们不懂指数函数的本质。也不懂"一"的力量。再举一个例子。开玩笑的 我放过你。另一个例子 这个例子捕获了我的想象力。在我小的时候。父亲给我讲象棋游戏的由来。有些人可能知道这个故事。象棋游戏的发明人。他在印度 去见当地的国王。国王很喜欢这个游戏并说。"我要如何奖赏你?" 象棋发明者说。"不必 真得不用奖赏"。国王说"不 我想奖赏你。我要怎样奖赏你?"。于是象棋发明者说。"好吧 我希望。在第一个方格上。我想要一粒大米。在第二个方格上 我想要两粒大米。第三个方格四粒大米 以此类推。那就是我的要求"。国王问"你确定只想要那么多?我可以给你更多奖赏。这是一个很好玩的游戏"。发明人说"我确定"。国王指示助手。去实现他的愿望。当他们开始计算。需要多少粒米填满所有方格时。从第二格到第63格。他们发现米粒。能厚厚地覆盖全世界。再一次 误导…。没有理解指数函数 因为我们不理解。指数变化的本质。 

最后一个例子。你们认为需要折多少次。你们面前都有一张纸。你们认为需要将纸折叠多少次。才能碰到月亮?月亮离我们24万英里。需要折纸多少次。才能碰到月亮。41次。如果你们有一张纸。再在下课之前折叠41次。你就能碰到月亮。我不理解人类登月。有什么大不了的。在我看来很简单。得出的结论是。我们低估影响改变的能力。因为我们低估了指数函数的增长。我们活着的每分钟都在影响世界和他人。问题是我们选择哪个方向。是成为推动改变的力量?深思熟虑?成为实践理想主义者?还是只有好的意图但不付出。必要的努力发挥正面指数函数的作用。这门课的最后一份作业。是由你们做一次报告。选修和专修学生都要做。你们要准备一次讲座。我们在组织课堂时。心中就有这个想法。你如何影响改变 那些被你改变的人。希望能进一步影响其他人等等。从许多方面 此想法取自电影"让爱传出去"。我想插播一段片段。引出接下来的话题"让爱传出去"。为那些没看过电影的同学。 

电影"让爱传出去"。深刻地描述了。人类网络是指数函数的观点。我们低估了自己影响改变的能力。因为我们低估了指数函数的增长。如果世界只是个巨大的失望会怎样?除非你把在这世上讨厌的事物。翻转过来 

你们可以从今天开始。这是我。这是另外三个人。我准备帮助他们。他们每人再帮助三个人。再每人帮助三个人。但必须是很重大的帮助。他们无法自己做到的事。一个想法可能改变世界。车坏了?很敏锐的观察力。我能帮你。你给我一辆崭新的捷豹但不要回报。你可以称为陌生人之间的慷慨解囊。你对我儿子说了什么。让他把一个流浪汉带回家?我有新闻 相信我?Chandling and Moss的高级合伙人往外送新车?把爱传出去。帮三个人三个大忙。不可能把两个人放在一起令其彼此喜欢。就是它。把爱传出去。喜欢这个主意。你可以改变一个人。想进来留下吗?这很复杂 进来。本来就该很困难。我不在乎你的烧伤。Eugene 如果那是你的顾虑。那是你顾虑吗?我不能。我很难过。别说你有多为我难过。也许你害怕被拒绝。我无法拒绝你 你变得太快。你还会向外传递爱吗?为了我 再给她一次机会。要来不急了。我想穿那条绿色的裙子。不 你很香。真的? 是的。"把爱传出去"是洛杉矶兴起的一场运动。别说笑 一场运动?要是我们不改变世界 你会让我们不及格。你可能只得一个C。 

影响三个人。每个再影响三个 以此类推。经过20度分隔。就能改变世界 整个世界。第三个前提。内在因素和外在因素。很多研究表明。基于外在因素改变快乐很难。对幸福的研究。心理学家用这个词来描述快乐。已经进行了很多年。直至最近 多数完成的研究。使用的都是问卷方式。很多人置疑问卷的。真实性和价值。因为它是主观的。我们是在测量真实的东西吗?过去几年。呈现的情况是越来越多心理学家。使用大脑扫描。使用功能性磁共振成像 脑电图及其他手段。他们的有趣发现是。两者有很大的关联性。在所谓的客观手段 如大脑扫描。脑电图 功能性磁共振成像等生理学手段。和人们对自己的幸福的评估。换言之即快乐的主观水平。两者有很大关联 

这在很多方面。给多年来进行过的研究提供了可信度。在拥有技术进行更全面的研究之前。我会和大家分析讨论各种研究。有些使用大脑扫描。其他是自我评估。两者同样有意义有价值。我们会深入讨论研究的进行方式。当我们谈到冥想等现象。或像是Richie Davidson 通过扫描大脑。显示八周的冥想项目。可以产生重要改变。或年轻的Joshua Greene。他对道德进行研究。显示大脑里有道德中心。手段更成熟。更有趣 有趣的是。它验证了许多至今通过不全面的方法。所完成的研究的正确性。像自我评估。Daniel Gilbert所作的研究。他教初级心理学的"有效预测"。 

以下是他的一个研究。他所做的是去访问。接受终身职位审核前的教授。他们要么得到要么得不到。他问他们"你会多快乐。如果你得到终身职位?"。他们说"喜出望外。这是我多年努力的目标。这将会是。将是梦想成真"。"你会快乐多久?"。他们说"余生都会很快乐。因为这是我们努力很久的目标。它会让一切变得更简单。我可以停止不成功便成仁的竞争。我可以更加享受工作。这将改变我的人生"。他问他们"如果得不到终身职位会怎样?如果被拒绝呢?"。他们说"我们会很难过。这是我们多年来努力的目标"。"那你们会痛苦多久?"。"可能直到我们在别处获得终身职位。即使到了那时 可能也不够"。如果在一所学校无法获得终身职位。就没有可能在更高学府取得。经常是在别的院校获得。那些比拒绝你的学院。差一等的院校。所以那会让人难过很久。Gilbert再去访问他们。这时他们已知道终身职位审核结果。有些得到 有些没得到。他问"你们感觉如何?"。得到的说。"喜出望外 我们没从这样快乐过"。"你们的快乐会持续多久?"。"余生都会倍感快乐。我们成功了"。然后他去找没得到终身职位的并问。"你们感觉如何?"。他们很难过。而且确信。会难过好久。 

三个月后 他再去访问他们。六个月后再去。Gilbert和他的同事发现。几乎每一个人。不管得到还是没得到终身职位。都恢复到之前的幸福水平。如果他们之前是快乐的。六个月后也是快乐的。如果他们之前是不快乐的。不管他们是否得到终身职位。他们都是不快乐的。换言之 情形看似这样 然后回到基础水平。或者看似这样 然后回到基础水平。他们对中彩票的人做了同样的研究。"如果你真得中了一千万美元。真得能让你更快乐吗?"。是的 更快乐一段时间并不长久。西北大学的Philip Brickman做了这个研究。 

经过六个月。人们恢复基础的幸福水平。遇到严重意外的人。意外导致瘫痪。经常地 一般性的…。再次强调 这是平均而言。都会恢复到基础幸福水平。他们以前是快乐的。一年后 他们也会快乐。如果不快乐 将维持不快乐。极端情况对我们的幸福影响很小。伊利诺伊大学教授Ed Diener。就快乐水平作了大量研究。他自70年代开始研究。他的研究显示…。他和其他人 包括诺奖得主Daniel Kahneman。表明财富对幸福水平的影响很小。就像中了彩票。对我们是否幸福影响甚微。并不是说一个没有足够食物。一个无家可归 流落街头的人。每个月多有一两千美元。那当然会让他或她更快乐。但超出基本需要后。当我们的基本需要得到满足 即有食物。住所 基础教育。这些需要一旦得到满足。收入的影响就微乎其微。几代人之间没有变化。我们这一代比父辈富足。比祖父辈更加富足。我们不是更快乐。这是全球现象。 

不论是中国 英国还是澳大利亚。或美国。幸福水平几乎是静止的。焦虑水平和抑郁水平。正如第一堂课讲到的 有明显的加深。收入水平的影响很小。总得来说 外部环境的影响很小。回想你们的亲身经验。你们感觉如何。回想自己。考入哈佛时的体会。信箱里收到一个大包写着"已录取"。在4月1日或12月未。当你拿到大包时。有何感受? 可能是喜出望外。可能是你人生的亮点。在那一刻 如果你们像我一样。你们也会想"成功了。我会快乐很长一段时间。因为我在高中很努力。很多时很困难。很多时很痛苦 但完全值得。我考上了哈佛"。第二天 你还会有这种感觉。因为学校的人开始谈论你如何考上。你感觉很棒 不是吗?可能整个高三都会感觉很棒。当然难免起起落落。但总是来说是高水平的快乐。这种快乐将持续一生。来到这里 经过新生周。你知道快乐会持续一生。因为身边都是杰出的人。大学生活很丰富 派对不断。你说"人们对哈佛的想法都错了。这里其实是个派对大学"。你们很确定这一点 不是吗?不仅是个派对大学。而且未来四年甚至以后 都会过得很快乐。因为你的人生被这封录取信改变。没错吧?事情也许从开课第一天就发生变化。但没有完全不同。因为这是购物期 妙不可言的阶段。你们选择课程。比去商场更过瘾。有三千门课程任你挑选。在哈佛的第二周妙不可言。这种情况将持续下去。余生将是一浪高于一浪的快乐。事情突然开始发生变化。轻微地 非常轻微地。要写第一篇论文时 情况彻底改变。或期中考试来临时。你恢复基础幸福水平。 

如果你在高中。初中时经历过很大压力而且不快乐。一般来说最好的预言家会说。在哈佛呆上一个月 你会有同样的感觉。外部环境影响甚微。同样地 居住地的影响也很小。人们会想"如果我搬到。尤其是在白天。要是搬去加州 我会更快乐?" 错误。加州人不比麻省人更快乐。刚去到一个温暖的地方。会感到放松 幸福水平高到值峰。但很快我们会恢复基础水平。和以前的我们一模一样。我会进一步说明。虽然还没有这方面的研究。我敢和你们打赌 下面的说法是正确的。我们的幸福水平不会有所不同。不论我们是生活在河边。还是呆在quad (哈佛举行毕业典礼等重大集会的地方)。这个例子把理论推至极限 但是真的。我喜欢这件衬衫 什么?不能和大狗一起跑就呆在院子里。这里有韩国学生吗?好的。你在这么冷的日子赶来。很出色 我很感激。我们真是受宠若惊 不会有所不同。微乎甚微的区别。我们在哪里 我们生活在哪里。收入水平 是否中彩票 是否取得终身职位。入读理想中大学。找到理想工作 大四学生。可能已经找到。没错 你们会感觉极至的幸福。我肯定拿到录取信时。你们的幸福感也达到峰值。但很快又恢复到基础水平。对基础幸福水平很重要的一样东西。它是一个外在因素 那就是民主和压迫。生活在民主制度下的人通常明显。比生活在独裁制度下的人更快乐。 

以女性为例。生活在压迫制度下的一般。没有生活在自由国家的女性快乐。生活在达尔福尔的人。肯定没有生活在丹麦或美国的人快乐。这些是能产生区别的。极端情况。比如我举的无家可归者的例子。收入当然会改善他的境遇。移民到自由国家。当然会让他们更加幸福。但在这些极端情况之外。增加或减少的外部环境。产生的区别很小。这即是好消息也是坏消息。坏消息是 看来不论我们怎么做。都无关紧要。我为什么还要努力争取好工作?我为什么要拼命入读这所学校。 

如果这些都无关紧要 总要经历高低潮?答案是。是的 它不会改变我们的幸福水平。但并不意味着。我们不能提升自己的幸福水平。很多人说。一般的快乐或不快乐水平。是由于人们有过高的期望。如果能降低期望。降低压力水平 我们会更享受生活。我不在乎成绩是不是B。不在乎会让我更快乐。如果我降低期望水平。我不在乎从事什么工作。只想快乐。你可能会快乐。降低期望水平会略微快乐些。但从长远来说行不通。下周我们会谈到这个问题。长远来说行不通。问题不在于是否降低那些高的期望。这一点不重要。问题在于正确和错误的期望。而不是低的和高的期望。那不会影响我们的幸福水平。能影响幸福水平的。是我们的期望是正确还是错误。错误的期望认为进入某间公司。获得升职。找到理想的伴侣。就能让自己更快乐。搬去加州或哈默堡。这本身不会让我们更快乐。那是错误的期望。正确的期望是相信内在的改变。这些事情不会让我们快乐。事实上 我们的准备和体验幸福的可能性。主要由我们的心境决定。不是我们的地位或银行账户状况。 

我们要改变我们的认知 心境。要改变我们对世界的诠释。以及我们的遭遇 成就 失败。重要的是我们选择领悟什么。精力集中的焦点是什么。重要的是转变。而不是外部信息或外部成功。1504这门课程的重点。是关于转变。正如我在第一课中提到的。我想进入第四个前提。这堂课最重要的一个前提。在很多方面。我们能理解人类智力发展历史。仅仅基于这个观点。应不应该接受人类本性?我们能否完善它?它能否改变?这个工作或者说这些观点。基于Thomas Sowell的成果。Thomas Sowell 哈佛58届学生。目前就职斯坦福的胡佛研究院。我心中的一位知识分子英雄。他的工作和作品的贡献在于解释了。人们为什么会选择一个党派反对另一个。为什么会选择一种生活方式摒弃另一种。这本书帮助我更好地了解自己。更好地了解别人。不论是政治上。心理上还是哲学观上。对政治感兴趣的同学。如果说有一本书是必读的。读完Marva Collins的作品后 就是这本。Thomas Sowell所执的观点是什么? 

他把人分成两个阵营。那些认为本性受约束。有局限性的观点的人。和认为人性不受约束。无局限性的观点的人。认为人性受约束的观点的人。相信人性无法改变。它是永恒的 我们有一些本能。有一些欲望。它们是固有的 无法改变。是什么样就是什么样。作为人的物种。只有与生俱来的东西。我们的缺陷不可避免。无法改变。不论好坏 我们只能接受。我们只能接受缺陷。执有人性受约束的观点的人认为。我们的本能 欲望。我们的本性是永恒的 唯有接受。但他们没有屈服 而是加以引导。如何引导?通过建立某种政治机构。引导有缺陷的。不完美的人性向好的方面发展。执有这种观点的。哲学家和心理学家创造了它们。创建了人生哲学 心理学 各种机构和体系。来引导我们有缺点的不完美的人性。 

在哲学史上认为人性受约束的人。包括汉密尔顿 亚当斯密。学习经济学的都耳熟能详。Fredrick Kayak等。这些人认为。"我们的本性是受约束的。是有局限的" Edmund Burke说的。最准确阐述这种观点的。是培根的话。他是17世纪时科学运动之父。是一名哲学家。"号令自然 必须遵守自然"。不论是物质本质还是人性本质。我们都需要遵守它。本质是不变的。还有人性不受约束的观点。也许你更乐观。更加乐天 人性可以改善。可以改变。取决于我们去不去做 本性可以完善。不需要接受缺陷不可避免的观点。可以完善它。有办法解决这些缺陷。完善不完美的地方 我们的职责。机构的目标。不论是政治机构。教育机构 各种体系。各种组织 个体哲学家 心理学家。职责都是改变人类本性。完善它 把它变得更好。致力于此的哲学家。有托马斯杰斐逊。卢梭。萧伯纳 德沃金及其他杰出思想家。 

Thomas Sowell的贡献是。绘制了整个世界智力发展历史。展示出人们执有不同观点。两种截然不同的政治观点。最好地阐述了本性受约束观点的人。法国哲学家政治家Benjamin Constant说。"命运召唤我们进行自我完善"。执本性受约束观点的人。是在政治上 并非总是。但一般都是支持资本主义的人。比如亚当斯密。开创了"不可见的手"的理论。把不完美 有缺陷的本性导向好的。执本性不受约束观点的人。倾向乌托邦主义或共产主义。并非总是 有时是。为什么?让我们来改变人类本性。自利不是好事。长远来说是有害的 所以。我们需要改变它 而认为本性受约束的人。他们会说。"我们也许不是那样 但本性难移。不能改变它。我们尽去完善它 把它导向好的"。两种皆然不同的观点。基于人的观念得到两种皆然不同的手段。我为什么要给你们讲这个?这不是政治科学课程。因为它与心理学息息相关。它与心理学息息相关。我们认识了现实吗。对人类本性执不同观点。最终将影响我们心理。这是非常有意义和重要的。 

我来解释。首先。有些人在政治上执受约束的观点。但在心理学上执不受约束观点。并不总是一致。但多数时是一致的。这门课提倡的心理学观点。是本性受约束观点。换言之 那是我相信的观点。我会用三个研究加以证明 从今天开始。直到下个学期。人类本性是固有的。我们有欲望。我们有生而有之的本能。或上天赐与或经几百万年。在进化中形成。这些本性不会很快改变。在有生之年是不可能的。它是固有的。不论好坏 我们都要接受它们。在接受这个本性后 我们唯一能做的。是通过研究首先来理解它。理解它后。好好地利用它。通过研究 探究。通过内省来理解它。然后好好地利用。如何引导我的本性? 

现在我想演示一个案例研究。说明我所谓的本性受约束的含义。为何它对快乐。幸福 长久的成功如此重要。当论及到我们的心理时。人性受约束观点很重要。我谈及的话题是准许为人。原因有三。第一 它说明了Thomas Sowell的观点。在心理学领域的政治方面。第二 因为我认为它是。快乐和幸福的最重要支柱之一。第三个原因。因为它引起人们。对何为积极心理学的误解。当我开始教授这门课时。有六名学生。我记得有一天坐在学生宿舍里。独自用餐 一名学生走进来说。"能一起吃吗?""当然" 我们共进午餐。他对我说。"Tal 我听说你教一门有关快乐的课程"。我对他说"是的 没错"。他说"我的室友选了你的课"。我说"那太好了"。六名学生中有了两名。他说"Tal 你现在要多加小心"。我问"为什么?"。他说"Tal 你必须小心"。我说"为什么?"。他说"因为如果我见到你不快乐。我会告诉他们"。第二天上课时。我讲了这件事并对学生说。"我最不希望你们以为。我总是保持快乐 或者你们。在期未或学年未。那门课程要讲一年。你们到学年未时。会一直保持快乐"。 

只有两类人。体会这种持续的快乐。不会体会到痛苦的情绪 像愤怒。嫉妒 失望。悲伤 不快乐。抑郁或偶尔的焦虑。两类人不会体会到这些痛苦的情绪。一类是精神病人。因此不会感受到痛苦的情绪。第二类。没有痛苦情绪体会的 是死人。没错 所以。如果你们体会到这些情绪。那是个好兆头。说明你不是精神病人 也还没死。然而在当今的文化中。我们不准许自己为人。也没有体会痛苦情绪的自由。我们为这种无能付出高昂代价。为拒绝接受。本性受约束。生而有之的事实 付出高昂代价。孩子和婴儿。回到婴儿期。我们准许自己为人。我们知道那是自然的。事实上根本不去思考。自然而然地经历起起落落。后来当我们停止准许自己为人时。当我们的"面子"变得重要起来的时候。当我们开始发觉其他人在看我们时。时刻评价我们时。人们远没我们以为的那么注意我们。那时我们停止准许自己为人。为此我们付出代价 包括精力水平。幸福感 快乐感。创造力 最终以成就大小作为代价。我举一个。准许自己为人的例子。准许为人。不是说大家应该像那样。 

我的意思是我们都需要一个空间。生活中要有一个地方。在那里我们准许自己为人。可以是和挚友在一起时。我们关心的人。最重要的是面对自己时。写日记时。这时我们准许自己为人。准许自己哭泣 快乐。如果不那么做。我们付出代价。我们需要一个无条件接受自我的地方。我得到的最好建议。或者说是我们。是我妻子和我 当David。我们的大儿子出生时 Tok Shapiro医生给的。David在凌晨一点出生。早上八点时。他来查看我妻子的情况。查看婴儿的情况。不知为什么 他没询问我的情况。一切都正常。在他走出房间时。他转过身说。"还有一件事。在接下来的几个月里。你们将体会到每一种情绪。极至的情绪 那没关系。是很自然的事。我们都经历过"。然后他走了出去。这是妻子和我。得到的关于抚育孩子的最好建议。为什么? 

让我举个例子。一个月后。我开始偶尔。对David产生一种嫉妒。因为这是妻子和我在一起后。第一次有人夹在中间。另一个人比我获得更多的关注。不管我怎么哭。我嫉妒他 然后五分钟后。我对他产生最强烈的爱意。一种我从没感受过的情绪。平时我会想。"真是个伪君子。事情不对劲。前一分钟充满嫉妒 转眼体会无尽爱意?"。这完全合情合理。完全正确。这是为人的一部分。因为我脑子里装着儿科医生的建议。准许为人。那帮了我很多。我体会嫉妒 接受它。然后享受和赞美。我对David产生的积极情绪。准许为人。这里有自相矛盾的地方。Daniel Wegner对"反语处理"做过研究。当我们压抑一种自然现象时。那种现象只会加强。 

我用一个实验加以说明。在接下来的十秒钟里。不要想粉色大象。在接下来的十秒钟里不要想粉色大象。知道我说的是什么吗?就是大耳朵的小飞象?在接下来的十秒钟里不要想粉色大象。我肯定没有人想一头粉色大象 是吗?事实如何?多数人都会想粉色大象。因为当我们企图压抑一种自然现象时。比如提起一个词时浮现出相应形象。只会加强它。压抑自然的痛苦情绪也有此效果。企图压制它们时 它们会加强。当我听说Marva Collins的事迹。并决定以教书为终身职业时。我知道必须克服一个问题。那就是我个性内性。站在听众前面我会很紧张 对我来说。超过五个人就算众多听众。但我必须克服这个问题。我知道必须克服它。所以我会站在听众面前 在那之前。我会对自己说。"不要紧张。不要焦虑。今天不能焦虑。不要紧张。不要紧张 不要"。事实又是怎样?我即紧张又焦虑。但在读过 Victor Frankl。有关"矛盾意向"的阐述后。我开始准许自己为人。现在当我去上课时 紧张感。因为我准许自己为人。在课堂进行三小时后会消失。但紧张是可以控制的。每次上课前 我仍会感觉紧张。其实这是件好事 它可以控制。我能应付它 能控制它。准许为人。 

拒绝本性导致次优表现。不论在情绪上还是外在表现。想象每天早上醒来后对自己说。"我拒绝接受万有引力定律。万有引力定律很麻烦。要被迫上下楼。我只想早上时飘去剧院。我只想下午时飘去餐厅。更加容易 更少痛苦"。想象过这样一种生活。你真的能拒绝接受万有引力定律吗?那会是怎么一样生活?首先 你可能活不长。如果你不接受。人会从半空中掉下来的事实。即使你幸存下来。比如你在一楼。就算你真的幸存下来。你的生活将是处处受挫。拒绝接受现实事物。不论你喜欢与否。 

所以我们接受万有引力定律。不仅如此 我们利用定律发明游戏。你们中喜欢运动的。教室里多数人都做运动。跑步时需要万有引力定律。你能想象哈佛和耶鲁。在没有万有引力的情况下进行橄榄球赛吗?你能想象没有万有引力情况下的篮球赛吗?我不是在说乔丹 而是普通人。我们接受它 并利用它发明游戏。然而说到情绪时 我们却不这么做。这是关键点。人类本性的痛苦情绪。一如物理世界的万有引力定律。"号令自然 必须遵守自然"。航空航天工程师。想制造飞机 必须遵守。加以考虑 学习。研究理解万有引力定律。心理学家也是如此。人类本性受约束的观点。号令自然 必须遵守自然。然而我们却不那么做。我们的文化不那么做。为此我们付出高昂代价。我们正在经历我所谓的"大骗局"。我们是那种嘴上说"你好吗?"。嘴上说"很好 棒极了"。事实上我们不是那么好。更合适的回应是"我正经历不顺。恰逢困境 我很难受"。然而我们不想承认事实。我们不准许自己为人。因为我们觉得是自己有毛病。如果我们有这样的情绪。每个人都说。"很好 棒极了 妙不可言"。当我们被问到"你好吗?"时。我才不要作唯一的扫把星。我才不要作唯一的丧气鬼。于是我说"很好 棒极了"。就这样我们帮忙壮大了这个大骗局。正是这个大骗局导致严重的抑郁。严重的抑郁是快乐付出的终极代价。这很好地解释了。为什么今天有那么多人。感到抑郁 达到45%。 

全国大学校园都如此。没有足够的"准许为人"。我不是说要我们肆无忌惮地表达情绪。有人走向我们。在William James大厅的电梯里。我说"你好吗?"。"感谢关心。这得从我三岁时说起"。我不是这个意思。在William James大厅也许会遇到这种事。因为那部电梯。运作得太慢。但不是那样。我不是那个意思。我的意思是一个空间。生活中的私密空间 和挚友一起时。和家人在一起时 最重要的是面对自己时。我们要准许自己为人。我说的不是屈从。远远不是那样 不是说。"我很抑郁 这是毫无办法的事。我接受我的本性。接受我的状态。就这样吧"。我说的不是屈从。我说的是主动接受。那是什么意思?它意味着理解有些事。我无法改变 有些事我能而且应该去改变。我们会集中讨论的一点是"差别"。在有关"改变"的课上中会着重谈。情感 行为和认知的差别。心理学的ABC。情感代表情绪 行为代表行动。认知代表思想。无条件地接受。准许为人 主要与情感相关。与情绪相关。它们是现实存在 就像万有引力定律。那不代表我们要接受我们的行为和认知。 

举个例子 我可以…。因为以前的经历。嫉妒我最好的朋友。那种情绪本身不代表我是个坏人。这是人类本性。我从没遇到过任何一个。从来没有或没有体会过…。也许达赖喇嘛可以。除了他。谁都嫉妒过别人。如果说达赖喇嘛真的无欲无求。那也是修行几十年的结果。嫉妒是人性的一部分。嫉妒没有好坏之分。 

愤怒也无好坏之分。抑郁和焦虑也无好坏之分。那是人性的一部分。问题在于。我要选择怎样的行为。去表达情感?道德在此介入。我可以选择道德和不道德的行为。对待最好的朋友 孩子或一般人。我仍会嫉妒最好的朋友。但选择对他宽容和亲切。认知也是如此 我们会深入讨论认知。当谈到认知行为治疗时。我可以感受它。但不代表。我要屈从于对那种感受的想法。我们会频繁谈到沉思。其实沉思痛苦的情绪。没有多大帮助。书写描述更有帮助。向别人倾诉更有帮助。要胜过总是想着。被女朋友甩了多么的惨。我还没在快餐店找到新目标。沉思没有帮助。我没有必要接受。所有不理智的想法。下周我们会更深入地探讨。如何认知性地重塑自己的想法。但情绪 情感即情绪。是无法改变的。重要的是要真实面对现实。这也是本门课的重要主题之一。从很多方面来说这门课不该叫"积极心理学"。我在你们选了这门课后才挑明。想退课也晚了。这门课不是讲"积极心理学"。而是一门"现实心理学"课程。 

因为积极心理学可能表示。我们只聚焦于积极的起作用的部分。忽略其余不起作用的。我们要做的就像是改变钟摆幅度。比起21比1的比率 两方更均衡。我们做得更多 除了专注于积极的一面。与此同时。也认同痛苦的情绪。和美好的情绪一样都是人性的组成。我们越早接受越好。那并不代表在1504课程结束后。或修读完另外100门课程。然后就可以把读过的书丢了。并不是说 你就不会再感受到痛苦的情绪。只是你的心理免疫系统会变得更强。希望在本学期未就见成效。心理免疫系统会变得更强。那不表示我们不再得病。而是意味着我们更有抵抗力 即使得病。也能很快康复。极其快乐的人和极其不快乐的人。区别不是在于一个会伤心。难过 焦虑或抑郁。而另一个不会。两类人都会。区别在于他们能够多么迅速。多么快速地从痛苦情绪中恢复过来。换言之 我们的心理免疫系统有多强。我们的心理免疫系统会加强。当我们准许自己为人时。很多同学可能读到过这首诗。听说过这首诗 它非常好地阐述了。主动接受含义的基础。它已成为AA运动的正式圣言。"主啊 请赐我安详 接纳我不能改变的事物。请赐我勇气 去改变我可以改变的东西。并赐我智慧去认识这两者的差别"。认识两者差别的智慧。幸运的是 这样的智慧可以通过学习获得。通过研究和在这个学期。进行的深刻反省中获得。 

现在我想做个练习。我想做一次群体冥想。希望你们理解。不只是在研究的认知层面上。还要在内心情感层面上去体会。准许自己为人的真正含义。感觉不自在的。可以不参与。认为没问题的 一起来做。如果你觉得很勉强。我仍建议你一起来。如果你从没冥想过。这是你们尝试的好机会。我对你们的一个要求是。如果你不做。不参与的同学。请保持安静。此外 我强烈建议你参与。现在我要大家一起。一起进入无条件接受境界。在座位上尽量坐直身体。 

如果可以 后背在靠背上放松。双脚舒服地放在地板上。感觉舒适 闭上双眼。转移呼吸的焦点。把思想的焦点 转移到呼吸上。向腹部深深吸气。然后呼气。再一次深深吸气。缓慢 平稳 安静地呼气。安静地重复呼吸。如果走神了 回到呼吸上。深深缓慢地吸气。深深缓慢 平稳 安静地呼气。多数人吸气的深度不够。我们没时间赞美我们的呼吸。我们的精神 我们的存在。联系 关联。心灵与身体的桥梁。情绪和思想的桥梁。脑和心的桥梁。继续深呼吸。继续深呼吸的同时。将焦点转向情绪 转向感受。你们好吗?感觉如何?注意你的情绪。不论是怎样的情绪 不论感觉到什么。让它流过你的全身 

自然地。体会那种情绪。你可能先体会到一种情绪。然后又体会到另一种 这没关系。不论是怎么情绪 接受它。体会它。准许自己为人。都没关系 呼吸。继续观察体会出现的情绪。不管是平静或快乐。不管是焦虑 困惑 无聊或喜悦。不管它是什么。继续向腹部深呼吸。平稳 缓慢 安静地呼气。让那种情绪像呼吸一样流动。随着呼吸的加深。用你的心眼。看自己走出这间教室。走在校园里。体会你的所有感受。都是人性的一部分。它们在那里。不论好坏。在你走在建筑物和树木之间时。在你看到朋友和同学时。让那些情绪流遍你的全身。自由地 轻盈地。通过体会这些情绪。不论是起是落。你正在做的是。作一个人。继续让呼吸和情绪流动。任其流动。如果你真的能准许自己为人呢?如果你真的准许自己为人呢?放飞想象。生命变得更加轻盈。更加简单。不是想办法战斗打败我们的本性。我们接受它。我们接受自己。我们接受出现的一切情绪。深深 缓慢地吸气。缓慢 平稳 安静 平静地呼气。安静地深呼吸几次。拥抱这份安静 静止。拥护你自己 你的情绪。下次呼气。深深 缓慢 安静地呼气。睁开你的眼睛。如果你身边的人睡着了。轻轻地把他或她唤醒。放飞想象。想象你将体验怎样的生活。你将拥有怎样的生活。如果你真的。准许自己为人。它是健康人生的支柱之一。不论是心理上还是身体上。试试吧。每天提醒自己一两次。准许自己为人。同样也准许别人这样做。你们应得的。 

周四见。 

第5课-环境的力量 

我想介绍一位来自"为美国而教"的代表。这个组织一直致力于做好事。在世界上宣扬善 我想…。这门课 过去两年一直向他们提供支持。我们希望继续下去。有请。 

大家好 能听到吗?非常感谢。我只占用90秒。我叫Josh Bieber。好的 我感觉很好。我是为美国而教的成员 现在为组织工作。负责新址开展。大四时。我勉强认为这是个机会。招聘到期日 固执的组织招募人员。打来电话 听电话时我差点睡着。我本不会应聘。但我还是磨蹭了两小时完成短文。多亏那个打来电话的人。这是我人生最美好 最快乐的两年。 

请再给我一分钟。希望大家对组织有两个了解。我想你们知道组织的目的。一是这个国家对学校里的。孩子们的教育是不平等的。低收入孩子在离开小学前。已落后多年。其中不到一半能高中毕业。这非常不公正。完全的不公正。二 你们能带来改变。我开始也怀疑组织。和自己的能力。但我知道这是真的。如果你走进一个五年级教室。学生们从第一天开始就落后很多。恨学校 因此也恨你。九个月后。他们成绩优异。恳求更多作业。希望与你共度周未。让你觉得一切都有可能。我在此足以说明。"为美国而教"对我来说是一个。真正让我实现想法和希望的地方。参与实现我们都想见到的改变。我愿意鼓励所有同学。不管你们计划明年做什么。请大家超越那些想法 考虑这个机会。它是我可能做过的最好的事。最后期限是明天。很容易做 不需要推荐。申请就行。给自己机会。像我一样爱上这份工作。最后我想说。现在坐在教室里的同学们。可能获得同龄人中最好的教育。世界上最好的教育。我们的学生和学校需要你们。请加入我们。感谢给我的时间。祝大家上堂精彩的课。 

大家好 只说几句。如果这里有公开课学生或本科生。不能在这周来上课的。今晚七点半有一堂开放课。是为公开课学员制作的录像。有时间和兴趣上课的。请在课后来找我。谢谢。现在我想介绍两位挚友。积极心理学的支持者。可能比任何人。更致力推广积极心理学。以严谨和有趣的方式。我请他们简单介绍一下。宾州大学应用积极心理学硕士课程。唯一的硕士学位。世界上第一个硕士学位。目前还有其他几个。他们会简单介绍一下。利用15分钟时间 介绍一下硕士课程。然后再开始今天的课程。希望让人兴奋。有请James和Debbie。 

非常感谢。很高兴能来到这里。和你们共度几分钟 真是乐事。在这样特别的课堂上。在这里你们将积极心理学的科学。和应用合二为一。当然你们很快就知道。Tal是位杰出的教授。能把复杂的概念。变得简单易懂 又不过份简单。而且发人深省。让讲座富于感召力。使我们想出去应用学到的理念。我想知道有多少人。对将积极心理学应用于。生活中感兴趣。请举手?好 很好。有多少人选择这门课。是因为有兴趣在未来事业中。应用积极心理学。在工作中? 好的。很好 好极了。希望你们起码能那么做。那些有兴趣。深入学习积极心理学的学生。在研究生程度课程获取知识。但同样能够。将所学带入事业。我们很高兴能够向你们介绍。一下宾州大学的。积极心理学硕士课程。谈谈将积极心理学。从课堂带向世界。 

你们知道积极心理学刚刚诞生十年。始于1998年 Martin Seligman。美国心理学协会当时的主席。积极心理学是他主持开创的。四年前。Martin Seligman创办积极心理学中心。在宾州大学。积极心理学中心的使命。是推广研究 培训。教育和传播积极心理学。2005年一月。我加入积极心理学中心。任教育主任和高级学者。2005年二月。Debbie Swick加入中心任教育助理主任。我们的任务是编制硕士课程。并在秋季推出。我一月加入 Debbie二月加入。三月完成论出。印刷并出版。申请的最后期限是四月。我猜有点乐天派。我们不知道是否有人报读。不知是否有人感兴趣。但我们不需要担心。我们收到超过一百份申请表。秋天时收了36名学生。后来的收生情况越来越好。这是我们当前的学生。 

2007年秋天 开课第一天。今年一共收录41名学生。我想多介绍一点学生情况。我们有三名毕业生或在读生。是这门课的助教。让人兴奋。Debbie Coen和Elizabeth Johnston。请站起来。还有Elizabeth Peterson 很好。看左上角。两年前 Gabriel修读这门课。在我们来推广硕士课程时。现在他正在我校读硕。这门课和我们的硕士课程。有很多合作。我们的学生年纪从22岁到62岁。来自美国各地。佛蒙特 弗罗里达 加州。来自世界各地。我们的学生来自英国 挪威。瑞士 印度。马来西亚 日本 香港。南韩和新西兰。25%到30%���������������轻人。刚刚大学毕业。希望在进行职业训练之前。先深入学习积极心理学。其余学生是专业人士。来自各行各业。在第一班里。学生中有来自苏格兰的。全国前社工总监。 

那很有趣。肾移植医生。摩根大通的前副主席。律师 非牟利发展计划的主任。学校的常务董事。Carlbrook Academy今年选择了我们的课。人力资源主管。咨询师 执行教练。甚至音乐家和喜剧演员。班里的学生可谓五花八门。 

我想用两分钟。简单介绍一下教学设计。再简介一下课程内容。本课程是一年制全日制学习。九月开始 八月毕业。我们的教育设计是混合模式。每个月 学生。有一次现场教学。其余时间。学生通过远程学习模式完成。这是一个职业硕士学位。教学重点放在。积极心理学理论。及在各行各业的应用。目前来说。积极心理学。还没有独立的执照或证书。学生来自教育。商业 法律 医疗等各个领域。都有自己的文凭。让学生每月来校一次。进行现场教学的设计。允许学生继续全职工作。我刚说过。课程是全日制的 你可能奇怪。"怎么可能一边工作一边全日上课"。请记住这是一个职业学位。高级管理人员教育模式。就是为那些。全职工作的人员设计的。同时上全制日课程。多数学生继续。他们从事的工作。继续全职工作同时又能上课。这种模式的另一个优势。是不需要学生居住在宾州附近。学生从全国各地赶来。即让我们惊讶且欣喜的是。远在欧洲和亚洲的学生。也决定赶来。我不知道现在你们如何来上课。但我们的一些学生。每月从南韩和新西兰赶来。参加积极心理学课程。可以想象。为课堂带去的兴奋和活力。也给教授施加了压力 我们要确保。能传授值得绕半个地球赶来学习的知识。 

学生的多样性。提升了教学体验。居住在海外的学生。赶来上课。我们还聘请多位教授。不局限于宾州大学。还有其他大学。我们请来最杰出的研究员。和积极心理学的从业者。简单概述2008年秋的现场教学。春季有五个。现场教学周未。其余时间是远程教学。简介一下课程。每位学生秋季上四门课。春季上四门课 然后是总结性课程。秋季时。课程集中介绍积极心理学的基础理论。Martin Seligman教。积极心理学课程概述。Angela Duckworth教研究手段与评估。让学生。真正把握积极心理学科学很重要。理解结论背后的研究手段。我教积极心理学原理。第四门课是"通向美好人生的方法"。请来多位杰出的研究人员。讨论他们的研究。春季 我们的课程。不再以理论为焦点。而是转向积极心理学的应用。因为这是应用积极心理学硕士学位。不知道你们是否学习了"特长与美德"。Chris Peterson的"价值和行为分类"。如果还没有学习。我肯定不久会学到。他也会教授有关自己研究的课程。我教积极干预应用。课程也包括一个服务教育部分。Karen Reivich和Judith Saltzberg Levick。教积极心理学和个体。即在与他人的人际关系上。如何应用积极心理学。不论是在工作中 还是与朋友家人在一起时。 

最后一门课。不知你们是否熟习"肯定式探询"。如果不知道。我想Tal会在本学期向大家介绍。David Cooperrider是这个领域的先驱。将积极心理学方法。带入组织 并致力寻求改变。不只是个人。而是整体水平的改变。最后要修的一门课是总结性课程。开课时间是夏季。这是一个独立课程。允许学生结合。课堂所学。进一步。应用于他们各自的职业。我们让学生进行研究。量化研究和质化研究。我们让学生。就感兴趣的领域写文献观后感。他们计划写的书的进展计划。或实验计划等等。以上只是课程简介。Debbie会占用几分钟时间。给大家看一些照片。过去几分钟展示了太多文字。Debbie会展示一些照片 更详细地介绍一下学生。交给Debbie。谢谢。 

大家听过了James的课程简介。我会介绍一下课程安排。和内容。这是一张开学周的照片。开学周时 同学初次见面。他们来宾州五天。我们的课程从早上八点到下午五点。我知道听上去很疯狂。但我们相处得很快乐 你们看得出。学生们兴趣浓厚。因为我们收录来自全球。各行各业的学生 在这个世界级学院。他们有机会进行真正的交流。班上有41个学生。一起上课。更像是论坛而非讲座。同学们有深入交流。休息时 午餐时。他们结识。来自世界各地的同学。宾州应用心理学硕士课程的。另一种交流形式是团体交流。每个人。和另外三五个人组织一个团队。你和他们合作不同的课题。远程学习可以做很多事。可以经常和这些人交流。这是你获得的另一种深度。当你与来自。世界各地的人合作时。我们很想利用总结性课程。作为踏脚石 让学生。更顺利地展开应用。很多人借助它。创建工作室。 

有一位学生把积极心理学初级课本。翻译成日语。她的译作得以出版。那个计划让她获益良多。还有学生。在学术刊物上发表了总结性课程论文。被邀请进行讲座 这是一位学生。在积极心理学峰会上作讲座的照片。我们录取学生进修硕士课程。他们可以做各种各样的事。我们给他们打下坚实基础。帮助他们加以应用。申请者打来电话时。询问得最多的问题是。"这个学位有什么用?"。我会告诉大家 这张快照上。Sasha Lewis Heinz 照片中的第一个人。目前她正攻读博士课程。在哥伦比亚大学进修发展心理学。读博前 她攻读了宾州应用心理学硕士课程。打下坚实的基础。那是她想应用和致力研究的。当她继续攻读博士时。上面第二名学生是Saniel Mimen。她在投资公司工作并利用业余时间。她挤出时间。创建了积极心理学每日新闻网站。内容囊括宾州硕士同学会撰写的。有关积极心理学的研究和应用文章。他们真的在进行研究和应用。另外 她还和。其他多名硕士同学合力。参与到培训中来 参与我们。积极心理学中心的项目。中心有各种项目 去英国培训他人。向学校推广积极心理学。我们派去参与项目的是硕士生。因为他们有所需的教育背景和基础。去完成这些项目。 

我们希望继续。安排同学参与。各种机会。Caroline Miller 照片中的第三个人。她是一位作家。正准备出版她的第二本书。她也是一个演说家和教练。攻读硕士学位后。她获得很多授课的机会。她的方向是目标设立。和运动心理学。硕士课程大大扩展了她的能力。并非所有学生都是哈佛毕业生。但这三位都是哈佛毕业生。这张快照颇有趣味。我们希望有更多哈佛毕业生参加我们的课程。James 你来总结吧? 

非常感谢 Debbie。我们的推广名为"从教室走向世界"。宾州应用心理学硕士课程显然是一个途径。将积极心理学带向世界。再简单介绍另一个机会。我们刚刚创办国际积极心理学协会。它将成为一个重要国际组织。帮助积极心理学的研究人员和应用者。更便利的交流与合作。这是照片。记录了第一次董事会议。在十月举行。照片上没有的是Tal。他是董事会成员之一。也是美国乃至全世界。众多杰出积极心理学家之一。协会设立特别的学生会员。我鼓励你们了解一下。国际积极心理学协会。网址是www.ippanetwork.org。加入并把握积极心理学领域。最新发展动态。很希望有时间回答你们的问题。但Tal准备了很多精彩内容。请继续上课。希望大家知道。今天下午有一个咨询会。在哈佛大厅103教室 从三点到四点半。我们非常期待。有兴趣的同学过来和我们谈谈。到时我们可以更详尽地。解答你们的问题。下课后 我们也会留下几分钟。我们带来宣传册。欢迎有兴趣的同学来取。在我们的网站上。可以了解到更多信息 pennpositivepsych.org。也可向列出的地址发邮件。希望我们能在宾州应用心理学硕士课程上见面。同时。祝你们本学期取得成功。深入学习积极心理学的科学。并将之应用于生活。思考如何把积极心理学。从教室带向世界。 

好的。成为宾州应用心理学硕士班的一员是种特权。因为在那里共度一年的。是来自各个领域的杰出人物。我们将会谈到的很多人。比如Bob Fredrickson 过一会我们会谈到他。或者以后会谈到的David Cooper。当然还有Martin Seligman。都会去那里给大家讲座。你将和他们亲密相处。真是一种特权。今天我想结束有关基础前提的讨论。讨论最后一个前提。还要再看这个吗 不。这个也不用看了。可以了。 

最后也是第五个前提很重要。是一个哲学前提。但我想在课程开始时介绍。你们好理解我讲课的起源。这门课程的来源。因为很多人说。好的 快乐是重要的。我们寻找它 我们有宣言。国家宣言。关于快乐对我们多么重要的个人宣言。但那不代表它是重要的 问题在于这个"是"。它是重要的但不代表我们应该去做。我认为它不仅重要。而且也应该是重要。首先要阐述这个"是"。快乐 不论我们喜欢与否。不论它是有意识地。还是潜意识地 不论它是明显的还是隐晦的。对多数人来说。不是全部但对多数人来说。快乐是最高追求。我们有宪法保证我们追求快乐。我们投入大量努力。大量时间思考自己和他人的快乐。二千多年前的亚里斯多德。"快乐是人生的意义和目的。人类存在的最终目标"。William James。1890年在"宗教经验之种种"中写道。如果要问。人类生命最主要的担心是什么?我们应该得到的一个答案是。"是快乐"。如何获得 如何保持。如何重获快乐。是多数人时刻怀有的秘密动机。是他们愿意忍受一切的目的。你们可能听说过William James。这里有幢大楼以他命名。他在二百多年前谈到快乐。亚里斯多德在二千多年前也谈过。而且不只是西方人。达赖喇嘛 "不论一个人是否相信宗教。不论一个人相信这种还是那种宗教。我们生命的最终目标是快乐。我们生命的主要活动是寻找快乐"。 

快乐是重要的。不论是有意识的。还是潜意识的 明显的还是隐晦的。问题是它确实重要。但快乐应该是重要的吗?快乐的道德维度呢?还有很多事情在发生。世上有很多重要的事情要做。为什么把快乐作为最终目标。作为主要担心。为什么由它来决定生命活动?换言之 幸福有什么好处?积极情绪有什么好处?大量研究都尝试回答这个问题。暂不说研究。感觉好就是感觉好。想到快乐 注入内心。这是亚里斯多德的"同一律"。一即一 感觉好就是感觉好。这本身就是辩解理由。如果能感觉好 为什么不感觉好呢?证明。快乐为什么是不重要的重担可能落在。执否定观点的人身上。我们会在以后谈到这个论点。但首先的关键问题是快乐本身是好的。几乎无需辩解。然而除了感觉好。快乐还对我们的人生有帮助。对人际关系有帮助。对他人有帮助。Barbara Fredrickson做了这项研究。在宾州大学任教的学会成员。你们曾在上面见过她的照片。Barbara Fredrickson的观点是。积极情绪有一个进化理由。它们的目的不仅仅是让人感觉好。比如它们有助我们超越。现在的思考范畴 拓展我们的思想。有助我们建立人际关系。帮助我们建立能力。这门课的一个主要概念。就是积极情绪。积极心理学。就是要建立能力。我们使用的两个类比。是加强我们的免役系统或加强"心理引擎"。获得更强大的忍受能力。不只是从负到零。还要从零到正。这是Barbara Fredrickson的观点。 

我想进一步引用她的文章。也是你们这周要阅读的。"我们应该努力培育积极情绪。在自己身上和周围人身上。不能只将它们作为终极状态。还要将其作为手段实现心理成长。改善心理和身体的健康。我称其为积极情绪的扩建理论。因为积极情绪似乎能扩展人们的瞬间想法。和动作指令库并建立持久的个人资源。通过体验积极情绪。人们实现改变。变得更有创造力 更博学。适应性更强 更易融入社会。获得更健康的人格"。体验积极情绪有诸多好处。是双赢的。它让人感觉好。对我们有好处 对整个社会也有好处。一会我就会解释。她谈到的问题包括。积极情绪有助我们克服消极情绪。体验消极情绪时。我们的意识。我们的思维变得狭窄 收紧。只专注于一件事 

比如。那可能是件好事。一头狮子朝我冲过来。我不想考虑宾州大学硕士申请的事。不想考虑室友说了什么。我只想考虑这头狮子。我的意识变得狭窄 收紧。进入"对抗或逃跑"模式。有狮子朝我冲来时是件好事。但它变得不是很好。如果我的意识继续狭窄。收紧超出威胁 超出困难。我们知道经常会进入下行螺旋。一个恶性循环 当我们进入狭窄收紧的模式。随便举个例子。女朋友离开了我。我的思维收窄缩紧。一心只想着被女友甩了。结果就是。我感觉悲伤。因为我一心只想那个 悲伤。是一种痛苦的情绪 一种消极情绪。导致进一步的收窄缩紧。那有可能。并非总是 但可能一直持续下去。直到变成抑郁。那时我将很难脱离这种下行。恶性的循环。积极情绪正相反。它们扩展建立。扩建导致积极情绪。积极情绪再进一步扩建。所以是个良性循环。我眼界宽了。关注其他人。关注其他事 思考现在我能做什么?我能去哪里?我把时间用在哪里?很经常的是。积极情绪把我们带出下行螺旋。形成一个上行螺旋。积极情绪可以是看一部滑稽电影。可以是"一次深呼吸"。探讨"心理 身体"时会谈到深呼吸。积极情绪可以是与朋友交流。一次愉快的交流。积极情绪。能把我们带离这个下行螺旋进入这个上行螺旋。而且不用很久。 

困难在于将之与准许为人相结合。去体会情绪 实现转移。但不会进入下行螺旋。在发生小小意外六个月后 仍心情低落。我们会谈到什么是合适的时候。如何找到合适的时候。以及如何在。思考痛苦情绪。和让痛苦情绪滑入沉思到平衡。沉思并非总有帮助。它有助克服消极情绪。还有创造力 我们的眼界更宽。能产生更多联系。看到以前没有看到的联系。有很多关于"抑郁创造者"的说法。如果你想拥有更高水平的创造力。必须承受抑郁。其实并非如此。躁狂抑郁病人一般更有创造力。但一般在躁狂阶段。在抑郁阶段 我们的思维收窄。我们失去创造力。当然有例外 但总体而言没有。 

实际上有这样一项研究。在内科医生 医生中间进行。向他们提出一个。与肝脏问题有关的很困难的问题。那是一个病人的病症。医生们被随机分成三组。第一组是对照组。他们必须解决这个问题。第二组听一段有关。医学人文观的声明。为什么作医生如此重要。第三组得到糖果。让他们获得愉快放松的好心情。第三组得到糖果。并获得积极愉快的心情。明显比另外两组有更好表现。他们考虑更多选择。得出更好的问题解决办法。这是这个领域中的一项研究。再举个例子 小孩子。一群小孩组成对照组。有人教第二组小孩回想。让他们大笑或微笑的经历。第二组。在学习任务上比对照组表现更好。因为他们被置于积极情绪。它是双赢的。它在很多方面定义了"没有痛苦就没有回报"。不论是对医生的教育还是在学校。在一般的工作场所也一样。因为动力和精力。不需要研究就能明白。你知道感觉好时。动力和精力更充沛。 

当然有很多研究支持这一点。最终达到成功。他们调查专业人士。能更好地控制自己的情绪。获得积极情绪。脱离狭窄收紧进入扩建的人。长远会取得更大成功。他们不是没有痛苦情绪。没有的都是死人。那些人经历痛苦情绪。但同时能够更好地。把自己 他们的意识 思想。他们的感受向积极方向转移。越快乐的人越成功。因为他们有更多精力 工作更努力。因为他们在追求。而不是在逃离。这叫"靠近而非逃避目标"。我们将在探讨"目标"时谈到。还因为他们建立更好的人际关系。他们更开放更宽容。他们的创造力更强。所有这些因素最终导致更高层次的成功。积极情绪不仅有助成功。不仅能让我们感觉好。还有助我们获得幸福。乐观的人。不是指盲目的乐天派。而是脚踏实地的乐观主义者。平均而言 明显更加长寿。保佑你们。免疫系统更强。所以对身体健康也有帮助。现在的问题是 道德问题。那么其他人呢?我谈论经营自己的生活。追求自己的快乐 这怎么不是自私?答案是"是的 那是自私"。当我对自己说。"我想更快乐"。我对自己说"我想更快乐"。那是一件自私的事。它是坏的吗?是不道德的吗?在我们的文化中。自私和不道德成了同义词。那是个问题。这就是为什么 因为这是头号。将自私等同于不道德是头号原因。多数是在潜意识里。导致不快乐的头号原因。人们为追求自己的快乐感到内疚。人们在自我感觉很好时会内疚。我怎么可以? 我怎么敢自我感觉良好?我怎么能追求自己的快乐。当世上有那么多痛苦时?世上确有很多痛苦。我们要如何回应? 

首先 快乐是正和博弈。不是零和博弈。也不是负和博弈。我的快乐不是从别人那里夺来。那将是负和博弈。我有更多 你一定更少。甚至不是一个零…。或者我更少。你也会更少 那也是负和博弈。或零和博弈 我更多 你就会更少。饼就那么大。不是这样的。它是一个正和博弈。为什么?因为快乐是感染的。如果我更快乐 我更有可能。对他人的快乐和幸福作出贡献。感受快乐 换言之。也是一种道德状态。对他人的幸福作出贡献。佛在几千年前就谈到它。"一支蜡烛可以点燃千支蜡烛。蜡烛的生命不会被缩短。分享绝不会减少快乐"。就像传递光明。你快乐并努力争取快乐。就是间直接对他人的快乐作出贡献。就像上次大笑的婴儿 把你们逗笑。笑是感染的。一般来说 努力争取快乐的人。再一次 不是那些总是保持快乐的人。那种人将很难建立人际关系。因为他们是死的。一般来说 活着的人。努力争取快乐。体验生活的变迁 起起落落。但努力争取快乐而且越来越快乐。有更好的人际关系。更宽容 更接受他人。更容忍他人和他们自己。 

很多研究表明自我帮助。换言之 努力为自己争取快乐。有助我们获得幸福。也能引导我们对他人更加宽容和亲切。一位重要的研究者。积极情绪领域的首批研究者之一。Alice Isen进行了这个研究。她反复证明。感觉好如何对我们和他人都有益。反之亦然。这是快乐很美妙的地方。别人快乐你也快乐。因为帮助他人也是帮助自己。记得你们这周的任务。一些同学已经知道。就是去做超越以往所做的事情。一天内多做五件好事。多做五件好事。这是Sonja Lyubomirsky做过的一项研究。我在第一堂课提及她的书。《快乐之道》。她很好地。证明人们。不论是在一周内多做五件好事。可以更多。不必局限于五件。或者人们在一天内多做五件好事。都能让他们更幸福。帮助他人也是帮助自己。我说到的一件事 只说了一半意思。即最自私的所为就是善举。只说了一半意思。因为两者有内在联系。两者间有自我实施的循环。帮助他人就是帮助自己。帮助自己继而帮助他人。不要把它看作自私。有人可能会这样简单考虑。不该把自私等同于不道德。我们应该把它视为。人性的美妙之处。是我们应该颂扬的本性。我们的快乐与他人的相关连。我们通过同情网络和他人相关连。那是人性的美妙之处。一样值得我们进一步颂扬的东西。 

如果不颂场。如果我们不欣赏人性的那一部分。人性的那一部分将贬值。欣赏有两个意思。一是懂得感谢 二是传递。如果我们欣赏人性中。欲望中好的东西。如果我们欣赏本性中那一部分。它将升值 我们将获得更多。如果我鄙视它并说。"很可怕。我从帮助他人身上获得利益"。那么本性中那一部分将随时间贬值。你们正在读我书中的冥想部分。书中有更详细的阐述。也给出了哲学基础。因为在很多方面有别于"数数和思考"。那种方法在20和21世纪处于统治地位。思考道德。感谢好就是感觉好。也能让他人感觉好。 

我想以介绍一位。一生致力于。传播快乐的人来结论这个前提。圣雄甘地 这是关于他的一个故事。一个女人像很多人一样。来寻求甘地的建议。她长途赶来。还带着儿子。她拜见甘地。在他面前说 她对他说。"我长途赶来 因为我儿子有个问题。我儿子吃太多糖。希望你能告诉他别吃太多糖。因为糖损害他的健康 牙齿。他会听你的 他很崇拜你"。甘地看着她说。"夫人 你能一个月后再来吗?"。她不明白为什么但还是听了他的话。因为他是甘地。她离开 长途返回。一个月后又回来。再次与甘地见面。她在他面前说。她说"一个月前我来过"。他说"是的 我记得"。她说"你能告诉我儿子。不要吃太多糖吗?"。于是甘地注视着那个孩子说。"孩子 不要再吃太多糖"。就是这样。那个女人很困惑 她鼓起勇气。说"圣雄 非常感谢。我肯定他不会再吃太多糖。但为什么不在一个月前告诉他。在我长途赶来的时候?"。他说"夫人 因为一个月前。我也吃太多糖"。是的 我知道这是个很深奥的笑话。需要时间 很高兴你们消化了。甘地说过一件事。出自他的精彩自传。《我的对于真理的实践经历》。"成为你想在世上见到的改变"。只有这样才能实现改变。 

我和想大家作个小练习。特别对男生来说是个困难练习。但请忍耐。感觉不舒服的。就不要做 希望按下面的指示做。让你的姆指和中指。尽可能形成一个直角。可能有点痛。尽可能形成一个直角。看着我 就像这样。直角。再让这两个手指。中指和姆指。尽可能保持直角时围成一个圆圈。男性比女性更难做到。男性的柔韧度更小。围成一个圆圈。看上去像只兔子。如果能投影的话。中指和姆指。好 看着我。像这样 尽可能围成一个圆圈。不是完美的圆但要尽力。把这个围起来的圆圈。看到吗? 这是个圆圈。把这个圆圈放在面颊上。另一个地方。可能需要一会。我看到的多数同学都放在了下巴上。我很清楚地说是"面颊"。问题就在这里。多数人都是照你做的做而不是照你说的做。请记住这一点。我不认为课堂上有任何人。会告诉我。"我的人生目标是让别人不幸。我真得想那么做。我想要世上所有人都不幸"。这里不会有人真得说出那种话 希望如此。我们在座的多数人 不论做什么。从现在到未来 都是理想主义者。我们想为世人造福。想传播快乐。但记住 人们照你做的做。不是照你说得做。你也许想传播快乐 通过你的言话。但最好的方法。传播快乐的最佳方法是努力争取自己的快乐。因为那时你就成为榜样。 

这也适用于领导。领导最重要的不是说的话。而是如何以身作则。为人父母最重要的。不是告诉子女"诚实很重要"。而是你自身有多诚实。如果你想传播快乐。"成为你想在世上见到的改变" 以身作则。这就是五个基本前提。它们构成这门课的基础。我们将在未来两个月。扩展这些前提。最重要的是如何将研究。原则运用于生活。 

现在开始下一课。"信念即自我实现预言"。我必须承认。这个话题点燃了我的想象。当我是孩子时想到它。当我还是一名运动员时。那时我明白了信念的力量。后来它挑起我对心理学的兴趣。我想从一个特别的故事开始。在很多方面 它是我听到过的。第一个有关心理学的故事。那个故事让我明白。心理学对幸福和成功是多么重要。 

成功对14岁的壁球手来说。是我生命中最重要的东西。故事讲的是Roger Bannister。举下手。多少人听说过他?好的 只有少数。听过的可以重温一次。Roger Bannister是个跑手。跑一英里。直到1954年 在四分钟以内跑完一英里。被认为是不可能的。事实上医生证明四分钟跑一英里。是人类能力的极限。生理学家进行实验。在科学上展示证明人类能力极限。是四分钟跑一英里。不可能少于四分钟。跑手们证明了医生和科学家…。证明了他们是对的。一英里跑四分钟两秒。四分钟一秒 但没有跑手能少于四分钟。从一英里跑计时以来。当人们开始计时跑时。那是不可能的 医生和科学家证明了。跑手 运动员。世上顶尖选手证明了医生们是对的。然后Roger Bannister出现。他说"四分钟内跑完是可能的。我要做给你们看"。说这话时 他是牛津大学的医学博士。也是一名出色的跑手。顶级跑手 但时间远高于四分钟。他的最好时间是四分12秒。自然没有人把他当真。但Roger Bannister坚持苦练。不比别的跑手练得更苦。但和世上别的跑手一样苦练。而且有进步。他突破了四分十秒。四分五秒。跑到四分两秒时停止。像所有人一样。无法低于四分二秒。他不是世上最好的跑手。但也是佼佼者。但他还是说"有可能。在这件事上人类没有极限。我们能在四分钟之内跑完一英里"。他坚持这么做 坚持练习 却一直失败。直到1954年 1954年5月6日。在重回故校时。Roger Bannister用了3分59秒跑完一英里。轰动一时 登上全世界的头版头条。"科学遭到挑战""医生遭到挑战"。"不可能成为可能"。它成为梦想一英里。现在听这个。数十年来 自从开始一英里跑计时以来。没有人突破四分钟界限。那被认为是不可能的。但在5月6日 Roger Bannister特做到了。六周后 澳大利亚跑手John Landy。一英里跑了3分57.9秒。第二年 1955年。37名跑手在四分钟内跑完一英里。1956年 超过三百名跑手突破四分钟界限。 

这是怎么回事?运动员们更加努力训练?当然不是。是有了新的技术 新的鞋子?不是 是信念。信念是多么强大。不是因为跑到那个时间。就说"不好 超过速度极限了。放慢速度吧"。根本不是那样 他们尽了最大努力。最大可能。然而他们的潜意识限制了他们。阻止他们突破界限。那不是医生。生理学家和科学家们声称的身体界限。而是心理界限。Roger Bannister攻破了那个要塞。意志和心理上的要塞。信念即自我实现预言。它们经常决定我们的表现。我们表现的多好或多糟。经常决定我们的人际关系有多好或不那么好。它们是人生成功和幸福的头号预言。我们将会谈到。今天和下节课我们要谈的是。信念如何形成现实。它如何发挥作用?信念力量背后的机制和科学是什么?因为在很多方面 这听上去像神秘主义。一部分是很神秘 仍无法理解。但我们会谈到我们所知的。信念为什么会起作用及如何起作用。可惜 人们对乐观主义有很多误解。因为自助运动在很多方面。是在告诉我们信念力量是怎么回事。提到《思考致富》那本书。我们会讨论各种秘密。它们如何通过思想创造现实。其中有些是事实 但仅是部分。我们将连通象牙塔和主街。说明那个信仰背后的科学和危险。最重要的是如何应用?如何加强我们的信念。如果真有这么紧密的关联。如果希望。乐观主义和信念真的。有着非常紧密的关联和预测力。如果它们决定我们的产出。无论在体育场上。还是在工作地点 在人际关系上。如果它如此重要 那么该如何加强信念? 

我们会谈到Bandura对自我效能的研究。谈到Nathaniel Branden对自尊的研究。如何才能将梦想变成现实。可以是政治梦想。我们将谈到马丁路德金的"梦想法"。他是如何做到 做了什么。也可以是个人梦想。我们会谈到Herbert Benson和Bandura。再一次引用佛的话。"境随心转。全由意念升起。我们的念头造就了世界"。这是几千年前的一个主张。现在我要做的。是专注于这个主张的科学基础。 

我会从皮格马利翁开始讲起。"Pygmalion"是古希腊语。皮格马利翁是个雕刻家。成年后他开始寻找理想中的女人。他想结婚。他在雅典城里寻找。找遍整个希腊。找遍整个希腊帝国。去希腊帝国以外寻找 寻找他理想中的女人。一个能和他结婚的女人。但没有找到。不论去哪里找。这是可以理解的。那是1879年之前 雷德克利夫学院还没成立。远在哈佛成为男女合校之前。他找不到理想中的女人 于是他返回雅典。他对自己说。"我不再寻找理想中的女人。我要雕刻一座雕像"因为他是雕刻家。"我会以她的形象雕刻一座雕像"。他雕刻出那座雕像。当他看着她时。他被激动之情和悲伤淹没。因为他找不到她。于是他开始哭泣。宙斯 雅典娜 特别是阿芙罗狄蒂。俯视他 深感同情 于是把雕像变活。当然他们幸福地生活下去。这是这个词的由来。然后萧伯纳借助这个词。创作了一部情节类似的戏剧。后来被改编成音乐剧《窈窕淑女》。说的是Higgins。一位博士 语言学家。如何把一个卖花女塑造成贵族。当然故事讲完时。是她塑造了他 改变了他。非常精彩的故事 当时很重要的故事。因为它挑战了整个阶级体系。人生来就有阶级 不能。也不应该被改变。在当时是很重要的戏剧 今天也是如此。皮格马利翁和人是可以被改变。被改造的。 

1960年代。Robert Rosenthal曾在本系担任多年主任。现在就职加州大学河滨分校。把皮格马利翁的思想应用于课堂。他是这么做的 Rosenthal随便走进几间学校。去到学生中间。让他们接受一次测验。然后去找他们的老师并告诉他们。"你们的学生刚接受了一个新测试。一种新的学业测试 叫作快速迸发者测试"。测试能让我们找出。将在新学年在学业上。取得巨大进步的学生。在新学年取得巨大进步。换言之 他的意思是。测试找出最有潜力的学生。他发现…。他对这些老师说的是。"这只是仅供参考 不能向学生透露。我们不希望学校有歧视行为。只是让你们知道 这些学生有巨大潜力。这是一项新发明的测试。只有你们知道就好"老师们不知道的是。对学生进行的测试。只是普通的现成的智商测试。老师们也不知道这些所谓的。"快速迸发者"或者潜力巨大的学生。都是从一个帽子抓阄抓出来的。他们都是普通学生。和别的学生没有区别。但老师们以为他们有巨大潜力。Robert Rosenthal离开学校。学年结束时再回来。他发现 他查看他们的英语成绩。"快速迸发"学生有明显进步。比任何学生的进步都快。他查看他们的数学成绩。因为英语不够客观。也许是老师们。觉得某些学生英语更好。于是他查看他们的数学 客观的成绩。这些学生也有明显进步。比其他任何学生的进步都大。接下来才是有决定性的结果。Robert Rosenthal又对所有学生。进行了一次智商测试。他发现被标签为。被随机地标签为"快速迸发"的学生。智商在一年间有很大增加。而且在长期研究中保持着增加。 

这让人不解。智商。本来是生而有之的。它是恒定的。从出生到死亡时都不会改变。或者说他们这样认为。但它却因为。老师对学生的信念而改变。信念即自我实现预言。这项研究说明了什么?老师们被骗了吗?突然产生幻象? 不是的。是因为他们之前被欺骗。幻象是他们看不到。就在他们眼前的东西。即每个学生都有潜力。Robert Rosenthal来到 可以说是骗了他们。"骗"他们注意到一直在眼前的东西。在那之前 他们没注意到车上的孩子。可以这么说。之后 突然他们在一些孩子身上。看到一直都有的潜力。他们欣赏那种潜力 潜力得到升值。他们灌溉 播撒阳光。种子开始发芽生长。这正是Marva Collins每天在学校里做的事。她看到存在的潜力。她不是发明家。没有脱离现实。是人们看不到他人身上的潜力。也看不到自己身上的潜力 我们会讨论到。他们没有看到整个现实。只看到一部分。完全无视车上的孩子们。我们知道无视一部分现实是多么容易。就算他们就在我们眼前。经常需要被问及。问及我们之前忽略了什么。不论是对风险人口的研究。或者是人口结构。或者只是对我们的。人际关系和自己提出负面问题。或者没有看到。每个孩子身上都有的潜力。如果我们看到它。只要我们看到它。欣赏它 灌溉它。播撒阳光。它就会感恩 成长。Rosenthal做的只是将他们注意力。转移到一直在那里的东西上。在工作场所也是如此。几百几千次地重复着。 

皮格马利翁效应。曾在一个工作场所里重演过。主管或经理被告知 这些是。极有潜力的员工。那些员工也是随机挑出的。但真的变成最有潜力的员工。并取得更大成功。他们的保持力增加 工作表现出色。在公司更有可能被升职。并留在公司里。这都只是期望产生的结果。角色调换后结结果一样。 

Jamison在1997年做了一项很有趣的研究。她想知道角色调换后是否有相同结果。她去到由同一名老师负责的两个班。开始上课之前。只告诉一个班。这位老师受到。从前学生的极大好评。他们作为心理学家。给这名老师极高的评价。然后他们离开两个班。学年结束时 情况如何?首先 干预组给这名老师的评价。高于对照组。而且学生对课业。付出更多时间 成绩比对照组好。因为他们相信。他们被欺骗相信。这位老师的水平比实际情况高。换言之 学生看到老师身上的潜力。学生们表现得…与老师无关。老师没有更好的表现。但学生表现得更好了。当他们有更高的期望时。当他们相信自己的老师时。如果各位想学好1504这门课。你们知道该怎么做了 是吗? 

真得管用 信念即自我完成预言。我们创造我们的现实 歌德说过。"人是怎样便怎样待他 他便还是那样的人。一个人能够或应该怎样便怎样待他。他便会成为能够怎样或是应当怎样的人"。现在我想谈一个相关话题。它对心理学家很重要。对生活应用也很重要。那就是我们创造的情境。或为我们创造的情境的重要性。在很多方面 心理学家。肯定地说是社会心理学家。都会被在情境之下完成的。一系列研究所影响。但多数是消极的情境。修读或没修读过初级心理学的。可能都听说过"阿施从众心理实验"。个体对群体存在从众现象。很多同学可能听说过。Milgram的"权力服从"。如果没听说过。我不会在这里深入介绍。阅读了解一下 去网上找一下。两者都是心理学领域。最精彩的研究 在街上随便找个人。在实验者的指示去电击另一个人。甚至电击到另一个人尖叫恳求停止。因为实验者说实验必须进行下去。而且实验者经常穿着一件白大褂。就像医生或者实验者。因为他们说实验必须进行下去。经常的 人们会。多数人 美国人都会继续电击那个人。甚至电击到另一个人哭泣。恳求放他出去。仅仅是因为那一句。"实验必须进行下去"。情境的力量 对权力的服从。这项研究的目的是为了证明。为什么大屠杀只发生在德国。那里的人更容易服从权力。他们发现全世界的人。都倾向服从权力。不论是在美国还是德国。全世界都是如何。如出一辙 

情境的力量。Philip Zimbardo的监狱实验。在斯坦福进行 他们做的是。没读过的找来读。我不会深入介绍。他们从街上随便找来一些人。分别扮演典狱长。狱警和犯人。这项实验预计进行两周。证明进入角色的效果。一周后 他们发现。实验必须终止。因为随意找来的 随机分配的典狱长们。变得非常残暴。他们侮辱犯人。进入犯人角色的人受到侮辱。就像犯人经常感受到的侮辱。这解释了Abu Ghraib监狱丑闻。时事现象。伊拉克战俘的遭遇。随便一个人一旦进入角色。他们进入得太深。Zimbardo不得不在一周后终止实验。可以去视频网站上看到。这是很重要的。知道很好 知道很重要 但不够。因为情境如此强大。为什么只强调消极的东西?为什么不创造积极的环境。让我们获得更快乐更讲道德的生活?正像很多积极心理方法一样。被21比1的比率忽视。 

我要和大家分享这个领域的两项研究。两项研究都由本校的Ellen Langer完成。顺便说一句 我现在要讲到的这项研究。是即将推出的Langer教授记录片的主干。扮演Ellen Langer的演员 Langer是。心理学系的第一位女性终身教授。扮演Ellen Langer的是詹妮弗安妮斯顿。电影有望在一年后上映。但电影讲的是我要谈到的这项实验。实验在1979年进行。Langer的实验是。她找来。75岁以上的男人 把他们送去一座别墅。那是间以1959为主题的别墅。虽然实验在1979年进行。音乐来自1959年。他们读的杂志都是1959年前后的。日报是1959年的。所有的一切都是1959年的。就算他们要进入角色。和Zimbardo的实验一样。要扮演1959年时的角色 仿佛年轻了20岁。这是个心理学实验。在实验前后进行各种测量。他们发现。在别墅待了一周。在一周结束时。在一周结束时。心理和生理年纪都减小了。比如 他们在各项测试中变得更灵活。他们变得更强壮 他们手掌 双腿。身体都变得更强壮。他们的记忆力有明显改善。他们的智力水平。在实验前后进行测试 再与对照组相比。在仅仅一周后就有明显改进。她测量指骨间的距离。人越老 骨骼间空隙变得越小。指骨变得更紧。一周后 他们的手指变长。他们变得更快乐。变得更加自立。更少依懒他们 不论是他们自己。还是家庭成员的评估都是如此。他们变得更健康。他们的视力和听觉有明显改善。在短短一周时间内。就是因为他们进入了强大的积极的情境。与在外面世界遭遇的。典型情况和偏见不同。仅仅通过"扮演"某个角色。他们变成那个角色 正如Zimbardo的犯人。在短短一周内真的变成犯人。 

她的另一项研究。她在书中有叙述。《念》 我强烈推荐这本书。她测量人们的视力。给他们一幅普通的视力表。测量他们的视力 记录结果。然后她把让同一群人。这次让他们穿上飞行服。同时让他们坐进飞行模拟器。然后给他们看完全相同的视力表。相同的距离 相同的视力表。唯一的不同是 他们坐在。飞行模拟器里 穿着飞行服。他们坐在那里看视力表。然后她又进行了一次视力测试。40%的参与者的视力有明显改善。仅仅是在改变情境的情况下。相同的距离 相同的表 什么都一样。只是情境不同。问题是我们如何创造积极情境。如何创造一个利用角色自我改进的情境。 

我想介绍两个研究。先说第一个再说第二个 关于环境。首先是Barge的研究。影射分为潜意识和有意识影射。比如你看着一个屏幕。在几毫秒内闪过一个词。这个词让你有了准备。有很多关于如何利用。典型消极情况进行影射的研究 比如。使用偏见。或者积极的 但对积极影射的研究不够。Barge做了下面的实验。他用与"老"有关的词影射人。比如"老"这个词。"拐杖"。"佛罗里达"。他用这个词影射人们。他用与老有关的词影射人们。然后他对比他们和对照组的。智力和记忆力测试结果。用与"老"相关的词影射过的人的记忆力。他们的表现比对照组差。第二 他观察这些人。记录他们从实验地点。走到电梯的时间。还找来盲测员。那些不了解情况的人。评估他们的走路。用"老"影射过的人行走起来。真得比其他人更弯腰驼背。走向电梯的速度明显慢。虽然不知道自己被与"老"相关的字影射过。但他们走向电梯的速度。比没有受到影响的人慢。完全是潜意识的。然后他们用与"成就"相关的词。影射人们。受到影射的人。用与"成就"有关的词进行潜意识影射的人。测试结果比对照组好。他们的记忆力得到改进。面对困难任务时更有持久力。问题是…下次课我会谈到的是。如何创造有意识的。和潜意识的积极环境。让我们带出最有道德。最成功的自我 并且去欣赏那个自我。帮助环境带出最杰出的自己。下堂课见。 

第6课-乐观主义 

希望各位度过了个愉快的周末。好了 直入主题。上回说到哪儿了。上回我们谈了环境的力量。谈到了信念是如何成为自我实现的预言。以及我们是如何通过环境。对他人产生期望及信念。这些信念会变成现实。还谈到了为什么说社会心理学本质上是从。某些重要的实验开始的 比如米尔格伦实验。又称权利服从研究。比如津巴多的监狱实验。试验中 囚犯被关在……。学生扮演囚犯及看守的角色。完全进入了角色 以至于。一周后实验被迫中止。而原计划是要进行两周。接着我们提出了一个问题。既然环境的力量如此巨大。能使人改变。那能否创造出一种强大但积极的环境。将人的最佳状态充分调动起来?答案是肯定的。我们对此做了深入分析 并谈到了。Langer教授以及她的研究。她让一群老人置身于仿20年前的环境中。要求他们装作回到20年前的生活。这些老人竟然变得年轻了。他们的智力水平提高了。记忆力提升了。视力听力也改善了。手指骨骼间的距离变长了。这是年轻的迹象。 

他们自评及互评都认为。自己更健康 更年轻了。就因为他们装作自己处在1959年。而不是1979年 这就是积极环境的力量。她还进行了另一项实验。对象要进行视力检查。他们被安排在了飞行模拟器里。做完全一样的视力检查。因为环境的不同 测试结果竟然提高了。然后我们接着继续讲 上回结束的地方就是。分析影射 影射是……。可以从意识及潜意识两个层面进行。是指在我们的意识或潜意识中植入。一粒种子 一种信念 一个词或一幅画面。以及如何对我们行为产生影响。 

Bargh在纽约大学时。进行过一项研究 他用"年老"的相关词。影射实验对象。如果你们还记得 这些词包括皱纹。年老 佛罗里达。这种影射导致的结果是。实验参与者走路更慢了。更佝偻了。研究者进行评估时。并不知道每个对象的实验条件。也就是说不清楚他们是被影射了对照词。随机词 还是年老相关词。所以通过影射 人们的行为。的确会发生改变。环境的力量 外界的力量。然后Bargh将实验更进了一步。决定用使用积极的影射 比如成就。毅力 成功。比如这些词。成就的近义词。实验对象被影射这些词后。智力及记忆力测试的成绩。比没有影射的对照组好。 

两位荷兰研究人员Dijksterhuis和Knippensberg。也进行了影射实验 他们找了一批人。请他们为下一次实验进行描述。不是本次实验。让他们为下一次实验。描述三种原型的典型日常行为及其性格:。足球流氓 秘书及教授。描述这三种原型。他们被告知 这是为下次实验做准备。需要这三种原型的描述。但他们不知道的是 通过对原型进行描述。自己已经被影射了。他们分别被影射为足球流氓 秘书。或教授。影射后 他们接受了智力测试。及记忆力测试。描述足球流氓日常行为的人。表现得最糟。无论是记忆力还是智力。然后是描述秘书的人。表现最好的 是描述教授的人。而他们之前并没有更努力学习。什么也没有改变 除了发生了影射。环境中有东西植入了他们的思想。现在的问题是。我们如何为自己创造一个积极的环境。让自己受到积极的影射 变得更幸福。更成功等等。方法有很多。我跟大家分享几个。其实这将是你们这周的作业之一。即创造一个积极的环境。 

举几个例子。找一些你爱的人或地方的照片。这些东西很重要 因为即使你不看它们。潜意识中还意识到它们的存在。虽然只是在潜意识层面。人们被影射时 看不见……那些词。比如"成就"这个词出现在屏幕上25毫秒。根本不可能看清 但大脑却记录了下来。所以即使你挂了照片但没去看它。或没意识到它 念到它。它仍然会对你造成影响。所以找些你爱的人的照片。让你开心的东西 无论是纪念品。鲜花 艺术品 你最喜欢的艺术品。我家中和办公室都挂了。我最爱的画 由我最爱的艺术家创作。让我跟你们分享一下。我今天带到课堂上来了 它们价值连城。我给它们投了保 所以不用担心 瞧。这位画家处在艺术生涯第一阶段时的作品。如你所见 充满了有力的线条。非常强烈 非常从容 真是绝了。另外一幅。仍是那位画家 进入艺术生涯第二阶段。印象派对他作品的影响已经清晰可辨了。不过你仍能从中看出同样的特点。同样的力量 同样强烈的情感。这些都挂在我家墙上。 

我还想向你们介绍一位画坛新秀。她还没进入第二阶段。因为她才一岁大。这幅画是在她妈妈帮助下完成的。但是我还是挂在墙上 每天提醒我。虽然我已经看过几百次了。但还是挂着 让我想起我的孩子们。我挂了些其他的……。名气不比他们的画家作品。你们可能没听过这画。这是罗丹的《沉思者》。也挂在我的墙上 是我最喜欢的雕塑。还有罗丹的另一幅作品。提醒我生活不仅要沉思 还有其他的。让生活更美好。这是罗丹的《吻》。所以找些东西挂在墙上。创造一种让你觉得温暖。觉得开心的环境。名人名言 我喜欢名人名言。你们很多人肯定也喜欢。已经有人把他们最喜欢的名言发我了。我墙上也有一列名言。就算我没看它们 也能感受到。比如 让我看看有没有带来。好 在这里呢。这句说亚当斯总统说的。"耐心和坚持总能奇迹般地。扫除困难和障碍"。将"坚持"用于影射。插一句 Bargh的实验也囊括了这个词。受到影射的人的确变得更坚韧不拔。乔治.艾略特 原名玛丽.安妮.艾凡斯。我最喜欢的作家 她写过"推动。人类进步的重任 不能等待完人来完成"。大约三周后 我会讲到完美主义。对我个人来说是非常重要的私人话题。可以说她是我的榜样。加缪说过"在深冬里 我终于发现在。心里有个不败的夏天"。 

这句话帮助我走过了艰难困苦。亨利.戴维.梭罗。"如果一个人能昂首挺胸地朝着梦想前进。努力实现他想象的生活。他会与成功不期而遇"。同样地 我都挂在了墙上 还有其他的。其中许多名言你们都在课上读到过了。它们不断影射我 提醒我。为我创造出积极的环境。我们会在下一节讨论艺术及其重要性。把喜欢的书放在手边 听听音乐。要集中注意听。不要只当背景音乐 要让音乐激发你。无论是摇滚 还是痞子阿姆 都可以。多看看激励你的电影。这些东西能帮你创造积极的环境。有助于成长 获得成功与产生幸福感。其中大部分都是在潜意识层面上进行的。这就是影射的力量 对于影射的研究。这也是积极研究的意义。如果我们总是关注"消极"研究 焦虑。压抑 精神分裂症 就会被这些研究影射。被这些工作影射 创造出现实。你们有些人可能知道。这与冰山理论有这千丝万缕的联系。即当你度量某种现象时。就是在改变这种现象。当你度量某种现象时。也是在改变自己。所以通过研究积极环境 我们也是在。对自己影射积极的东西。今天包括下周 我谈到的很多内容。其实都是源于自助运动。自助运动确立于20世纪。并从各个方面展开。其理论基础为我们通过思考创建了世界。我们的思维创建了时间。1930年代相关书籍开始出版。比如《思考致富》。 

从30年代至今销售超过6千万本。而且仍不断再版。至今还是畅销书 不论在美国。中国 印度。非洲还是欧洲。它对人们产生了巨大影响。作者Napoleon Hill这样说过。"只要你想得到 只要你相信 你就能够做到"。只要你想得到 只要你相信 你就能够做到。非常鼓舞人心。对各类人群 对大众产生了很多影响。6千万本销量 史上最畅销书籍。Henry Ford也发表过类似的看法。"不论你认为自己行还是不行 你都是对的"。许多人都这一理念吸引 为什么。因为它鼓舞人心。让他们觉得没有什么不可能的。另一本极具影响力的书。Norman Vincent Peale的《正面思考的力量》。 

"要有美好的希望 并全力以赴去追求。要有远大的梦想 并全力以赴去实现。要有宏伟的期望 并深信不疑"。多么让人为之一振啊。只要你思考。只要你相信 就会获得富足。无论是精神的富足。还是物质的富足。这一理念对大众一直有着巨大吸引力。时至今日仍是如此。当今最畅销的励志书。全球销量第一 没错 《秘密》。成功的秘密就是 所谓的"吸引力法则"。即 你的生活是被你吸引过来的。无论你想象什么 无论你相信什么。有一条强大的讯息。本书已经卖出了几百万本。就在我们说话当口 仍在热卖。而它传达的讯息其实很简单:相信就会成功。构想就会实现。在现实中就会实现。 

让人信服 鼓舞人心。但是 这条讯息正确吗。这我们讨论过 还有Roger Bannister的故事。他在4分钟 准确说是3'59跑完了1英里。6周后John Landy将这一成绩缩短到3'57.9。一年后37名运动员都能跑到4分内了。其中有一定奥妙。一但他们相信能办到 就有可能实现。但问题是。当今的心理自助运动的发展趋势。当今的自助领域强调某个真相。将这个真相过分鼓吹。没错。某种意义上我们的精神的确能创造现实。但是这不是全部。我们创造了现实 或者说是共同创造。还会受到外界或者内在的某些影响。比如《秘密》。如果你相信自己可能会成功。那成功的可能性会变大。但离不开刻苦勤奋与坚持不懈。而且失败避免不了。你要从失败中学习。 

所以这只是等式的一半 而这些自助书籍。传达的讯息往往言过其实 效果甚微。甚至可能会造成毁灭性后果。比如 如果一切皆建立于吸引力法则上。我创造了一切。一切都因我而变 一切都是我的错。举个极端点的例子。如果3岁小女孩被虐待 那是她的责任吗。30岁的男子被酒醉驾车的司机撞成瘫痪。是他自己的责任吗。有些事情是我们无法控制的。所以一切因我而变。一切都由吸引力法则决定。这种说法短期内可能能让你。坚定信念 获得幸福感 但是长期来看。只会让人感到挫败 内疚 不快乐。离成功越来越远。因为如果我真的相信一切都由思考决定。这种想法就会成为我的心态习惯。只要有信心。钱就会滚滚而来。爱情就会降临于我。这抹杀了勤奋工作……。坚韧不拔以及失败的作用。它们也是成功 幸福 完满人生的必要组成部分。我给大家放一段短片。是我在网上看到的一段广告。于是我向苏格兰皇家银行索要了原版。因为它抓住了当今人们对精神力量的。某种误解。很诙谐的一则广告。 

Douglas 也是参加会议的。 

- James Aderson - 你好。这儿风景真不错。那是什么鸟?- 怎么回事 – 

不知道。没事的 各位 我上过一门课程。- 什么? - 正面思考。正面的思考会产生正面的影响。明白不 正面的思考会产生正面的影响。看着我 没事的 来。(紧急启动)。语言不能代替行动。这就是思想的力量。创造现实 苏格兰皇家银行集团。 

- 我真厉害 - 真的耶。它抓住了部分真相 但不是全部。我们看看严肃研究。对精神力量的解释。一种神奇的现象。但还是需要通过实验来研究其原理。Albert Bandura 你们本周的阅读任务。做了许多关于自我效能的研究。本质上是自信的学术术语。以下是他进行几十年研究后的结论。去年他获得了授予积极心理学家的。研究者奖项。进行了很多这方面的研究 他是斯坦福大学的教授。"自我效能感影响人们生活选择。动机水平 机能的质量 对逆境的适应力。对压力和沮丧的抵抗力。相信自己的人 充满信心。在生活的不同领域都表现出色 他还指出。认为自己高效能者。与认为自己无用者。想法感觉完全不同。他们创造自己的未来 而不是简单地梦想。换句话说 他所说的是。对生活充满信心的人。他们创造自己的人生 与那些。由他人摆布的人不同 他们适应力更强。对困难与挫折的应对力更强。换句话说 能力更强。我们之前讨论过应对挫折的问题。这项研究的重要性在于 经过长年累月的。研究发现 这些能力是可以习得的。可以慢慢培养。否则研究就是无用的 至少对我们来说是。 

如果通过研究他得出结论。说这种能力只是天生的。有些人天生就有适应力基因。其他人没有。幸好情况不是这样。他的研究表明这些都可以习得的。但要花时间和经历。我们以后会讨论如何培养。重要的是 这些能力是可以锻炼的。我们可以改变它。Curry做过一项关于运动员的研究。下面一定坐着许多运动员。她证明了一旦成为运动员。56%的成功是由期望度决定的。成功取决于你坚信自己能取得成功的程度。记得第一或第二节课上。我提到了John Carter的研究。有关商……哈佛商学院毕业生的。他预测……。有两个因素会影响他们的长期成功。某些毕业生会格外成功。一是他们不断地提问。二是他们相信自己。 

他们相信自己能行。但需要意识到的是。不是所有这些差异。都能用希望 信念或乐观来解释的。很多自助书籍想引导我们这么认为。但当然这些是成功的一部分。Nathaniel Branden从事心理治疗50年了。他做过许多严谨全面的哲学及心理学研究。致力于对自尊概念的探究。等我们进行到自尊课时。会详细讲他的研究。现在我只引用他的一句话。"我们每个人自尊的水平。也就是自信的程度。对我们的方方面面有着深远的影响。比如我们的待人接物。能取得的地位 能取得多少成就。还比如在个人领域。我们会与谁坠入爱河。我们与配偶 孩子及朋友的关系。个人幸福的水平"。物质富足与幸福感。或生活地点与幸福感。属于低度线性相关。而自尊与幸福感属于高度线性相关。大多数研究发现两者的相关系数。可以达到0.7。我的论文也获得了相同结论。所以自尊非常重要。内在因素很重要 我们提高自信水平时。发生在我们身上的是。我们在转变精神 改变"容器"。这样我们看世界的角度不同了。有了不同的理解与感悟。处事待人的方法不同了。庆祝成功。从失败中学习 转化为机遇。但最重要的是我们的内在。 

Branden说过"自我概念即命运"。与"只要你想得到 只要你相信 你就能够做到"。非常相似 Napoleon Hill说的。唯一的不同之处是。他对此进行了严谨细致的思考与研究。以证明信仰的局限性。这一领域的另一个研究。大多数都听说过。安慰剂效应。你头痛 去看医生。医生说 我给你开种药。非常有效 新出的。能治疗你的头痛。你吃了药 头不痛了。你不知道的是 她给你的其实是糖球。但是因为你以为。因为你相信它能治好头痛。还真治好了 真是神药啊。于是Herbert Benson 在河对岸。医学院的老师。对此做了一系列研究。我着重跟大家讲两个。他找了一些处在妊娠期。胃部不适的妇女。她们都有呕吐恶心的症状。这其实挺常见的。尤其是在怀孕初期的三个月。他告诉她们。"这里有一种新药。能缓解你们胃部不适。治疗呕吐症状。非常有效。于是他把药给了她们。她们的胃部就好了。当然 她们不知道那只是安慰剂。普通的糖丸。 

然后他将实验更进一步。找来了另一组有同样胃部不适。呕吐症状的妇女。把药给她们。告诉她们可以缓解胃病。但是这次 他给的不再是糖丸。而是小剂量。没有危险性的小剂量的吐根。你们可能知道吐根……。是催吐时服用的药物。如果发生食物中毒。就需要服用吐根催吐。他给这些病人服用了吐根。这一实验通过了道德审查 没有风险。他给这些病人服用了吐根。并告诉她们能治疗呕吐。而她们也的确恢复了健康。本来应该出现的呕吐症状不仅没有出现。由于服用了吐根更该呕吐的 实际上呕吐却停止了。我给大家读一段摘录自他的《永恒的治疗》。我非常推荐这本书。"引人注目地是 病人的恶心感与呕吐反应。完全终止了 且根胃内水球测量。胃胀现象也消失了 恢复正常。因为她们相信她们服用的是止吐药。竟然逆转了催吐药的作用。 

虽然很多人在医药箱储存了吐根。以防食物中毒发生。这些胃部不适的孕妇。却抵御了本该让她们更难受的催吐药物。单凭信仰就治好了病痛。但这不是说药物没有效果。我不推荐大家回家以后。给你们的室友。服用吐根 甚至氰化物。然后告诉他们这是维生素C。他们一般不会感觉好。但这一实验。及其他类似实验表明。我们不应忽视精神的强大力量。而需要研究并理解它。Benson在他的书中还写到了另一个实验。这项实验是在日本进行的。实验对象对某种植物过敏。他们被蒙着眼睛带进实验室。研究人员用过敏植物碰触他们一只手臂。然后用另一种植物碰触另一只手臂。但没有告诉他们哪边是过敏植物。理论上被过敏植物碰触的手臂。会发出了疹子 出现过敏反应。但是 实验还有另一个条件。他们被告知会过敏的植物。碰触了他们的左手。不会导致过敏的植物。碰触了他们的右手。 

但是Benson将两者对调了。即真正会导致过敏的植物。碰触的是他们的右手 而不会导致过敏的。碰触的是他们的左手。所以按理说 应该右手起疹子 而非左手。但是因为他们误以为左边的才是。过敏植物 所以左手起了疹子。而右手没有。因为精神的力量。但同样的 这并不是说外在因素不重要。也不说明生活中的一切都是被吸引过来的。但是这的确证明了精神。对构建我们的生理 情感 认知。以及外在环境起了重要作用。信念常常会成为自我实现的预言。但它是如何作用的呢?我想跟大家分享一个模型。是我根据这一领域大量研究建立起来的。它基本上解释了我们的信念及希望。是如何影响我们的表现的。无论是安慰剂效应。还是我们学习 体育运动 恋爱关系等。生活各个方面的表现。换句话说 这盒子里是什么。这上面应该有哪些内容?是什么斡旋于信念期望。及实际表现之间?我想谈谈两个机制。第一个相对简单易懂。即动力。如果我相信我能行。各位可以回忆一下自己的经历。当你相信你能做好某事时。会非常有动力。如果你认为自己希望渺茫。那就很有可能就此放弃 无所作为。你不仅事前会受到激励。无论是训练 还是练习。而且正式进行时也会动力十足。 

想想Marva Collins的研究。她激励学生。让他们相信他们能获得成功。他们一生中能有所作为。这是他们的责任。这改变了他们的命运。因为他们有了动力 所以非常刻苦。所以动力非常重要。除了动力外 另一样东西也会影响我们。即一致性或相合性的概念。那是什么意思。我们每个人对世界都存在心理图式。世界应该怎么样 会怎么样 是怎么样。比如 我的一个心理图式。表明把东西放在半空会掉下来。所以我认为如果我放开这东西。它就会掉下来。图式认为我的数学能力。我可能是个优秀的数学家。我在这一领域是否有潜力。对其他人的态度也有图式 我喜欢这人。那人真好。还是那人不好 残忍。所以我们对他人 对自己。对自然现象 都有图式。图式就是内在。还有外部世界。现实发生在外部世界。比如 半空中我松开手 它就会掉下来。这一事实与我的图式无关。它发生在外部世界 不管我的图式如何。我在数学上的实际潜力。也与我的图式无关。 

某个人对他人。是慷慨仁慈还是可憎无礼。也与我的图式无关。所以内部和外部两个因素。共同作用就是关键。我们的精神不喜欢内部与外部。存在差异。我们的精神喜欢两者一致相合。如果不相合。如果不一致。我们就会产生异议感。觉得不适。感觉不舒服 不对劲。所以我们常常不惜一切 让两者统一。要么改变外部现实。要么改变自我思维以符合外部现实。我们不喜欢不一致。有好几种方法。可以重建一致性。我给大家介绍几个。我会讲4个 都是互相关联的。互相联系 还有重合之处。 

但这样能帮助你们就此进行思考。帮助你们理解内部与外部间。发生的对话。两者差异发生时。我们首先能做的就是更新图式。这是第一种做法。比如 原来我不知道松开手东西会掉下来。现在我知道了 于是我更新图式。又比如 我原来以为那人很可恶。但我看到他们对别人慷慨仁慈。于是我更新图式 他是个好人。第二种做法。通常比第一种更普遍。即忽视或抛弃外在信息。忽视或��弃。��符合��式的��部信��。我们不喜欢不一致。于是直接扔掉这种不一致。也就是忽视 抛弃它。第三种方法。主动验证。我们会主动寻求验证信息。 

想想2和3 想想现实世界。今年正是选举年。想想你不会投谁的票。或者你以前没投过谁。我们是怎么做的?通常 比如。我看到布什总统做了件好事。为我信仰的某个事业奋斗。但我之前没给布什投票 举例而已。于是多数情况下 我会选择。忽视抛弃这条信息。然后寻找有利证据以支持……。坚定我讨厌布什的立场。这同样适用于今后的选举 比如希拉里或奥巴马。如果我不喜欢他们 我就会寻找相应证据。证明我为什么不喜欢他们。即使他们做了什么我喜欢 重视或信仰的事。通常我会选择忽视。我们所做的 就是创造另一个现实。因为我们在质疑。布什做过什么恶。希拉里或奥巴马做过什么恶。当我们问这些问题时。我们已经完全忽略抛弃正面的东西。记得我问你们几何性质吗。你们没看见车上的孩子。虽然他们就在你们眼前。所以质疑会创造现实。而为了保持一致性。我们常常会采取二者中的一种方法。要么忽视抛弃 要么寻找有利证据。无论积极或消极 来支持我们的图式。是希拉里 奥巴马好 还是布什好。或是其他任何看法。第4件事直接得多 即创造新的现实。 

1954年5月6日前的现实是什么。当时的现实是 4分钟。是人类的极限 不可能突破。爱迪生做……抱歉 Bannister做了什么。他说可能 于是改变了外在现实。不仅是他自己 还有世界上其他赛跑者。突然间创造了新的现实。起初的图式是4分钟不可能跑完。或者说4分钟不可能跑完。而所有人都拥护这个图式。Roger Bannister认为是可能的。他改变了现实 不仅为自己。也为其他赛跑者。事实是 当干劲十足的人达成后。干劲没那么足的人可能也会寻求统一。我们需要解读自己表现。成功还是不成功 好还是坏。有两种解读方法。一是客观解读。我考试得了A还是C。跑步我是第三名还是最后一名。周五与普林斯顿大学的冰球赛。是赢了还是输了?哈佛加油 到底发生了什么? 

这就是现实 客观。但同时 还有由信仰决定的主观解读。举个例子。有个关于托马斯.爱迪生的故事。爱迪生于1870年代。与科学界都在研究灯泡。如何用电发光。整个科学界都在研究这一课题。但一无所获。爱迪生也不例外。当地报纸的一名记者。前去采访爱迪生。他当时已经非常有名了 发明了很多东西。他们谈了各种话题。然后开始讲灯泡问题。那个记者对爱迪生说 "爱迪生先生"。您致力于灯泡研究许多年了。整个科学界都在进行相同研究。但毫无所获"。当时爱迪生已经进行了5000次实验。这位记者也知道 于是他对爱迪生说。"爱迪生先生 您已经进行了5000次实验。失败了5000次 放弃吧"。 

你们可能知道 爱迪生听力不好。事实上 他有一项专利就是助听器。所以他对记者说"你说什么。那个记者重复"我说你失败了5000次。放弃吧" 爱迪生回答"我没有失败5000次。是成功了5000次。我成功证明了哪些方法行不通"。同样的客观现实 表现 5000次失败。但却有完全不同的解读。那个记者也其他科学界认定。这是不可能达成的。但爱迪生的主观解读。却是他走向成功的敲门砖 因为是可能的。事实是 爱迪生在发明灯泡前就宣布。他将在1879年12月31日。展示灯泡。这是在他发明灯泡前。1879年12月31日 爱迪生向世界展示了。用电发光。同样地 1962年 肯尼迪总统说。60年代末人类将踏上月球。那时我们连登月的物质条件……。科学技术都还不具备。他向全世界放话说。人类能踏上月球 然后创造了现实。这与大量的辛勤工作是分不开的。事实上 爱迪生失败了不止5000次。才最终发明了灯泡。他不是干坐在实验室说。"相信就能做到"。而是我相信 而且我会加倍努力。满怀斗志地工作。他的一条名言是"我从失败走向成功"。爱迪生是史上最富创造力最多产的科学家。这并不是巧合。 

他一生申请了1097项专利。当今世界的发展 大半要归功于他。史上最成功最富创造力的科学家。也是失败次数最多的科学家。这并不是巧合。还有Dean Simonton的研究。等我们谈到完美主义时会精讲。Dean Simonton的研究证明。史上最成功的科学家与艺术家。也是失败次数最多的。"Babe"Ruth是职业生涯中全垒打次数最多的球手。他也是三振出局最多的。其中有5个赛季 三振出局次数最多。史上最成功的人往往是。失败次数最多的。这一句话我会反复在课堂上重复。学会失败 从失败中学习。学会失败 从失败中学习。成功别无他法。学习如何工作别无他法。要学习如何在科学 政治或艺术界。取得成功 别无他法。正如Dean Simonton指出的那样 纵观历史。那些有过最多次挫折 最多次失败尝试的艺术家。也是最终看来最成功的艺术家。成功没有捷径。只是坐下来思考。构想和相信是不够的。我们要做的比这多得多。 

Martin Seligman 他被认为是积极心理学之父。做了一些有关乐观主义和悲观主义的研究。他发现。他发现 就目标设定而言。悲观主义者不论在他们的短期目标。还是长期目标上都很现实。现实是好事。短期目标和长期目标都很现实。而乐观主义者很现实 抱歉 乐观主义者与之相反。在他们的短期目标设定上并不现实。但对于他们的长期目标就很现实。这是怎么回事? 让我们再来看这个模型。 

我们先来看悲观主义者。他们有某个目标 他们的期望低。信念不高 他们不认为自己能做好。积极性低 他们的大脑寻求一致。他们的表现通常取决于。他们的信念和期望。他们的理解是 我早跟你说了。我早就跟你说了我做不好。于是其他人都异口同声说是的 你早就跟我们说了。你这样现实真是好啊。但有时悲观主义者超出了自身的期望。取得了成功。那么又会怎么样呢?这时的解释是 低水平的信念。那么解释就是 只是走运而已。 

是因为火星转到了金星的前面。50年难得一遇 不会再发生了。或者 今天是我的幸运日 或者是因为她今天表现得不好。于是 大脑在寻求一致。一次又一次地重复这个循环。然后再一次变得现实。不成功的现实 但是现实。于是他们在短期目标上。同样也在长期目标上很现实。 

我们现在来看乐观主义者。乐观主义者一开始有着很高的信念。很高的期望 积极性 非常高。他们的大脑寻求一致性 表现 没那么好。没期望中的那么好。换言之就是 不现实。但是 由于信念水平很高 他们的解释。其主观的解释是。好吧 如果我从中吸取教训了会怎么样? 这是个机会。我这次其实做得有进步了 就当吸取教训了。他们依然保持着很高的信念。很高的期望。积极性很高 大脑寻求积极性。他们的表现依然不好。没有他们所期望的那么好 不现实。但解释依然是 如果我从中吸取教训了会怎么样?我这次做得好多了。我指出了哪些方法是不可行的 然后他们继续。但然后外界的声音就会争吵说。"不是吧 真的么 你为什么就不能现实点。像你的好兄弟或者好姐妹 悲观主义者那样"。 

但是他们相信他们能做到。于是一次又一次地继续坚持 继续努力工作。5次 10次 有时5000次 甚至10000次。久而久之 直到他们带来了"不现实的现实"并让它成真。和他们的信念相一致了。所以即使在短期看来 现实可能不一致。我可能接下来10年跑一英里。都要4分12秒或者4分2秒。但我终将到达目标 让我不现实的。"期望"和目标变成现实。经历很多失败 经历很多努力付出。经历很多坚持不懈。Seligman就乐观主义和悲观主义的内容所展现的。并非只是某种遇事过分乐观的自我感觉良好。你知道 构想然后相信之类的方式。而是关于我们如何解释事件。 

比如 我如何解释我的失败?最后的结论是 大惨败 放弃。还是一个通向成功的机会?所以他 我们下次课会谈到。他把解释一件事分为是。永久的还是暂时的。我没发明电灯 那么这可能意味着。这是不可能的 我永远也做不到 或者也可能意味着。这是向成功的更进一步。告诉我了什么是不可行的。我想要的一份工作被拒了。这意味着我永远都找不到工作。或者这是个暂时的挫折 来看看我学到了什么。来看看我怎么样能重新寻找并。最终找到我理想中的职位。也要看人的态度是一概而论还是有针对性。悲观主义者认为事物是……。他们一概以消极眼光看事情。我不是个好的科学家还是……。 

我只是这一次的这个实验没做好。或者如果考试得了低分 我就是白痴。看什么都是一概而论 不是全部就是一无所有。而相反可以认为我只是具体这一次考试没考好。我从中能吸取什么教训并且下次如何能考好。这就是乐观主义和悲观主义 而这种态度可以习得。很多有关这方面的研究……表明学习主观地。像乐观主义者 解释事件可以获得更高层次的成功。你们中的而有些人可能知道Matt Biondi 1988年汉城奥运会。许多人都认为他能追平Mark Spitz。7块金牌的记录。第一场比赛 银牌 巨大的失望。第二场比赛 铜牌 巨大的失望。大家 我记得很清楚。人们谈论着他如何受到压力的影响。他不是我们所期望的那个人。Martin Seligman在那时说"不 他将会成功"。为什么? 因为他衡量过他和许多其他运动员。在乐观与悲观方面的水平。Matt Biondi一直是个乐观主义者。 

他知道他会将这些"失败"理解为 我指的是。奥运会上的银牌和铜牌 算是失败 对吧?这些相对于他期望赢得7块金牌来说 是失败。但他不会把它们视为永久性的失败。把"我发挥失常 我不能在奥运会上实现目标"。视为 "我再也不能游好泳和赢比赛了"。而只是视为暂时的和特定的失败 他也正是那么做的。并赢下了接下来的5块金牌。这让很多人吃惊 但Seligman和他的研究小组没有。他们非常清楚他的理解方式。屡"败"屡战。这同样也和社会和身心健康有关。比如我们的免疫系统 心理上和生理上的免疫系统。因为乐观而实际上增强了。适应力水平 Karen Reivich的很多研究表明。如果你教人们积极地 乐观地理解事物。他们就更有可能成功。如果有谁有兴趣了解更多这方面的内容 看她的书。《适应力因素》 Karen Reivich写的 很不错的书。她在书里讲到了一个为期。几个星期的项目 孩子们此后两年都。对忧郁的经历免疫。 

那些学习如何乐观的人。她在短短两周时间里对他们进行指导。并和剑桥Summerville五年的研究做对比。在短短两周时间里 那些被教导。乐观地理解事物的人变得更快乐。学过的人患上忧郁的可能性。减小了八倍。有忧郁经历的可能性减小了八倍。只是简单地学习一种不同的理解方式。我没有失败5000次。我成功地发现了什么是不可行的。所以如果我这次没做好。我下次或者下下次会做得更好。学会更积极地理解事物的人。事实上活得更长。我们将会谈到一些这方面的研究。你会读到一些这方面的内容 修女研究等等。积极地理解事物或者更乐观的人。事实上活得更长。那并不是说所有的悲观主义者都短寿。而所有的乐观主义者都长寿。那只是这个等式里的因素之一。还有其他的因素。你知道 也有不现实的乐观主义。我们稍后会谈到。你知道 一个乐观主义者说。"无论如何 我会活到120岁"他有吸烟的习惯。但一天抽120支烟很可能活不到120岁。所以这不只是只是信念 还有很多因素。 

最后 正如我屡次提到的 它是可以习得的。我们将讨论要如何学习它 我们能怎么学。而在你的课后论文里 你们将研究这个问题。不是这周的 是下周的论文。或者是通过分组作业 或者独自完成。一种能被习得。并有所有这些积极效应的方法。好吧。但是有一个问题。我要如何区分这两者。现实的乐观主义和不现实的乐观主义。用盲目乐观主义做例子。对于自助运动而言。是不是所有我认为我能实现的。我构想的都能变成现实?我的自我概念是否我的且是我唯一的命运?或是否存在所谓的不现实信念?确实存在这样一个东西。下面是个例子。她的叫Rena 是个歌舞女郎 至少对我来说她是。Rena和我有很多共同点。我们都十几岁 我12岁 她14岁。我们都5英尺高。她5英尺6英寸 我5英尺1英寸 

我爱上了Rena。Rena也爱她自己。现在 只是做个边注 我有一个很棒的妈妈。很好的妈妈 她相信我。而事实上当我们三个。我们三兄妹放学回家时。我们经常会跟妈妈讲学校里。发生的所有事 好事还有坏事。我有一天回家 意识到我是真心地 疯狂地。爱上Rena 她就是我的真命天女。我回到家 告诉了我妈妈。我告诉她 "妈妈 我那么地爱她"。于是她说 "那问题是什么?"。我就跟她解释了她比我大。学校里的那些大男生也对她有兴趣。她比我高多了。而我妈妈对我有信心 她采用了。自我应验预言的那一套。尽管她从没读过Napoleon Hill。她对我说 "Tal 只要你想得到。只要你相信 你就能做到"。尽管她从没读过Branden 她对我说。"Tal 你的自我概念就是命运 而Rena就是你的命运"。她告诉我如果我真的真的认为我可以。相信Rena会喜欢我 她就真的会喜欢我。我妈妈激励了我 让我相信自己。 

于是我去接近Rena 我做好了准备。我认真做好了准备。虽然我那时还没有刮胡子 但我喷了些须后水。然后我继续行动。我可以做到 我知道我能。我走进Rena 但这时音乐停止了。尽管我期望很高。尽管我预想我能成功。我闭上眼睛 我看到自己成功了。她没有兴趣 她不喜欢我。但是 但是 我们成为了很好的朋友。事实上我们关系那么好 她会跟我谈论。她那些肤浅的高大的男朋友们。当我妈妈告诉我她真的对我有信心。她认为我是-- 她用意第绪语对我说。她说(意第绪语)。在意第绪语里的意思是 "你是个大发现"。我真的 真的就是。我那犹太母亲心中符合犹太教规的皮格马利翁。好烂的比喻 我知道。那句话花了我不少时间想出来的。你们最好表示一下喜欢它。但是……谢谢 谢谢 但这没有用。再一次 别误解我 我非常感激我妈妈。简直就像是全天候的驻家Marva Collins。我是那么地感激我妈妈 但是 这一次。这可能不是现实的信念。那么什么是现实的?我们如何区分什么是现实的 什么不是?下面是一些对我们有帮助的想法。斯托克代尔悖论。它认为要找出矛盾。让矛盾化解 这并不容易。我不会给你们一个公式 告诉你们:。好吧 就这样……算这个 如果是大于7的。就没有问题 低于7的 可能就不是这样。不存在什么公式 但是 仅作为一个启发式。当你思考信念时 存在一些需要思考的东西。 

James Stockdale是名海军上将 他是。越南战争被俘人员里级别最高的战俘。他开始注意到一个现象 他在那关了很多年。最终被释放并得以活着向人们讲述往事 而其他许多战俘没能做到。他看到许多人幸存 许多人死亡。战俘的生存条件非常恶劣。他发现那些活下来的人有两个特点。第一个 他们相信他们能出去。他们坚信终有一天他们会重获自由。他们会再见到自己的家人 朋友。这种信念支持着他们走下去。同时 也就是第二个特点。他们有着现实的信念 能对情况做出估计。如果这两点有一点没做到。他们就不大可能活下来。如果他们没有信念 他们不相信他们会活下来。他们很可能在恶劣的环境下死去。另一方面 如果他们认为。他们能在这年的圣诞节之前就出去。并真的相信他们能在这年的圣诞节之前离开。他们同样也不大可能幸存。因为他们这年的圣诞节没有被放出去。他们失望 气馁 无助。通常情况下 他们会死去。所以关键是这两点 相信你会出去。但同时明白自己需要。应对现实 残酷的现实。 

James Stockdale如是说。"你不能把坚信自己最终会在。输不起的情形下取得胜利的信念。同直面现有残酷现实的行为。相混淆 无论是何种残酷现实。现实的乐观主义。现在的问题是你可能有时会错。有时你可能真的相信什么。相信你有现实的期望。但没有成功。但就像我们谈到的那样……我们接下来会谈到。这也没太大关系。因为即使没有成功 我们能很快恢复动力。继续努力工作。而关键就是尽可能地保持平衡。这就是Maslow所说的高水平信念。和与现实联系间的完美整合。这个不是答案。只是一个启发 让你们记在脑子里。积极的思考不是秘密本身。是的 它是这个秘密的一部分。是的 它是成功公式的一部分。是的 它是幸福公式的一部分 但并非全部。错误的乐观主义迟早意味着幻灭。愤怒和无望。成功的秘密是。这是一个经研究证实的秘密。多年对不同文化 不同地方。不同年龄的成功人士的研究。这个秘密是乐观 热情和勤奋。纵观历史 你很难找到哪个成功人士。是不具备这三个品质的。乐观 信念 坚信他们能做好。坚信他们能成功。热情 热爱他们从事的事业 并且勤奋工作。 

正如爱迪生所说 没有什么能代替勤奋。这三个要素 就是这个秘密……很简单 很明确。你们都知道 还记得我第2次课上说的么。我不会教你们新的知识。我希望使你们想起你们已经知道的知识。乐观 热情和勤奋。那运气是否也有作用呢? 当然了。Jefferson说"我非常相信运气。我发现我越努力工作 我的运气就越好"。运气背后实则也有科学奥妙。你们会在《幸运要素》里读到。在书中 拥有这三要素的人。相信自己能成功的人。热爱自己的工作并勤奋努力的人。事实上获得了更多的运气。用梭罗的话说就是不经意间有了更多的运气。好的 那么为什么不是所有的家长都。教我们要有高的期望呢?如果这对我们来说有好处 如果这和成功有关。和幸福 和身体健康有关的话。为什么不是所有家长都对我们说不管你们做什么。你都能成功 你都做到。如果你梦想 你就能做到。为什么父母和其他关心我们的人。常常要降低我们的期望呢?因为他们关心我们的快乐水平。他们关心我们的自尊。有很多种信念……人们相信。高期望自然地会导致失望。所以……高期望导致失望。这听起来是对的 

这就是。William James在19世纪写下的公式。他说自尊等于我们的成功。除以抱负 换言之就是 我们的成就……。我们有多成功 我们取得了多少除以。我们假装或期望所能取得的 我们的目标。换言之 如果我们有很高的期望 很高的抱负。我们就有更大的可能伤害自己的自尊。例如 比如说我的渴求 我的抱负。我的期望是一个月挣4千美元。重申一下……我只是举了个数字化的。例子 这样便于理解。让我们假设我的希望是一个月挣4000美元。那是我的抱负。但实际上结果我一个月挣到了2000美元。根据William James的公式 我的自尊。而我们都知道自尊和快乐。是紧密相关的。我的自尊 我的快乐水平。根据这个公式便只有一半。另一方面 假设我的抱负。我的期望是一个月挣1500美元。而事实上我一个月挣了1500美元。根据这个公式 我的自尊。我的快乐水平是1。尽管我挣的更少了。 

我这里举的是数字化的例子。你们可以想想其他例子。尽管我实际上挣得少了。我的自尊和我的快乐程度却变高了。所以要降低你的期望。其实不尽然。过去35年的研究发现。这个公式实际上是错的。情况不是这样的。这才是实际的情况。让我们画一张图 横坐标是时间。纵坐标是我的自尊水平和快乐水平。让我们从我的基础水平开始 就在这里。然后我参加了积极心理学的期中考试。我考得没有我期望的好。我的自尊会发生什么变化? 我的幸福会发生什么变化?当然它会下降 这很自然。我们都想成功或者说大多数人想成功。所以如果低于我的期望它就下降了。但很快 如Daniel Gilbert。有关有效预测的研究表明……。 

很快我们又回到我们的基础快乐水平。接着一个星期后 我又考了 比如说55号数学课的期中考试。我考得很好 比我期望的成绩还要高。我的自尊和快乐曲线又会怎么变化? 它上升了。但然后很快它又降下来。然后我们假设我是12岁 我在街上走。我遇到了Rena 我约她出来玩。她拒绝了。我的自尊和快乐水平会发生什么变化?它下降了 轻微地。然后很快它又重新上升。然后我很幸运 我走在街上。我以为我看到了Rena 但她有些不同。原来是她的孪生妹妹。我约她出去 她拒绝了。再一次 发生了什么变化? 轻微地下降。但然后我恢复了 又再次上升。你们是不会相信的 她们是三胞胎。我碰到了老三 她答应了。她说我也爱你。我的自尊和快乐水平发生了什么变化?它轻微地上升 对吧? 甚至高出了黑板。但然后很快又下降回基础水平。如此反复 生活中的悲欢离合。 

不管我是得到了终身职位还是没得到。不管我买彩票是中奖了还是输钱了。在幸福的一个基础水平上下起起伏伏。现在有个好消息也有个坏消息。好消息是我能承担更大的风险。她拒绝我不是世界末日。我失败了或者没能得到理想的工作。也不是世界末日。没有关系 我会恢复 我们都会 所以这是好消息。坏消息是……我就卡在这了?我根据我的基因生来就是这个基础水平。并且不能提高吗?我如何提高这个快乐的基础水平?这是可能的 我们会讲到……。整个课程就是关于如何提高快乐的基础水平。但这有一个发现。是从专门针对自尊的研究中得来的。你下课之后马上能做的一个实验。来提高自尊 就是去面对而非逃避。去面对意味着把自己置身于风险线上。不是要你在恐慌时冒险。而是在你放松时放手一搏。去承担失败的风险 去解决 去处理。去面对那些对你来说重要的事情。参加戏剧试演 讨论时说出未曾表述出的观点。去你十分想去。却从没去过的地方 去尝试。因为那些经常面对问题的人有着同样。或者有时频繁的起起伏伏。但他们的起伏线条是这样的。依旧是起起伏伏 和普通人一样。但基础水平提高了。为什么基础水平提高了? 我将以此结束本节课。为什么当我们不是逃避而是面对时。基础水平提高了? 

三个原因。首先是自我知觉理论 Daryl Bem……。我不记得他是达特茅斯学院还是康奈尔大学的了。Daryl Bem指出我们得出关于自己的结论。和我们得出关于他人的结论的。方式是一样的。比如说看到一个人走到大家面前。发起谈话或者在会上发言。或者去参加一出戏的试演并一次又一次地尝试。我关于这个人的结论就是这人很有勇气。这人肯定有很高的自尊。我可能也是用同样的方式。推出关于自己的结论。通过观察自己某些方式的行为。我会得到关于自己自尊或者。仁慈或者慷慨或者其他方面的结论。自我知觉理论 如果我面对了 我尝试了。那我就认为这是一个有勇气的人。这是一个敢于尝试的人。得到关于自己的结论就是我肯定有很高的自尊。 

我的自尊还有幸福水平。久而久之就会上升。自我知觉理论。我们得出关于自己的结论。和我们得出关于他人的结论的方式是一样的。就是通过观察行为。我们在面对失败后会意识到。真正来自于失败的痛苦远小于我们想象的。我们认为与失败有关的痛苦。我克服失败 是的 我是没做好。或者她是拒绝了我 但我克服了。但我一直在想的话 我就会认为它将一直持续。正如那些终身职位或者非终身职位的教授。以为他们的薪水将永远不变 其实并非如此。所以当我意识到实际上的痛苦。比和失败有关的痛苦远远要小时。我变得更自信了 我能应付了。我实际上比自己想象中要更有适应力。我的自尊水平上升了。我的快乐水平也随之升高。最后还有更多的成功。因为成功别无二法。学习走路没有别的方法。在艺术上或科学上或商业上或政治上。取得成功没有别的方法。学习失败 从失败中学习 我们星期四再见。 

第7课-逆境还是机遇? 

我们今天的课。要把自我实现预言的信念讲完。并开始讲下一课 关于专注。和专注如何创造现实。我们上一课讲到哪里?我们讲到Rena和其他的一些事。我们讲到模型 Dan Gilbert的模型。Philip Brickman模型。成功或失败过后 会有大起大落。但我们会恢复过来。我们一生基本沿基准的幸福发展。其中有起伏和变迁。所以问题是。或说有好消息也有坏消息。好消息是 我们不必太过担心。我们能冒更多风险 也更容易回到基准线上。失败往往 虽然不是所有的失败。往往只是使幸福水平暂时的下降。还有自尊水平 然后会迅速回升。所以这是好消息。坏消息是 如果我们总是回到原来的水平。那我们为什么要努力寻求幸福呢?而答案。这整个课程本质上就是对它进行回答。 

我们如何提升基准水平?一个方法是去面对 把自己置于风险之上。我们这么做的原因。是因为这样能产生积极的影响。结果 幸福基准会上升。意思是仍然有起有落。但起伏是像这样斜线上升的。而不是在一条直线上或者。平行线上徘徊 我们去面对时能想象我们的行动。想象自己在努力 通过自我知觉理论。我们对自己做某种总结。我一定很勇敢。我一定是自尊心强的人。我一定是非常渴望成功的人。等等。然后我们给自己做出的结论。就像我们给他人的结论一样。这就是自我知觉理论。第二个原因是一旦我们经常经历失败。我们会意识到那其实并不那么糟糕。不像我们想象的那么糟。我们脑中想象的失败的痛苦。比实际的痛苦要多得多。也许不是即时的 而是稍后才感受到。当我们恢复后 就知道我能应付它。我一定是一个有适应力的人。最终 努力去面对后会有更多成功。学习失败 或在失败中学习。成功没有其他途径。 

我现在要说的是。我们如何变得更乐观。我们谈过优点 谈过长寿。乐观的人实际上活得更久。我们谈过更高水平的成功。更高程度的快乐更能克服忧郁。乐观的人。患上忧郁的机会是八分之一。他们会更快乐。再说一遍 乐观是指诠释的方式。而不是盲目地自我感觉良好的方式。我们如何变得更加乐观?我想谈谈三个方法。首先是采取行动 尽管去做。把自己置于风险之上。第二点 我要讲想象力。形象化的力量 你们许多人熟悉这点。尤其对运动员而言 作为一项技巧。我们要谈论这个技巧。最后一点是认知疗法。至少研究表明 是当今最成功的。最有效的治疗干预手段。我们要讲一些基本要素并进行归纳。我们先说行动。这是基于Albert Bendura的研究。你们会了解到自我效能。他的基本概念上是要有坚定的信念。他将它描述得更学术化。更科学化 并研究了数年。我们谈过自我效能的许多益处。 

Bendura所做的大部分工作是反对。人们称为"自尊运动"的活动。或者说他反对的是。被称为"感觉良好运动"的活动。所谓的 告诉小孩他们有多好。他们有多棒。每天早上站在镜子前对自己说。"你好精神"十遍 你将更成功。你的自尊会提升 你会变得很好。这不管用 相信我 这不管用。我都试过 实际上经常。事实上长期以往会造成伤害 会伤到自尊。它实际伤害了那些学生的积极性。他们经常被夸奖说"你们很棒"。"你们都很了不起"。我们会讲如何对赞扬更有辨识力。如何正确地赞扬。在我们过几周谈论Carol Dweck。关于心态的文章的时候。同时Bendura表明。Carol Dweck表明。还有其他许多人都表明 这还不够。通常情况 只给你积极的言论 积极的肯定。无论来自外部或内部。这甚至是有害的。这也许其中的一部分 但不是全部。Bendura说的是 我们需要的是行动而不是空谈。我们对任何事都要付出努力。我们想变得更自信 就需要去应对。我们要把自己置于风险之上 很简单 为什么?因为努力和面对必然能获得成功。至少比以前。以前不行动 不去面对时要成功。即使只是一点点成功的贡献。也有助于提高自我效能。有助于增加自信 

而自信反过来。反过来激励我们更努力。我们开始更加相信自己。这是个很简单的模型 我是说任何人。一个五岁小孩都能做出那个模型。然而考虑到对我们生活的影响。考虑到对教育的影响。我们应该仅是对学生不停地说你们很棒。你们真厉害 还是应该像Marva Collins那样。让他们努力奋斗获得更多成功。让他们想象自己回归自我概念理论。想象自己努力奋斗。想象自己投身其中。面对失败 重新站起。它和我们身体免疫系统运作方式相同。当我们身体不适 当我们生病时。我们的身体感应到抗体。我们实际上会免疫得过的病。我们的身体通过失败获得免疫力。在心理层面也相同 失败会被低估。我希望你们拥有的东西之一。我是很真挚的。希望你们能多经历失败。我不是随便说说 我真这么想的。 

另外 我希望当你们失败时。你们也能学习用不同的方式去诠释。而不像大多数人那样诠释失败。因为没有别的方式取得成功。成长的途径只有这一条。像这样的生活 只是在逃避。我们会继续谈这个。当我们谈论完美主义时。健康的生活 真实的生活。快乐的生活基本都看起来像这样。一个带起伏的螺旋 不是一条直线。当我们看到自己去面对 努力奋斗时。我们对自己做出结论。我们的自信增长 我们的动力增加。我们的信仰上升 等等。以一个上升螺旋的趋势发展。Soren Kierkegaard说过"勇于冒险或许会一时失足。却步不前则会迷失自我"。当我们去面对 去尝试 满怀期望时。一时失足而导致失败。不可避免 而我们不去应对 选择逃避。却步不前则会迷失自我。因为这个模型倒过来也成立。当我们不断地逃避挑战。当我们总能处理好困难经历。当我们不允许自己失败时。会形成一个向下的螺旋 影响自尊。我们的成功和幸福。 

第二个技巧 想象成功。有件事我想和你们分享。要和你们分享让我有点紧张。但既然已经说开了 无论如何我都会说。尤其上节课真是我们关系中的转折点。我要告诉你们些事 但别告诉别人。尤其别告诉我的客户。或将来的潜在客户。我演讲前会非常紧张。我会很紧张 其实对我而言。我无论何时站在许多观众面前。心里会七上八下的。我觉得这很难。我说的许多观众是指多于五人。所以这是个真正的挑战。当我和客户们聊天时是个挑战。这一直是一个挑战 从我决定。当一名教师的那一刻起 当我听说了。Marva Collins的事迹时就决定成为一名老师。我说"这就是对我的召唤" 后来我就成为了老师。我曾说 "如果我在人们面前会这么紧张。我要如何成为成功而有效率的老师?"。 

记得成功的三要素。成功的秘密? 是乐观和自信。是热情 热爱你所做的 还有努力奋斗。我有其中两个要素。我非常努力 实际上我的座右铭。来自Richard Hackman教授。他是我在这里读本科时的论文导师。他是一个非凡的教师。他走进教室后会说。准备 再准备……然后顺其自然。所以我遵照他的建议 我仔细看教案。我准备好所有教案 然后不再去想它。我准备得很充足 我很努力 没问题。我有热情 我爱学习。这是我职业生命中最重要的事。但我缺少第三样 缺少乐观。缺少自信 我问我自己。我要怎样提高第三个要素。因为这是幸福的一个重要部分。记得对运动员的研究吗?56%的成功。是源自乐观主义或自信。还记得John Carter二十年前的研究吗。关于哈佛大学商学院毕业生的。将最成功的学生和其他人区分开的。两要素之一。一个是相信自己 自信。这是成功的一个重要因素 而我缺少它。 

我怎么办? 我回想到。打壁球时的事情。我的教练过去教育我要形象化。我开始阅读这方面内容。我查看文献时 看到很多关于这方面的研究。实际上在几世纪前宗教实践时就出现。被东欧运动员引入到体育运动中。他们会想象自己 比如。在跑道上跑然后站在颁奖台上。领取金牌。越来越多的心理学家把它应用起来。销售人员很有效地使用它。想象自己真的谈成了生意。经历困难 去做 并获得成功。那让它更有可能成为现实。为什么?形象化背后有什么原理?为什么这么重要?为什么它这么好?为什么它这么有效?为什么被这么广泛而成功地使用?答案归结到我们大脑的结构上。这个研究是由心理系主任。Stephen Kosslyn教授做的。他说当我们看某样东西时。比如说我现在看着我的手。我大脑中的某些神经元被激活。实际上记住了这只手的形状。我的视觉皮质 后部。现在。如果我闭上眼想象我在看这只手。即使它不在那 如果我想象看这只手。相同的神经元会激活 

换句话说。我的大脑无法区别。真实的事物和想象的事物。这解释了为什么梦境会那么生动。在漆黑的深夜。你正梦到中午。在院子里散步。我们的想象。至少对于大脑而言与现实没有区别。当我想象成功时 从某种意义上说。我在为大脑做一个模拟。我可以说是在欺骗大脑认为。这是真的事情。我的大脑不知道两者的区别。潜意识无法区分。真实的事物和假想的事物。如果我在脑海中反复想象着成功。记住大脑不喜欢不一致性。换句话说。如果我想象着成功并坚持不懈。不止设想一两次 而是一次又一次。大脑不喜欢不一致性。因此它会让外部现实。和内心的想象相符。为什么?因为大脑不知道。真实的事物和假想的事物间的区别。就像Thomas。就像Roger Bannister四分钟内跑完一英里。在那之后。每个人都有了和以往不同的想法。四分钟跑完一英里是可能的 然后37人成功了。第二年超过300人实现了。因为他们有了不同的想法。我们可以通过场景模拟做到同样的事情。因为大脑无法区分。真实的事物和假想的事物。 

不是总能成功。不是百分之一百管用。但很多时候都很管用。我每次演讲总要设想一下。我想象自己站在观众前。冷静 激动 热情。当我正式开始演讲时。现在我不知道这是真的还是想象的。但我假设这是真的。正式开始演讲时 我确实有同样的感受。我仍然会紧张 仍然会焦虑。但这是健康的兴奋。通过练习提高了我的信心和自信水平。就像一个……一个飞行员。会希望能在模拟环境中先练习。然后再飞越大西洋。思想是个模拟器 所以它能管用。关键是…… 这有一个许多人都犯过的错误。无论是运动员或是……。许多关于形象化自励书籍中的记载。关键是不仅仅只关注结果。加州大学洛杉矶分校的Shelley Taylor。做了研究 把学生随机分成两组。一个学生想象在考试中得到A。一遍遍地这样想象。第二组想象他们得到A。但也想象他们在图书馆坚持努力。为考试做准备 最后得到了A。第二组想象了过程。和结果 他们更加成功。考试成绩更好。想象过程和结果。这是你们这周的作业。想象你们实现目的。你们实现目的的过程以及目的本身。史上最有影响力的演讲 或者之一。是马丁路德金的《我有一个梦想》。这个演讲我听了不下12次。为什么? 

因为除了它很美 押韵和重要性之外。也表达了它的意图。它的意图就是描绘出一幅成功的画面。因为这是他为这个国家做的。并为后来人所定义的远景。因此首先 我想从中摘录几句。是对一个更美好的未来的想象。他说得很明确 没有隐藏。 

"我有一个梦想 朋友们 今天我对你们说。在此时此刻 我们虽然遭受种种困难和挫折。我仍然有一个梦想。这个梦想是深深扎根于美国的梦想中的"。这都是关于想象。但通过想象他创建了更好的现实。看他如何谈论这个过程。他不是仅仅谈到结果。我们希望最后达到平等。 

他谈的是我们如何获得这个结果。让我读几句摘录的句子。"现在绝非侈谈冷静下来。或服用渐进主义的镇静剂的时候"。多优美 我是说 他这里说的是。我们别只坐着等待。想象会发生的事情。让我们做些事情 让我们也着眼于过程。他明确地说。"在争取合法地位的过程中。我们不要采取错误的做法"。这里他在提建议 提议。这个过程中什么该做 什么不该做。"我们斗争时必须永远举止得体。纪律严明 我们不能容许。我们的具有崭新内容的抗议蜕变为暴力行动"。他在描述一个过程。"我们要不断地升华到。以精神力量对付物质力量的崇高境界中去"。我们要做的另一件事。当我们想象成功时 尽可能逼真点。尽量使人有所感受 为什么?因为感受越真实。我们的大脑就越相信这是真实的。再次 马丁路德金这点上也做得非常漂亮。让我读几句摘录的句子。"我们将来也不满足 除非正义和公正犹如江海之波涛。汹涌澎湃 滚滚而来"。你们能感受到那江海波涛。你们能看到 能想象到。"我们不要为了满足对自由的渴望。而抱着苦味和仇恨之杯痛饮"。他引入了味觉 苦味等同于仇恨。人们会联想到它 让它真实 具体化。"有朝一日 阿拉巴马州的黑人男孩和女孩。将能与白人男孩和女孩情同骨肉。携手并进"。你能想象到孩子们手拉手围成圈。再次让它变得真实 你可以想象到 看到它。顺便说下 这是所有著名演讲的特征。它们不是抽象的。看看那"伟大的沟通者"罗纳德里根。 

看肯尼迪。他们演讲很精彩 因为他们创造了一幅图画。他们利用了感官 创造了一幅成功的画面。 

"我梦想有一天 甚至连密西西比州。这个正义匿迹 压迫成风。如同沙漠般的地方。也将变成自由和正义的绿洲"。你真的能感受到那种酷热。你真的能感受到他描述的那种汗如雨下的感觉。这描绘出了一幅好的 成功的画面。这让演讲变得很精彩。最后 要唤起情绪。你不想只是做一个认知练习。你要对自己正在做的感到非常兴奋。否则将会失败 什么都没有改变。这些词语间的联系 情感和行动。没有情感就没有行动。如果你想让自己或别人行动起来。你得引起情感共鸣。让我们稍微看看那段历史性的演讲。 

"让自由之声响彻每个山岗。如果美国要成为一个伟大的国家 这个梦想必须实现。所以。让自由之声从新罕布什尔州的巍峨的崇山峻岭响起来。让自由之声从纽约州的崇山峻岭响起来。让自由之声从宾夕法尼亚州的阿勒格尼山脈上响起来。让自由之声从科罗拉多州冰雪覆盖的落基山响起来。让自由之声从加利福尼亚州蜿蜒的群峰响起来。不仅如此。还要让自由之声从佐治亚州的石岭响起来。让自由之声从田纳西州的了望山响起来。让自由之声从密西西比的每一座丘陵响起来。让自由之声从每一片山坡响起来。当这成真 我们使自由之声响彻时。当我们让自由之声从每一个大小村庄。每一个州和每一个城市响起来时 我们将能够加速。这一天的到来 那时 上帝的所有儿女 黑人和白人。犹太教徒和非犹太教徒 基督教徒和天主教徒。都将手携手 合唱一首。古老的黑人灵歌。 

"终于自由啦!终于自由啦!感谢全能的上帝,我们终于自由啦"。我看过十几遍 每次看都很激动。假想你最后说 终于自由啦 终于自由啦。全能的上帝 终于自由啦 有同样效果吗。无聊 枯燥 你必须引发情感来创造行动。不论是在想象中或是对全国人演讲。另一个技巧 另一个我要说的技巧。第三个技巧 第一个是 努力行动起来。投入其中 去面对。第二个 使用想象力。使用思想的模拟器。一个我们所知的最强大的模拟器。第三是 认知疗法。认知疗法已被成功地运用四十多年。Martin Seligman是认知疗法的。创始人之一。他和Beck一起做研究。Beck是这技巧的正式创始人。认知疗法的内容如下。它的基本前提是思想驱动情感。比如 外部发生一件事。我感知到那件事。我感知那件事后的行动是。对那事进行评估。也就是对那件事进行思考。思考后就唤起了情感。 

比如 事件 有只狮子向我奔来。我的评估 天 它要吃掉我。引发的情感是恐惧 然后自然就导致行动。逃跑或勇敢地和狮子搏斗。或另一个例子 一个事件。一个美女在街上走 我的评估。是我的漂亮太太 引发的情感是爱。事件 评估 思考 情感。然后导致行动。认知疗法认为如果我们想改变情感。无论是忧郁或焦虑。我们要干预的是。评估和想法层面。 

如果能改变这个 我们的情感也随着改变。特别地 认知疗法认为。我们要恢复理性意识。我们要恢复真实 当然这不是说。比如 我有。一个非常重要的试镜 我很紧张。认知疗法不是说停止紧张。我们改变你的评估 或你的想法。因为那是自然的 又或者我站在狮子面前。感到害怕是自然的 重要的 健康的体验。但是。它说的是我们的想法有时是荒谬的。我们上节课提到过 比如。我有一次考试成绩不好。我立刻认为自己很笨。或认为我考试永远不会取得好成绩。或者 我约别人出去 被拒绝了。突然觉得没有人需要我。这个事件引发一个荒谬的想法。对情景的荒谬评估。 

那只是一次性的 短暂的失败。而那会引发不好的情感 比如放弃。没人会再需要我了。而认知疗法做的是恢复理性。通过说"是的 这很伤人 很痛苦"。但你知道 哈佛还有3200个其他男人。认知疗法极其有效。比其他干预手段都更有效。虽然它对于。极端棘手的精神病理学病例不太管用。但对于大多数精神病理学病例 多数的焦虑症状。多数的忧郁症状 它已经被证明是最成功的。且起效最快的方法 它不仅有用。而且最有效 最迅速。相对其他干预而言。 

总体上 不是人人有效 但大多数有效。最后 这是我们能学会的重要的内容。我们提过。五年的剑桥-萨默维尔研究失败了。Karen Reivich在一个两周的项目中。教授这些认知技巧。那个两周的项目旨在帮助市中心的。危险人群中的孩子们。降低陷入忧郁和吸毒困境的可能性。增加幸福的可能性。这个为期两周的认知疗法或技巧的课程。结果在长远看来是成功的。所以这是可以学到的技能。这就是重要之处。不是说很容易。不是说你听了一次课。下面的十分钟我要讲一些概念。就能改变你的生活 你要运用它。你们将要把它运用到之后两节。之后两节课中 但你们要做的更多。 

我给你们讲下认知疗法的概要。这三个M是首字母缩写。我所做的是从本质上把。我刚提到的Karen Reivich的成果。Martin Seligman的成果。Seligman的老师Aaron Beck的成果。还有David Burns的成果 你们也许有人知道他。《好心情手册》的作者 也是Aaron Beck的学生。把他们的成果结合起来。成为三个M 每个M有二个建议。你们会看到它们互相有联系。区别不怎么明显 但你们会注意到。你们总能联系上三M中的一两个。在你们行动之前先问自己。"我在哪处扭曲了事实?我在评估方面 或者。想法方面哪里歪曲了事实?"。导致了某些不必要的消极情绪。因为如果我恢复理性。我就感觉不到同样程度的情感。所以当你们学习三个M的时候问自己。三个M指的是非理性想法的三个陷阱。 

第一个是放大。就是夸大发生的事情 比如。归纳法是天生的本能。一个孩子看到这个 会叫它凳子。然后他们在酒吧看到另一张凳子。当他们一岁时出去。再看到另一张凳子 看过20张凳子后。开始形成一个观念 认为这东西就是凳子。现在他们看到一张。以前从没见过的凳子 他们也知道是它什么。就像我看见一个人在街上。以前从没见过他 我知道这是个人。因为我从其他发生过的事件中。归纳出这个也是一个人。这就是归纳法 是健康的 重要的。我们就是这样形成观念和语言的。 

但有时我们做得过头 归纳过度了。比如 我期中考成绩很差。归纳为 我不聪明 我不能成功。这个学生就是归纳过度了 他的评估。他的想法是非理性的。或他拒绝了我 因此大家都会拒绝我。而不是实事求是地说 "看看周围。1504课堂上还有有很多帅哥"。另一件事我们经常讨论的是。全有或全无态度。我们会在讲完美主义时讨论它。它是把失败小题大做。而不是把失败当做一次机遇。一个跳板 而是视之为世界末日。我大一时得了B 现在要无家可归了。永远找不到工作 全有或全无。要么全是A 要么就是彻底的失败。我们会多次提及它。因为这是阻止我们去面对的一个关键要素。阻止我们投入进去。它的另一面。同一个硬币的另一面 是极小化。这是我经常做的。这是我仍要警惕的陷阱。我至今仍会考虑。虽然我几年前就注意到它了。 

直到现在我才能够迅速醒觉自己在这么做。极小化的第一个概念是。Karen Reivich称之为"隧道视野"。比如 我的课堂里有620个学生。620个中618个在看教材。一个在看天花板 在看灯。620个中的另外一个在睡觉。隧道视野指的是当我关注那个睡觉的人。对自己说。"我的课肯定讲得让人无聊 犯困"。或反过来 620个学生中618个睡着。一个在开小差 一个听得十分入迷。我只关注那个人 我说我是个好讲师。我非常受鼓舞。然而不管怎样这都是不现实的。认知疗法把现实主义引入这个等式。我也经常因此而困扰。我最近刚遇到这个情况。你们许多人看过Jon Stewart脱口秀。在Jon Stewart脱口秀中 他讲得太快。对我而言超级快 我感觉就像落后了两拍。还在深思和考虑着两个之前的问题。"这会是个很有趣的研究"。而Jon已经领先了10步。我落后了 感觉很迟钝。我也说了些后来让自己后悔的话。只是一句话。没什么大不了的 但我专注在这上面。我后来飞回家 飞回以色列。在飞机上我一直在想这事。忽略了其他的事情。Jon Stewart脱口秀完了后二小时。这本书成了最畅销书 我完全忽略了这个。我根本没去考虑那。我一直在想那句话 我怎么会说出来的?为什么没有更好的准备? 等等。然后我突然说"等一下" 隧道视野。我把它缩小了 我把视野扩大。然后我说"我说了一句让我后悔的话 超赞啊"。这是喜剧 很有趣 许多人在看。你的许多朋友都看过 都很喜欢。这是个很好的机会 我看到了更大的方面。 

允许自己做回一个人。会感到后悔 但从总体上。我能用比以前更理性的目光看待它。像我说的 我仍然在用这些想法。过去。我可能要花几个月才不会想这句话。但是现在 视野变窄 评估想法 非理性的影响。能更迅速地恢复。消除积极或者消极的影响。有隧道视野的人不会去留意其他618个。专心听课或睡觉的人。忽视了积极面的人看到那些。但说他们不重要 它们很微不足道。比起我的失败或我做的错事 比如。我经常会遇到的一个困境。我大四时申请了奖学金。我得到去剑桥的奖学金。我得的是John Elliot奖学金。剑桥有四个奖学金。John Elliot和Fiske 还有Pembroke。还有一个是John Harvard奖学金。我得到了John Elliot奖学金 四个之一。我得到后不久 我就得到消息了。并不是我无视了它。但我想为什么没能得到John Harvard奖学金。它们之间除名字不同外没任何区别。但那是我的脑袋里立刻想到的。对我而言是很重要的经历 因为我说。"看看你" 我在那个时候开始意识到。我们的评估的重要性和威力。它是怎样决定或甚至毁掉。本该是一件值得庆祝的事。但一旦我们认识它 一旦了解它后。我们可以怀疑它 可以质疑它。可以重组它 也能看到积极的一面。此外 认知疗法就是要我们实际点。 

第三个M是虚构 或称捏造。这是我们无中生有的时候。比如 我们个人化或者归咎他人。被虐待的人身上经常能看到这点。受虐待的妻子经常会说。"是我做的 我的错 我做的不对"。而不是理智地说。"这个家伙虐待我 我要离开"。那是不理性地个人化。它也会有别的方式。我有次考试成绩很差。我因此责怪我男朋友或女朋友。那也是不现实的 我得承担责任。要记住Nathaniel Branden的例子 没有人会来。承担起责任而不是责备别人。Marva Collins和她的学生交流的。重要的观点之一。不是过度个人化。或过度责备 而是面对现实。 

最后是情感的推理。我满怀妒嫉 因此我肯定是个坏人。或我在试镜前。遇到了危险 所以肯定很危险。选取一种情绪 让它成为现实。而不是把情感只当做情感。不必符合现实。而是我对现实的评估。或者 我非常害怕失败 因此。意味着失败是很危险的。实际不是的 我会马上回到这个话题。所以要有正确的评估。更正轨迹 更正这些错误。更正"我很平庸。心理学或数学我学不好"之类的想法。因为我有那种感觉 我的评估就是。我肯定真的很平庸 学不好。而不是理性的想法 让我试一下。看看事情会怎么样 更正扭曲的想法。再说一次 关键是真实化。我们如何通过我们的提问变真实?我在这要给你们举几个例子。你们要自己去练习。这些是其中几个。 

首先 我的结论和现实相关吗?第二 那合理吗? 它们是相关联的。我忽略了什么重要的事吗?比如 之后那本书销量很好。比如1504教室还有别的男孩 等等。什么重要的迹象。是我需要考虑的?进行辩论 举出证据 现实化。但小心别只是进行几何形状的扩大。公交车上还有孩子。这是这些问题做的 让我们放开思绪。它们打开我们的视野 创造现实。我夸大了什么?我在贬低什么?我陷入困境了吗?我忽略了什么进展很顺利的事吗?我忽略什么进展不顺的事吗?这不是不计代价地盲目乐观。是要联系现实 最终缩小范围。大的画面是什么。结论是 Ed Diener和Martin Seligman。对幸福无比的人们做了重要研究。这是对高端的研究。看那些百分之十最幸福的人。我们能从他们那里取经。他们的研究结果很有趣。 

首先 这些人经历痛苦。不比其他人少。比如最不幸福的10%或中间的10%。他们经历过痛苦。他们与剩下的人之间的区别是。由于不同的诠释 他们恢复得更迅速。所以他们沮丧时 他们很乐观 不说。"这个月剩下的日子我都会沮丧"。他们说"我很沮丧 没什么大不了"。我能从中学到什么?发生了什么?我怎样让自己感觉好点?然后他们比悲观者恢复得更快。悲观者会沮丧很长一段时间。事件发生 世上有很多事发生。经常不受我们的控制。我们不能创造所有遇到的事物。有些事件是不好的 有些是消极的。更重要的是我们之后怎样对待它们。怎样评估它们 事实上。我们的评估成了自我实现预言。因为悲观者会说"我现在心情不好。会持续很长时间。不会消失" 而乐观者会说。"这是暂时的 会消失的"。并相信我们的自我实现预言。 

如果我认为它会持续六个月。它更有可能 不一定。但更可能持续六个月。比我说"那也会过去的"。我们对待信仰的方式。是造一个向下的忧伤沮丧的螺旋。或一个向上的拥有积极情绪的螺旋。并越来越多。Barbara Fredrickson所说的。"扩大和构造"现象大部分取决于我们自身。没有捷径 通往幸福 成功。和自身更高水平的信念的道路并不容易。那需要付出和努力 无论是形象化。还是努力奋斗和积极面对。或是学习如何消除非理性思想。并不是弄明白一次。然后就一劳永逸。我们一生都必须持续努力。如果你想继续努力。我们必须一直致力其中。下个内容 讲"关注"的问题。因为我们关注某一要素。或事件与否 有很大的区别。它对我们整个生命的宏观层面有巨大影响。我现在要进行详细的阐述。对一些我们经常讨论的。关于某事的理解和学习螺旋进行更高层次分析。 

Ed Diener 我刚提到他的研究。他也是国际积极心理学协会的。第一任会长。他来自伊利诺州 经过多年研究 他发现。对幸福而言 人们感知世界的方式。比客观环境更重要。如果你想一下的话 很有道理。我们认识许多看起来拥有一切的人。无论是财富 或很好的朋友或家人。他们的梦想都已成真 但仍很痛苦。而有些人拥有的很少。经历了一次次困境。而从没停止为生活感到欢庆 还有些人。他们拥有一切并很感激这一切。不觉得理所当然 他们很快乐 有些人。拥有的不多 永远都在抱怨。换句话说 不只是内在的……抱歉。重要的不只是外在。更重要的是内在。幸福不是取决于我们的处境。或我们银行账户里的金额。更依赖于我们的思想状态。依赖于我们选择关注什么。因为我们的感情。由外部和内部状况一起决定。如何去诠释很重要 

比如。看爱迪生如何对待5000次的失败。对他而言 评估和常人不同。内部评估 主观的评估。我证明了5000种方法不管用。要关注什么由我们自己决定。我们庆祝成功或视为理所当然?我们觉得失败和困难都是灾难。或是成长的机会? 爱默生说过。"对不同思想而言 世界既是天堂又是地狱"。莎士比亚说过。"事情没有好坏之分 只是思想使然"。我不是百分之一百同意莎士比亚。但从很大程度上看他是对的 住在达尔富尔的人。就像生活在地狱中 很难改变环境。并在那样的处境中表现得积极乐观。对住在集中营的人也一样。所以有的外部情况会有影响。一个人 无家可归的人。如果给他们足够的钱买房子 买食物。接受基础的教育 他们当然会更快乐。因为外部情况的改变。然而 在基本需要和基本的自由外。主要和我们如何认知现实有关。以及我们如何让生活变成天堂或地狱。我们做个练习。来帮助我们理解"专注创造现实"这一观点。我有个问题问你们 有多少几何图形?不幸 这个很棒的练习 只能做一次。 

记住这点 你如何创造现实。基于你提的问题。当我们看这幅图时我们在思考。当然不是每个人都在看那个。时钟上的公共汽车上的那个小孩 对吗?但它就在你面前 而你没看到。你们许多人没有看到这点 是专注的问题。我们经常抱怨生活 我们认为。事情很糟糕 很可怕 而没意识到。我们是它的起因 要么是在想象它时。然后我们的信念变成自我实现预言。或因为我们只专注于不对的方面。而这并不意味着。没有什么客观的糟糕的情况。但很大程度上 我们一起创造了现实。我的一个学生。两年前在1504教室 向我推荐这本书。《深夜加油站遇见苏格拉底》是本好书。我刚开始看 它有部同名电影 对的。我昨天才开始看 我看到这个……。这个寓言很巧合 我想读给你们听。作者Dan Millman讲了一个故事。我和他在中东一个建筑工地认识。 

当午餐哨吹响时。所有的工人都会坐在一起吃饭。每天Sam都会打开午餐桶 开始抱怨。"该死的家伙"他会大叫。"又是花生酱和果冻三明治。我讨厌花生酱和果冻"。他埋怨着花生酱和果冻。日复一日。直到一天工作组里的一个人终于说。"Sam 如果你这么讨厌花生酱和果冻。为什么不叫你老婆。给你做点别的?"。"什么我老婆?"Sam回答说。"我没结婚 我自己做三明治"。我们经常自己做三明治。而没意识到这点 因为我们问的问题。我们问的问题决定我们的现实。花生三明治 几何形状 或车上的孩子。或香肠 无论什么 我们创造我们自己的现实。如果我们理解这点 我们能改变。改变我们认知的方式 改变关注的事。改变我们的问题。无论是辩论非理性评估。或是关注积极面。还是创造新的现实的问题。 

我们将在这里讨论。本质上来说有两种原型。在如何认知性地重建 加强 创造现实等。多种途径有着两种原型。两个原型分别是积极者和消极者。首先消极者 消极者是。总是专注不成功的事情的人。花生酱和果冻 专注进展不顺利的事。专注交往中的问题 他或她自己的问题。工作的问题 总是爱抱怨的人。我们都存在于消极和。积极两个极端之间。没有人。总处于某一个极端。每个人都在这个集合当中的某处。我想要鼓励的是 从消极向积极的方向。逐渐移动。因为这样有很多好处 我等下会讲到。 

消极者。这名字出自梭罗的书中。消极者甚至在天堂都能找到缺陷。极端消极者总会经历辞职。感到无助感 为什么?因为他或她开始真的相信。糟糕的现实真的存在。不受他或她的想法支配。而没意识到是他们创造了那个现实。他们开始相信无论他们做什么。无论找什么工作 总有个糟糕的老板。无论和什么人搭档。他们都很糟糕 不顾及别人。无论去什么饭店 服务总是很糟糕。他们总是吃同样的花生酱。和果冻三明治 是他们自己做的。他们向现实屈服。那同时当然也让他们的存在。变成自我实现预言。消极者用柠檬榨柠檬汁。抱歉 用柠檬汁做柠檬。 

而积极者 加大洛杉矶分校的Julian Bauer做的研究。积极者相反 专注于成功的事物。积极者专注生活的光明一面。在乌云笼罩中找到一丝白光。用柠檬榨柠檬汁。不因讲师说太多陈词滥调而和他争吵。and doesn't fought an instructor for using too many clichés.。积极者能在普通中找到奇迹。爱默生就是一个模范积极者。专注于积极面和进展顺利的事上。这里的问题也许是。积极者也许会孤立地……。这是人们经常看到的一种批评。只看到生活的光明面。这样的话的确是一个问题。 

这节课不提倡只专注积极面。它专注于提倡现实。现实包含好的和坏的食物。现实包含几何图形。和车上的孩子 关键在于要同时关注两者。分离。分离地看待事物的积极者 不是一个原型。在长期条件下。它当然也不会使人心理健康。Nathaniel Branden说过。"世界上需要尊重现实"。我们不能只凭专注创造现实。我们是共同创造者。我给你们读一段William James的话。"但这些实在的因素 虽然都非常固定。而我们对付它们 还是有一定自由的。拿感觉来说。感觉的存在 我们虽不能控制。但是在我们的结论里 我们注意哪个。着重哪个 毕竟得凭我们个人的利益来决定。着重之点不同。结果构成的真理可完全不同。事实全同 我们的看法可各异。同一个滑铁卢之战 具体情节绝无二致。而英国人看来是"胜利"。法国人看来是"失败"。同样 宇宙也如此 乐观主义者看作是胜利。悲观主义者看作是失败 我们对现实抱有怎样说法。全看我们怎么给他配景"。 

这是William James在1890年说的。理解我们如何共同创造现实。这是我最喜欢的R.L. Sharpe的诗。"是不是很奇怪 王子和国王。在木屑圈裡跳來跳去的小丑。一如你我的凡人。怎么会是永恒的打造者?人人都拿到了一份规则表。一大块不成形的东西 一袋工具。在生命流逝之前 人人都必须雕出。一块绊脚石 或是一块垫脚石"。经常我们在开始得到的东西。是一大块石头 我们怎么办?我们把它凿成漂亮的"大卫"?凿碎多余的石头?或者它只是我们通往成功的障碍。通常这是由我们决定的 因为我们共创现实。石头就在那 我们不用负责。但我们用它做什么 我们要负责。 

认知重建。是要学习乐观地诠释事情。换句话说 要积极的肯定值得肯定之处。Ann Harbison 我同事。以前也和Philip Stone一起教书。说过"永远不要浪费一次好的危机"。危机在发展中有潜力。我们将谈到的其中一本书。《热情婚姻》是关于婚姻关系的。作者David Schnarch 谈论婚姻中的考验。他说 成功的婚姻。婚姻关系不是一直。和睦融洽的 往往会有争吵。而是经历过磨难。有过危机 有过分歧的婚姻关系。就像黑格尔的否定之否定规律。往往会因为经历过这些困难而获得更好的发展。他说没有其他办法。学习失败或在失败中学习。维持长期的关系必须靠这个方法。关系看起来不像这样。它必须有起有落。从不错失关系中的一个好危机。不是说每个危机都要解决。 

有些关系可以结束 也应该结束。但多数危机能被解决 需要处理好。在个人层面也是一样 Warren Bennis的研究表明。他来自南加州大学。在这里的商学院执教过几年。他鉴定了一群领袖人物。无论是他称为"极客"的年轻一代。或者是30多岁或更大的。他称之为"怪杰"的老一辈。他出过一本叫《极客与���杰》的���。他在���面讨论���。这两���领导风格 领导方式 激情以及。兴趣爱好的区别 有许多的不同之处。比如。极客 年轻一代会谈论工作生活平衡。怪杰们则不知道那是什么意思。而男人工作。女人照顾家庭和男人。则是对怪杰而言的 对老的一代而言。 

年轻的一代 谈论工作生活平衡。他们如何结合生活中所有的元素。当然在极客中的女人数比怪杰中多。两代人有无数不同点 但有一个相似点。所有的伟大的领袖。无论是30岁或85岁。他们都经历过艰难的困境。留下一段充满惊险的往事。我们大多数人经历过困境。但有些人。能从这些困境中找到益处。从而成长。永远不要错失一个好危机。无论你在经历的是。职业上或人际关系上。或是内在个性上的危机。四年或五年前。我还是Leverett宿舍的住宿导师。我们举办了开学前活动。我们讨论了即将来临的一年的情况。和招聘事宜等等。那一年是很糟糕的一年。高盛在裁员 没有校园招聘。瑞士信贷 第一波士顿银行也在裁员。德意志银行建行以来第一次裁员。市场环境非常糟糕 我们讨论时每个人。都在估测市场的形势和他们能做什么。非常严肃的话题 

然后一个学生。我的学生Sean Fieldscoy当时大四。他坐在那里 他学了积极心理学。他举起手说。"Tal 你描绘了一个非常悲观的。消极的画面。但作为一个积极心理学老师。你能否说点积极的?"。观众中有些窃笑 然后是一片寂静。每个人都看着我。我不知道说什么 我第一反应是说。"Sean 事出皆有因" 没错。然后。当我说出口前 我心想。"其实我不相信"。我不相信事出皆有因。但虽然我不相信事出皆有因。我的确相信有些人能够因事制宜。这里有一个很大的区别。因为当我说"我不满意刚发生的事"。我不喜欢它 我更希望市场是好市场。人们就业很容易。许多人争相向我提供工作。当然 我更情愿那样。所以事出未必有因。但有些人接受这种情况。并且能够因事制宜。对任何危机而言都一样 你们以为领袖们。无论他们80或90岁。会主动在生活中寻求严酷的考验吗。会想让这些困难发生吗? 

你觉得处在关系中的伙伴会说。"我们寻找考验。我们吵一架让我们的关系更健康吧"?当然不会 但如果真的发生这种情况。他们会充分利用它。有些事会发生。我们要么无动于衷要么从中获取利益。我再举一个领袖的例子。他是一个模范积极者。我们很荣幸能和曼德拉聊天。南非前任总统 他不可动摇的勇气。他的信仰和正直鼓舞了数千万人。看到自己生命中的可能。我很高兴 很荣幸在此之前几次见过。曼德拉先生 我告诉你们。和他在一起真的 和你在一起。就像同时和庄严和优雅在一起。我想知道一个人在狱中度过27年。被压迫者迫害入狱。出来后没有铁石心肠。没有漠不关心。而怀着愿意原谅和容纳的心。 

我记得一晚吃晚饭时和你聊天。你对我说 我们对压制者的憎恨。太过强烈 我们没有看到和他谈话的价值。那么你看到了什么价值 要放下仇恨。并开展对话?首先我想说 这是一个很好的策略。把一生最好的时间花监狱中。虽然看起来很讽刺 但有好处。如果我没有入狱 不会有……。我不会完成毕生中。最困难的任务 那改变了你自己。我不会有那样的机会。我有那个机会是因为 在监狱里你有。在监狱外工作所没有的。机会 可以坐下思考 这是重要部分。但你需要27年吗?你可以用几天 几周 一个假期。你需要27年吗?好。他花了27年是有意义的吗? 

当然他经历时 很痛苦 很可怕。但他能改变这个。甚至在那个情况下 看到积极方面。当我们改变那个。那些知道他经历过什么事的人。当他开始看这积极方面。即使在监狱也有机会。当他改变思维模式 换句话说。不是外部情况。外部情况确实开始改变。另一个积极者。"悲观者在每个机会中看到困难。乐观者在每个困难中看到机会"。你不会想跟他争论。我想在结尾给你讲一个故事。一个关于我几年间经历的个人故事。一些自传细节。一些你们多数人不知道的事。我要告诉你们。先从消极者的角度。我有轻微的注意力缺陷障碍。这让事情变得非常困难。我很难集中精神 我的脑子经常游离。我真的不断地与它斗争。 

当我读高中时。我很想进入一个数学的赚钱项目中。没通过入选考试 我很失望。我没做到 成了专业壁球运动员。是我11岁以来的梦想。那是我一直会想的事情。那时这是我最重要的事。我想当一个专业选手。20岁时。我在以色列马上就要服满兵役。我在壁球场受了伤。拉坏了背部的肌肉。医生让我选择 或者动手术。这有很大风险 或者放弃职业生涯。我选择放弃我的职业生涯。一个我儿时的梦想 

这个打击很大。我去了剑桥。哈佛毕业后 我参加一个博士项目。我是那个项目中唯一没通过的。我被赶出了这个项目 很没面子。像浪费了一年。因为我甚至没拿到硕士学位。这真的是个很艰难的经历。在哈佛 作为研究生。我和其他学生一起 参加资格统考。在所有参加那个考试的研究生中。我是唯一没有通过的。又是一个非常丢脸的经历。走在William James楼边 大家都知道成绩。这很难堪 不只那样 我被临时。临时选择留在这个项目中。他们对我说 你不仅要通过明年考试。要和下一届一起考。你还要出色地通过考试 否则就出局。剩下的这个学期。忙着于论文和写作的同时。我几乎快毕业还得回去看最基本的知识。把所有教材再复习一遍。 

天啊 我太不幸了。我想说一些我生活中的某些自传性事实。我有轻微注意力缺陷障碍。这很棒 知道为什么吗?因为 这实际迫使我只能学。我喜欢的东西 我关心的东西。因为其他的事情 我就集中不了精神。所以就像我有一个内部机制。强迫我专注最终的东西。专注让我快乐的东西。我只学我热爱的东西 这很棒。我读高中时。非常想进某个项目。一个可以赚钱的项目 但没能进去。我入选考试没通过。这最终是塞翁失马。因为它让我有更多时间练习打壁球。我最终赢得了国际冠军。在第二年。如果我进了那个项目就不会做到了。当我在剑桥大学时。我在一个博士项目 我是班里唯一。据我所知。数年来唯一被赶出这个项目的人。这其实是个很重要的经历。我哈佛毕业时 很自负。我认为自己比谁都神圣。这是个非常令人羞辱的经历。对我而言尤其重要。因为第二年我去亚洲工作。在那里 自负是最糟糕的事情。所以这个令人羞辱的经历。对我后来的生活有很大帮助。也帮助我正确对待事情。 

当我来到哈佛 我是项目中唯一。没通过资格统考的人。这是沉重的打击 很艰难。但结果是个很好的经历 知道为什么?因为我得把相同的教材。这些厚厚的书和文章重新学习一遍。不只是达到基本水平。我要学得非常好 因为这是条件。是教授让我留在这个项目组的条件。这非常好 因为我把教材融会贯通。我的社会心理学 现在大概。比我认识的任何人都好 这帮了我很大忙。它帮我把这个1504课程整合起来。没什么地方比这里更让我喜欢待的。天 我真幸运。谢谢 同样的事实 不同的诠释。只要记住 我们经历的大部分事情。我们参与创造 下星期见。 

第8课-感激 

在上节课结束之前。我跟大家分享了我生平的一些事。后来有人问我"是真的吗?"。没错 是千真万确的。这个故事我是从两个不同的角度来讲的。第一个是消极者的 第二个是积极者的。大家要记住的很重要的一点是。我们所说的积极者并非。不会感到痛苦或失望。愤怒或羞辱 恐惧或失望。至少无法追求自己的目标。是一件很让人失望的事。如果这个目标碰巧是关于职业生涯的。或者是如果自己是唯一一个。没有拿到学位的人。而不得不在Willam James大楼再呆一年。就像脑袋长了一个角那么羞愧。这种事没有快乐可言 非常痛苦。但是积极者……对不起。积极者和消极者的区别在于。积极者明白。这世界是不会事事如愿的。但我们可以扭转坏事。事情会好转的 一切会变顺利的。可能需要点时间。可能要过一段时间才能看到曙光。可能要过一段时间才能忘了羞辱痛苦失望。但一切不好的事都会过去。也就是说 积极者明白。这些感觉是暂时的 他会允许自己。有人之常情 他会明白。事情最后会好转的。我明白这点 发生的都发生了 这就是人生。允许自己有人之常情 包括允许自己。感受这些负面情绪 允许自己失败。我们稍后会深入讲允许自己失败。等我们讲到完美主义时。因为完美主义对失败有强烈的情绪。如果做不到最好的 那就是最差的。如果你不是完美无瑕 那你就是一无是处。 

要么就是非凡 要么就是平庸 没有中间地带。积极者明白。人的本性决定了。我们会有痛苦的情绪 人的本性。或者说是人就会有失败的时候 但失败会过去的。做积极者有很多好处。第一个好处就是幸福感多了。积极者感到更多幸福 还有其他好处。例如。波莫纳大学的Suzanne Thompson做了以下的研究。她找那些在加州大火中。失去家园的灾民。那时候很多灾民……那是一场很大的山火。很多人去失去了家园。火灾过去 她去采访他们。她把积极者和消极者区分开来。积极者并没有说"我很高兴火灾发生了"。而是说"这场天灾也有好的一面"。"我可以重新开始 火灾给了我一个新起点"。"现在我更喜欢我的家 我的家人都安全无事"。"这让我很欣慰 这是好事"。所以积极者看到的是好的一面。随后她跟进这些受访者。那些跟消极者区分开来的积极者。长远来说 他们感到更幸福。他们有更多积极情绪。焦虑的情绪更少 身体也少患病。所以对身心都有好处。有很多研究针对积极心态对身体的影响 例如。Glenn Affleck找一些人做过一个研究。这些人有过心脏病发。有些人把病发看成大灾难。世界末日。另外一些人当然不会为病发感到高兴。但他们会说"这是一个警钟"。"心脏病发也有好的一面"。"因为它告诉我要好好照顾自己的身体"。 

或者病发让他们对自己的价值观有所改观。看到积极的一面的人。把心脏病发视为一个警钟的人。再活八年的几率更高。而且再次心脏病发的几率也更低。这听起来很显然而见。因为他们会改变生活方法。但原因不只是改变了生活方式。我上次介绍的UCLA大学的Julienne Bower。他研究过一些艾滋病患者 找出其中的积极者。不是那些说"太好了 我有艾滋病"的人。而是那些说"因为患了这个病"。"现在我更懂得感激一些事"。"因为患了这个病"。"我更关心那些真正重要的事"。"因为这个病 我和一些人更亲近了"。四到九年后 她跟进这个研究。她发现积极者的存活的几率更高。活下来的可能性更高。Laura King和Minor做的研究表明。我们可以把人培养成积极者。所以除了那些天生的积极者。或者消极者以外……我们知道积极与否和基因有一定关系。但除了天生的以外 还可以后天培养。他们找来一些生活中受过创伤的人。让他们写下这些创伤。写下从创伤中"看到"的好处。那些写下从创伤中"看到的好处"的人。身体和心理都更健康更幸福。另一个研究的对象是癌症患者。那些参加研究的女性写下她们的病情。她们还会写。"患上癌症带来的好事"。写出这些的女人。看医生的次数更少。战胜癌症的几率也更高。通过改变她们的观念 改变她们的关注点。这些女人没有说"患上癌症最好了"。也许有些人会这么说 但大部分不会。 

他们会说"但愿我没有癌症"。"但我确实患上了。关键是我们怎么处理 怎么看待这个病"。然后她们跟自己说。"我们和家人更亲近了"。"我现在更感激生活了"。"我可以享受一下花香了"。"现在我和家人乐也融融"。"现在我知道谁才是真正的朋友了"。她们从坏事中找出好事 而不是把坏事看到好事。但她们会尽力从坏事中看到好的一面。这种心态甚至会影响到他们的寿命水平。有很多人研究乐观。积极心态和长寿之间的联系 例如。在Mayo诊所那839位住院病人中。研究人员用他们作研究对象。他们区分开积极者和消极者。两年后那些乐观的积极者。存活率高出19%。到目前为止 在研究积极心态和乐观领域中。影响最深最有趣的实验就是修女实验。这个实验从多方面表明圣经里说得没错。快乐真的能使人长寿。这个修女实验是从1932年开始做。1932年 178位修女完成受训。她们的年龄大约为22岁。这些即将开始传教的修女受到方方面面的测试。其中一个就是她们要写自己的短小传记。这个资料几十年前就收集起来了。最近心理学家才打开这些资料。想对它进行研究。想弄明白。有几个修女活到了今天?活了多久?这个实验在1932年展开 她们当时22岁。他们想找长寿的预测因素。所以他们看她们的传记写得多深奥。也就是说考察她们的智力水平。跟长寿一点关系都没有 

他们看居住环境。看看居住环境的污染程度。会不会影响她们的寿命。没联系 住加州的和住波士顿的没分别。他们研究她们的虔诚程度 信仰程度。她们那时才22岁 对长寿没什么影响。只有一样东西。跟她们的寿命有联系。那就是积极情绪。研究人员所做的是。看她们写的传记。他们不认识这些女人。所以这是一个完全盲法研究。双盲研究 研究人员看这些传记。把传记分为四类。最积极的 最不积极的。中间还有两类。然后他们比较最积极那类。和最不积极的那类。他们得出以下的结果。我先给大家读一段她们写的传记。让你们了解积极的传记是怎样的。最不积极的是怎么样的。这一段是积极类里的Cecilia O'Payne写的。"上帝为我的人生开了一个好头"。他恩赐我生命以无尽的价值 过去一年"。 

"我在圣母学院受训 这一年我过得很快乐"。"现在我热切期待接过圣母的圣衣"。"接受主无上的爱"。快乐 欢乐 爱 这是一个积极的人。下面这位属于最不积极的一类。你们会看到 她并不是特别消极。但她并不关注于积极情绪上 没有欢乐。Marguerite Donnelly"我出生于1909年9月26日"。"家里最大的孩子 五个女孩 两个男孩"。"我在女修道院会受训一年"。"教化学 第二年在圣母学院教拉丁文"。"蒙主恩赐 我会尽力完成使命"。"传教 修道"。非常写实。没有积极者的那么积极。没有Cecilia O'Payne那么乐观。现在我们来看看这些资料。85岁时 我要重申 她们85岁。是很多年前的事了 最积极那类有90%。还活着 而最不积极那类只有34%。两个数据相差很大。这并不说明消极者就不会。活到120岁 也不能说明积极者就不会。30岁死于心脏病发 当然个别例外是有的。但平均来说。在这个长寿研究中 最能解释。两组相差如此大的数据的因素。就是积极情绪 总体的积极性。再过九年 她们94岁的时候。最积极一类中有54%还活着。而最不积极那类只有11%。结果很显著 你也可以拿一些传记来。分析它们 你不知道谁还活着 谁死了。只根据积极心态这一因素。你的预测就能达到很准确的程度。你能预测谁会长寿。谁还活着 谁已经去世了。 

现在我看这些资料。还有很多关于长寿 身体。健康等资料 我觉得"太棒了"。"积极心态真的很有用 当乐观主义者真好"。但我有两个问题 "为什么不是每个人都是乐观主义者?"。如果我们能更幸福更健康。为什么我们不都去做乐观主义者?这是第一个问题。第二个问题是 如果我想当乐观主义者。我该怎么做?第一个问题:为什么不是人人都是乐观主义者?第二个:怎么变成乐观主义者?我会回答这两个问题。先来看第一个问题。为什么乐观主义者这么少的主要原因是。我们认为乐观主义是不切实际。我们怎么知道?谁让我们觉得这是不切实际。谁让我们觉得乐观主义者不切实际?最主要是媒体。我们在媒体上都看到什么?仇恨。流血 不幸福。恐怖主义 如果有人说。"我是乐观主义者 我觉得世界很美好"。什么?你脑子不正常了吗?你不切实际 你盲目乐观。你看看这世界发生的可怕事情。在这样一个世界里 你怎么乐观?你怎么积极?快乐幸福从哪谈起?在这样的一个世界里谈幸福心理学?你太盲目乐观了。很大程度上来说 Thomas Hobbes的话说得没错。人生短暂 残酷 可悲。在现代世界 这种悲观情绪。比乐观情绪更切合实际。 

我们来看看几个新闻头条。这是我几周前为本节课备课时。找来的新闻头条。委内瑞拉 飞机失踪 航班延误。选举 科索沃再次发生暴动。新国家诞生 发生暴力事件。我们关注的却是暴力 地震毁了几百个家园。土耳其入侵伊拉克 <血色将至>。(血色将至VS老无所依)两部电影竞逐奥斯卡。这算是好消息 Tori Spelling……。继续 刚才是CNN的新闻。这是……什么?我忘了哪间媒体了。应该是路透社 伊朗核问题悬而未决。还是土耳其 大使馆抗议 欧盟。之类的 都是消极新闻 这是福克斯新闻。关注点是什么?小孩母亲想脱身。一个母亲抛弃孩子。它没有关注那几百万个。几十亿个抱着孩子的母亲。而是关注想脱身的母亲 他们报道原因。是什么让这位母亲想抽身离开。之类的 这么多负面新闻。在这样一个世界里 我们怎么乐观?那太不切实际了吧?接下来让大家看一位专家。我最喜欢的心理学家 Ellen DeGeneres。Ellen DeGeneres(2003年<此时此地>片段)。我前几天在看新闻。"由Paxal(抗抑郁药)赞助" 看完真的要吃 非常聪明的广告。 

我小时候每天只播一次新闻。错过了就没得看了。现在全天候都有新闻 这还不够。主持人在讲 下面还有滚动字幕。你一边看主持人 一边看滚动字幕。你还一边上网发表评论。"不 我要说不……"。又有一条滚动字幕不停播放。"情况持续恶化" 还有完没完。还有本地新闻 "天啊 他们想你"。每个新闻节目都看 是吧?你看了自己想看的还不够。他们还会想噱头来吊你胃口。这种做法太可恶了。"这算得上全世界最可怕的东西"。"有可能就出现在你的餐桌上"。"今晚11点为您揭晓"。是……青豆吗?我觉得这些主播很可怜。我们可以关掉电视 但他们的工作就是。读这些报道和字幕提示机上的字。他们不知道下一条新闻是什么。他们的情绪大起大落。"无一生还",下一条。哪种糖能帮你减肥?继续下一条。有个小行星正撞向地球。但先让我们来找找城里哪家批萨芝士最多。有令人不安的研究表明研究很令人不安。她真的是我所认识的最聪明的心理学家。这个学期我还会放很多她的录像。所以说媒体是罪魁祸首。确实媒体把大量关注点放在负面新闻上。但这样也不全是坏事。因为在公民社会 媒体的职能之一是。把应该纠正的错误行为公诸于世 鼓励人们。去行动 去改变 让世界变得更美好。但是我们要认识到。媒体报道时并不是就事论事。而是单独强调事件中的某几个方面。所以就会涉及媒体偏见了。 

媒体偏见并不是说 左倾的CNN。和路透社对右倾的福克斯和华尔街有偏见。这不是我所说的媒体偏见。我说的偏见是一种倾向于负面的偏见。媒体关注突出的是负面。它就像一面放大镜 而不是眼镜。我们需要记住这点 并对此加以纠正。媒体报道时没有就事论事。纽约时报头版上的不是全面的事实。华尔街日报上的头版不是全面的事实。它们只是突出事件中的某个方面 加以放大。它强调负面 关注负面。土耳其发动的战争 科索沃的流血事件。或者说母亲弃儿 仇恨事件。媒体这样做会造成。正面新闻得不到足够关注 大家想想。这两种做法正是我们之前讲过的。三种心理扭曲中的两种。放大消极 缩小积极。也就是说媒体扭曲了我们的观点。它实际上让我们变成了悲观主义者。尤其是现在一天24小时都有新闻。我们不停地受到消极的轰炸。一个接着一个。我们能从哪里听到正面新闻?新闻报道最后30秒。没错 这世界还是有点好事的。这样做只是为了让你笑一笑 让你明天。或下一小时 继续收看更多的坏消息。一个接一个的负面消息。扭曲 造成认知扭曲。这就是我们之前学认知治疗。和认知行为时 谈到的心理陷阱。 

我们成为悲观主义者 有没有人想为什么大部分人。是悲观主义者? 当媒体对我们有那么大影响力时。我这样说不是贬低媒体的作用。我这样说是为了突出。我们需要认清并且反对的一个方面。我们怎么反对……等一下再谈这个。媒体突出什么?欺诈行为 例如玛莎家居。安然 世界通讯公司 突出它们的欺诈行为。而不是去报道世界上每天都发生的。另外那几百几千亿宗的正当交易。此时此刻就有很多正当交易了。强调负面 无视正面。在我们身边每时每刻都在发生的。几百几千亿的正当交易。其他正在发生的事呢?我们……。或者说媒体推动我们。从一些负面例子中进行联想演绎。而无视了其他几十亿人。贡献他们的人生为世界传播幸福。例如派送食物的救助站。让世界变得更好。或者在后院里写文章。1800多名哈佛学生在PBHA组织里当志愿者。想想我之前谈过的错误刻板印象。不管是在哈佛还是在全美国。很多人花了很多时间去帮助别人。 

但我们关注的却是几个负面的。这会在我们头脑里形成一种观念。我们会觉得这个世界是一个很糟糕的地方。现在我们特别关注恐怖主义。但那几十亿个。想生活在和平中的人呢?这并不代表。我们要无视恐怖主义 或其他负面新闻。这是媒体一个很重要的职能。但同时我们不应该无视了积极的新闻。媒体的关注大部份都是那些伤害别人的人。例如关注强暴。无视了几十亿人的正常性爱。这不仅发生在童话故事里。世界各地的人……我不认识他们。但愿他们不在这个教室里 这是谷歌搜来的图片。不只是世界各地的人 在哈佛这里的人。我不知道他们现在还做不做这个调查。但我在网上找到最近的一个是04年的。我不知道后来发生了什么事 你们当中。有些人不相信在哈佛也有人。在这里在做爱的。这证明了的确有这些人存在。他真可爱 那婴儿很可爱。好了。我并非说我们在无视消极的新闻。无视社会存在的问题。我们应该关注这些问题。我们应该改善世界 但同时。我们还要明白 那些跟乐观主义者。和积极者说"实际点"的人犯了以偏概全的错误。因为脱离实际的不是积极者。相反是消极者。因为世界上好事多过坏事 多得多。事实上 看到积极占"整整一半"还不够。看到一半还不够 要看到那90%。因为世界上好事比坏事多得多。我们需要关注它们 不仅因为这样做是好事。不仅因为这样做能让我们更健康更幸福。还因为关注能创造现实。 

我们都明白这个道理。当我们看到公车上没人照看的小孩。当Marva Collins说以下这句话时。"我们要关注人性中伟大的种子"。我们在恢复性研究里看过。当关注点从"为什么这么多人失败了"。转变为"为什么情况这么困难,但有些人能成功"。我们就能明白这个道理。关注创造现实 大家想想。当我们看到诈骗罪时。看到报纸上经常报道的不正当行为时 我们开始相信。我们开始相信。"你想成功?"。你想成为财富500强的CEO?像安然那样的公司?那你就要犯诈骗罪。除此以外,没有别的成功道路了 为什么?因为另外那几百万个。靠正当途径成功的人 没有被报道。大部分都没有被报道 我们关注的是负面。我并非说揭发者存在并不重要。报道负面新闻并不重要 但同样重要的是。我们还要关注那些正直而且成功的人。因为他们才是大多数 超过90%。因为如果不这样做 那些人就会想。"如果我想成为成功的商人"。"那我就要做违规的事"。 

当他们要作出道德决定时。当他们在35岁当上经理时 他们就会想。"成功的唯一途径就是不要太守规矩。这就形成了自我实现预言。这里有多少人……说真的 请大家举手。想过进入政界 成为州长。总统或首相的?有多少人想过……除了我。好的 很多 我们知道。在哈佛有很多。除了那些举手的人。有多少人听过别人说。"你不要进入政界 你太诚实了"?太老实 我们经常听到这样的话。这就成了自我实现预言。因为很多本来会从政改善世界的诚实人。有了这种想法后就不从政了。因为别人都说"政客是骗子"。政客不是骗子 有些政客。是骗子 但他们只占少数。大部分都是诚实工作。想让世界变得更美好的。他们确实会犯错误 毕竟他们也是人。但他们的意图是好的 他们有道德操守。虽然我们都关注坏事 但世界上还是有很多好事。这确实算是一个奇迹。在我来看 真的是一个奇迹。这体现了人性的强大。我们人性的善良因子的强大。因为我们能保持这份善良。还有 我们脑袋里不停有个声音在说。"为什么我不这样做 为什么我还没看书?"。"为什么我还没写完论文?"。"为什么我要跟他说这种话?"。"为什么她要跟我说这种话?"大部份都是消极的。 

我们多久才听到这样的声音一次。"我今天跟别人聊得很开心?"。"我刚做完这么多工作 我太高效了"。"我做得很好 我刚拿了A-的成绩"。"我都没怎么花精力"。 

当然 在这门课是不可能的 在别的课程上。为什么我们脑袋里很少这种积极的声音?因为我们的观念都关注在消极上。我们下周再来谈改变。我们会明白媒体在我们脑袋里。形成了一条神经通道。因为它让我们关注于消极问题上。所以我们慢慢就变成了消极者。怎么改变?我们等一下再谈这个问题。关注于好事上是很重要的。你知道人们怎么评价甘地吗?他为他国家做的最重要的一件事是。他使印度为自己感到自豪。他使印度为自己感到自豪 这是伟迹。美国呢?你们为美国感到自豪吗?我们住在这里 这里的大部分美国人。你们让美国为自己感到自豪吗?你们知道美国也有缺点 不是一个完美国度。它从建国起就在犯错 以后还会继续犯。希望还会犯很多年。但我还是说这是一个伟大的国度。很多人想来这里生活 为什么?因为这里的自由和机会。虽然不完美 但这是一个伟大的国度。我们感激美国吗?我们感激它的伟大之处吗?因为如果我们不感激 我们会付出很大的代价。现在国内很流行"抨击美国"。当我们专注于抨击时。当我们专注于消极时。我们没有改善现实 反而使其恶化。没错 我们需要有批判精神。这个国家的伟大之处之一就是。你可以站在大街上 或者写一篇文章。批评政客 制度 其他人。这就是这个国家对世界的重大贡献之一:言论自由。但同时我们有这种自由。我们应该把这种自由用于专注。社会中的好现象。因为感激很重要 不管是感激一个国家。一段关系 一个人。一个学生还是一个老师。感激很重要。 

我从字典查来"appreciate"的意思:重视其价值。认同我们身边的人或世界中最好的一面。肯定过去和现在的优点和潜力。感激那些为生物。赐予生命 健康 活力和优点的事物。appreciate的第一个意思就是感激。这是一种好行为。但它不只是一种好行为。我们以前谈过这点。"appreciate"第二个意思是增值。经济增值 银行里的钱增值。这个意思很重要。因为当我们感激好事时 好事增值。当我们感激生活中的好事时。当我们感激其他人的善良时。当你感激这个国家的伟大之处时。好事就会增多 可惜反之亦然。当我们不懂欣赏时 不管是我们自己。我们国家 还是我们的恋情。好事就会贬值 这点我以前就讲过了。有没想过为什么很多恋情不顺利。蜜月期过后就撑不下去。人们都会问"出什么问题了?"。或者"我们怎么能改善这段关系?"。如果我们单纯问这些问题。我们就会专注于问题和弱点上。我们就会无视优点。顺利的地方。当我们不专注于优点上 优点就会贬值。在我们眼里甚至看不到这些优点的存在。就像在你眼里 公车上那个小孩。是不存在的一样。本质上 感激能形成一个良性循环。 

我来解释一下 我来随便举一个例子。例如 星期一早上 或星期二早上。你正走向1504室。突然你看到你的一个朋友。你知道这个朋友是个很真诚的人。你信任的一个人 你朋友看着你。然后说"你看起来真精神"。这时候你会有什么感觉?你会觉得自己很精神。你走路会走得更得意。然后你走进这间教室。另一个朋友看到你说"你看起来真精神"。这时你会有什么感觉?更精神了。然后你进来 坐下。另一个一个月没见面的朋友。"你真精神" 这时你有什么感觉?你精神到不能再精神了,往复循环。你这天就会越过越好。现在假设是另一种情境。星期四早上你走进1504课室。突然一个你信任和重视的朋友。看到你跟你说"天啊 你怎么了?"。 

本来你没事的。但现在有事了 你走进教室。有人看到你说"天啊"。你会有什么感觉?你会觉得更糟了。你刚坐下看到又有人看着你。完全震惊了。这时你的感觉跌到谷底了。恶性循环 越来越糟。感激能形成良性循环。我们知道有时候 一日之始听到的。一句话一个词能让你过上愉快的一天 甚至一周。这个研究表明一句话如何。影响一生 这是Bandura做的研究。一句感激之言就能让我们有力量支撑下去。但这有一个关键:感激一定要出自真心诚意。等我们谈到自尊时 我们会谈到。虚假的感激会带来多大的伤害。我不是说你们的男朋友或女朋友。穿着同一身衣服出去玩了一晚。第二天回来问你"我好看吗?"。你当然不能说"很难看"。我们说的不是这种灰色地带。感激一定要真诚 要跟实际相关。等我们讲到自尊时再深入谈这点。但关键是你要明白。你总能找到可以感激的东西。想想刚才的癌症病人研究 或者艾滋病人研究。连他们都能在这么悲惨的境况中。找到值得感激的事 当他们感激时。当他们从坏事中看到好事时 好事就会升值。越变越好 

Oprah是真正的积极者。"你专注的东西会变大 当你专注于生活中的好事上"。"你就能创造出更多好事"。"机会 恋情 甚至钱都会向你滚来"。"当我学会感激"。"不管我生活中发生了什么事"。因为我们专注的东西会升值。再次说明 我一直在课堂上。用这个比喻:种子。如果我们不给它浇水 它会怎样?如果我们不让阳光照射它 它会怎么样?这种子就会凋萎死去。很可惜 人类的大部分潜力就是这样被扼杀的。大部份恋情和国家的潜力。如果我们想它成长 想它开花。我们就要让阳光照射它 给它浇水。Marva Collins正是这样对待她每个学生。身上的伟大的种子。在经典的皮格马利翁实验中我们看到。当老师的专注投在那些。被挑为"潜力生"的潜力上时。当老师专注于他们时 就等于投射了阳光。当阳光投射在他们的潜力上时。这些学生就开始变聪明。他们的潜力就发挥得更充分。那我们怎么改变这种错误的思维呢?我们该怎么做?我跟大家说一些。社会层面上的做法 和一些个人层面上的做法。在社会层面上 我们可以创造好消息。现在对好消息的报道不够 有人说这好消息没市场。很无聊 没错 有很多是没市场。部分原因是因为好事太平淡无奇了。但我们可以报道伟大的技术突破。让我们的生活变得更美好。 

为什么不报道那些伟大的医学进步。为什么不多点报道和平?健康?大家都知道和平代表这个故事没有吸引力了。有战争时才有吸引力。那何不更多关注这些领域?这是一份在网上成立的报纸。好新闻网络 我强烈建议大家去看看。几年前才开始的。它让大家更加关注好现象。大家都知道爱因斯坦。53年那一天柬埔寨获得了独立。柏林墙 它会提醒我们。自己历史上的好事和今天的好事。他们会谈论一些很好的消息。另外一个叫Geemundo 我不知道这是什么意思。这是另一个专注于好现象的网上报纸。除了看华尔街日报和纽约时报。何不一大早就上来看看。看看这些网站 抵消我们的负面思维。艺术 对于改善世界非常重要。对于改变我们的内在思维非常重要。看看文艺复兴画家弗米尔画中的光。在看不到多少曙光的时候的一束光。因为那时是中世纪 黑暗时代。但画家专注于一束光这个好现象上。专注在人的潜力上 为文艺复兴奠定基础。一切从艺术开始 又如在18 19世纪 浪漫主义。贝多芬或雨果 他们都遭遇很悲惨的困境。但他们专注于崇高伟大的事业上。专注于人的潜力上 亚里士多德曾经说过。"小说比历史更重要"。因为历史只是一五一十地描述生活。而小说则是描述对生活的憧憬。对生活的憧憬 人的潜力。这为世界上更高层次的自由和平等。奠定了基础 

又如上世纪30年代的美国。很大程度上 好莱坞鼎盛时期的导演Cukor。Capra Sturges正是在大萧条时期。拍出他们的代表作。人们会去电影院看这些电影。鼓励他们 帮助他们度过困难时期。这段时期同时也正好是……。一战和二战中间 全世界都陷入困境。正在复原的世界突然又陷入另一场大战。但这些艺术家还是不停地鼓励人们。他们并非无视问题 而是在改善。这是我最爱的电影《苏利文的旅行》。现代艺术也是 如果我们走进画廊。感受到艺术对我们的鼓励 不管是摄影。雕塑 电影 还是书 这样不是很好吗?还有更多的例子。艺术家的作用之一也是突出。社会中的问题 使我们对其加以改正。但如果艺术家 如果一个人真的很想。把好事发扬光大 那么光是专注于不好的现象上。是远远不够的。因为如果我们不突出好事。如果我们不感激好事 好事就会贬值。基本上这就是幸福心理学的本质。大家想想幸福心理学家做的两件事。们做的第一件事是。研究好的现象 第二件事是。他们专注于优秀中的优秀。通过这样做 引用Miriam的话。他们使优秀民主化。因为他们专注于好事上。他们没有忽视每个人的优秀之处。没有忽视那些优秀不凡的人。没有忽视世界上的Marva Collins。没有忽视高危人群中那些不屈不抗的孩子。 

除了媒体外 还有一个原因导致我们。忽视了好事 专注于坏事上。我们会适应于习以为常的事。驱动我们 吸引我们的是不平常的事。是反常 而不是常态。因为世界上有很多好事。好事太平常了 我们就习惯了。我们就再也看不到好事 而坏事。是一种反常 总能吸引到我们的注意力 为什么?因为大自然把我们造就成这样 我们是变化探测器。我们是变化探测器 当有变化发生时。我们的头脑就马上专注于变化上。有些不同的事发生了 不管是我们看见。还是听见。这是给那些睡着的同学听的。使我们惊醒的是反常 适应性是好事。大自然把我们造就成这样是好事。因为它能帮我们逃离险境。例如 当我们听到有狮子。向我们走来 或者想悄悄接近我们时。它能帮我们闻到毒味。一些不同的事 发生了变化的事。还帮我们听到熟睡的宝宝的动静。这太厉害了 沉睡时 你知道……。可能房子塌下来了 但你不会醒来。但如果宝宝在哭 你就会马上醒来。我们都是变化探测器 大自然把我们造就成这样。或者说上帝把我们造成这样 让我们能更好地生存。某些程度上来说 适应性是好事。首先 如果我们不适应。我们就会听到各种各样的吵声 我们会变疯的。因为我们现在周围有很多嘈杂声。除了我们在听的声音外。打字声 呼吸声。偶尔外面传来的汽车声。 

如果我们什么都听得见 我们会疯的。所以适应性是件好事。这就是为什么有些人住在高速公路旁边。晚上还能睡得着。因为过了一两个月 他们就听不到汽车声了。我们有亲戚住在协和地铁站附近。可以说他们就住在地铁站上方。我们去他家吃晚饭 有辆地铁经过。房子震动了 声音非常大。一开始。我们看着对方 他笑了笑。第二次地铁经过时。房子又震动了 非常吵。我禁不住问"你不觉得烦吗?"。他问"什么?"他不知道我在说什么。我说"地铁刚经过"。他说"我们现在都注意不到了"。"因为每隔二三十分钟就有地铁经过"。他们没注意到 所以适应性是好事。在困境也这样。我个人遭遇过最大的困境是。1997年9月19日。当时我住在新加坡。我最好的朋友从印尼过来看望我。那时她对我来说是世界上最重要的人。她的飞机坠毁了 胜安航空MI185。在座可能有些人记得 我的世界崩溃了。我这辈子第一次。觉得我不想活了 我没有了活力。我打电话给家人和另外一个人。他对我的人生产生过很多影响。Nathaniel Branden 出于很多原因。 

首先 他是一个著名的作家兼心理学家。他当时刚去过新加坡。他刚见过Bonny 我打电话给他。因为他是心理学家 我认为他能帮我。还因为他45岁时失去了妻子。她在泳池里中风发作溺水。他回家时在泳池里发现她溺水了。他经历过这种痛苦 比我早25年。所以我想听听他的建议。我在电话上连话都说不清楚。因为我不停地哭 他跟我说。"Tal 我现在跟你说的话可能没意义。但我说的是真的 你会撑过去的。我知道你会的 我们会撑过丧亲之痛 虽然很难。很痛苦 哭吧 把情绪释放出来。但你会撑过去的 我们都能克服痛苦的情绪。我们都能 因为如果我们不能 那只有主能救我们"。因为如果我们不能 就只有主才能救我们。过了很久。大约三个月后 我才可以重新投入工作。过了一整年以后 我才感觉到。我以前有的那种动力和热情。现在我想起Bonny 我会苦笑一下。但我学会感激我们度过的美好时光。感激我有荣幸认识她。适应性是一种好事 很重要的一种能力。因为如果我们不能适应 就只有主才能救我们了。但适应性也有坏的一面。有坏的一面。当我们适应了 我们就会习以为常。这不是好事 因为我们。我们对重视的家庭习以为常。我们对友情习以为常。我们对在餐厅里为我们端上的饭菜。感到习以为常 这是多么奢侈。 

我们对坐在旁边的人习以为常。我们对在学校里上的课程习以为常。因为我们适应了。在处理消极的事情时 适应性是好事。但当我们对生活习以为常时就不是好事了。我们需要问的一个问题是。能不能做到两全其美?能不能适应消极的同时。不对积极的东西习以为常。有一个犹太传统故事里讲到一个男人。住在一个犹太小村里。东欧的一个犹太小村 他过得很苦。他住在一间小房子里 家里有很多孩子。他妻子整天在耳边唠叨。他们不停地吵架 生活得太糟糕了。那个男人想改善他的生活 所以他去找拉比。把他的情况告诉他。他说"我们活在小房子里 挤满了孩子"。"没有隐私"。"我妻子整天在唠叨 拉比 帮帮我"。拉比说"你后院有鸡吗?"。他说"有"。"下周 把鸡带进屋里"。"拉比 你在说什么?我们没有地方……"。"把鸡带进屋里"。男人很虔城 信任他的拉比。一整周都把鸡放在屋里。鸡毛 鸡粪 臭味 什么问题都有。孩子们吵得更凶了 他妻子更唠叨了。"你为什么要这样对我们?房子本来就够小了"。"拉比吩咐的"他们继续吵架。 

这一周终于结束了。他跑到拉比那里说。"拉比 帮帮我" 拉比说"情况怎么样?"。他说"更糟了 家里没地方了 我们整天吵"。"很臭 太糟糕了"。"孩子 你后院有牛吗?"。"有 拉比 我们有牛"。"把鸡和牛都带进屋里一周"。"拉比 但是……""听我说的做"。所以他把牛带进屋里 这一次也很糟糕。他鼻孔里都是这股臭味。他们一周不得安睡 太糟糕了。一周过去了。他整个人衣衫不整 憔悴不堪。浑身臭味 他去找拉比说。"拉比 帮帮我 情况太糟了 糟多了"。"你家后院有马吗?"。"有 我知道你想怎么做了"。"把马带进屋里"。他又把马带进屋里。太糟糕了 马乱踢乱叫。到处地跳 把东西摔坏。家里一片混乱 吵声不断 糟多了。终于一周过了 他跑去找拉比。拉比说"现在你过得怎么样?"。"拉比 太糟了""好的 下周"。"你把动物都带出去"。"然后再来见我?"一周过去了。他来见拉比 拉比问"现在情况怎么样?"。"拉比 太好了 家里宽敞多了"。"气味太好闻了"。"我们一家人乐也融融"。"孩子很高兴 非常感激你 拉比"。问题来了。 

我们要等到情况恶化。才得到感激眼前和身边的好事吗?我们什么时候开始感激健康?我们什么时候开始感激健康?等我们或旁人的身体现出问题时。我们什么时候开始感激生活?当我们有危险时 当我们失去亲友时。我们需要问自己一个问题。"一定要等外界发生一些异常的悲剧时。我们才开始感激习以为常的东西吗?"。我们身边和内心都有无尽的幸福。就在我们身边 餐厅里。坐在旁边的同学 在你家里的房间。我们身边我们心里。有很多好事值得我们感激。但我们都把它们习以为常 真的要等悲剧发生吗?答案是不 如果我们把感激当成一种生活习惯。如果我们培养感恩的习惯。正如媒体培养我们成为消极者一样。我们可以把自己培养成积极者。我们可以培养感恩之心。 

因为当我们感恩时。我们不再把好事习以为常。G.K.Chesterton说"你们在餐前感恩"。"但我听音乐会和歌剧前"。"在看戏剧哑剧前"。"在我打开一本书 画素描前"。"画画前 游泳前 剑击前 拳击前 走路前 玩乐前 跳舞前"。"把钢笔头醮进墨水前 都会感恩"。"感激能带来人类感受到的"。"最为单纯的快乐"。大家想想。想想你上次感激别人时。当你向别人表达你的感激时 你有什么感觉?你使对方有什么感觉?或者当别人感激你时?你会有一种飘起来的感觉 形成一个良性循环。前提是这种感激是真诚的。我们感激得不够。我们的感激之情表达得不够。不管是对我们吃的食物 还是写作。不管是对朋友 还是对家人。这是在关于感激的力量方面最好的书。这不是一本教材 但我等一下谈到的很多观点。在这本书里都有所研究。Brother David Steindl-Rast写的《论感激》。他还办了一个很棒的网站 谷歌他的名字就能查到。我们怎么培养感激?他的建议很简单优雅。"何不先审视平凡的一天?"。有什么事是你不自觉地就会去处理的?有什么事是毫不费力你就会全心投进去的?也许是你早晨喝的第一杯咖啡。让你感到温暖 睡意全无 或者是带你的狗。出去散步 或者是跟孩子玩骑背背。培养感激需要经过一次又一次的练习。直到变成第二天性。直到变成习惯 直到我们把一次的感激。变成我们性格的一部分。这是有可能做到的。 

其中一个方法就是每天都找出。一两次件 有意地专注于这些事上。不管是在餐厅喝的第一杯咖啡。还是走向教室的那段路程。还是你独自在房间里闭着睛神集中精神。听十分钟音乐。欣赏你最喜欢的歌曲。不仅花时间去变成红酒鉴赏家。还要成为生活的鉴赏家。这门课很大程度上就是要教我们感激生活。他继续写道。"感激之心能量度我们活得多充实。对习以为常的东西 难道我们不是死了一样没感觉吗?"。"麻木就如同死亡" 大家想想。斯坦福精神学家Irvin Yalom做过很多研究。研究患不治之症的病人 他找来那些只剩三个月。六个月 最多一年的人 研究这一类人。他同样也发现这些人。也有这样的感受。"我这辈子第一次觉得自己活着"。"我这辈子第一次觉得自己活着"。为什么?"因为我这辈子第一次懂得呼吸的价值"。"这么多年来 我第一次感激我丈夫"。"我妻子 我朋友 我孩子" 感激花草。聊天 我这辈子第一次懂得感激这些东西。而以前他们根本不关注这些东西。他们关注别的东西 关注困难。关注苦况 大部份都关注消极的东西。记住 在我们眼里。公车上的孩子是不存在的。在大部分人眼里。他们生命中 恋情中美好的东西。并不存在 因为我们并不感激它们。当我们对它们习以为常时 在我们眼里。它们就不存在 我们对它们的感觉得麻木的。对于这些东西 我们像死了一样没感觉 我们需要一个警钟。像患上绝症一样 才能醒悟吗?才能让我们关注我们内心或身边。美好的东西吗?为什么?为什么还要等?有很多人研究感激的价值。你们之前也看过了。 

你们看过Robert Emmon写的文章。还有加大分校Michael McCullough写的文章。我把这两个研究合总结在一起。他们的研究就是随机挑出一组人。把他们分成四组。第一组每天晚上睡觉前。都要写下至少五次他们感激的事。大事小事都行。第二组写下至少五件生活中遇到的。与别人的争吵 坏事。第三组写下至少五个。你比别人优秀的地方。第四组是对照组 他们可以随便。写下一天中遇到的任何事 测量标准是。看他们的乐观 幸福程度。身体健康程度(过去参加这个研究的六个月。他们多久去看一次医生)。他们对别人有多慷慨和仁慈。最后还有他们达成目标的可能性。他们为自己设下的目标。也就是说他们有多成功。结果最差的那组是与人争吵的那组。生活中发生的五件坏事。结果最好的那组 最快乐的那组。最乐观 最有可能达成目标。对别人最大方最仁慈 最健康的那组。就是每晚睡觉前。写下他们感激的至少五件事的那组。对身体和心理都有好处。 

这个研究在很大程度上是。他们后来做的一连串实验的开端。这本书最近才出版。Robert Emmons写的《谢谢》非常好的一本书。书中追踪了很多研究和方法。能帮助你变得更有感激之心。把感激培养成一种生活习惯。对身体的好处 包括心率变异性。它能预测我们是否能长寿。预测我们是否健康。当我们感激时 副交感神经系统功能增强。使我们变平静 从而加强免疫系统。当感激成为我们的性格 还有很多好处。所以感激不只上一种心情 也是一种性格。我们怎么培养感激?通过一次又一次的感激 我每晚都这么做。从1999年9月19日开始到现在。从1999年9月19日一直做到现在。这个研究2002年才开始 而我早就开始了。当Oprah叫我这么做时 我就开始做了。真的 从那时起一直到现在。我和大家分享一下我昨晚写的。昨晚是26号 上帝 家人。我深爱的妻子Tomush 瑜伽-我练瑜伽 David-他很棒。Sherial-宝贝(我的孩子们) 办公时间。我昨天在办公室过得很开心。这个不能说 对不起 这个也不能说 对不起。这门课有时候需要打马赛克。需要花点时间才……。这个也不能说 开玩笑的 我妻子做的汤。她做的青豆汤很好喝 我昨晚喝了。大事 小事。大事 小事。现在也有东西值得我感激。我感激Shawn Achor。这个班的首席助教。我感激他的工作 他的热情。我感激Debb。另一个班的首席助教。我感激所有的助教。他们为了你们的教育投入了很多时间。为了让你们在这门课学得幸福。 

我感激Barry 他在幕后。大家不常见到他。但这个课堂很多东西都要靠他。感激他的支持 他的谦虚 为我们做了这一切。我感激我的学生。因为没有你们 我今天就不能站在这里讲课。做我喜欢做的事。如果没有你们 我就做不了我最喜欢做的事。谢谢你们 我感激你们。这样做的关键是坚持。这样做矫情吗?绝对矫情。这样做能起作用吗?绝对能。我还会每天晚上和我的孩子David这么做。我怎么做?我问David 他现在三岁半了。"你今天有什么开心事?"他回答我 然后又问我。"你今天有什么开心事?"。这个方法是我从一个朋友那里学来的 他是手工匠人。他前几节课有来过。他来看我 有一天我去他家吃晚餐。他听到我讲课的内容 他跟我说。"我有些东西要给你看" 他有两个小孩。他们都站在旁边说着。五件他们感激的事。他们都是很小的孩子 Daniel和Maya。我看着他们 我感动得流泪了。非常温馨感人。从那时起我也和我的孩子这样做。我妻子也会定期和我这样做。她容忍我的矫情 但这样做很有用。 

因为我们不会对好事习以为常。关键是要保持新鲜感。怎么保持新鲜感?这个练习的缺点之一就是。它很容易变得一种习惯 失去新鲜感。我们找不到新鲜感 无法用心去练。我们把这种练习当做家常便饭了。关键是 Lyubomirsky说。一周做一次吧 对有些人来说。一周做一次比每天做会更好。但是这样做是有代价的 每周练一次。很难变成我们的一种性格。我们怎么保持新鲜感?每天都做。我怎么保持新鲜感?很幸运。有很多人都在研究这个问题。有些是Lyubomirsky做的 有些是其他人做的。第一个方法是变着来做。例如我写Sherial有关的事 有些东西。是我每天都写的。我女儿 我儿子和我妻子。还有上帝 我每天都写这些东西。我也会写其他事。我可以写这些事的不同方面。某一天我可以写Sherial的微笑。她才一岁还没有牙齿 笑起来很可爱。另一天我可以写。她第一次走路。变着来写 另一天我可以关注别的……。这一周我可以专注写我的工作。下一周我可以专注写个人私事。诸如此类。变化是生活的调味剂 很有效。 

第二个方法 Ellen Langer说的用心。她把用心定义为创造出新的差别。其实跟变着来练差不多。看我以前没留意过的东西。我没正眼看过的东西。这也是维持爱的一个办法。有些人说随着时间过去。他们就会适应一段恋情 其实。我们每次都能从别人身上看到不同的东西。不管是父母 还是情人。还是朋友 创造新的差别。用心去看 专注在上面。通过这个新的专注产生新鲜感。图像化 William James大楼的Steven Kosslyn。我们系的主任做过一个研究。表明孩子会在脑海里把单词形成图像。所以当他们看到……假设是母亲。他们脑海里马上形成母亲的图像。很多时候我们说这些词 如果不是有意识地话。都不会在脑海里形成图像。这就是为什么孩子的思维比我们慢。因为他们还在脑海里把单词变成图像。但大人要有意识地才能形成图像。这就解释了为什么孩子活着这么有童真。为什么他们会感激简单的东西。看到飞机都这么高兴。谈到他们在日托所做的开心事。他们活得很有童真。我们长大成人时。我们适应了 我们对这些东西都麻木了。除掉这些麻木的方法之一。就是在脑海里形成图像。当我们形成了图像。我们就能重新像孩子一样看待事物。下节课我们再继续谈这点。 

在这节课结束前 我想和大家分享一个故事。这个故事是关于一个。我认识的人中最当之无愧的积极者。我的人生模范。她教会我学会专注于事物好的一面。我外婆1915年在罗马尼亚锡盖特出生。她的童年很普通 她是最小的女儿。她有一个姐姐 五个哥哥。他们过得很好 她的很多哥哥。成为有名的拉比 小提琴家 音乐家。她是个过着快乐平凡生活的小女孩。直到1940年希特勒入侵罗马尼亚。希特勒入侵罗马尼亚 她的生活被颠覆了。她全家人都被带到集中营。从一个转到另一个 最后去到奥斯威辛。在奥斯威辛 她有十几个外甥女。和外甥 他们都被谋杀了 她五个哥哥和父亲。都在她眼前被谋杀了。战争结束时 她和她姐姐活下来了。英国人来到奥斯威辛解救集中营的人。他们当时带着一个医生。医生会一个一个检查集中营里的人。当然大部分都死了 那些还活着的人。他会说这个带走 这个留下。因为他们资源有限 地方有限。他们认为活不下去的病人。就会留下来 当看到我外婆时。她当时只有26公斤 也就是54磅。我外婆长得很高大 但她只有54磅。他说你留下 然后看到她姐姐。她就在她身旁 她姐姐36公斤 79磅。他做这个动作 说把她放到车上。当然她太虚弱了 起不来。 

所以英国士兵抱起我外婆的姐姐。她的名字叫Shanti。他们抱起Shanti 她抓住她妹妹。她不肯放手 他们想松开她的手。虽然她很虚弱。但他们无法松开她的手。于是医生说 好吧 两个都带走 所以士兵把她俩都带走了。他们把她们抱上车。他们很肯定我外婆会死去。一个月后 她活下来了。虽然她体重没有增加。他们以为她随时都会死去 但她顽强求生。三个月后 她体重开始上升 她活下来了。半年后 她回去锡盖特。看看她的家变成什么样了。到达后 她从马车上下来。一个叫约瑟夫的男人看到她。他看到一个弯着腰 头发还没长回来的女人。但他认出了那双眼睛 因为那双眼睛充满骄傲。充满生命力 他认出了我外婆Goldie。他收留了她 两周后他们结婚了。外婆恢复健康 怀孕了。生下一个死产儿 她又怀孕了。再一次生下死产儿。他们从罗马尼亚搬到特拉维夫。他们被英国人抓住。因为当时以色列是英国托管 他们被英国人抓住。被送回到塞浦路斯的集中营。在集中营里。我外婆在塞浦路斯又怀了一个孩子。这次她生下了一个女儿。后来又生下一个男孩 也是死产儿。我外婆不能再生育了。她的身体吃了太多苦。我外婆活了下来 生活得很好。 

Shanti也是 Shanti不能生育。但她把我妈妈视如己出。她把我当成外孙一样 我记得。那是1988年10月。Shanti几年前去世了。她去世那天 我们去墓地。在她的墓碑前祈祷。我和我外婆去 我们一起走。我外婆站在墓碑前。墓碑上不仅刻着Shanti的名字。还是她兄弟父母的名字。她看着墓碑。我听到她跟Shanti说。"我们真幸运 不是吗?我们很幸运。你看 Talik也来了" 她叫我Talik。然后她跟她说我干什么工作。她告诉她我兄弟姐妹在做什么工作。她们的外孙 我父母。跟她说生活有多美好。她在说美好的事。她偶尔会哭 想念Shanti。她是一个有血有肉的人 她站在那里。有时说依地语 有时说希伯来语。偶尔会跟我说一两句。然后又跟救了她一命的姐姐说两句。她跟我说"走吧" 然后我们一起走。我们周围有几千个坟墓。我们在小道上向车走去。那时是十月 很美丽 太阳西下。风在吹 树在摆动。然后她停下来 向天空望去。她站得很骄傲 很稳重。她紧抓住我的手 抬头看天 再看着我。她说"Tal 这个世界真美好"。"可惜我们都要离开" 她笑了笑 我们继续走。我外婆见过成千上万的尸体。她亲眼看着她家人被谋杀。她生了三个死产儿。她没有无视生活中可怕的事。她怎么能无视?但是同时。她也拒绝无视生活中美好的事。生活中美好的事。她心怀感激 她撑下来了。我外婆告诉我 这是一个美好的世界。我相信她 谢谢大家。 

第9课-积极情绪 

首先我想向进修学院的同学问声好。每天都有350名学生观看讲座。我想向他们问声好。我想特别邀请进修学院的同学。及在座的各位来听谦逊的Tal。他没告诉我们他上了《60分钟时事杂志》 但没关系。但周六晚上。Tal将担任女子队。荣誉篮球教练。周六晚七点对抗Cornell大学队。那是本赛季最重要的比赛。六点吗?周六晚六点 本赛季最重要的比赛。希望能见到在座的每一位。及进修学院的同学去观赛。进修学院的同学们 我有免费票给你们。有一百张票。以示敬意。我们有可敬的同学和可敬的球队。著名的Kathy Delaney-Smith 获奖的。国内最著名的教练。女子篮球队也在这里 上来。这就是可敬的同学们 上来 她们害羞。来吧 上来 见见大家。我会和所有朋友和家人去观战。Tal会在你们的手上签名。 

好 准备好了吗 

好了。 

这是最重要的比赛。她们是去年的长青藤联盟冠军。好 把礼物扔出去吧 扔出去吧。扔 快扔 快扔。好的 谢谢你们。谢谢 周六见 祝你们好运 谢谢。好 这不是我安排的 好啦。好的 身上还有别的东西吗?好。谢谢你们 也感谢你们。我们继续谈"感激"。那么……上次课结束时我讲到我的榜样。我的一个典范。比任何人都重要 一个真正的积极者。当我谈到我的祖母时 我说……。我告诉你们她并不无视无益的东西。她并不无视邪恶 坏的 生命中消极的东西。但同时 她也拒绝无视积极的东西。换言之 她坚持保持现实。现实地生活 与周遭保持联系。当周遭充满邪恶的坏的东西。她与之保持联系 与此同时。她也与好的东西保持联系。积极者不是盲目乐观的人。区别很大 我们上次提出的问题。我想以"感激"结束今天的讲座。然后开始讲"改变" 

第一是。为什么没有更多的积极者?为什么没有更多的人保持乐观?大家看到研究 人们活得更长。更快乐 更健康 更成功。为什么不是人人都是积极者?为什么我们不都是乐观主义者?如果它是有回报的。用终极货币形式 以幸福来回报。也用硬货币来 以成功来回报?为什么没有更多的积极者?问题的答案 在很大程度上。是因为媒体的原故。媒体正在放大 聚焦。放大消极的东西 使其最大化。由它占据整个屏幕 整个版面。同时最小化积极的东西。在很多方面 媒体把我们变成消极者。而我们需要反击 如何反击?宏观层面上可以收看好的新闻频道。微观层面上可以培育富有启发性的艺术。艺术在历史进程中改变了世界。不论是在黑暗年代。在向文艺复兴过度的时期。或者是在18和19世纪兴起的浪漫艺术家。致力铺设通向自由之路。亦或是上世纪三四十年代。好莱坞的复兴 带给人们更多希望。艺术扮演着非常重要的角色。希望能继续如此。然后我们谈到大的层面 宏观层面。在微观层面上。我们谈到不要坐等坏事发生。不要等悲剧。或等外界有事发生 我们才知感激。 

因为我们感激 就其定义而言。就是不以某事为理所当然。我们还分享了有关"感激"的文章及研究。做这个修习的关键。正如我所说 我每天都在做。从1999年9月19日开始。每一天 都进行信仰修习。修习的关键 不将此修习视为理所当然。是一直保持新鲜感。如果你想每周做一次。每周一次用心去做。比每天去做并将之视为理所当然要好。好的 做过了 但最理想的是每天做。因为每天去做才能养成习惯。那样才能改变思维。每天去做的话 关健是要有变化。每天思考不同的方向。思考家人的不同方面。如果你每天都给家人写信。一周讲述工作 下一周写家庭。每次都要用心。专注以前没有注意过的事情。发现新的特点 Ellen Langer如是说。并在脑中想像。当你想到你的女或男朋友。书写感激之情时 在脑海中想像他们。当你想到。刚在食堂吃过的一顿美味可口的饭时。想它 在脑海中想像 让它变得尽可能真实。孩子就是这样思考。所以孩子们每天都感觉新鲜。他们不把任何东西视为理所当然 他们视生命为奇迹。爱默生曾经说过。"如果星星每千年闪烁一次。我们都会仰视赞美这个世界的美丽。但是因为它们每天都在闪烁 我们将之视为理所当然"。 

在孩子眼中不存在理所当然。其中的一个原因。是因为他们不会自动地概念性地思考。他们的思维是感知感性的。他们和现实保持联系。通过在脑海中想像 我们就能做到。这项杰出研究完成于这里。就在William James大楼的八楼。做到"感激" 及实现即将谈到的"改变"的关键。我们会用超过两次讲座专门谈"改变"。利用今天和下周整周。关键是去做 亲力亲为 没有捷径。不是因为你们听了有关"感激"的讲座。你们理解。你们真正理解积极者的含义。那不会……。仅仅理解不会让你成为积极者。你需要去做 需要去经历。只有这样 经过一段时间。你才能开始越来越多地看到世界的积极方面。才能反击当下禁锢着。大多数人的思维方式。也就是消极者的思维。William James在1890年。说过需要21天改变一个习惯。可能有点过于乐观。可能需要更长时间 但不妨尝试21天。看会怎样 有些人。我知道有些人已经在上次课后开始尝试。有些人立即见到益处。当然益处可能会消失。然后又在六个月后出现。但尝试一下 起码尝试21天或一个月。这也是你们下周的任务。作为每周任务的一部分。 

从今天开始 不要等 实现改变没有他法。因为你所做的是一点点凿掉多余的石头。那多余的石头 那些强加给我们的限制。例如通过周围的观念。通过今早读到的新闻 大多数的交谈。还有内心的审视。以及外在的交谈等等。��所做的事。��你做这个简��的修习时。��如此有效的��因。记往经常用心修习的人更快乐。更健康 更宽容 更和蔼 更成功。修习行之有效的原因是因为。你所做的事是一凿一凿地。削走多余的石头。在我的书中 我谈到O'Hart Cummin。我的导师 在课堂上。也谈到过他几次。他曾经给我讲过一个故事。在他不比你们大多少的时候。在他20来岁时 离开以色列。他在欧洲生活了几年 最后住在荷兰。过了一段时间后 他变得无家可归。住在一棵树下面 寒冷难耐 当时是冬天。身无分文 没有一分钱 也没有朋友。悲惨极了 然而出于某种原因 他说 好吧。他抑郁了几周时间 然后他说。"我不如试试"。拿出一张纸 在这张纸上。写下让他感激的一切。 

他写下来的东西包括 他说。有贝多芬第五交响曲 他酷爱音乐。他写下在以色列的父母。他写下香草冰淇淋 至今仍是他的最爱。他想到家乡的朋友们。所有这些事 他写了长长的一个名单。写下世上让他感激的事物。他将那次经历视为他人生的转折点。为什么? 因为他开始专注于别的事物。不再对无奈可悲的处境耽耽于怀。而是专注于各种可能和美好的事物。大家也来试试。随便说一句 他现在55岁了。仍随身携带这张纸 虽已褶皱。但仍装在他的钱包里 提醒他。世上有太多美好的事物。想想我们所过的生活。就在200年前。谁有钱听最爱的音乐家的演奏。欣赏最爱的演员的表演 最爱的戏剧?谁有钱做那些事? 只有王室。即使是他们也不能随心所欲。只能看在城里的音乐家演奏。只能看逗留在城里的演员。今天这一切都在我们的指尖上。在我们的MP3和DVD播放机上。想像……想想我们的奢侈生活。国王和王后也望尘莫及。 

然而我们习惯了 我们适应了。有时这是件好事。因为我们也适应困难的经历。问题是我们如何学会。学会适应痛苦。但又不被伤害或变得容易受伤害?变得漠视我们所有的特权。如何才能保持感激之情?那就是要用心。思考我们拥有的美好的东西。不论是在朋友身上。在一部我们想看的电影中。或是过一会。食堂提供的午餐。感激修习如此有用的一个原因。因为我们所做的是共同创造一个现实。我们问得最多。或被问及最多的问题是"哪里不对了?"。"有什么需要改进的?""我的弱点是什么?"。这远远不是重要的问题。如果这些问题。是我们唯一提出或关心的 那么好的事物是不存在的。当我们问"我要感激什么?"时。即使每天只问一次。这么做的本身创造了……祝你身体健康。这么做的本身创造了一种新的现实。让我们开始看到曾经无视的东西。我在做这样的修习时。我已经修习了很长时间。今天我才会注意到不做修习时。忽视的事物 我会说。"今晚我要写一写这样东西。它是那么美好"。 

开车兜风时。看到笼罩在周日夜色中的Hampshires的群山。晚上我要记下它。没有修习我不会注意到它。就我而言 这些东西是不存在。就像公交车上的小孩子不存在一样。当你没有问对问题时。显然表达感激也是重要的。不仅仅对自己 也要对他人。这方面也有很多研究。这是……你们正在或已经读过。Seligman的有关向他人表达感激的论文。可以是写信表达 拜访或致电。表达感激的关键。这不是一张简单的"感谢字条"。亲爱的妈妈 感谢你如此伟大 爱你的儿子。那不只是简单的感谢字条。它需要坐下来思考。 

"我能感激妈妈什么?她都为我做了什么?这些年来 她给与了我哪些?"。认真地去思考 认真思考。"三年级的老师都为我做了什么?"。不是跟朋友聊天时说。"三年级时我有个很棒的英语老师"。坐下来认真思考 我要感激什么?她或他都为我做了什么。对我的人生帮助良多 让我成为今天的我?我的室友总是尽其所能帮我。我真得很感激 要去静思。而不是在学年或学期未。说"你很好 棒极了"。真正用心去思考这些人。你生命中的重要的人。为你做了什么 然后再表达出来。不要视之为理所当然。不要认为他们理应知道你心怀感激。"是的 妈妈当然知道。爸爸当然知道我感激他们 他们很棒"。不要视之为理所当然 表达出来。 

写信 致电或面对面说都可以。当今最有效的一种干预形式。是向他人表达感激之情 尤其是这种方法。写一封表达感激的信然后去拜访。收信人 再把信读给他们听。俗气? 同意 难为情? 有时是的。你无法想像。人们做感激拜访时获得的结果。但即使你不去拜访他们。你觉得害羞 我建议去拜访。如果你觉得害羞。那就把信寄出去再打电话。致电前思考几分钟。"我要说些什么?我到底感激什么?"。不论是对你的父母。或朋友。或是一年级老师 去做吧。或者是对你们的教练 表达感激时。是最让我们感到幸福的时候。想想吧 相当不可思议。进一步思考会发现它是双赢的。因为显然你从中得益。研究已证明了这一点 表达感激时我们感觉很好。对方也会感觉很好 他们的获益良多。于是你创造了一个双赢的局面 一个上升的螺旋。因为对方也更可能向他人表达感激。最好的方法是以身作则。成为你期许见到的改变 如甘地所说。想让他人充满感激?先以身作则表达感激。他们才可能承接下去。向他人表达感激 你不仅仅。在你和对方之间启动了一个上行螺旋。对方和他人之间可以承接上行。传递出去 传递出去。 

这种干预的不利之处是。虽然它能让幸福感达到峰值。但一个月后 幸福感会消退。时间不同 可能是一周或三个月后。但平均来说一个月后会消退。幸福感的峰值。保持这个峰值的关键就是经常去做。每周做一次 两周做一次或一月做一次。一周打一次感激电话 把它变成仪式。下一周写信表达感激。再下一周 感激拜访 以此类推。还是要有变化 形式的变化很有帮助。但把它作为一种仪式经常去做。下周谈"改变"时。我们会谈到"仪式"的重要性。因为在很多方面。那是唯一的真正持久的改变。有一项惊人的发现。是Sonja Lyubomirsky的发现。她发现如果我们写了那封信。就算不寄出去 幸福感也能达到的峰值。寄出去当然会得到回应。峰值会更高。但仅仅是书写 感受感激之情 敞开心扉。这样做的本身也有助提升幸福感 下周。你们的功课就是一封感激信。我们强烈建议你们去拜访对方。把信读给他们 或者最起码。把信寄出去 如果你们相隔很远。 

但即使是写了信。却不好意思寄出去。那同样能提升你的幸福感。我想再深入谈谈。如何表达感激 再概况谈谈。如何应对痛苦和积极的情绪。再次引用Sonja Lyubomirsky所做的研究。她邀请参与者谈论。分享生活中最糟的经历。和最好的经历。她所做的是把他们分成四组。第一组 写出来。写出三方面 一是影响 也就是情感 二是行为。即当时你做了什么 三是认知 即当时的想法。连续三天 每天一次 每次15分钟。第一组只是写出来。第二组对着录音机说。讲述最好的经历 另一组。讲述最糟的经历。第三组 只是去想。连续三天 每天沉思15分钟。她观察他们身体和心理的健康状况。她观察他们的健康程度。观察他们的身体。他们认为自己有多健康。他们有多少种病症。她还观察他们的情绪 他们有多快乐。观察期是实验前和实验后的四周。经过三天 每天15分钟的干预后。 

还有第四组 是一个对照组。她的发现如下。一共有三或四组 每组作两项研究。书写 谈论和沉思。研究一是最糟的经历 研究二是最好的经历。书写的人 我来解释一下。书写最糟经历的人事实上感觉更好了。身体上也更健康。这是经过四周后与对照组相比的结果。 

对着录音机说的那些人 还记得吗?谈论他们的感受 他们的经历。谈论当时的想法和现在的想法。15分钟 连续三天。在谈论之后感觉更好了。那些想的人。沉思的人 不去谈论。不去书写 他们感觉更糟了。一个月后身体状况也更糟。研究二。他们书写最快乐的经历。我们看到相反的趋势 换言之。那些连续三天书写他们的经历。如何经历以及汲取的教训的人。每次15分钟。一个月后身体和心理上都感觉更糟。那些谈论这三样的人。即影响 行为和认知。一个月后同样感觉更好但身体状况更差。而那些只是想的人。沉思他们的积极经历的人。连续三天 每天15分钟想像它的人。一个月后感觉更好更健康。四周后的结果 于是她进行第三项研究去了解。这是很惊人的结果 出乎她的预料。于是她进行第三项研究去了解其中的机制。这是怎么回事 为什么。当我们书写和谈论消极经历时。我们会感觉更好更健康?增强我们的免疫系体?但当我们思考或沉思时。发生什么事?思考积极情感时却产生相反的结果。 

她的发现是。分析和重现之间是有区别的。当我们分析一次经历时。当我们理清它时 它真得有帮助。痛苦的经历和消极的经历。所以心理治疗才有帮助。心理治疗中最有帮助的 除了技术。除了心理治疗师的多年学习。除了心理治疗师的经验 这些都很重要。但不是最重要的 最重要的是他们的感同身受。换言之 他们是不是好的聆听者?当我们觉得可以谈论或者分析痛苦情绪时。我们感觉更好 身体更健康。当我们只是坐下来。沉思痛苦的情绪却不去理清它时。我们经常进入下行螺旋。我们收窄压缩 感觉更悲哀。进一步收窄压缩。还记得Barbara Fredrickson的研究吗? 

不断恶化 相反 虽然不清楚原因。当我们分析积极的经历时 真正地分析它。尝试理解为什么会发生等等。连续这样做三天。它确实没有帮助 我们不明白为什么。也许是因为让我们回忆起。经历的快乐之处。但当我们只是沉思积极经历时。只是想也会提升我们的幸福感。感激修习就是重播快乐经历。就是如此 换言之。就是说回想最快乐的日子。能自动导致积极的结果。那么准许为人说的是什么?有什么区别?因为准许为人也是为了。实现长期的快乐和健康。区别是这样的 准许为人。我们准许自己感受各种情绪。只要有必要。失去亲人时 我们准许自己感受情绪的时间。远比考试失利的时间要长。但我们准许自己感受经历。然后我们自问。"现在最有效的行动是什么?"。 

最有效的行动之一。是去分享那次经历。最有效的行动之一。是写日记 接受每天的行为。不要苦苦沉思痛苦的情绪。我思考的一件事。也与祖母的经历有关。为什么会那样 一会我们会谈到PTSD。为什么会那样。从越南回国的很多美国人。30%的越战老兵都患有严重创伤后遗症。那影响了他们整个人生。但经历过大屠杀的人。同样是可怕的经历 经常是更可怕的经历。经历过大屠杀的人 就百分比来说。患有PTSD的要少得多。严重创伤后遗症 

为什么会有这种区别?我知道在以色列 50%的老年人。在我出生时 50%的人口都是大屠杀幸存者。我没看到有很多严重创伤后遗症患者。但30%的越战老兵都有PTSD。现在陪审团正在调查。二次伊战老兵有多少患PTSD。数字一定也很惊人。 

为什么? 为什么有这样的区别?我认为这就是解释。越战是遭到反对的战争。很多老兵回到家。回到家乡后 不会公开谈论它。他们把经历藏在心里。他们所做的事情是 他们就在这里。他们反复回忆最可怕的经历。在脑海中反复播放。结果情况越来越糟。相反。同样可怕的大屠杀的幸存者。他们回到自己的村子。回到自己的国家。他们去以色列 在那里做什么?他们谈论。谈论他们的经历 多数人并非所有人。多数人会和朋友谈论他们的经历。和家人 不停地谈论。他们在这里 随着时间推移。这帮助他们继续生活下去。这项研究给出的最好建议 你们都知道。正如我所说。这门课你们学不到太多新东西。只想提醒你们 互助组的帮助有多大。与朋友 家人分享的帮助很大。和心理治疗师聊天 写日记 书写。 

下周我们会谈到日记的作用。两者对积极和消极经历。都是最有效的干预形式。只是以不同的形式书写。一种是分析 另一种是简单的重播。表达感激不应该等到感恩节。我们不应该只等每年一次 在11月的那天。去表达感激之情。坐在桌旁轮流说感激什么。它应该是一个习惯 首先因为它行之效 有帮助。其次因为它是道德的。这点很重要 我会贯穿始终强调。道德也是实际的。而实际在多数情况下 也是讲道德的。两者密不可分 

David Steindl-Rast。上次课我提到的那本书"我们生活在一个给与的世界。带来满足的是感激之情。心灵对此给与人生的简单回应。就是圆满"。就像一个孩子 诗人Galway Kinnel写道。"感激地活着和死去 若无其他美德"。Cicero写道"感激不仅是最大的美德。也是其他美德的发源" 为什么是其他美德的发源?我经常想起这句最爱的话 但是为什么?想想吧 如果我们不知感激。我们就会将事情视作理所当然。如果我们不知感激生命中美好的事物。认为它们理所当然 那就会对它们视而不见。对我们来说 它们是不存在的。美德为什么是高尚的? 是它的好。如果世上没有好。那么美德将不再高尚。所以我认为……。Cicero所谓的。"它是所有美德的发源"的含义。另一个重要的积极者是雷德克利夫学院学生。曾经的雷德克利夫学院学生 Helen Keller。Helen Keller。能让我们睁开双眼感激周遭的事物。她在精彩的自传中讲了一个故事。关于一个曾去剑桥看望她的朋友。当时还有很多树林。那位朋友在树林里散步。朋友回来时 Helen Keller问她。"你看到了什么? 留意到什么?"。她的朋友回答说"没什么特别的"。 

Helen Keller在自传中。对"没什么特别的"作出回应 我来念念。"我奇怪怎么可能在树林里。走了一小时后 却什么都没看到。我虽看不到却发现许多东西。精美对称的叶子。银色白桦的光滑树皮。松树粗糙的树皮。我一个瞎子也能给看得见的人一个提示。像明天就会瞎掉那样用你们的眼睛。听声音的乐曲 鸟的鸣唱。一个管弦乐队的强劲弦律。仿佛明天你会突然聋掉。像明天就会失去触觉那样触摸每样东西。闻花朵的香气 品尝每一口的气味。仿佛明天你将永远失去味觉和嗅觉。充分利用每一种感官。每一面的美好 愉悦和美丽。这世界向你展现"。每天两次花一分钟时间留意周遭的一切。花一分钟的时间。在上课的路上看看美丽的草地。青翠的树 美丽的雪。晚上用一分钟去回忆。回想你度过的一天。写下让你心怀感激的事物。今天午餐时。略比平时吃得慢些去品尝。体验品尝的滋味。 

因为我们所吃的食物是种特权。朋友是我们的特权。家人是我们的特权。我们不应视特权为理所当然。因为我们不知欣赏的就会贬值。我们不需要经历一次威胁 一出悲剧。才去感激周遭和内心感受的一切。时刻提醒我的 我的周围环境。时刻在我眼前提醒我的。是我家人的一张照片。特别是中间的祖母。祖父看到她弯腰走进来。病倦不堪 他把她揽在怀中 我办公室里放着这张照片。经常看它来提醒自己。我本可以现在给你们讲这个故事。原本安排在讲座最后。但我知道讲完故事 我会心力憔悴。我将无法继续讲座。所以我会在讲座最后讲这个故事。但这是给了我一个提醒。它激励我创造一个积极的环境。激励提醒我要作一个积极者。 

接下来我想做的是。现在我想占用一点时间。本学期第一次占用时间 不会是最后一次。占用一点时间让你们有机会审视自己。去反省。分组时也有机会。我想现在教室里做。我要你们用几分钟时间回想。在脑子里记下来你所感激的事物。生命中美好的事物 给你们两分钟时间。计时。好 希望大家课后能继续。现在我想做另一件事。有点让人难堪的事情 我先表示歉意。但我想让大家现在。和你旁边的同学分享。不必分享所有事。可以隐去一些事情。你想有保留 完全可以。用几分钟时间分享一下。和身边的同学 如果是三个人。简单快速地分享一下。我给你们两分钟时间。分享你们感激的事物。可以详细说一件事也可以读你列下的清单。 

(学生在分享)。 

再给你们30秒 30秒收尾。好了 好了 大家可以课后继续。我建议大家课后继续。很简单的干预却获益良多。只要我们能专注于积极的方面。就能同共创造更多积极的现实 现在我想做的是。在开始关于"改变"的讲座之前。我想就期末报告说几句。期末报告。你们将要做的。是一次时间在20到30分钟的报告。进修学院的同学。和在校生都要做。20到30分钟的报告 而且要交上来。积极心理学中的任何主题都可以。随便说一句。也可以是用积极心理学干预。治疗抑郁和焦虑 可以是关于"感激"的。可以是关于身心联系。可以是关于灵性和宗教。可以是关于自尊。我敦促大家去做的。是找到对自身最有具意义的。你想研究的主题。记往最私人的也是最普遍的。如果你对灵性问题很感兴趣很狂热。就写有关灵性的报告。如果你想更好地理解。更好地应用于人生。把积极心理学干预用于解决 我不知道。公开演讲焦虑症或其他问题 都可以。 

报告内容对个人问题涉及的越多。你从中得到的也越多。你们要把书面报告交上来。20到30分钟的报告内容 大约要写10到15页。这是你们最终要交上来的。既要交书面报告也要交幻灯片。我们不会因你的幻灯片。有多漂亮多精彩而打分。但我们希望你们能提交幻灯片。那将作为你们报告的大纲。打分的唯一部分。是你最后提交的东西 也就是你的报告的文稿。但是除此。幻灯片是必选项。就是说必须做幻灯片。但幻灯片不会参与打分。是给小组成员作的报告时用的。进修学院的同学。将对社区的其他人作报告。 

这样做是为了传递出去。找到一个主题然后传递出去。我的演讲生涯就是这样开始的。我自己有个感兴趣的主题。我给壁球队作了演讲。我跟和我很亲近的人讲。我有了一个主题 最后发展成演讲。我们希望那些没有经历的同学。首先去综合。你感兴趣的主题的资料。那些能改变你人生的东西。然后再传递出去。必须有参考书目。这不只是个人故事或自传。必须有参考书目。当然可以加上能突出你的论点的故事。最有效率的沟通方法。大概就是通过故事。十年后 你们也许不会记得。关于自我效能的研究 但你很可能。记得Roger Bannister的故事。人们记得故事 与故事产生联系。但与此同时 这是一份学术报告。 

我们希望即有学术性又通俗易懂。所以要有参考书目。可以选择以下手段。可以截取电影片段 如果报告中有电影片段。如果是十分钟的电影片段。它不会被计入报告时间。尤其是当你截取20分钟电影片段时。除非你准备来次讲座 那也可以。要包括练习。如果这是个工作坊 你会做哪些练习?你会让。参与者相互表达感激吗?你会让他们写篇日记吗?你会让他们在课后出去锻炼身体吗?所以还要包括练习 下面是提交日期。这是……你们的最后期限是。或者说你们报告的生命线是。3月20日 在春假之前。你们只需要告知助教你们的主题。一个字 一句话 "感激"也好 "灵性"也好。是"身与心" 身体练习 都可以。通知他们一声。一两周后 你们可以改主意。我们只希望你们开始思考。你最想做的主题 那是3月20日的任务。 

4月7日 让你们享受几天春假。放假时不必操心课业。虽然我们希望你们能抽时间思考。放假期间阅读些材料。4月7日交一页大纲 你的主题的草稿。这个也不计入分数。只是为你们自己 确认是否通过。所以必须交上来 只是为了你们自己。好让你们和助教商量 5月3日之前。向至少三到四位同学作一次报告。如果你是进修学院学生 就给别人讲一次。征求他们提供意见。得到反馈意见 最好是书面的。到时我们会谈到具体的过程。这个也不计入分数 只是为了你们自己。既为你也为参与者。因为你们在传递。可以是一份草稿 就是一份草稿。我也会给出意见。"我觉得还缺一个故事。我感觉你的研究资料不足"。或者"再加入更多活力"。不论是什么 向听众获取意见来帮助你。再过一周就是期末报告的截止日。有什么问题吗? 如果你有问题。别人也可能有同样的问题。关于期末报告有任何问题吗?好的 想一想 请说。 

不 不是分组进行 其实你们会……好的。问题是你们在哪里作报告。你们会被分成组。和另外三到四个人 可能是三人一组。然后相互作报告。想用幻灯片。做完整的报告 非常好。只想读或说出来。也完全没有问题。可以选择对你最有帮助的方式。再强调一次 这个设计是为了你们自己。是为了你们自己 希望那些。听你报告的人也能从中受益。做过这个设计的学生们。过去两年我们一直有这么做。我教的第一年只提交了普通报告。学生们非常喜欢这个过程。从中获益良多 他们保留了这种作法。因为你可以用这个方式。在修完这门课后继续向外传递 还有问题吗? 是的。塔尔宾夏哈尔博士。是的 很好的问题。 

是否要收窄到一个主题。而不是笼统的人生灵性体验?这完全看你们 你可以先概况。在报告结束时再说。有一些很有趣的研究。如果你们想了解更多。可以去那里找找 也可以做些特别的报告。比如去教堂的好处。或有氧锻炼的好处。可以专注于一个主题也可以更笼统。比如有氧锻炼的心理益处。即可以专注于一个主题也可以很概括。由你们决定 只要你觉得更感兴趣。我们经常会发现。当你们着手准备讲座 查资料。阅读很多研究时 那时你们才会明白。"仅就这个问题 我就有很多资料。我准备专注这里"。或者我很想概述整个主题。然后在后面的报告中。或者我会在以后专注这个主题编一堂课。或者网站 很多学生。根据他们的报告开设了网站。那里还有人举手 或者你只想问候一下?是的 他们听不到你说的话。世界各地的进修学院的学生。你们可以在任何地方选两三个人。家人 朋友。任何人都可以 但要是你的"美丽敌人"。他们会给你真实的意见。因为他们关心你 好了 还有吗?好的 还有问题的。可以问你们的助教或问我。祝你身体健康。 

改变 可以说。整门课都在讲改变。这门课就是关于"改变" 我在第一堂课就说。如果我认为人不能改变 我就不会上这门课。我说到改变时 我说的是不同的层次。可以是培养一个习惯。我想每周锻炼三次 为什么?因为几周后 你们会看到。每周锻炼三次的效果。和很多有效的心理药物相同。或者你想培养的习惯是。我希望人生的精神层面更丰富。改变也可以是"我想更快乐"。我希望不要那么焦虑。一般的焦虑或者特定的焦虑 例如考前。不论你希望生活有什么样的改变。总有一定的模式和方法行之有效。另一些方法行之无效。未来两堂课我们要做的是。两堂半课 是区分哪些行之有效。哪些行之无效 以便让你们将之应用于生活。 

首先 关于改变 我们谈过很多。让我们来回顾一下曾经讨论过的问题。首先 在第二堂课上 我们已经提到。改变很难。很多研究都证明了改变有多难。双胞胎研究 还记得结论吗。尝试改变快乐水平。如同试图改变身高一样徒劳。虽然他们后来也说。"不一定是不可能和徒劳的"。但他们的研究证明了很难做到。双胞胎研究 快乐是一种随机现象。还有Daniel Gilbert的关于有效预测的研究。在哈佛得到终身职位 或考入哈佛后。我们的幸福感上升 当我们遭到拒绝时。经历下行。但很快我们又恢复到幸福感的基准线上。人生的多数时候。幸福感都是在基准线上下波动。获得上升抛物线很难。通过去面对可能做到 但很难。 

剑桥-萨默维尔研究表明。五年的干预 最终导致消极改变。即使有最好的心理学家 心理治疗师。社工和各种计划进行干预。实验人群中还是有更多人酗酒。我们知道改变很难。不论是个人还是社会层面。然而我们也知道改变是可能的。我们知道很多人争辩说。改变是不可能的 他们犯了普遍性的错误。只看平均分而不看例外情况。不看那些实现改变的人。当我们研究改变时 是例外证明了改变是可能的。如果改变是可能的 有些人实现了改变。或者是通过治疗 读一本书。参加学习。或者是通过交谈。有时只需要一句话。就能改变他们的整个人生。他们是例外。但问题不再是改变是否可能。问题是"如何才能实现改变"。这个问题我们将在讲座中。更深入在更高层次上讲。 

改变是什么样的? 我们从微观上说。去到大脑的层面 改变到底是什么模样的?改变发生时 我们的大脑有什么变化? 直到1998年。神经科学家认为大脑是不变的。我们生来具有一定的神经元 一定的通道。大脑在三岁之后不会生长 不会改变。开始时大脑可能会改变发展。但三岁之后 不会再改变。这个理论直到1998年仍站得住脚。也就是在不久以前 这个理论让人相信。并实际证明了。"快乐是随机现象"的正确。你生来具有一个的基因。生来具有一个的倾向。具有一定的神经元通道。有一些经历对你产生深刻影响。正如弗洛伊德的论点 而剩下的人生。就是围绕那个水平波动。不会改变 

直到1998年。科学家开始发现大脑其实是改变的。并得出神经可塑性的概念。神经可塑性是指神经元有可塑性 它们会改变。不仅如此 不只大脑的通道会改变。一会我们就会看到。他们还得出另一个概念。即神经形成 神经元是发展的。在人的一生中不断生成 直到我们死去。因此 事实并不是。我们生来具有一定的神经元。然后就一直走下坡。传统的认识让我们相信这种说法。今天仍有很多人相信这种说法。不 神经元是发展生长的 在人生过程中不断出现。原来大脑在很多方面很像一块肌肉。使用它就不会失去它 使用它能让它再生。让肌肉更发达 问题是。要如何使用它 才能让它帮助我们。变得更加快乐。我们来看看大脑到底是什么样子的。 

大脑里有数以百万计的通道。神经通道 不同神经元在大脑里的连接。看上去就像这样。你们能看到有些很薄 是新生的。有些更厚 更健全。它作用的方式很像自然界。即有大河 宽敞的渠道。也有小河 狭窄的渠道。原理是这样的。每次神经元之间发生联系 换言之。你的思维模式有特定的路径运作。那个路径会成长。像河流一样 每次有水流过。河流都会变得宽一些。没有水的时候。神经元不工作时。打个比方 它会萎缩一点。使用时神经通道得到扩展。不使用时就会萎缩。如果一条新的神经通道刚刚生成。因为神经是有可塑性的。神经通道不断生成 因为有神经形成。新的神经元不断生成 开始时很薄。举个例子 我学法语 学了一个新词。我的大脑里形成新的联系。如果我只听了这个词一次。神经通道会分裂 会消失。 

如果我不断重复听到那个词。最后通道会变得越来越厚。我说的比实际过程简化得多。但它会变得更厚。我就会一生记得这个词。就像形成了一条新的河流。不再是涓涓流淌的小溪。可能在某一天消失。这是我们理解改变思维的。关键方面。并最终提升我们的幸福感。神经通道会自我巩固的 像河流一样。想一想 下雨时 下倾盆大雨时。水流向已有的河流。流向渠道 流向我们建好的沟渠。如果之前什么都没有。那么形成一条河流或小溪会更难。我们的经历也更倾向于流向。已建成的神经通道并进一步加固它。而不是生成一条新的神经通道。所以如果我们想记住什么东西。和其他东西产生联系是个很好的方法。和已有的神经通道 已有的记忆产生联系。 

让我再解释一次。理解这一点很重要。理解这一点 其实……。这是Carol Dweck 斯坦福的教授的研究结论。理解神经可塑性。理解我们如何实现改变。能让我们更容易成功。能让我们更加快乐 理解这一点很重要。已建成的神经通道吸引更多活动。从而变得更厚。没有建成的通道。一条小溪更容易消失。 

当它变得更宽 就更可能保留下来。不仅保留下来而且能继续生长。它是自行巩固的 习惯就是如此。一种做法被反复巩固。就变成一种习惯 比如打网球。你反反复复运用正手击球。开始时你要考虑怎么打 必须集中精神。手腕翘起来一点。知道球拍应该停在哪里。但一段时间之后 挥过几百几千次后。你不用再考虑怎么打。它已经刻成槽 我使用"刻成槽"这个词。因为事实上你创造了一条新渠道。只有有球飞过来。你不需要再考虑。它被吸引。你的思维被引向那个或那些特别的通道。很多通道 告诉你"举起球拍击球"。完全是自动的 音乐也是如此。经常练习演奏的人。或经常练习一种运动的人。他们的大脑真的会改变形状。更多神经通道在那个区域形成。"让我演奏C调"。"让我的手指这样移动"。根据我看到的乐章。特定的神经通道被创造出来。大脑在那些区域变成更厚。更多经验流入 做得越多 经验流入的越多。直到它变成沟槽。让我不需要考虑就能演奏降C调。完全是自动的 

他们做过一项研究。这是做过的首批研究中的一项。让他们发现大脑其实是改变的。他们让伦敦的出租车司机或新手司机。他们如果想获得执业证。他们的许可 他们必须学习伦敦地图。伦敦地图。比纽约地图复杂得多。去过伦敦的人都知道。但他们花很多时间研究地图。他们发现司机们的大脑。他们的视觉皮质的一部分发生变化。在司机学习伦敦地图前后。因为他们在使用那部分大脑。他们有经验。这些经验在大脑中形成沟槽。当有人说"牛津广场"时 他们立即知道。要走哪条街才能去到牛津广场。他们的大脑变得更大。那改变了他们的神经可塑性 那是个好消息。是非常好的消息 

那意味着我们可以控制。我们可以引入改变。在大脑中既有健康通道也有不健康的。比如说。一些消极的通道 一个总是忧心重重的人。一个事事担心的人 只要有事发生。他们立即转向。"那对未来意味着什么?"。"那对我的期望意味着什么?"。"那对今天意味着什么?" 时刻担心各种事情。甚至经常为好事担心。他们会立即把好事理解成需要担心的事。或者发现缺陷 消极思维会通过那条大脑通道。再说一次 通道是自行巩固的。我会通过那条通道不断寻找我的缺陷。 

我还记得自己的亲身经历。当我获得本校的奖学金时。我得到了奖学金 但我的大脑立即开始想。"但我为什么没得到另一个?"。我有那么多事情值得感激。但作为消极者 神经通道。经验流向的河流是条最大的河。当时对我来说 最宽的河流就是消极的。也有积极的渠道。积极的渠道是什么? 积极者。一个即使在困难环境也能发现恩惠的人。也许不会说塞翁失马 焉知非福。但他会从发生的事情中寻找恩惠。乐观主义者。乐观主义的渠道更大更宽。经验流向那里。今天我们对大脑的认识更多。这些渠道存在在哪里? 

我给大家举个众所周知的例子。我们都知道。经常使用左侧前额皮质的人。和经常使用右侧前额皮质的人相比。要更快乐 更易受积极情绪的影响。对痛苦情绪的适应性更强。前额皮质这部分的活动更多。与右侧前额皮质相比。右侧活动更多的人。和左侧活动更多的人相比要更压抑。今天我们知道这些 是因为有功能性磁共振成像。因为我们有脑电图。能看到大脑两侧的运作。但几百年前 我们已经知道这一点了。我们知道这边与积极情绪有关。前额皮质这侧的活动越多。与痛苦情绪相关联。 

怎么知道的? 因为人们发生意外时。意外发生在他们身上。左侧前额皮质受损时。他们经常变得更抑郁。因为右侧的活动更多。相比而言。另一方面如果意外发生在右侧。右侧受损 人们经常会变得更快乐。我不建议以此为干预方法。不管有没有父母在家 都不要在家尝试。但是如果走在街上 碰巧摔倒。最好往这边摔 大家记往了。需要记住的更重要的事是。相同的刺激往往造成不同的反应。世界不只是外在世界。还有一个内在世界。记往对同一个头脑来说 正如爱默生所说。同样的世界……对不同的头脑来说。同样的世界即可能是天堂也可能是地狱。要看那种经验流向哪里。多年来 我变得。更像一个积极者 通过修习。通过沉思 通过书写 写日记。通过身体锻炼 我变成了积极者。为此我付出很多努力。 

我人生最大的财富。是我娶了一个完美的积极者。我的妻子Tommy 她不需要后天的努力。举个例子 我们参加一次派对回来。派对上有个人一个劲地说话。你们遇到过吗……一个说个不停的人。我正要和Tommy抱怨。说个人就是个话痨。她在我开口前说了什么? 她说。"你看那个人多好。他对所做的事是如此热情?"。在我眼中多话 在她眼中是热情。她自动地发现恩惠。这里有基因因素 我们会谈到基因。让某些人比其他人更善于发现益处。但与此同时 我们可以加以培养。我们可以一条通道一条通道地塑造。我们所做的就是改变大脑。我们会在讲到"念"和"冥想"时讲这个问题。冥想 比如经常做瑜珈。能改变我们大脑里的通道。让左侧相比右侧更活跃。让我们更易受到积极情绪的感染。对痛苦情绪有更好的承受力。 

我们可以改变大脑 利用神经可塑性和神经形成。下周我们会谈到两种类型的改变。第一种是渐进的改变。这种改变是一点点凿掉多余的石头。这种改变。就像自然界的水冲刷石块。历经时日 让石块变得更薄更光滑。渐进的改变 这是世上最常见的改变。这是健康的改变 没有捷径 需要时间。然而改变的过程可以和结果一样让人享受。想想学习演奏乐器。它需要时间 是相同的过程。因为我们在大脑里创造新的神经通道。当我们学习演奏乐器时。我们享受十年的学习过程。最后在Leverett学生社交室里举办的音乐会上演奏。 

我们可以享受最后站在台上演奏前的乐趣。所以改变本身也可以很有趣。我们即能享受过程也能享受结果。它需要时间 很多时间。也有实现改变的剧烈方法。如果用水冲刷石头。来比喻渐进的改变。可以用拿大锤。砸石头来比喻剧烈的改变。就像摩西那样 剧烈的改变。不需要太多时间 立即发生。但要记住的重要的一点是。剧烈的改变不是一步奏效 剧烈的改变。经常需要很多准备功夫 打个比方。你需要很多力气去挥大锤。还要有力气砸下去。所以也需要时间 两者都需要时间。记往这一点很重要 为什么? 

Martin Seligman说过。"相信有捷径通向满足。绕过个人力量和美德的训练。是很蠢的 它导致很多人。在坐拥巨大财富时感到抑郁。精神饥渴而死"。这是当今世界不快乐的。一个主要原因 我先提示一下。我会在结束前多讲一些。有多少同学读过。Stephen Covey的《高效能人士的七个习惯》。好 不少同学读过 很好的一本书。就算不是最好也是其中之一的自励书。Stephen Covey在书的开头。谈到他的博士论文的研究课题。他所做的是。看过去200年来成功学著作。他发现1930年是个分界线。1930年之前 19世纪和20世纪初。自励在于性格改变。 

从内在改变自己。需要努力挣扎 跌倒再爬起。经历苦难。一步步地改变 缓慢渐进地实现。1930年剧烈改变 从性格改变变成一步奏效。想发财。如何结交权势 现在就做。秘密就是 快速改变 立即改变 轻易的改变。1930年代以来。我们看到人们幸福感不断下降。人变得更加抑郁 更加焦虑。其中一个原因就是。人们希望相信他们能发到一步奏效的办法。没有一步奏效的办法 需要时间去实现改变。但改变的过程可以是很享受的。如同最终实现改变一样乐趣无穷 令人兴奋。下周我们就会谈到这个美妙的过程。祝大家周未愉快。 

第10课-如何去改变 

早上好 今天要讲的是变化。今天很多事都会发生变化。我们之间的关系会发生变化 敬请关注啊。 

上节课。上节课结束的时候 我们谈到很多种不同变化。第一种方式 是渐进式的。好像滴水穿石。要花时间的一点一点 慢慢的。第二种方式 是突发式的。好像用大锤开山劈石。凿出一条路 一个隧道。对于这两种变化 或是所有形式的变化。必须要注意的是。我们讲的重点是要将变化持续下去。而不是突然一变之后又恢复原状。因此要注意的是。变化既不是药到病除的灵丹妙药。即便当我们举起大锤。就是这举起来的动作。也需要我们在之前做很多准备的功夫。因此 举例来说 渐进式的变化。包括每天都做感恩练习。慢慢成为一个更容易找寻美好的人。逐渐的看到更多积极的事物。一点一点的构建起新的神经通路。 

还记得我们之前说过的两个重要概念。神经塑性和新生神经元吗?因为通过了解和理解我们大脑的变化。我们自身也将更容易产生变化。Carol Dweck做过相关研究。我们之后谈到完美主义的时候还会提到她。另外大锤的例子和探索经验是一回事。真知灼见不是凭空产生的 而是经过。长时间准备后千锤百炼而得来的。大家都知道所谓的灵感来自于99%的汗水。所以说 没有灵丹妙药 而且。正是那种对灵丹妙药的笃信和渴求。导致了今天抑郁症如此高发频发。因为大家都很郁闷 很失望。发现灵丹妙药不管用的时候。就觉得是自己出了问题。我在书里也写了 很可能他们认为。"如果做完这五件事 我的人生就会美满了"。结果五件事做完 我却没有变得更快乐。于是我就开始怀疑自己。没有灵丹妙药 任何事都需要时间。没有所谓什么"幸福生活五步走"。 

在我们讲变化过程之前。我们先要理清一些概念。首先我们要问的是。我 或者你 真的想要改变吗?我是不是真的想要改善我的个性。或者我不喜欢的性格或行为 。这不是一个无关紧要的问题。也不是象征性地问自己一下。因为表面上我们可能会说当然了。但是潜意识里却有东西在阻挠我们。 

我给大家举个例子吧。早在80年代的时候 Langer。和Thompson曾做过一个实验。这个实验是这样的。他们找来一些学生或参与实验的人。然后问这些人。他们想不想要摆脱自己性格的某个方面。比如说古板 或者容易轻信别人。或是冷酷 他们问这些人。想不想要摆脱这些性格。想不想在这方面变的更好。是不是真的能改变自己 即是说。如果改变对于你来说真的很重要。那你能不能最终变得不那么古板。不那么容易轻信别人 不那么冷酷。这就是他们最初问的两个问题。想不想改善 想不想改变。第二个问题 能不能改变。然后 在他们问了这些问题之后。实验还有第二个阶段 在这个阶段。他们让这些人去评估。评估下列正面性格 比如。言行一致对于你来说是否重要。值得信赖 是不是重要。以及 被认为是一个严肃认真的人。对于你来说有多重要。下面我们来说这个实验的有趣的结果。所有给这些正面性格打高分的人。这些黄色的是正面性格。认为这些性格很重要的人。反而比较不容易去改变自己的负面性格。大家明白这其中的奥妙么?也就是说 比如 。我想要摆脱古板的个性。我不喜欢自己那么古板。但是 与此同时。言行一致在我看来是非常重要的品格。我反而不容易去改变自己的古板。因为在我的意识里 它们是相互关联的。对 我不想要古板下去了。同时我潜意识里有个小人却在说。我想要言行一致。别摆脱古板 因为我把两者结合起来了。我把两者联系在一起了。因此不想舍其一。因为我不想把婴儿和洗澡水一起倒掉。对于我来说两者是紧密相连的。同理 还有轻信和值得信赖。这两种品质 可以说 值得信赖的人。另一方面很容易轻信别人 这种性格。极端化以后 人会变得容易轻信别人。然而 我却不想丢掉轻信别人的习惯。因为同时我不想丢掉我值得信赖的优点。同样的例子还有冷酷 我的冷酷。可能在潜意识里正反映出我的严肃认真。 

很多年来 真的是很久以来。我在思考分析 并写了很多。关于完美主义的东西。我不能理解为什么我自身的情况。长久以来并未好转。因为我知道自己深受完美主义其害。我读了研究报告 反省个人经历。我知道完美主义对我百害无益 有损于我的健康。从长远来看 也阻挠我成功。但是我却摆脱不了它 直到我读了这个报告。于是 我问自己。'OK 与完美主义相对应的是什么呢'。在我的意识里 它和什么联系在一起?完美主义和什么密切相联呢?和它密切相联的是动力和雄心。如果我需要自我定义的话 首先要做的。定义就是我有动力和雄心抱负。同时因为我不想失去这些品格。我的潜意识便阻止我。摆脱对完美主义的追求。我对完美主义的定义是。对失败的极度恐惧 失败会阻止我们。我们会用整整一周来讲完美主义。到时候会讲的更深入一些。只有当我明白了这两者之间。的密切关联。我才能够把它们拆开 分开。我只想保留其一。再比如 担忧和焦虑。其实我已经说过很多次了。我最终摆脱了焦虑的困扰。下次我会和大家说说具体的过程。 

自从很久以前开始。我上大学的时候 每当有壁球比赛也好。或者是考试也好。或是在分组讨论的时候发言之前。我都会非常焦虑。我不想再这样下去了 我不想要焦虑。然而 我却丢不掉这种担忧和焦虑。因为我同时很重视责任感。如果我会焦虑 说明我有责任感。我会在上课前更认真的做准备。而不是随便糊弄 偷懒。因此我把焦虑和正面品格责任感联系起来。还记得关于Brandon的故事吗。责任感是非常重要的品质。但是它却阻挠我摆脱担忧和焦虑。这都是在潜意识里发生的。我之前并没有意识到这些。 

还有 罪恶感。有好的罪恶感 也有不好的罪恶感。我不想摆脱罪恶感 因为。我不想变得对他人冷漠无情。我们经常会把这两者联系起来。还有一对概念 是精简化。这也是我的切身体会。我想要减少工作量 因为。我明白同时参与太多活动 有损于。有损于我的工作效率 创造力和健康。但是我却没办法做到精简。为什么呢 因为我将它和失去优势联系起来。所以我没办法做到精简。还有我们上周谈到的寻找过错。为什么人们不能停止寻找过错呢?为什么要维持悲观的态度呢。因为他们把寻找过错和现实主义联系起来。我不想放弃这种现实的态度。不想变成一个脱离现实盲目乐观的人。因此我不会放弃寻找过错的习惯。 

幸福。通往幸福路上最大的障碍。就是人们把幸福和偷懒联系在一起。因为当今文化中最为人所追捧。所信奉的真理就是"没有付出就没有回报"。如果我现在很快乐 就意味着我不再经历痛苦。就意味着我放弃了。意味着我不会成功。意味着我已经失去了棱角。意味着我不再有动力和雄心。所以我们在潜意识里选择不快乐。这样的话 我们就不会失去其它珍贵的东西。比如雄心壮志 比如长处优势。比如勤奋刻苦。当然 依我们现在来看。我们知道这些东西并不是密不可分的。正相反 比如说 快乐。我们从"拓延-建构"理论中学到。正面的情绪是和成功联系在一起的。等到春假之后 当我们讲到完美主义的时候。我会讲一些。从研究中得来的理论。大家可以把这些理论和你们个人联系起来。你们中间有很多人会发现可以。将其和自身联系起来。如果我们能赶走对失败的恐惧 学会面对它。学会冒险 从失败中学习经验。将失败看作是未来成功路上的垫脚石。这些都不意味着我们会相应的。失去动力和雄心抱负。克服对失败的恐惧 或担忧焦虑。或罪恶感 或种种不幸福的关键。通常 不一定总是 而是通常是关于理解。或更好的去理解这些品格的含义。 

举个例子。要理解 我想要摆脱对失败的恐惧感。不是一般的恐惧 而是消磨人意志的恐惧。因为每个人都在某种程度上害怕失败。但是要摆脱这种消磨人意志的恐惧。同时保存我的动力和雄心。就必须要理解两者是密不可分的。又或者 你们知道。我以前不会对人说不。这么短小的一个词 简单的一个词。有时候是那么难以说出口。为什么?因为我眼中的自己和我希望。别人眼中的自己是有同情心的。善解人意的 是老好人 于是。我把两者拆开了 它们并不一定要。连在一起 我可以善解人意的说不。因为很多时候当我对别人说好。我其实同时在对自己说不。从长远来看 也是在对亲密关系说不。所以现在我更好更深入的了解了。我什么时候能说不 什么时候不想说。同时 我还能保留自己的善解人意。和同情心 也不必在每次说不的时候。感到有种罪恶感。这里列出来的每种品格。都是可以被拆开的。大家可以想想有哪些品性。是你很久以前就想改掉 却改不掉的。对完美主义的追求?还是严酷无情?还是你想变得更顽皮活泼?也许你并不想摆脱这种个性。因为害怕自己会失去严肃认真的品性。其实你根本不需要失去。把婴儿留下 把洗澡水倒掉。前提是我们能搞清楚。我们到底想改变什么。 

下周的小组讨论活动 你们要和助教。一起做一个叫完成句子的练习。下面要说的正是你们在练习中要做的事。你们要识别……。这个练习是Nathaniel Branden设计的。它将帮助你识别。自己分别想要摆脱和保留的品格。很多时候 不管在显意识还是潜意识中。都有一个你自己制作的开关。可以引发洪水 打开通往新隧道的大门。大脑里新的通路 真正永久性的变化。我们现在要讨论三个截然不同。却又相互连结的通向变化的途径。这就是我们之前提过的 心理学ABC理论。ABC理论 A代表情感 情绪。B代表行为 行动。C代表认知 思维。我们今天要做的是。我们会分别讲这三个元素。在分别讲这三个元素的时候。我们会谈到渐进式变化和突发式变化。可以说 我们用两种方法解释三元素。情感 行为 认知 乘以渐进式和突发式。一共有六种变化的方式 这六种方式。彼此相通 把它们联系起来很重要。我们希望能三者兼得。为什么呢? 引用一句我们提过多次。以后也会再提的名言 Dryden和Poet说的。"我们养成习惯;习惯造就我们"。为了能改变习惯 产生永久性的变化。我们不仅需要一种稳定的变化模式。也同样需要一种外部介入的力量。仅仅把注意力放在情感上是不够的。放在行动上也是不够的。放在思维上也是不够的。我们要对三元素同时下功夫。A B和C三元素。大家可以想想看。就好象 我们把习惯看成一股洪流。一股我们思维的洪流 这股洪流是。由来自不同通路的神经元构成的。我们现在要做的是对抗这股洪流。为了做到这点 我们需要足够大的外力。因此我们要尝试尽可能多的方法。 

在我们讲A元素之前。我再问一个问题 我们想改变什么?我们能改变什么?不能改变什么?以Lyubomirsky Diener等人为首的。研究者宣称 快乐是由三部分构成的。在我们试图分析一个人的幸福感时。我们需要研究三个因素。第一个因素是遗传排列。不是遗传点 而是排列。我们都生来具备一些。特定的快乐和健康倾向性体质。有些人生来就是。一张笑脸 嘴里含着"笑汤匙"。另一些人则不然。我们都排列在这条钟形曲线上。有些人注定比其他人更幸运。我之前也提过几次 我就没那么走运。我一出生就在曲线上偏焦虑的那一端。嘴里也没含着快乐的小汤匙。但我的天性令我有资格教这门课。因为我通过努力和各种尝试。改变了自己的人生。我可以谈个人的经验。因为我有过那样的经历。现在有些人可能会说。"我们还有遗传排列 真是太可怕了"。对此 我想说 这不是什么好事。也不是坏事 这只是既定事实。就好象万有引力 无所谓好坏。想要支配天性 必先服从天性。我们要认识自己的天性。理解 并充分善用这些天性。 

我们健康中和快乐相关的指数差异。有50%并不是由遗传所决定的。当然遗传基因是有影响的。比如 双胞胎研究的案例。为什么有很多对双胞胎。在长期分离的情况下。彼此之间仍有那么多相似之处。因为遗传决定的 这作用无所谓好坏。也不是百分百由它决定 感谢上帝。只有差异中的50%。我们应该随时牢记 对于如何运用。这些基因 我们是有很多的支配能力的。你可以说成功是……我只是随便举例。我并不知道确切的数据。成功的篮球运动员。50%是由遗传基因决定的 比如。肌肉里有多少快肌纤维和慢肌纤维。或者 能跳多高 协调性如何。身高多少 这些都很重要。这些都是由基因决定的。但是 如果迈克尔乔丹从没练过篮球。我应该能比他打的好。如果他从没练过的话 而且我的确练过。换句话说 在遗传作用的前提下。不去努力争取幸福的人。即便他们有最好的遗传基因 也不会比。基因不如他们但努力争取的人快乐。所以说 后天努力至关重要。 

第二个对快乐差异指数。有影响的因素。是外部环境。当然了 外部环境很重要。我们生活在一个自由的国度。还是受人压迫 带来的结果是完全不同的。但总的来说 外部环境只要不过于极端。比如无家可归流浪街头 其实影响不大。在差异方面只有10%的作用 因此。遗传作用50% 外部环境包括居住地。包括收入 天气等等。另外这个不包括。有季节性情感异常的患者。对于他们来说 有没有阳光太重要了。我现在说的是大部分人 不包括有。季节性情感异常的人 因此除去极端案例。外部环境的作用不大 大概是10%。 

第三点。剩下的40%是由意向活动决定。意向活动包括我们的所作所为所想。我们对世界的诠释 和我们关注的焦点。简而言之 就是这门课的内容。包括我们从第一节课开始讨论的内容。以及下半学期我们将要讨论的内容。这些意向活动 我们关注的焦点。其实就是我们的ABC。这也是产生有意义的变化的根源。是我们将要关注的重点。如果我们能改变遗传基因 当然很好。是很好 但是不可能。如果我们能更好的控制外部环境。当然很好 但是即便是我们能控制。结果也不会被改变 虽然在场的。或是在家看视频的人中 有很多的确。能更好的控制外部环境。能改变结果的 能被我们所掌控的。是内部活动 是我们对世界的诠释。和我们的行动 好 现在我们就开始讲。这部分的内容 

A 情感 情绪 是一种连接。更多的是从逻辑上 语言上。将情感 动力 动作连接起来。情感 感动 移动 没有情感。我们将寸步难行 你们在书中读到了。或者之后会读到 关于Elliott的故事。Damásio做了关于他研究。Elliott失去了情感功能。并随之丧失了行动做事的所有动力。尽管他还有认知能力。我们要行动就必须要情感。现在我要从两个方面谈情感。首先 是渐进式变化 之后 是突发式变化。渐进式变化是一种专注冥想。专注冥想可以带来安宁沉静。被认为是最有效的疗法。有很多与其相关的研究。我们之后会用一周的时间讲专注冥想。今天我只想做一个简单的介绍。讲一下专注力的螺旋曲线。 

从许多方面来说Jon Kabatt-Zin。在这个学科领域做了很多重要的研究。他和另外一些科学家 包括。Tara Bennett-Goleman和哈佛医学院的。Herbert Benson 一起在这个被认为。很神秘的学科领域里做了很多重要研究。"对专注力的培养可以引领我们发现。身心纾缓 自身宁静和洞察力。的最高深的境界。通向这个境界的路径就在。你的体内 脑中 和你自己的呼吸里"。这就是专注力的奇妙之处。当我们讲到这些研究的时候 你们会看到。那些研究结果是多么令人难以置信。它们仅仅通过专注于呼吸。和身体的某个部位 通过亲临现场。改变了我们的思维方式 改造了我们的大脑。"所有人都有专注的能力。关键在于开发我们在此时此刻。全神贯注的能力"。让我们花一两分钟体验一下这种感觉。下面我想请大家做一个练习。还是那句话 如果你愿意做就做。请大家在椅子上坐好 将后背。靠在椅背上。尽量坐舒服了。两脚放松 平放在地上。如果你愿意 如果你愿意的话。闭上双眼。现在把注意力放在你的呼吸上。深吸一口气 吸至丹田。慢慢的 静静的 轻轻的呼气。再次慢慢的深吸一口气 直至丹田。慢慢的轻轻的静静的长长的呼出来。如果你的思绪飘走了 把它放回到呼吸上来。现在想象你在脑中观察自己的身体。你的前额 眼睛 鼻子 嘴 脖子 胸膛。你的后背 腰 双腿 一直向下。直到你的双脚 上下观察你的身体。同时继续深深的慢慢的轻轻的呼吸。在你打量自己的时候 寻找身上的。一个部位 比其它部位更紧绷。可能是你的下巴 脖子。可能是你的肩膀 你的胃部。你感到这个部位有些不适。可能是你的腿 膝盖 双脚。找到这个部位。它比其它部位更紧绷。把注意力放在这个部位上 继续呼吸。深吸一口气 将气引至那个部位。然后在你放松呼气的时候。把那种紧绷感也呼出去 放松。多深吸几口气至该部位。然后放松 呼气。重新将注意力放在呼吸上 深吸一口气。慢慢的轻轻的静静的长长的呼气。同时放松 深吸一口气 慢慢的轻轻的长吐出来。在下次呼气之后。轻轻的慢慢的静静的睁开你的双眼。另外 如果坐在你旁边的人睡着了。请轻轻的把他们叫醒。Tara Bennett Goleman。如果有人还在说梦话 请把他们叫醒。Tara Bennett Goleman 在她的。 

'情感炼金术'一书中写到"专注力。意味着看到事物的本来面目 而不是。试图去改变它们 关键是要消解我们。对负面情绪的反应 而不是抵制情绪。这就是所谓做人的权利。随着情绪 感受情绪。在这个过程中呼吸吐纳。虽然不一定总会这样 但是很多时候。如此一来 情绪就消解了。随着这种情绪的消解。令人痛苦的情感也得到了心理解离。还是那句话 春假回来以后。我们会谈到很多关于这个疗法的内容。这就是渐进式的变化。在我们待会说到研究报告的时候。大家会发现 即便这种渐进式变化。只是短短的8周的日常冥想。我们的大脑也会开始改变形状。我们的免疫系统得到增强。即便只有短短8周的日常冥想。而且不一定是一天5小时的冥想。有时候就是一天30或20分钟。已经足以带来变化。这个过程是循序渐渐的 缓慢的。希望大家可以用一生去实践。这是一个终身疗法。 

下面的例子是从突发变化的角度。分析我们的情感 所以。当我想填满这个3乘2模型的方盒子。我应该怎么去找情感突发式变化的。案例呢?我立刻想到了一个临床心理学。研究 是关于创伤后应激障碍的。有成千上万的文章。都是关于创伤后应激障碍。我们系也做了相关的研究。当我想到它的时候 我对自己说。'好吧 这是一个突发式变化的案例。非常突然的变化' 因为实际上。当我们有创伤之后 很多人的一生。因为创伤后应激障碍而从此改变了。它实际上改变了。我们大脑中的化学物质。它改变了我们大脑的结构。建立了新的神经通路 破坏了很多。以前的神经通路 全都是因为。一个单独的经历 就像是大锤一挥。很不幸的是这是一个非常常见的疾病。我上次已经说过了。30%的越南战争老兵有创伤后应激障碍。30%之多 80%的第一次海湾战争老兵。有创伤后应激障碍 我之前也提过。我们现在还没有关于第二次海湾战争。的确切数据 但是相信人数只会更多不会更少。在911之前 在纽约110号大街以南。有两万人有创伤后应激障碍。911刚刚过去之后。有6万人有创伤后应激障碍。这对人是有影响的。影响了我们的大脑功能。这种影响很多时候是终其一生的。这就是突发式的变化。伴随着创伤而来。比如911 或者在战争中的恐怖经历。从而影响了余生。影响了整个人生。这是一种休克疗法 消极的休克疗法。当我想到创伤后应激障碍的时候。我问'OK 这和积极心理学有啥关系'。这属于临床心理学的范畴 就是所谓的。消极心理学 但是其实有很大关系。 

首先 从1998年起开始有很多相关研究。仅仅是在10年以前 就是在。人们开始关注幸福心理学的时候。开始有很多关于创伤后成长的研究。研究发现 实际上更多的人在经历创伤后。得到的是成长 而不是应激障碍。然而 没人提起创伤后成长。每个人都知道创伤后应激障碍。但是再次说明了 这又是一个关于。人们不愿去关注正面事物的案例。如何正面呢 就是大部分人。其实是非常有能力去承受创伤的。这是一个非常振奋人心的事实。如果人们知道这是可能实现的。是常见的 如果人们不会对经历。创伤后成长感到愧疚 那么。我们就会有更多自我实现的预言。也会有更多人拥有创伤后成长的经历。在我经历了越南战争之后 有了。那样的所见所闻 我怎么会成长呢?这是不对的 再说一次 世事总有不幸。但是有些人能尽量从世事中受益。当人们开始关注事情带来的益处时。就会产生创伤后成长。 

那么 创伤能带来什么好处呢。发生这种事终归是不幸的。比如 我从患癌症这件事上学到什么呢。和家人的关系更亲密 更珍惜生命。更珍惜花草自然 更享受。和朋友在一起的时光 这件事。的确不幸 但是我却尽量从中受益。这就是价值发现。再或者 很多人能在写日记的时候。得到启示 我们下节课会讲到记日记。再或者 有些人能够分享创伤经历。还记得二战大屠杀的幸存者。和越战老兵之间的区别吗?对于不幸的经历 大屠杀幸存者拿来。分享讨论 把它写出来 越战老兵则。只是反复去想 

回到我们之前说的。Lyubomirsky的研究 当你只想不说。不把心里话倒出来。它会积压在心里 更容易形成。创伤后应激障碍 并不一定要这样。了解创伤后成长的好处。是至关重要的。那么现在我向大家提一个问题。既然这是讲幸福心理学的课程。大家应该还记得幸福心理学。关注的是有用的有效的事物。因此在备课的时候。我问了自己一个问题。有没有一个正面的和创伤相对应的元素。它的效果强大 但同时是正面积极的。并最终能像猛击大锤一样。瞬间改变我们大脑的运作模式?有这样的元素吗?换句话说 我想问。会不会有一种正面的经历可以创造。一条正向的通道 引领我们得到更多。健康 宁静 和积极的回忆 也就是说。和创伤后应激障碍正相反的反应。再重复一次:勤问必有所得 求索始于追问。当我问出这个问题 在我眼前立马开启了。一条我前所未见的通道。这就是我多年来一直钻研其中的。我心目中的最伟大的知识分子。马斯洛提出的高峰体验理论。我现在要说的这个理论 更多的。是一种假设性理论 而未经过实证的。我希望你们当中有人能。将这个假设性理论当成论文课题研究。或者以后 能鼓励别人研究这个理论。我在这个课上讲的理论。都是经过大量研究证实的。只有这个领域是假设性的。但还是请大家听一下 我讲完后。请好好想想这个假设性理论。因为虽然是有一些相关研究。但是不多 不足以证实其真实性。但还是请大家想想看它有没有道理。 

那么 让我们先了解一下什么。是高峰体验 马斯洛的定义是。'高峰体验是人类生命中最精彩。最幸福的时刻 最令人心醉神迷。欣喜若狂 极乐体验的高度浓缩。我发现 这种体验一般来自于。深度美学体验 包括创作时的喜悦。爱情 完美的性体验 为人父母。自然分娩 以及其他的人生体验'。每个人 或是绝大多数人都有这种体验。这种高峰体验。不管是你的男女朋友在一起的时候。还是在读一本好书的时候。还是在听你最爱的音乐的时候。这种体验就好象 嗯。许多神学家会说宗教体验。信仰体验 比如当你走在花园里。突然你看到了约翰哈佛(铜像) 也许���不是他 总之是一种集之大成的体验。这种体验就是人们所说的禅。与现在连接起来。感觉是如此美好如此完整。好像你已经完满了 拥有了一切。我和家人在一起的时候就有这样的体验。昨天晚上我和亲朋好友一起吃晚餐。你们知道吗 当我们围坐在一起。我感觉 这就是了 夫复何求。感觉如此完整 享受当下 这就是。高峰体验 但是这种体验一般不会长久。是会消逝的 它只存在于高峰一刻。但是 它是有震荡效应的。 

现在 也许你们已经知道我。接下来要说什么了 当我想到。高峰体验的时候 我发现它也许。只是也许 和创伤带来的影响是同等的。也许它是一种喜悦式的休克疗法。能够带来超越体验本身的效果。就好像创伤后应激障碍一样。 

时至今日 仍有人活在911的回忆中。这些回忆仍然影响着他们的大脑。创造着新的通道 新的神经通路。01年9月11日之前并不存在的通路。所以说 高峰体验是一种喜悦式的。休克疗法吗?如果它反面对应的是创伤。那么和创伤后应激障碍对应的就是。我管它叫高峰体验后通路。我知道 是土了一点 但是我觉得。是有道理的 相关的研究其实很少。但是我认为这理论是有效的。其中一个研究。是由马斯洛的一个学生做的。她研究是 生过小孩的妇女。她发现在一些时候。这种情况并不是大多数 妇女有时候。在生小孩的时候会有高峰体验。这种体验改变了她们的人生。体验带来的后果是 她们变得。更自信 更宽容 更友善 更快乐。一切都源于一个单独的人生体验。一种喜悦休克疗法 这种体验。发生在她们身上 对她们意义非凡。这个实验是在50年代初期做的。还是晚期 不 是50年代初。 

现在 这种情况对于男士也更常见了。因为现在的男士比起以前更可能。出现在孩子出生的现场 当我儿子。David出生的时候 我当场喜极而泣。对于我来说 那是一种非常强烈的。体验 绝对是一种高峰体验。下面是马斯洛对于高峰体验的阐释。他没明白说就是高峰体验后通路。但是也算是一种暗示。'高峰体验通常会带来一些后果。效果等同于心理疗法。前提是体验者目标明确。有自知之明。并且清楚自己的方向。一方面 我们当然可以只讨论。怎么分解症状 怎么老调重弹。分解焦虑 或其它问题 另一方面 我们。还可以去探讨怎么培养自主性和勇气。怎么培养奥运选手 幽默感 诸如此类。感知认识 身体感知 和其它.'。所以他的意思是说 如果我们做一些。事后跟进的工作 那么会产生一些具有。影响力的效果 而不仅仅是高峰体验。W. James在'宗教经验多样化'一书中。谈到改变人生的时刻 以及它们。怎样改变人生 高峰体验可以创造。新的大脑结构 虽然相关问题。仍待研究 但已有越来越多的迹象。显示这是有迹可寻的。如果我们知道该怎么做 高峰体验。可以产生和创伤相对应的正面效应。下次我们会详细讲这部分的内容。 

现在我们要说一个很关键的问题。首先 我们会有高峰体验吗。我们能有更多的高峰体验吗?其次 在我们经历了高峰体验之后。我们怎样做才能获得高峰体验后通路?因为正如同 并不是所有人都会经历。创伤后应激障碍 事实上大部分人没有。同样 大部分人在高峰体验过后。并未获得高峰体验后通路。怎样增加首先是高峰体验发生的概率。其次是高峰体验后通路的概率。因此 怎么增强高峰体验。这堂课基本上讲的就是关于这个内容。现在我要讲几个概念 当中有。我们以前讲过的 也有之后要讲的。 

首先 全然为人 接纳 接纳你的情绪。为什么呢 因为如果我们不能接受。痛苦的情绪 如果我们不能让自己。全然为人的 我们就阻断了自己的。情绪通路 而所有积极和痛苦的。情绪都流动于同一条情绪通路。当我们限制其一的时候 我们也限制了。另一方 因此如果我们开启通路。接受自己全然为人。无论是出于悲伤或极度的喜悦。允许自己哭泣。我们就能开启通路。让自己更容易感受到积极的情绪。看似矛盾。但这就是全然为人的隽语。当我们允许自己感受痛苦的情绪。我们就会更容易感受到积极的情绪。还有就是 专注力 感受当下。举个例子 很多时候 当我们听音乐。就是全神贯注的听音乐的时候。不是放着背景音乐 在一边发短信。做功课 和朋友聊天。而是全神贯注的听我们喜欢的音乐。很多时候 我们会有临场的高峰体验。根据马斯洛的理论。这两个条件能令我们最可能。获得高峰体验。顺便说一下 我相信这也是一种天赋。我从我一岁的女儿Sherio身上看到了。这种天赋 每当音乐响起 我们随之起舞。她就会开心的笑 这不是我们教她的。人对音乐和舞蹈的爱好是天生的。我们需要花时间去发掘这种天赋。制定一个有意义的目标。当我们有所追求 做自己爱做的事情。做自己觉得有意义 重要的事情。我们更容易获得高峰体验。最后 我们当今文化中最严重的问题。时间 当我们匆匆忙忙赶时间 倍感。倍感压力和焦虑的时候 我们很难。获得高峰体验 这些都是高峰体验杀手。无论是在做爱的时候。还是听音乐的时候。或者和朋友一起的时候。当所有这些发生情况的时候 都一样。 

那么一旦有了高峰体验 我们应该如何。增加获得高峰体验后通路的可能性呢。首先 我们可以重演画面。还记得吗 我们说过大脑。并不能分辨真实和想象。它可以将发生过的画面多次重播。当我们多次重播该画面的时候。神经通路得到了巩固。这条通路由大锤凿开。之后我们通过画面重演来巩固它。我们还可以把经历写出来。下周我们的心得报告作业。就是要大家写自己的高峰体验。当我们记录这种体验的时候 只是描述。而不是分析 记住Lyubomirsky说的两者差别。正面的情绪和经历不适于被分析。所以 只要去描述这种经历。再次巩固神经通路。花时间去重演 花时间去重播它。最后 就是要采取行动。通过采取行动 我们能巩固最初的体验。即高峰体验 巩固神经通路。所以说 当我有了高峰体验之后。我心里会感到柳暗花明 豁然开朗。现在我得付诸行动。这也是我今天要讲的第二部分-行为。 

20年来 自从我还未满20岁的时候开始。我就一直在教和参与各类关于。自我提升心理学 幸福心理学。临床心理学的课程 讲习班和研讨会。我发现 不管是在别人的讲习班上。还是自己的讲习班上。如果课程效果好的话 大部分人。在周末或期末离开的时候。如果之前他们在这里。之后他们都会经历一个高峰。但是一般人会分为两组。两组人都会经历高峰。第一组人 很不幸这组人占大多数。在经历了高峰之后 最后回到了原来的。幸福基点 我们对此并不陌生。第二组人 不是多数。是一小部分人 在经历了高峰之后。这种高峰并不持久。但是当他们落下来的时候。他们的幸福基点比之前高出来。之后尽管也有起起伏伏。但是都是在这个新基点上下起伏。不用说 我当然很想。知道为什么有些人是这样。另一些人是那样。因为我想让参加我课程的人能拥有。持久性的变化 而不是昙花一现的改变。只是短暂的感觉良好是不够的。这既对不住他们为课程所付出的努力。也对不住我为课程付出的努力。所以我很想弄明白两者之间的差异。下面就是我发现的这两组人。之间存在的最重要的差异。第二组人所具备的显著特点。就是在讲习班或课程或研讨会结束之后。会立刻做出行为上的改变。如果是一个学期的课程 他们不是在学期。结束后才做 而是每节课或几节课之后就做。他们不会边等边说'好吧 等课程全结束了。我到时再重新评估人生 看自己能做什么'。而是立刻采取实质上的行为改变。比如 做一些上课讲过的练习。学着承担以前不敢承担的风险。总之 立刻改变 而不是等着。而这些立刻改变的人并不会因此。而拥有永久的高峰体验。但是他们的幸福基点会提高。而不像其他人 因为没有改变。而回落到和以前一样的水平。所以 现在每当我有长达数天的。讲习班和研讨会 我都会在一开始就。给大家介绍这种变化的模式。 

我们从很多心理学研究中学到。态度和行为是密切相关的。这一点我们以前讲过 提到过。所以说我们都有自己的态度 无论是关于。心理学的态度 还是关于其他人。或是关于自己的态度。这些态度影响着我们的行为。比如说 如果我对心理学 或是幸福心理学。抱有好感 相对于对心理学。完全没好感的状况。我更有可能去选1504这门课。或者说 如果我对一种想法。或对一个人有好感。比起对其毫无好感的情况。我更有可能和他成为朋友。所以说 态度影响行为。这是显而易见的 很容易理解。但是一些心理学家 包括Alice Eagly。Daryl Bem和其他一些人。他们发现行为同样也能影响态度。这是一条双行线。 

所以说如果我有某种行为模式。我的行为很可能会改变我相应的态度。这个问题我们以前也讲过。为什么呢?因为如果说我们有某种态度。我们的行为就好像外在的世界。我们的大脑不能接受。内在和外在不一致的情况。因此如果我们的行为是这样的。那我们的态度就会被拉下来。和行为相对应 如果我们的行为是这样。大脑不喜欢不一致的情况。它就会让两者达到对等。让两者之间达到一致 谐调。方法就是要不然改变我们的行为 或者。这点很关键-更多时候 改变我们的态度。我们在场的所有人 不管你是。18岁还是80岁 我们都有自己的习惯。我们最初养成习惯 习惯之后造就了我们。习惯是一种行为模式 思维模式。行为 行动比语言更有力。如果我们在这堂课之前 有某种行为习惯。这堂课改变了你对某件事的态度。如果因此你的态度变得和行为不相符。下课之后 你的大脑寻求一致性。你的态度会被拉回到原来的状态。除非你改变自己的行为。所有的课程 不管是1504还是公正课或。心理学1 所有的课程 讲习班 研讨会。所做的都是改变态度。在这门课上。态度的转变就是 比如说 自尊。什么是自尊?是一种我对自己的态度。高自尊就是正面的 低自尊是负面的。 

价值发现者和错误发现者的不同。在于他们看待世界的态度不同。我把这个看成是积极的 能带给我。幸福的呢 还是像错误发现者一样觉得是消极的。完美主义 是一种对待失败的态度。这门课能做的 我和你们的助教所能做的。就是激发或鼓励你们在态度上的转变。通过介绍这些研究 让你们相信。正向思维的作用。但是 如果你们不在行为上做出改变。比如去做感恩练习。比如这周要交的感谢信。这个练习是应该定期去做的。除非之后有相应的行为。否则你的态度会回复原状。你的态度和行为都和上课。或是接受疗法之前没有两样。必须要有行为上的改变。 

现在我要给你们看一些研究。这些研究的结果显示了行为改变的影响力。这个研究实验的对象是。朝鲜战争期间的美国战俘。当时就在街对面的MIT 科学家E.Schein。发现关押战俘的人最终改变了美国战俘。对于共产主义 对于中国人的态度。这些关押他们的人只是很简单的。要求他们把共产主义的优越性。写下来。我们知道 我们不喜欢共产主义。如果喜欢的话 我们也不会打仗了。但是要求他们只写优越性。写给你的狱友看。写信回家 告诉家人你被如何对待。当然确保他们只能写好话 所以他们。并没有要求战俘们撒谎 虽然战俘确实。时常写些谎话 他们要求战俘只写优越性。长此以往 态度发生了转变。他们对关押者的戒备少了。好感多了。因为他们的态度变了 通过写信。和朋友聊天。还要做报告 每一次都在。说"我觉得这样很好"。长此以往 他们的态度发生了转变。变得更积极正面了。 

认知距离-这个词。你们中间上过心理学1的人可能听说过。认知距离的产生是因为两种想法。之间不一致 两种信念相冲突。或者说行为和信念相冲突。我们不愿看到这种情况 我们想要一致。认知距离理论是讲这种冲突。必须要被消解。我们消解它的方法就是让我们的。态度 我们的想法和信念 尽量符合。我们的行为 因为行为比语言更有力。自我认知理论 我们已经讲过很多次了。这次我不打算详细说了。就是我们认识自己。并从中衍生出自我结论。脸部回馈假说。如果你现在皱眉或者轻柔的微笑。你体内的化学物质会发生变化。这种变化反映了你表情的变化。引用一句名言 是……一行禅师说的。他的名字不好念 他是一个佛教僧侣。"有时你的欢乐是微笑的源泉。但有时你的微笑也可以成为欢乐的源泉"。 

所以说实际上 我们面部表情的变化。能够对我们的身体和情绪产生影响。这也解释了为什么演员 不管演的。是什么角色 都经常会入戏很深。因为脸部回馈于身体其它部位。在你做面部表情的时候。你体内释放了相应的化学物质。William James的一段话。"吹口哨壮胆绝不仅仅是一种比喻。同样 无精打采的坐一天。哀声叹气 语气阴沉。都会加重你的忧郁。舒展眉头 发亮双眼。挺胸收腹 声音洪亮。友善的夸奖别人。除非你的心如钢铁。否则不会不为这样的人所动"。William James所说的已经不仅仅。是脸部回馈假说。而是一个很少有相关研究的假说。身体回馈假说。如果你整天这样坐着 而不是骄傲的。挺胸抬头 你的情绪会因此受到影响。它会从几个方面影响你对自己的观感。首当其冲的是 你向自己传递的信息。也就是自我认知理论说的。但是同时 还有别人是如何看你的。如果你握手是这样软绵绵的。而不是这样坚定有力的握手。你在向与你握手的人传递某种信息。那种信息最终会回到你这里。一是通过别人对你的观感。一是通过你的自我认知。 

你们知道 之前有个学生选了这门课。她当时是冰球队的。去年毕业了 今年夏天我碰到她的时候。她走过来 同我握手。然后我哭了 因为实在是太疼了。我想她绝对是专心听课了。所以千万不要太过分 坚定也得友好。我是说真的 我觉得她把我的手骨。握碎了好几块 但是这真的很重要。因为握手传递了信息。我以后绝对不会找她的麻烦 永远不会。我现在怕死她了。这个握手传递的信息就是力量和信心。如果我们走路时昂首阔步 我们在传递。一种信息 如果我们走路时弯腰驼背。我们同样也是在向外界传递一种信息。外界因此会给予我们回馈。这种信息我们也会传递给自己。并产生自我回馈 

最后我想给大家。讲的是Hammerly做的研究。算了 我还是下次再讲这个研究。因为我今天还有一些重要的事情要讲。所以下节课一开始我就会讲这个研究。这个研究报告很长。但是又很重要 所以下节课。我要讲的第一件事就是这个研究。今天下课前我还要讲一些内容。这些内容足以改变我们的人际关系。弄假直到成真。David Myers在这个属于幸福心理学。的领域里做了很多相关的研究。他在研究中发现 很多时候即便我们是。假装快乐 假装很有自尊 像William James。说的假装高兴 我们的情绪也会因此改变。问题来了:那我们怎么能全然为人呢。好 首先 总有些时候。我们不想弄假直到成真。我们想要哭泣。我们想要难过 并表现出来。但是 同时 我们还得要。准时出门 参加派对。即便我们一点也不想出门。这里的区别就在于积极接受。我仍然接受我的情绪。接受我的痛苦 并感受它。同时选择用最适当 最对自己有益。的方式去面对。所以说 我能接受。被女友抛弃这个现实 我能接受。自己感觉糟透了的现实 但是三天之后。照样出门到Oaks参加派对 疯玩一通。因此 正是因为脸部回馈假说。因为身体回馈假说。因为自我认知理论。行为影响了我的态度。在Oakes狂欢一夜之后。我实际上感觉好多了 也更健康了。好吧健康不一定 但是感觉是好多了。 

下面我要播放一段Marva Collins的录像。当中她谈了一些自己的经历。"我觉得也许我现在这样是因为。我的父母和他们的坚持 我想要和他们一样成功。和我的祖父母一样成功。在以前 成功的黑人是很少见的。所以我认为是我父母和祖父母的决心。他们……我们会抬头挺胸。我和比我小14岁的妹妹。去教堂的时候 如果我们没有挺胸抬头。我妈妈会说。'你们今天这么了?老低着头走路?'。她会在我上学的时候。在街上喊'抬起你的头来!'。我常听见别人对她说。'我一眼就能在操场找出你的小孩'。我们就是在这种骄傲的氛围里长大的"。抬起你的头 直着走 如果你仔细观察。Marva Collins 你会发现她就是这么走路的。她的姿势就是如此 并因此给她的学生和其他人。也给她自己传递了一种信息。我现在要跳过一些内容。所以大部分行为上的变化都是渐进式的。那什么是突发式变化呢。突发式变化关注的是如何应对。关注的是承担风险。当我们去应对的时候。我们需要冒险。需要做一些不太愿意做的事。换句话说。就是 离开舒适区 进入学习区。 

现在我要给大家讲两个我自己的故事。这两个故事乍听起来毫不相干 但如果。大家听明白了的话 就会发现它们息息相关。第一个故事不轻松 另一个也不轻松。我的第一个人生记忆。大家知道当人们开始记事的时候。记得的第一件事就是。第一个人生记忆。我的第一个人生记忆是发生在。1973年9月 那时我快满三岁了。那天是赎罪日 一年当中最神圣的一天。我记得我当时在家 突然电话响了。我的父母是非常正统的教徒 我从小。成长在正统的家庭里 我们的电话在安息日。周六都不会响的 更不要说赎罪日了 但是。这天电话响了。我记得他们当时跳了起来 向电话奔去。我爸爸拿起了电话。我当时站在父母中间。我爸爸看着我妈妈 小声说着什么。我听不清 但是我看到我妈妈的脸色。变了 我从她眼里看到了恐惧 我哭了起来。我爸爸把我抱起来 他对我说。我要离开几天 不过我会回来的。现在我知道当时发生了什么事。当时 我父母接到我叔叔打来的电话。他当时在军队服役 在情报部门。他说战争爆发了。在以色列一年当中最神圣的一天。年轻人 以色列军队。当时都不在边防线上。完全出乎大家的意料。5个阿拉伯国家同时向以色列宣战。我爸爸把我放下来 走进他的房间。一切仍历历在目 我看到他拿出自己的军装。穿好 拿出他的M16自动步枪 背好。穿上鞋 系好鞋带。我妈妈一直在他身边帮他整理。他们交谈当中偶尔会向我微笑一下。但是我感到一种紧张不安 我不明白为什么。然后我们一起下楼走到我爸爸的车旁边。他当时有一辆老式的绿色福特跑天下。他把鞋油拿出来 涂在车前灯上。为什么呢 也是后来我才知道。因为在夜里开车的时候 你开着车灯。车灯不能太亮 以防会遭到空袭。所以他涂了车灯 我就在一旁看着他。然后他把我抱起来 又一次的。抱着我说 "我过几天就回来"。然后他上了车 我开始控制不住的哭泣。我们的邻居 Sharlon。他年岁太高不能服役。他站在一边。我们一起看着我爸爸开车走了。他把我抱起来 我还在哭。他看着我说。"Tal 你长大以后想和爸爸一样。去当兵吗?"。我说 "我想"。他说 "那好 士兵是不会哭的" 于是我不哭了。 

之后将近20年 我再也没哭过。我上了哈佛大学 我开始学心理学。你们知道我要讲什么是吧。我开始学心理学。我知道自己想要做的。想要做的最重要的事情之一。就是找到自己女性化的一面。用荣的话说 就是阴和阳。雌雄同体。我想要找到自己女性化的一面是因为。我知道压抑情感是不健康的。我压抑痛苦的情绪。同时也压抑积极正面的情绪。因为是同一条情绪通路。但是我在一个非常强调男子气概的。文化环境中长大 男人不哭 士兵不哭。我们很坚强 可以搞定任何事情。这是我长大的文化背景。而且 那是我人生的第一个回忆。之后我反复听过同样的话无数次。传递的都是同样的讯息。这样不行 表露情绪一点也不爷们。我现在想起来。那时候我输掉了对于我来说最重要的。壁球比赛 全国冠军。之前的一年我是冠军 当时我很震惊。我本不该输掉 本该赢的。之后我和女朋友一起回家。我们就是一起消磨时间。突然间她哭了起来。我问她 "你哭什么?"。她说 "我哭是因为你不哭"。因为她明白这场比赛对于我有多重要。这就是我在童年 青少年阶段的体验。要坚强 要像个男人。然后我来到这里 我明白我要释放出来。我要找到自己女性化的一面。这就是我的第一个故事。 

现在让我来讲第二个故事。我以前的一个学生 你们当中可能有人认识她。她是04届毕业的 叫Lindsay Hyde。Lindsay Hyde在我这里学习。我们也有一对一教学 经常一起学习。Lindsay Hyde是'顽强女人顽强女生'。组织的创始人 一个非常棒的组织。隶属于菲利普斯布鲁克斯内务协会。现在已是全国性的组织 我也是筹款委员会。的一员 有兴趣的人可以去他们的网站。swsg.org看看 Lindsay邀请我给。'顽强女人顽强女生'的导师们做报告。这些导师都是些哈佛学生 以及来自。波士顿其它院校的女会员。于是我去做了一个报告 照片上的就是Lindsay。这是她和一个三年级学生在一起 她。还有其他几位哈佛学生是这个学生的导师。他们一起做了很多很棒的事。我非常相信这个机构的理念。几乎从筹备的最初阶段我就支持他们。所以我做了一个报告 这个报告是关于。模范人物的重要性 关于诚实。我跟他们讲了Marva Collins。讲了他们要做的重要工作。我觉得 对于我和所有与会者来说。整个报告进行的非常顺利。最后 我收到了一份表示感激的礼物。礼物是一件衬衣。这可不是普通的衬衣。是一件粉色的衬衣 这个颜色我平时。不穿 当然在以色列我从来没穿过。因为这是一件'顽强女人顽强女生'衬衣。这衬衣不仅是粉色的 而且。尺码还很小。给三年级学生穿应该比较合适。但是他们还是对我说。'我们希望你能收下这份表达我们谢意的礼物'。然后我就犯了一个错误 我当然是在开玩笑。当时我在教1504课程 我开玩笑的说。'我会穿着这件衣服去上课的' 然后我笑了。这就是我收到的那件衬衫。我在1504班上的另一个学生。Tory Martin 她在Lindsay走后。接管了这个机构 她当时是主席。Tory Martin说 '我会确保你兑现承诺的'。然后我又笑了 然后我差一点就哭了。我说 '你什么意思?什么确保我兑现承诺。我是开玩笑的' 她说 '你看。你不是刚给我们讲过诚实的重要性么'。我的确是讲了 你们看 这里 诚实。尽管这么靠后也能看见 我是讲了。我说 '好吧 我讲了 我现在怎么办'。诚实可是我的核心价值之一。然后我的眼前闪过了我的整个童年。然后我就想到了丘吉尔说过的一句话。'在逆境中寻找转机'。于是我一直在寻找转机。现在转机来了 就在这件粉色衬衫上。让我找到自己女性化的一面。 

女士们 先生们 2008年春装展。\[音乐\]。我懂 我懂的 你们也许不会相信。但还是相信我 当我说。这个对于我来说是在我的舒适区之外。实际上这已经都不在我的学习区之内了。已经到我的恐慌区了。上课之前 备课的时候。你们知道我都是要做笔记的。所以我在这里写下要干什么。我跟助教们说。我刚写进笔记的时候 每次看过去。我都感到肾上腺素在胃里飙升。这就是在舒适区之外。现在我也是在舒适区之外。这就是为什么我要在下课前。而不是刚上课的时候干这件事。关键是要达到最适度的不适感。我可能的确是有点太吓人了。但是关键是最适度的不适感。为什么呢?想要改变 别无他法。我可以想一天想破头。一直说我要找到女性化的一面。我要更勇敢 我要释放出来。我可以想一整天。什么也不会发生的。什么也不会发生的。除非我们有真正的实质的行为改变。真正的实质的行为动作。当我们做到这一点的时候 天高任鸟飞。 

谢谢大家 下周四再见。 

第11课-养成良好习惯 

大家早上好。首先今天在座的有低年级父母吗。欢迎 欢迎 非常高兴你们能来。我更高兴你们周二没有来。我……要先宣布一点事。周二之后我一直问自己。"他们以后还会不会认真听我的话了" 我希望如此。或者"他们明天还会喜欢我吗"。 

好了 还是先宣布点事情吧。首先 我收到很多关于论文的邮件。你们写的感谢信。也需要提交给你们的助教。作为论文 所以我的回答既不肯定也不否定。好了 也就是 对 你得提交。不过如果那封信过于私人。或者由于任意原因你不太愿意提交。给你的助教。就直接给助教发"我写了那封信"。我们会信任你的。所以不是一定要提交。虽然最好是能提交。 

一个半礼拜之后就要期中考。所以关于期中考我说几句。我们打算期中考。只出选择题。会有…… 不太记得了 多少道 50? 50。50道选择题。有75分钟的作答时间。这样就可以留出十分钟。到十五分钟的时间来组织秩序。考试不会太难。要愚弄或是考倒你们。而是会考得很直接。 

第一节课我曾告诉过你们我对考试的态度。如果你们还记得的话。过去 我不太习惯弄考试。因为我记得我本科时就很讨厌考试。己所不欲勿施于人嘛。不过后来我意识到考试确实是有价值的。不是在于高低分化。或者成绩之类的。而在于期中考或期末考可以让你安下心来。整合所有学过的知识。如果是家庭作业 你可以找你所需的答案。你只掌握了部分的知识。但有期中考或期末考。你就能坐下来从第一课看到最后一课。而这可以帮助你整合知识。进而吸收掌握知识。 

请记住我反复强调的。这门课是基于螺旋形的知识体系的。也就是说我们在第一课中讲的内容。与第三课相联系 与第七课相联系。与第二十四课有联系 所有东西都是内在相连的。而只有你静下心来。当我在讲某个知识时 我知道。接下来的三周我会讲到什么 以及它们的联系。你们还不能 但当你们坐下来准备考试时。才能建立知识间的联系。才能从更高的层面真正。吸收掌握知识 所以我们才要举行考试。我知道这不好玩 但对教学有好处。静下心来把知识过一遍是很重要的。不管是期中考还是期末考。再说一次 不会很难 但一定要复习。考试会很直接 基于事实。可以想象得到 期中考不是要展示创造力。我们的目标是让你们回顾并掌握知识。有任何问题。可以给助教发邮件 他们会给予回复。如果没有 那就发邮件给我。 

好了 上回讲到哪了 其实我不太记得了。我好像忘了上节课的内容。不过有人跟我说我跳过了一些东西。以便最后能完成我的节目。有一样被跳过的东西 我想回头讲一下。在我们谈到身体反馈假说之后。还记得身体反馈假说吗。你握手的方式 言谈举止。还有面部反馈假说 包括微笑 皱眉。这向他人传递了特定的信息。然后他人再反馈回来给我们 与此同时。我们也在和自己沟通。 

我自信吗 骄傲吗 或者我害怕恐惧吗。很多时候我们需要。正如我所说"不断伪造方能成真"。因为我们的身体 行为 给我们的思想。以及情感提供信息 并影响它们。 

有一个很不错的研究。由Haemmerlie以及第二作者Montgomery完成。Haemmerlie和Montgomery做了如下研究。他们招募了一些内向的直男。招募他们进行研究 内向的性取向正常的男人。研究是按如下方式进行的。这些人被告知研究就是做一些特定的测试。他们被邀请到William James大楼。然后被告知"很不幸我们进度落后了"。"所以你得稍等"。"必须在研究开始之前稍作等待"。同时等候室在这里。有很多人也在等候做相同的测试。"等到了的时候我们会来叫你"。"可能得等一会 抱歉"。"我们会付你额外等待时间的酬劳的"。于是他就在等候室里等候。还有一个人和他一起等。一位女性 

他并不知道她其实是共谋者。是研究的一部分。他以为她和他一样在等候。也在等候进行测试。于是他和一个女人一起坐了12分钟。那个女人的任务是与这个内向直男。发起对话 并对他所说的东西。表现出极大的兴趣和兴奋感。于是他们互相倾听 互相提问。"哇" "真的吗"。他们在12分钟里谈笑风生。然后。那女人进去做他们以为的测试。另一个女人再进来 又是12分钟。她和这个内向直男坐在一起。表现极大的兴趣 对他说的话发笑。向他提问。诸如此类 聊12分钟。她进去做试验之后 下一个女人又进来。如此类推 一共六次。六个女人坐在这些内向直男身边。对他们所说的话表现极大的兴趣。开启对话 一共72分钟。然后再进行所谓的真正的试验。然后第二天。他们又被邀请参加同样的试验。再把完全一样的流程走一遍。72分钟里 他们和六个女人分别坐在一起。她们对他们的谈话内容表现极大兴趣。 

当然了 这个试验要研究的就是。这会有什么效果。这种行为对于他们的内向会有什么效果。而答案是非常大。接下来的半年里 跟踪调查发现。这些男人突然变得没那么紧张了。总体上来说 尤其是在女人旁边。他们没有那么害羞了。这些男人 很多是人生中第一次。开始发展感情 开始约会。在144分钟的干预以后。天翻地覆的变化 

但有一个问题。许多心理学实验的普遍问题是什么。就是必须向受试者报告结果。没错 你得报告结果。因此半年后 试验结束。研究者重新聚集受试者。告诉他们这只是个研究。而那些女人其实是试验的一部分。是被要求表现出兴趣的。很残忍对吧 但并不像预想的那样。这对那些男人没有影响。因为那时 他们已经外向了许多。他们与异性相处愉快。他们出去约会 他们不再那么内向。这开启了一个良性循环 已经完全没关系了。144分钟改变了他们的生活。至少从约会的角度来看。为什么呢 

回忆一下Bandura对自我效能的研究。没有比成功本身对成功更好的催化剂。当他们成功时 做得很好时。他们能看到自己做得很好 --自我知觉理论。关于对自我的认识和感知。使他们进入良性循环。并且保持这个良性循环。我们刚刚讲了"不断伪造方能成真"。再看看Marva Collins。先祖伦理。很重要的犹太教教义写道。"那些行为超越其智慧的人 他们的智慧将恒久"。"但那些智慧超越其行为的人"。"他们的智慧无法持久"。如果能记住我说过的。当我们参加讨论会或者上课。并从认知水平上理解它。即便有灵光乍现的时刻 有突破 也不会发生什么。除非我们的行为跟上了。这新闪现的智慧 为什么呢。 

因为我们首先塑造习惯 然后习惯再反过来塑造我们。假设我们在参加学术讨论或上课之前是这个水平。然后我们的态度改变了 但行为没变。态度会随着时间流逝被习惯又拉回来。所以只有我们随着时间改变习惯。开始做点什么 比如说。强迫自己做些事 解决些问题。或者开始做感恩的训练。或者平时经常写信。或者开始做运动。可以证明是最有力的干预。来解决焦虑 沮丧 以及多动症。除非我们很快开始做这些事。否则这改变就会是短暂的。我们很快会回到原来的状态。持久的长远的改变。态度的改变必须要使行为与之匹配。 

Dan Millman 我几周前在课上提过他。在《深夜加油站遇见苏格拉底》一书中。探讨了行动与改变的重要性。他说 我引述 "要改变人生轨迹。就要从两种基本方法中选择。其一 你可以指引你的能量及。注意力 用其整理思想 集中精神。巩固力量。释放情感以及想象乐观结果。从而使你最终获得自信 鼓起勇气。下定决心做出承诺。获得充分的激励去做你要做的事。其二 你也可以直接去做"。有时说起来容易 做起来难。但通常。例如跳入水中 这个行为。做出行动本身 和所有前期准备有一样的效果。或者效果更好 能形成良性循环。 

然后我们谈到了应对以及退出舒适区。家长们 你们可以暂时忽视这个。然后我以找到你们的。"最佳不适水平"结束。这是什么意思 什么意思。我们可以寻找改变的途径或方法。伴随紧张的连续统一体的行为改变。我们大多数人在大多数时候都蜷缩于舒适区。这很好 很不错 但是。当我们处于舒适区里时 很少发生变化。当跳出舒适区 就到了拉伸区。也就是最佳不适区。这就是改变实际发生的地方。再越过这个就是恐慌区。此时我们就有了焦虑和困扰。这通常是不健康的区域。对于改变来说不健康。因为这时通常我们都会回到原来的状态。可以打个比方。舒适区就像是结冰的水。拉伸区就是流动的水。而恐慌区就是沸腾的水。非常躁动。但难以控制 并且十分危险。一般最好是呆在拉伸区里。 

下周当你们读到"流动"时会读到这个。流动就是当你受到适度的刺激。适度的紧张。当你做的事 或者别的什么。既不会太难也不会太简单。我上节课最后的部分 很显然。正如我所说 你们所见 就越出了我的舒适区。可是它还不足以 可能够了……。但应该还没到恐慌区。为什么呢 因为我已经认识你们了。我们已经在一起一个多月了。已经过了撤课的最后期限 你们已经来不及了。所以我也不是冒了很大的风险。只是拉伸一下 而且这件事对我很重要。我是说对我个人来说很重要。因为它确实拉伸了我。使我更多的接触到了我的人性。或者比方说你要在健身房里锻炼。你意识到这很重要 你看了这方面的研究。你得知这方面的研究。你看到它确实有很重要的作用。于是你开始锻炼。如果你已经锻炼了五年。然后你打算开始每天跑八英里。问题就来了。你拉伸过度了 你可能会受伤。如果你继续坐在电视前 或者打游戏。那对于改变也不是什么好事。也就是呆在舒适区也许不是好事。拉伸就是"好吧 就从每天。走两英里开始 然后逐渐增加"。拉伸自己 但不要过度。或者如果你想给别人做讲座。传递知识 积极心理学或者生物学。无论是什么话题。在观众面前你都会紧张。不一定要从Sanders剧院开始。可以从你的朋友 队友开始。我就是从我的壁球队友开始的。我给他们做了第一次讲座 然后给家人做。 

逐渐走出舒适区。每次拉伸一点点 然后逐渐增加。这是改变的健康方式。但是有时候 有的时候。我们必须要进入恐慌区。为什么 因为有些东西无法。或者几乎不可能逐渐地改变。例如 成瘾。如果我是个瘾君子 那就很难说出。"我就少注射一点。今天少一点 明天再少一点"。通常都要很突然 然后我们就处于沸腾的区域。就处在恐慌区。时常都是这样 所以我们才需要帮助。在此时我们需要有人抓紧我们。安慰我们 保护我们。因为这是个如此反复无常的地带。并且十分危险。但一般来说 如果你想改变。健康的方式就是在拉伸区进行。仅从理论上来说 是不可能改变的。我已经说了很多遍 但还是经常有。 

上了这个课的人 他们中的很多人确实改变了。他们中的很多人确实说。"上这个课使我的生活变得更美好了"。还有些别的人则说。"我上了这个课 但它对我的生活没什么影响"。"这是个很有趣的经历 粉红T恤之类的"。"但对我的生活没有造成实质而持久的影响"。在每种……几乎所有 并不是所有 但大多数情况下。这是因为这改变并不是与态度改变。或与内在的理解和认知。相关联的行为改变。一定会存在着行动偏向。 

关于行动偏向 举几个例子。为了增加自信 我们得冒险。而不是说一下 或者想一下。或者站在镜子前对自己说。"我非常自信 我有自尊"。"我很棒 很好 很伟大" 这都不够。我们如何在生活中减轻压力。下周我们会讲这个。简单点说 就是少做点事 而不是多做点。我们会讲到为什么少做点事其实。不仅会带来更多的幸福感。也会带来更大的成功。更多创造力 以及生产力。但我们不能在理论上减少压力。要借由感恩走出阴影。创造更多积极的渠道。逐渐成为积极者。这需要时间 不是一夜之间的事。但好的一面是 当我们害怕时。当我们觉得难以采取特定行动时。我们可以运用内部模拟器。 

还记得吧 大脑是无法分辨。真实和假想的东西的。如果我们想象某物 用认知行为治疗的话来说。在我们受到刺激时。不管是通过想象还是实际行为。如果我们受到刺激 久而久之 我们就会变得自信。就像我告诉你们的一样 为讲座做准备。看到我自己 在我的脑海中进行这件事。我的大脑并不知道在真实和想象之间。有什么区别。逐渐地我们就变得更自信了。这是个好的开始 但这不够。但很显然是很实用。也很有益的 可供尝试和探索的方法。这些都很好理解。我想你们有些人肯定在想。事实上上节课后有些人。来找我谈过 说。"我们知道行动很重要 但有个问题"。"问题在于我 或者许多人都没有"。"足够的自律来完成这个行动"。"来保证每周跑三次或者五次步"。"我们很懒"。"玩游戏更有意思一些"。"相比于做一小时瑜伽 至少一开始是这样"。"所以我没有足够的自律性来改变"。 

我想让你们快速地举个手 说实话。举起手来 如果你相信。假使你能更自律。现在你们可能非常自律。也可能很不自律。但假使你能更自律。你就可以更幸福或更成功。假使你能更自律 说实话。我的手可是举得很高的。好 大多数人都这么觉得。我有一个好消息和一个坏消息要告诉你们。从坏消息开始吧。因为我喜欢积极的结尾。那么坏消息是。你们不会获得更多的自律。你现在拥有的就将是全部。就是这样 抱歉 真倒霉。大多数人 绝大多数的人。认为他们需要并且想要更加自律。认为他们现有的自律不够。 

大多数。几乎所有人都无法得到更多自律性。事实就是这样 这就是坏消息。好消息是 更多的自律事实上并不重要。不管是对于成功或是幸福感。就用你现在已有的自律性 你也可以。变得更成功 更幸福。怎么做呢 如果你把注意力从依靠自律性。来获得改变 转到引入例行公事上来。将你的注意力从自律转向例行公事。 

我接下来要讲的一个理念。完整的叙述来自于。Jim Loehr和Tony Schwartz所著的《精力管理》。一本非常非常好的书。他们所说的本质上是一种思维转换。思维转换。就是我们要停止试图得到更多的自律。因为自律本身对于成功 对于幸福。对于改变来说是不够的。正是由于人们依赖于自律来获得改变。这就是为什么。许多组织及个人的尝试改变的努力失败了。 

让我和你们分享一个有趣的研究吧。这个研究是Roy Baumeister做的。二十世纪以及二十一世纪初最卓越的。社会心理学家之一。非常伟大的研究者 他所做的研究如下。他逐个把一组人带进来。于是那个人来进行试验。然后同样告诉他们 "你得在这个房间等候。在这个等候室里等待实验开始"。于是那个人坐下来。在他旁边有张桌子。桌上放了一个碗。碗里装着刚烤好的巧克力脆饼。新鲜出炉的那种。芳香萦绕在空气里。就像动画片里画的 飘进他们的鼻子里。无比的香味。他们就坐在那儿。正挨着装巧克力脆饼的碗。试验者然后说。"我会在约十分钟后回来叫你去参加试验"。他们走出去 出去之前对受试者说。"顺便提一下 这些巧克力脆饼"。"是为下一个试验所准备的"。"所以如果你不介意的话 最好不要动它"。于是你就坐在那儿 非常渴望。但就是不能碰它。十分钟之后。试验者确实回来了 带你去参加试验。这个试验非常难。你要进行非常非常难的测试。大多数人都无法通过这个测试。并且这需要大量的坚持和努力。但大多数人。即使拥有坚持和努力也没法通过。所以评估结果时。研究者感兴趣的是。在放弃测试前 你坚持了多久。在放弃测试前 你坚持了多久。这才是对结果的评估 才是因变量。第二组随机挑选的人 进入同样的房间。坐在同一把椅子上 旁边桌上放着相同的碗。但里面装的不是巧克力脆饼。而是甜菜根。刚烤好 新鲜出炉。他们就坐在那儿。然后试验者正要走出去 说。"还有一件事"。"如果你不介意的话 别碰这些甜菜根"。"这是为了下一个试验准备的" 好。你坐在那儿坐了十分钟。十分钟后 试验者进来了。带你去参加"真正的"试验。你坐下来做了跟之前那个一模一样的测试。完全一样的谜题。然后再次看看 放弃测试之前你能坚持多久。你们自己想一下。你们觉得哪一组能坚持得久一些。巧克力脆饼组还是甜菜根组。你们自己想一下。顺便说一下 我在回答这个问题时猜错了。有一组显著地坚持得更久 显著地更久。而不是细微差别。甜菜根那组显著地坚持得更久。 

为什么 当时我也不理解。这个机制……。他还做了些其他试验来解释这个机制。这是因为巧克力脆饼组。已经用了自律来阻止自己动饼干。虽然他们十分想碰它。但他们不能碰 所以他们已经用了那份自律了。等他们去参加"真正的"试验时。他们就几乎不剩什么自律了。而那个迷宫 那个试验。那个测试需要大量的自律。这个研究所要告诉我们的是。我们都只有有限量的自律。问题在于我们将它用在了哪里。 

再问你们个问题。在座有多少人。确实制订过新年计划的?如果你们完成了每一个为自己。制订的新年计划的 请举手。举起手来。每一个新年计划。如果你确实制订过的话。请举高点 好吧 我显然没有。我还想问个问题。在座有多少人…… 这个问题很重要。所以如果你的答案是肯定的 一定要举手。在座有多少人今天早上刷了牙。举高你们的手 很高兴看到这个结果。你们可以课后找我聊聊。让我来想像一下今早上的场景吧。你早上起床 十分疲倦。然后你对自己说。"好 就是今天 就是今天"。"我要去做 我今天早上要刷牙"。对吗 你起床后真正给你动力的。我就来了 我就来刷牙了。你也刷牙对吧 就是今天。因为今天我要上课 我要刷牙。不 当然不是这样。你滚下床 几乎睁不开眼。你可能都不记得你刷了牙。因为你还是像梦游一样。 

为什么 为什么所有人。我觉得所有人今天早上都刷了牙。而没有人 六百个学生里没有一个。实现了他们的新年计划 为什么。因为新年计划依赖于自律。而刷牙是例行公事。我们每天都做 几乎是自动的。这就是例行公事 我们都知道刷牙很重要。如果不刷牙别人就不会跟我们说话。但我们也知道锻炼很重要。而世界上许多人。不是就哈佛来说 而是整个世界。不做锻炼 并为此付出了很大的代价。 

我经常感到惊讶。当别人夸我有不可思议的自律时。那时我还是个壁球运动员。这经常让我很惊讶。因为我不觉得我是一个有自律的人。我们家是没有巧克力脆饼的。即使有也会被Tommy藏起来。因为如果有 它们两分钟后就会消失。在这些方面我的自律为零。但对于壁球 我确实…… 人们觉得。我也这么觉得了好多年 ---我很自律。因为我早上很早起床去跑步。然后去学校 放学之后直接去球场。和教练进行训练 常规训练。然后打比赛 然后如你所见。去健身房健身。每一天都是这样。然后回家 做作业 上床睡觉。这是例行公事 这正是运动员所做的 例行公事。所以他们才能保持。从外界看来似乎是很高水平的自律。有时甚至是非常人的。但不是这样 这些是例行公事。 

Jim Loehr和Tony Schwartz说。"建立例行公事需要定义精确的行为"。"并在特定的时间执行它们"。"以深深扎根的价值观为动力"。举一些个人例子。锻炼 深深扎根在我心中的价值观。一直都是 尤其现在我知道了这个研究。关于锻炼的数据。所以我有件例行公事。每周跑三次步 每次30到40分钟。之后我就做伸展练习。我知道瑜伽和冥想对我有多重要。这对我来说是例行公事。早上醒来第一件事 深呼吸几次。假期回来后我们会更详细地讲到。这是例行公事。我每次来讲课。每次站在观众前面之前 我都会去跑步。今天早上我跑了步。周二时我加倍努力地跑。为什么 因为这可以缓解压力 这能帮助我。总体来说是这样 对于教这堂课则更是。我还有个很重要的价值观 我和我妻子的关系。看到这电影了吗 《全民情敌》 很好看的电影。这对我来说非常重要。因此我和我妻子有例行公事。每周例行的两次约会。当别人听说这个时 他们说。"拜托 那爱的真情流露呢"。对于我们在约会中所做的事。确实有自发的真情流露。但约会的时间次数是定好的。如果我或我妻子出差旅游。错过了约会 那我们就之后再补上。为什么 因为如果没有这件例行公事。我们就不是每周而是每十年两次约会了。 

我们共有的价值观就是我们的爱情。在现代社会。我们的时代中有许多相互冲突的需求。例行公事不但很重要 它们也是必需的。如果我们要做对我们重要的。我们所关心的事。别的例行公事的例子 写感谢信。每周一次 每月一次 去拜访别人。每月一次 或者每两个月一次。把感恩变成一件日常的事。这些是我们可以建立的例行公事。并且确实有很重要的影响。我们也有例行的全家一起吃的晚餐。每个安息日晚餐 周五的晚餐。都要全家人一起吃。这是件例行公事 是件很好的例行公事。每周我们都能聚一聚。因为家庭对我们来说是重要的价值观。有时 比如说。因为旅行或者别的原因 打破了例行公事。那就之后再补上。还有一些反面的例行公事。三个小时不查邮箱。我知道这很难 但这是很重要的例行公事。我们会讲到总是查邮件的影响。因为当我们持续和技术相连时。就经常会与生活中重要的事脱节。假期回来后我们会讲这个。 

例行公事 建立例行公事。人们听到例行公事时的。恐惧之一在于 他们会说。"这会降低我的生产力"。"就算不降低生产力 也会降低创造力" 并不是这样。事实恰好相反。如果看看…… 这是个历史性的研究。如果看看那些伟大的艺术家 不管是作家。海明威 或者是达芬奇。他们生活中都有例行公事。这例行公事就是 比方说。"早上七点到十点 不论如何我都要写作"。"晚上六点到十点 如果我习惯在夜里工作 我要画画"。他们有例行公事。而正是因为这些例行公事 他们才能创造。因为那样他们就能专注于主题。而不是万千使他们分心的事物。"也许我该干点别的"。"有些让我分心的东西" 不 有例行公事在。我现在就该干这个。几乎无意识地就去做了 就像我们刷牙一样。但一旦他们开始专注 就有了空间。有了创造的空间。 

现在关键在于花时间来建立例行公事。你可以看到 保持例行公事。需要一定的自律 但并不多。建立例行公事需要大量自律。因为我们时常会回到旧习惯。做个快速练习吧。抱起你们的双臂进行交叠。交叠一会儿。好了 继续交叠双手 不过这次。用不同的方式。也就是这只手换到下面 这只换到上面。对 对。我可以看到男生都不太懂。不过没关系。感觉如何 不是很舒服对吧。但这只是非常简单的一件事。交叠双手。我们仍然能感觉到很不舒服 有点不对劲。我们想回到原来交叉双手的方式。舒服多了。我们塑造习惯 习惯反过来塑造我们。所以关键在于。我们这还只是做了很小的改变 -交叠双手。那要改变那些我们养成了许多年。对我们的生活更重要的习惯。会有多么难呢。要改变它们非常困难 这需要时间。 

William James 我之前提到过。说改变习惯需要21天。Loehr和Schwartz在他们的《精力管理》一书中。说需要三十天。当然要看要改变的习惯是什么。给自己三十天开始学会感恩。给自己三十天。来建立经常锻炼的例行公事。关键在于。每三十天内不要建立超过两件例行公事。为什么 因为正如我们之前说的。建立例行公事需要大量自律。而我们的自律是有限的。这也就是为什么我们今天列出十项。想要做出的改变 但最终什么也没变。最终什么也没变。因为我们过度消耗了自律。最终崩溃了。开始一件例行公事需要大量自律。但一旦开始 当一个月后它已成为习惯。已经铭心刻骨 神经通路已经建立时。我们就可以建立下一件例行公事了。如果准备好了的话。 

不要超过两个 最好是一个。下周你们交的论文中。就要选择一样例行公事。这个月的两件。你们可以写一下。因为没有行为变化 也就不会有改变。Loehr和Schwartz说。"逐步的改变好过雄心勃勃的失败"。"成功建立在成功之上" 可以从个人水平上看到。可以在组织水平上看到。商学院的John Carter做了很多努力。来改变领导层的架构。他经常说你需要小的成功。然后逐渐积累成最终的成功。达赖喇嘛说"唯有长期坚持和熟悉"。"方能使事情变得容易"。"通过训练我们可以改变 可以重塑自我"。 

你们知道。我还在读大学时发生过一件事。是在我大二的那年。我没有…… 我当时在校队。没有在赛季里有很好的表现。我的大二简直一塌糊涂 还受伤了。我迫不及待地等三月一日。三月一日是……有时是三月二日。是赛季的最后一天。我迫不及待等那一天是因为那之后。我就可以不打壁球 专心学习了。我就可以做很多事 因为打壁球时。在球场上每天都要训练。两到三小时。周末我们经常要外出或在这里打比赛。早上 每周至少两个早晨。我们要健身或者举重。这很难捱 积了一大堆作业。我等不及获得没有壁球的自由。尤其是当时我还不是那么爱打壁球。等不及不再打壁球 专注于我的学习。追上阅读 写作之类的进度。三月一号到来了 我彻底不再打球。但与高效学习相反。我反而效率变低很多。 

我看到你们很多人点头。运动员们肯定都有同感。为什么 我的效率取决于什么。拖延 哇!突然升高。为什么 因为许多年来大家都跟我说。我是一个自律的人。而我也开始相信这一点。对 我确实很自律。看看我 我刻苦地训练 训练。每天像杂耍一样地训练六个小时。那可是非常的自律啊。然而突然间 壁球赛季结束了。我说"好吧 看看这自律的作用吧!"。但我却什么也没做成。为什么 因为我的例行公事被打破了。在赛季中 你们那些。在校队的 或是投身于音乐或社团的就知道。当你在特定组织里投入很多时。你的时间必须例行化。那样你就知道你在什么时候该训练 或学习。或者什么时候该开会。那之后你就回去学习。因为你只有两个小时。而你又想睡个还算安稳的觉。你有这样的例行日程 就会高效并具有创造力。突然之间这些例行公事都消失了。你说"好了 现在我要更努力学习了"。但恰好相反的事出现了 因为我们只有有限的……。直到我意识到。我的自律是有限的 这个事实之后。直到我明白人类本性……。本性就应该被遵循。那时我才真正开始变得高效率。因为那之后我又建立了新的例行公事。例行公事 非常重要且有意义。做出持久改变的唯一方法。现在来谈谈认知吧 

首字母C的这个词。我们已经谈过情感 也谈过行为。谈过它有多重要 既是剧烈的也是逐渐的。接下来让我们来谈谈认知。再次谈谈认知重建。也是造成改变的渐进方式。然后我们再谈谈造成改变的快捷方式。也就是"有了!"的体验 灵光乍现的一刻。首先 认知重建。正如之前说过的 诠释是一种神经通路。如果我像个消极者一样诠释世界。我大脑里的神经通路就会是消极的。生活中的体验。我就会消极地进行诠释。然后随时间不断加强。如果我是个积极者 对相同的经历。就会有非常不同的诠释。因为我大脑中的神经通路就非常不同。当然了。我们如何诠释经历是有不同结果的。不一定会有最好的结果。但作为积极者 抱歉我换了个说法。作为积极者 更能看到。每段人生经历中积极的方面。 

回忆一下同卵双胞胎的研究。以及其中基因到底有多大影响。那只能解释55%的差异。这是很多心理学入门教材中会提到的故事。是关于双胞胎 在同个家庭中长大。他们的父亲对他们及其妻子都很暴力。他经常酗酒 吸毒。非常非常可怕的童年 所能想到的最悲惨的童年。这对双胞胎与父亲一起在这个家庭中长大。然后他们离开了 离开家。从这个家里逃出去 然后到了三十岁。在他们三十岁时 一个正在。对双胞胎进行研究的心理学家拜访了他们。他先去找了双胞胎里的第一个。他看到的是 这个孩子结婚了。虐待家人 经常酗酒 吸毒。然后心理学家终于找到他。有那么一刻清醒着 然后跟他说。"怎么了 你在干什么 发生了什么"。那个孩子知道。这心理学家在做关于基因的作用。以及成长环境的作用的研究 他说"你认识我父亲"。"你知道我经历了怎样的童年"。"你还想让我变成怎么样呢"。那个知道成长环境的作用的心理学家耸了耸肩。然后他去找了双胞胎里的第二个。第二个孩子和第一个孩子正好是。一样的年纪。他去找了第二个孩子 也是三十岁。他走进他家 不敢相信自己的眼睛。如此和谐 如此平静 如此充满爱。充盈在他和他妻子 以及他孩子之间。他事业有成 家庭美满。他后来过了段时间再去找他 因为这可能只是个巧合。也许他只是装出来的 但不是 这是真的。事业有成 家庭美满。他找到这孩子 难以置信地问"为什么"。这个孩子也知道他的研究内容 他说。"什么'为什么' 你认识我父亲的"。"你知道我是怎么长大的 你知道他对我们做了什么"。"你还想让我变成怎么样呢"。"我知道他对我们的伤害有多深"。"你还想让我变成他那样吗"。 

同样的经历 同卵双胞胎 同样的基因。迥然相异的诠释 一个延续了。他儿时的地狱 另一个则创造了天堂。都是因为诠释的不同。应该怎么做 我别无选择。我就是这样长大的。那就是榜样 被动的受害者说。另一个则是 我不要变成这样。我不要像我父亲那样 主动的创造者。这都是因为诠释的不同。快乐 幸福 正如我们之前多次提到的。并不那么依赖外部的条件。不在于我们的地位。或是银行账户的状况。而是在于我们内心的想法。你们必须记住的是。很不幸 幸福是没有捷径的。很不幸没有简便的方法。如果有 我保证我会告诉你们的。如果我下周找到条捷径。即便是在春假之后 你们一定会收到我的邮件的。但我不觉得会有的 

通常。寄希望于找到捷径通常导致更多的不快乐。举几个认知重建的例子。这是由Tomaka进行的研究。关于我们将活动视为挑战还是威胁。因为对同一种活动。我们可以从认知上重建对其的理解。我给你们举个例子。当我…… 被剑桥拒掉之后。我当时申请了博士项目。这里是我申请的其中一个地方。我很希望能入选。我进来之后 我想回来是因为总体来说。在这里读大学时我挺开心的。即便有很多困难和困扰。我仍然很愉快 很想回来。在我进来之后。我突然开始有点担心了。因为我说。我大学时经历了很多焦虑。大学时经历了很多焦虑。我说 我不想再过那样的生活了。也许我应该去点别的地方。然后我回过头来 说。"好吧 哈佛不是对我的冷静心态的威胁"。"相反我要把它当成一次挑战"。这挑战变得很明确。我在日记里这样写。明确地想过并写下来。我博士的六年里 目标是要保持冷静。因为我对自己说 "如果我能在哈佛保持冷静"。"就可以在任何地方保持冷静"。然后我朝着这个目标努力 非常努力。这确实变成了挑战 而正是这个改变使我。从在这里的研究生经历获得了更多。即便遇到困难。各种各样的挫折失败 以及焦虑。但总体来说 还是很好的经历。 

我最近也做了这样的事。有时我刚下课就会去别的地方。三周前我去了佛罗里达。在那里进行了一次重要的谈话。是和一家我非常想进的公司。那是我的第一次会谈 是我梦寐以求的公司。我觉得这家公司做的事很不错。我非常希望能够顺利。在谈话之前我非常焦虑。然后我对它进行了认知重建。我说"好吧 这确实让人很焦虑"。"准许自己为人"。"把它当成一次挑战"。我有这么好的机会能和。这么棒的公司 这么棒的人对话。我一定得好好珍惜。将它从威胁转化为挑战 转化为机会。这对我起到了极大的影响。 

想想你们自己。有没有这样的事 是你要打的一场比赛吗?是你想约出去的人吗 是在公众面前说话吗。这有个学心理学1或者。心理学15的同学会学到的研究。由Schachter和Singer在60年代早期做的。今天伦理道德委员会肯定不会批准的。就像Milgram的研究肯定不会被批准一样。他们所做的实验如下。他们让人们参与实验。给他们注射肾上腺素。肾上腺素可以使机体兴奋。注射肾上腺素。但受试者以为打的只是维生素C。他们不知道这是肾上腺素。然后他们就坐着。等待"真正的"实验。 

他们坐在休息室里。正当他们等着时。试验者要求他们填一份调查问卷。记住他们刚被注射了肾上腺素。但他们并不知道被注射了肾上腺素。于是他们就坐着填调查问卷。在第一种条件下。问卷里有很刺激性的题目。可以这样说。比如说 其中有一道是这样的。"你妈妈和你爸爸结婚前。和多少男人上过床"。这是问卷的其中一道……。他们的研究放在今天一定不会被批准的。但那是在六十年代 所以他们填了调查问卷。而在他们旁边 旁边有一个同谋者。他们不知道他是实验的一部分。而这个人暴跳如雷。"他们怎么敢这么问" 非常非常生气。于是你也变得很生气。你甚至比对照组更加愤怒。对照组经历了同样的事。但他们并未注射肾上腺素。所以受试者看到自己。看到自己的机体非常兴奋。而他们将这种兴奋诠释为"噢 我一定是太愤怒了"。而他们确实变得更愤怒了 比原本应该愤怒的程度更高。对照组也很愤怒。但没有那些注射了肾上腺素的那么愤怒。 

而第二种条件下 同样接受了注射。做了个没有什么刺激性问题的问卷。他们旁边也坐了个同谋者。他们也以为他是来做实验的。这个同谋者偶然发现。地上有个呼啦圈 然后就开始摇呼啦圈。跳啊摇啊……。接着就有点疯狂了 他们很高兴 都在笑。而刚被注射了肾上腺素的人变得很疯狂。十分欢腾愉悦 比对照组快乐多了。对照组也和摇呼啦圈的人共处一室。只是没有注射肾上腺素。也就是说 他把肾上腺素的升高诠释为。"噢 我一定是太开心了" 而他们确实变开心了。换句话说 经常都是由于诠释的不同 在这个试验中。对某种生理症状的诠释 决定了我们的感觉。愉悦或愤怒 因为它们非常相似。两者都涉及肾上腺素的急剧升高。所以我们是怎样把一个情况诠释为兴奋。或者愉悦 或者愤怒的呢? 

这里还有另一个研究。是由Lee Ross及他的同事做的。他们邀请大学生…… 他来自斯坦福。大学生说出他们最慷慨。最善心的朋友 以及他们最好胜。最凶狠的朋友 然后告知他们是谁。然后他们联系这些人作为实验的一部分。他们想知道的是在一个可以选择合作。也可以选择竞争的游戏中 他们的行为是怎样的。所做的干预是将这些学生。随机地分为两组。在这两组中。有被认为很好胜的人。也有被认为很慷慨大方的人。在另一个组中。也是一样 一半人慷慨大方。另一半人非常好胜凶狠。至少他们的朋友是这样觉得的。第一组要玩一个游戏 这游戏叫做。"社区游戏"。第二组也要玩一模一样的游戏。游戏中你可以合作也可以竞争。但不叫做社区游戏。即便是一模一样的。相反被叫做了"街头游戏"(比较暴力的游戏总称)。他们想看看有多少人会合作。有多少人会竞争。他们也想看看是什么决定了这选择。 

不论这些学生 受试者。是被认为合作性或者竞争性的 都和结果没关系。毫无关系 他们是选择竞争还是合��。真正决定一��的是。他们玩��是"社区游戏"还是"华尔街游戏"���如果他们玩的是"社区游戏"。他们更倾向于合作。如果他们玩的是"华尔街游戏"。不论他们是慷慨善良。还是好胜凶狠。都会更倾向于凶狠而具竞争性。换句话说 如何表述一个境况。"社区"或"华尔街" 会起到很大的作用。威胁 机会 也许也正是如此。我们如何表述决定了所有区别。这个实验是由…… (部分丢失)。一个很重要的问题在于。我们如何提高志愿者精神 不管在哈佛还是别的地方。她想到的一个很棒的答案就是。使学生们及社会上的人。重新审视志愿工作。不要把它当成义务 我必须做的事。为什么不能将它视为一种荣幸呢。我很荣幸能帮助别人 这确实是种荣幸。 

还记得对善良的研究吗。幸福的最大来源之一。给予和帮助确实是种荣幸。如果人们这样重新审视。他们就会更愿意做志愿者。这对抚养孩子有意义。当然 对教育也有意义。对社会整体有意义。这是我即将深入讨论的一个话题。我会花很多时间讨论情感关系。不过先简要地提提 这个理解改变了。真的改变了我和我妻子的关系。它改变了。它正在改变我和我朋友的关系。以及我的学生和同事。我们卷入一段关系。大多数人觉得情感关系的重要之处在于。我们可以得到认可 有人能在背后做支柱。这在我们的关系中确实是重要的。得到认可很重要 在任何关系中都是如此。不管是师生之间 朋友之间。当然也在爱人之间。 

但是David Schnarch想说的是。如果我们想要长期的 成功的 健康的。有激情的关系 那首要目标。最基本的目标是通过关系被了解。被了解 而不是得到认可。也就是说 你想着。"我的伙伴如何才能更了解我呢"。当然了 要循序渐进。在第一次约会上 你肯定不会想说出所有秘密。循序渐进。想着这一点 慢慢地敞开心扉。能够逐渐逐渐敞开心扉的夫妇更能够。维持他们的关系以及激情。我还会就此讲得更多。我会用至少两节课来讲情感关系。这将是。成功的长期关系的中流砥柱之一。 

对于学生 也是一样适用。我刚开始教书时。我非常希望得到学生的认可。我如何能使学生觉得我是个好老师呢。我希望他们喜欢我。这对每个人来说都很重要。我们都希望被喜欢。但当我转换注意力 对。我仍然希望被喜欢。但我主要的注意力在于我希望学生了解我。我希望他们了解我对这世界。最有兴趣和激情的事物。这确实改变了我很多 不再是得到认可 也就是说。变得完美 而是作为一个人被了解。当然了 准许自己为人。而这实际上大大地促进了我的教学。我也更喜欢教课了。因为我们之间的关系不再有那么大的压力。不管是什么关系 只要我们抱着被了解的愿望。重在表达而非使人记住。我们会如释重负 而更好的是。这使得一段关系更健康。 

我还会讲得更多 因为这非常重要。是非常重要的话题 我们经常讲这个。我们如何理解失败 绊脚石?灾难? 还是作为一次机会 或成长的经历?这都会有很大的区别。春假之后我们会再学习。我们会讨论完美主义以及对失败的恐惧。 

最后讲一下。你们读到的Ali Crum和Ellen Langer做的研究。先给你们讲点这个研究的背景。我会简要地提一提 希望你们都已经读过了。Ali Crum从大一起就是我的学生。我是她的助教 那时我和Phil Stone一起教学。然后我和Ellen Langer是她的毕业论文导师。这是她的毕业论文题目。当Ellen Langer提出这个想法时。她告诉我和Ali。"我觉得这会是个有趣的研究"。我在会后把Ellen叫到一边。在Ali走了之后。我跟她说"Ellen我觉得这样不公平"。"我觉得让Ali用这个做毕业论文不太公平"。因为这是个非常困难的实验。如果你们没看过的话 一会儿就会听到了。"就算我们花费她大量的时间。"她也不会有任何成果"。现在你们的毕业论文可以没有结果。没有成果也没关系。但我说"干嘛浪费她的时间呢"。她跟我说"会有结果的" 我说"不会的"。而最后真的有结果。 

我从那时起就学会了不和Ellen Langer争辩。因为她总有你觉得不可能有结果的想法。但他们确实很喜欢我们1979年的。"回到1959"实验 或是那个视力测试。只是飞行模拟器提升了你的视力。就是这些看起来很无厘头的想法 她证明它们是正确的。这个研究也是如此。这个实验是 Ali去到酒店。和酒店的清洁工一起工作。她分别去跟两组人一起工作。跟两组人都说了锻炼的重要性。她说。"这是你们酒店搞的。只是为了让你们知道锻炼有多重要"。然后她测试了她们的各项生理指标。包括身体脂肪 血液样本。她们血液中有多少脂肪 体重。还有心理指标 沮丧 焦虑 诸如此类。而她所做的是。对于其中一组 她测试完后就不管她们了。而另一组。她对她们说--这就是干预手段。"你们所做的工作实际上就是种锻炼"。她计算了收床单所消耗的卡路里。抖床单 然后铺在床上。用吸尘器清洁要消耗多少卡路里。她估算了这所有工作。然后她给她们看了每天锻炼。所消耗的卡路里的数据。然后她说"你们做的其实正是健身"。这是实验的干预 然后她两个月之后再回去。 

两个月后 她又进行了同样的测试。她发现的有些结果。其中有些结果是预料之中的 有的则不是。血压显著下降。血脂显著下降。体重 两个月后。实验组而非对照组。对照组期间完全没有变化。体重显著下降。她们的自尊心上升 沮丧程度下降。焦虑水平下降 精力水平上升。所有都是这样。此时她问"你们做了什么不同的事吗"。"或者你们比以前做了更多的锻炼吗"。对照组与实验组之间。不存在区别。唯一的区别在于观念。她们可能工作得更努力了 我们不得而知。可能只是心理作用 我们不得而知。但事实是由于她们重新审视了。重建了她们自己的经历。从"我每天必须清扫三十间房间"。到"这是种锻炼 这对我有好处"。这就是区别所在。既是生理上的也是心理上的。这是急剧转变的一个例子。这就像一锤子砸下去的变化。这就是"有了!"的经历。有很多关于"有了!"的经历的讨论。研究 以及对这个领域的兴趣。灵光乍现的一刻 顿悟--如此珍贵 如此重要。而对此也有许多误解。因为人们都觉得这种顿悟是突然产生的 但其实不是。这是有一整个过程的。而这个过程从潜移默化开始。这是我们为顿悟准备的时候 是我们学习的时候。 

Howard Gardner做了许多研究。关于全世界的成功人士。他发现一般来说 对于那些。在某个领域成为专家。在某个领域具有创造力的人。他们都会下至少十年的苦功夫。这就是准备。这就是你将自己沉浸于那事物之中。看看贝多芬的例子吧。如果你们听过他的音乐 知道他的音乐轨迹。第一和第二交响曲 莫扎特 和莫扎特非常相似。第三交响曲 英雄 就是贝多芬了。他使自己沉浸于当时的音乐。研究它 学习它 然后许多年之后。就能独创自己的风格 具有创造力。以及改变整个音乐领域。作为浪漫主义音乐第一人。第二个阶段。在准备好之后 这也需要下很多功夫。无论是比尔盖茨所做的准备。即便他是在翘课 还是比尔·克林顿。在你成为某领域专家或具有创造力之前。都要做大量的准备。才会有顿悟的一刻。第二个阶段 孵化。在所有具创造力的人身上都可见。在你将自己沉浸于事物中之后 你什么也不做。你就是沉浸于其中。比方说 并不是巧合。我们许多好点子都出现在洗澡的的时候。阿基米德在泡澡时有了伟大的想法也不是巧合。莫扎特曾经每天花好多个小时 在萨尔兹堡晃荡。然后突然他就顿悟了。他说"我突然能听到交响乐"。突然他听到了所有。这是莫扎特的故事。但你也听说过莎士比亚会花很多时间驾着马车。到处晃荡 然后突然顿悟。"故事就应该是这样的"。然后他就可以去写 孵化是非常非常重要的。 

摩根说:。"我可以用九个月内完成一年的工作 但不能用十二个月"。J.P.Morgan他知道 你们知道。他可能是美国史上最成功的企业家。他知道休息对于创造力的重要性。当今的商业精英 领导者们都休息得不够。因为我们觉得这是浪费时间。我们认为如果只是坐着什么事也不干。我们就确实什么事也没干。但不是的 思想在工作。思想在工作 并且思想需要这样 才能获得顿悟。事实上 领导者。你们中的很多人都会变成领导者。比任何人都需要休息。因为他们需要创造力。他们需要思考公司的未来。机构的未来。适当休息是无价的。不只是对记忆力 对于创造力也是一样。我工作时做过一件事。作为顾问 就是。我会安排一段没有日程安排的时间。我们就出去玩 而时常。在短短半小时或一小时内 各种各样的点子就会冒出来。为什么 因为他们第一次有时间思考。我们常常一觉醒来就想到解决办法 这并非巧合。因为潜意识正在活动。 

Joseph Campbell说。"你必须有个私人空间 或者一天特定的时间段。此时你不知道今早报纸上讲什么。不知道你的朋友是谁。不知道欠了别人什么。不知道别人欠你什么。这才是你能够静下心体会。感觉到你是谁以及可能是谁的时候。这才是创造力孵化的时候。一开始 你可能发现什么都没有发生。但如果你有一个这样神圣的时刻 并且好好利用它。奇迹终将会发生" 很经常。就在这孵化的时间里 在这些休息时间里。在这你有机会反思自己的时间里。不管是写论文。还是出去玩 听音乐。灵光乍现的时刻通常都出现在此时。顿悟的体会 再一次的"我听到了交响乐"。 

某种程度上来说。你可以对此进行一个类比。你可以想想第一个阶段。或者你可以当做性交流来想想。比如做爱。准备和孵化就好比前戏。顿悟的时刻就是高潮。这个过程中要理解的很重要的一点是。你必须经过准备和。孵化才能到达顿悟的时刻。换句话说 必须要有前戏。男同学们 你们听见了吗。必须有前戏才能进入下一步。这很重要 这是创造力的过程的一部分。这意义重大 对于爱情来说意义重大。对于这个过程来说同样意义重大。而这之后。在你顿悟了之后 就是该进行评估的时候。你要问自己"这是个好主意吗 会有用吗。还是只是…… 你知道的 女人……"。很多主意 很多顿悟的体验 最终都没起作用。因此评估非常重要。这确实是好主意吗。这确实是个好故事吗 莎士比亚会这样问。这确实可以作为哲学命题吗 笛卡尔。这个花大量时间自省的人会这样问。这确实可以作为哲学命题吗。这是个好的商业计划吗 这会是下一个Facebook吗。或者还是只是我……。凌晨三点想到的。有人已经做过或者根本不会成功的主意。评估非常重要。世上有很多主意 但不是所有的都有用。回到做爱的类比上。有了第一个阶段 -前戏。有了高潮 然后之后。就有了评估时要问的问题。歌曲《你明天还会爱我吗》 The Shirelles合唱团。这是持久的财富吗。还是只是短暂的愉悦。我能相信你眼神的魔力吗。你明天还会爱我吗。现在就告诉我 我不会再问 亲爱的。你明天还会爱我吗。你明天还会爱我吗。你明天还会爱我吗。 

你知道她们是谁吗 不是问家长们的。同学们 你们认得出她们吗 The Shirelles。The Shirelles合唱团 我最喜欢的组合 真的。我女儿的名字 你们有些人知道 就叫Shirelles。我们想起一个。既在希伯来语里有 也在英语里有的名字。我妻子就想到了Shirelle。希伯来语的意思是"上帝之歌"。还有个The Shirelle合唱团。所以我们发现这是。这是最合适的名字了。你们知道我为什么这么喜欢她们吗。你们可以看到那个范儿 舞蹈的姿势。我喜欢那个时代 非常喜欢。总之 登录youtube 搜索The Shirelles。你们就能看到她们的其他歌了 这就是评估。 

马上就讲完了 这就是评估的时候。这会是持久的吗。是真实的吗。绝妙的主意? 完美的关系? 比方说。在那之后。这倒是没法和做爱类比 你得详细阐明。详细阐明那些主意。你要写出来 把交响曲写出来。你这时要把那个命题写出来。你这时要制订出来商业计划。我教这堂课 就经历了这整个过程。从准备开始。即便我之前做了充足的准备。我学过两遍基本知识。和Philip Stone教授一起教学。我是他的助教 两次 助教。即便我学了很多年社会心理学……。我会抽出点时间。将我自己沉浸于积极心理学的材料中。我读了积极心理学手册。这是个非常好的武器。如果有人接近 你可以用这个砸他。大概有这么厚 但是是很不错的书。 

我读了成百上千的学术文献。然后我就抽出点时间。在那段时间 我谈论积极心理学。但我不对此下太多功夫。我和我妻子聊 和Phil Stone聊。和我哥哥聊。正是这时灵光一现 正是这时我顿悟了。我说"这门课就应该是这样组织的。这就是那根螺旋" 然后我想到了PPEO。"峰值后体验" 这样的灵光一现。之后我就对其进行评估 我评估自身。我通过和他人交流来评估 看看。它是不是真的有用 然后我坐下来。花很多很多的时间把课的内容写出来。每堂课都写了出来 显然我没有照本宣科。但在准备时。在我准备上课时 我确实会看几遍。那就是详细阐明。这就是创造的过程。不管是准备一门课。或者运作一个机构。写本好书 或者发展一段良好的关系。 

下周见 周末愉快。 

第12课-写日记 

早上好 我叫Dana 我是大一的学生。我想跟你们说说。下周会在哈佛进行的一个社会项目。我爷爷是二战大屠杀的幸存者。他将要来哈佛讲述他的亲身经历。我想这会是一次非常特别的机会。让我们听到历史上一个感人的故事。他在其他大学和联合国做过演讲。他的故事挺扣人心弦的。演讲会于七点三十在纪念堂举行。下周的这一天 3月18日 周二。想了解更详细的信息 请查看Facebook网页。"在奥斯威辛集中营中幸存:一位幸存者的故事"。你们可以谷歌…。或在哈佛的网站上查找二战大屠杀。希望大家能够抽时间来听听。如果有问题请联系我 我叫Dana。我会随时回复 我是这活动的主办人。谢谢大家 谢谢Tal。 

大家早上好 今天。我们会先讲完"改变"。总结一下ABC 也就是影响 行为和认知。然后继续讲另一个相关课题 设定目标。这将是最后一个…。在放假前我要讲的课题。上次讲完了急剧变化。由探索而得的经验引起的急剧变化。这种经验来自顿悟感和洞察力。我们讲过五个阶段。之后有些学生联系我说。"并不是很顺利" 没错。这是一个理论 只是一个大纲。你写作的时候 当你思考…。或设定课程。或是写论文时。 

你们当中有人在写论文。你们常犯的错误是将它们混淆了。通常你们会先准备 然后写一些。之后休息一下 会有一些顿悟。再开始作更多准备等等。这只是一个轮廓 某些角度来看是很有用的。首先。我们光看着它就能明白。没有捷径。历史上所有的艺术家 科学家。商界名人都不是凭空出世的。他们首先得成为各自领域的专家。他们努力工作 将自己沉浸在知识里。这是你们首先要记住的一点。没有捷径 我们得先付出努力。还记得成功的秘诀吗?不光是思考 想象和相信。必须要有努力和热情这两点要素。所有的成功人士。再次提到 做好的自助书籍是自传。好的自传会教你如何成功。而不是那么简单的"五步"。第二个重要因素 比投入到准备过程…。更重要的是 空闲时间的重要性。今天的领导人所缺的…。他们所缺的其中一样东西是时间。时间是他们最需要的。有空闲时间去筹划 反思。或只是坐在浴缸里思考。熟虑 发呆的时间段。记得J.P.Morgan说过。"我可以用9个月而不是12个月来完成一年的工作"。那是要有所顿悟或灵感出现为前提的。这些灵感来自一夜好觉之后。或是充分休息一段时间之后。或是在洗澡过程中。下一个重要因素是。这个模型的组成部分是评估部分。人有很多好的想法。但极少数是可以有好成果。最终成为经商的好点子。或成为科学论文或书籍的好构思。或其他任何有作为的东西。这时候就需要评估。需要详细筹备。这也就是创作的必经阶段。 

真的需要百分之一的灵感。百分之九十九的汗水 没捷径可言。所以总结起来:ABC 也就是影响(情感)。行为和认知。我想谈一个小小的技巧。一个简单的方法 就是把有深刻。影响的事情记录下来。请大家举起手。多少人有长期写日记的习惯?好。这么多人有这个习惯 是件好事。从这个课程和大家之前。接触过的研究中知道 改变是困难的。有的项目持续五年了还是没有进展。通常投入的资金是数百万 数十亿。大概每年投入巨额资金来做研究。才能对各行业和机构有所改变。很多的资金投入都浪费了。说明改变有多难。而日记呢? 

有实质的研究结果证明它的作用很大。我跟你们分享一下某项调查。这项调查是由…。那些将日志带进科学领域的人来做的。有些像Ira Progoff这样的人。长期以来一直在讨论写日记的作用。但却是德州大学的Jamie Pennebaker。将这个想法带进了科学领域。并真正意义上研究它 他是这么做的。他选了一些参与者 并让他们。做以下的事情:连续四天。每天都用15分钟的时间 就15分钟。来写下最难忘的经历。他们知道其实没人会看的。就算被人看到。这些日记也是匿名的 绝对保密。所以完全可以看成是他们的私人日记。 

以下就是他们得到的指示。有点长 但值得一读。"连续写下你一生中最难过。或最痛苦的经历"。不必在意语法 拼写或句子结构。在日记中 我希望你能谈谈…。你对这经历的最深刻的想法和感受。写什么都行。但不管你选什么。都必须是对你有着深刻影响的。最好是一些…。你从未怎么跟别人讲过的事。其实挺难的 因为敞开了心扉。去触碰那些你内心深处的感情和思想。换句话说。就是写下你的经历和以前的感想。以及现在对它的看法如何。最后 你可以每次都写不同的痛苦经历。也可以整个研究过程都写同一个经历。每次你可以选择任何想写的痛苦经历。很简单的做法 很直接。 

如果有留心 就知道当中包含了ABC三个要素。它包含了"写出发生的事"。也就是行为。写下你的深刻感受 受到的影响以及你的情感。写下你的想法并分析它。这样 日记里就包含了ABC三个因素。 

他做了这项研究后 最初结果出来了。他首先看到的是焦虑水平。因此看到结果时 他失望了。他考虑要不要停止这项研究。因为这就是他的研究所得。大家看这幅图 这是焦虑水平。这是时间。他有一个对照组 对照组的成员。只是写下他们想写的东西。随着时间推移 焦虑水平没有改变。从这里开始 一条直线过来。而干预组 实验组的成员。经过四天 写下他们最痛苦…。最难过的经历后。他们的起点一样 当初分组是随机分的。他们的焦虑水平竟呈现上升趋势。每天这么写 在这四天内。他们的焦虑水平上升了。到了这个位置时。他考虑终止研究 因为非常失望。因为他自己觉得…。写日记对他有帮助。而也有其他人说过这一观点。四天之后 从第五天开始。尤其是第六天 第七天之后。奇怪的事发生了。他们的焦虑水平降低了。达到了原来的水平。还持续下降。最重要的是。在原来的水平之下保持稳定。在原来的焦虑水平之下保持稳定。他密切留意着这些参加者。一段很长的时间 长达一年。四次 15分钟 足足一小时。时间很短 效果是持续的。想想 我们之前也知道。 

有时候 简单的干预是很有效果的。想想我上周说过的害羞研究。12次 每次12分钟 让男士们与对他们…。有好感的女士们独处。这些害羞的异性恋男士发生了很大的变化。所以改变其实可以发生在极短的时间内。关键是我们要知道如何干预。而日记就是其中的一种干预方式。我不会要你们写这个作业。因为太私人了。我觉得不该布置给你们。但我强烈推荐你们试一下。严格按照Pennebaker的指示。不会占很长时间。连续四天都做 每次15到20分钟。想到什么写什么。一段时间之后可以降低你们的焦虑感。 

这项研究的其他结果。如果你想了解更多。他有一本很棒的书 叫《开放》。Jamie Pennebaker的《开放》。他们变得更健康。一年过后 他们回去看医生的人数。远比对照组的人少。就是说 这项研究增强了他们的免疫系统。不光是他们的心理免疫系统。同时还有他们的身体免疫系统。总得来说他们心情好了。比以前更快乐 更乐观 更积极。在他们记下最深刻伤痛后。他们变得更外向 不那么压抑和忧郁了。为什么会这样呢 我们说过积极情感。和痛苦情感都来自同一个输送通道。如果我们抑制着某些事情 或停止一些…。痛苦的事 我们常常会间接地。同时无意中抑制一些积极的情感。只要他们随时敞开心胸。让这些情绪尽情流露。那他们其实是打开了一条闭塞的通道。一条各种情绪…。包括痛苦和积极情绪流动的通道。他们也能体验到更高层次的快乐。他们会更阳光 更慷慨。就像你做感恩练习时所得的结果一样。非常相似。 

有趣的是 会有性别差异。男女都从中受益。但男士比女士受益更多。试想一下。为什么 仔细想想 其实不无道理。受益的女士 状况大大改善。但男士受益更多。为什么?一般来说。这是千遍一律的说法。但一般来说 女士敞开得更多。她们和女伴们谈得更多。她们有密友 并对她们倾诉一切。而在今天的文化中 男士还是比较保守。因为找人倾诉有点没面子。准许自己为人 不是一件很酷的事。尤其是有其他人在旁时。因此 在生活中 女士们。比男士们更容易获得支持。但尽管如此 拥有强大支持的女士。仍能从这项练习中获益。这让我想起了另一项研究的结果。我几个月前提到过的 在课程开始时。提到一般来说 女人。我指通常情况。或者说男人 在婚姻中比女人受益更多。类似的原因。因为女士们原来就有支持系统。而男人 通常这是他们第一次遇到一个。让他们觉得能安心倾诉的人 同样。尽管男女都在长期的恋爱关系中获益。但还是男士受益比女士多。他发现了因同一个原因而产生的性别差异…。存在于不同文化当中。这项研究分别在中国 日本 墨西哥和阿根廷做过。当然也包括美国 欧洲等不同文化。不同文化的人们都从该项研究受益。 

还有另一项研究 用的是相反的方法。Laura King是Pennebaker的学生。用相反的方法来做这项研究 她说。"我们来研究一下…。"看当人们…。写下最快乐的经历时会有怎样的结果。她的做法明确地根据。Abraham Maslow关于巅峰经验的研究。以下就是做法。三次 15分钟。连续三天 总共45分钟。"想出生命中最精彩的经历…。或多次美好的经历 快乐 兴奋的时刻。或坠入爱河极度欢喜的时刻。或者听音乐时的欢快时刻 或突然…。发现一本好书或好画或突然灵感乍现"。选择其中一个这样的时刻。想象自己正经历这一时刻。沉浸在所有与这一经历有关的…。感受和情感中。尽量详细地写下这次经历。尽量写下感受和想法。及当时产生的情感。尽量尝试重新经历那些情感。这个做法和Pennebaker的做法相反。结果呢?完全一样。那些写下他们巅峰经验的人。写下最快乐经历的人 去看医生的次数少了。就是说 这个做法增强了他们的身体免疫系统。让他们体验更多快乐。不论心理还是身体上都得到同样的效果。 

现在来看 你们有些人可能会想。"那Lyubomirsky的研究呢"。还记得Lyubomirsky的研究吗?那项研究表示 当你写下积极情感时。其实你感觉更糟。而写下负面情感。都会如Pennebaker所说 感觉会好些。不同之处在于。Laura King的做法主要是…。描述和重新体验。重现你的经历。而不是分析"怎么会发生这种情况"。而是它的发生过程 就是重现。当做法…。而Lyubomirsky的研究是分析这种经历。为什么会发生 你为什么会遇上之类的。这就是为什么随着时间推移 会产生负面的效果。但只是重现经历。只写下这次经历有多美好。重新体验当时的情感。实际上会让你受益。 

有趣的问题是为什么。为什么写日记能产生如此显著的好处?我们来理解一下这个过程 当中的原理。现在有几件事情在这里起作用。例如你带着积极情感。在回想其中一次经历。你其实是在增强神经通路。你再次重现 想象它。让它像重新发生一样。就像你面前有条河。记得两周前的类推法吗 你面前有条河。流过的水越多 它就变得越宽阔。然后越来越多水流进去。这是变化的本质 自我实现。因此。为什么写下 尤其是痛苦情感。但也不只是痛苦情感。之所以有所帮助是因为紧张感。我们之前讲过Daniel Wegner的"反语处理"。而他谈到…。当我们抑制非自然现象时 通常会适得其反。比如想像一头粉红色的大象。或抑制痛苦的情感。当我们准许自己为人。我们都倾向于释放感情 放开心胸。这就是治疗产生作用的原因 就是为什么和朋友分享。与人倾诉以及写日记能有所帮助。这就是抑制或压抑的概念。 

Pennebaker所说的另一件事是关联性。他发现的其中一件事是…。那些受益最大的人 他分析了这个测试。秘密地 他对测试进行分析。从这项研究受益最多的人。都使用了一些领悟性的词语或词组。也就是说到了第三天 他们都开始写到。"现在我知道了"或者"我明白了"。"我突然明白了"或"我意识到"。日记里提及到这些词或短语的次数最多的人。就是受益最大的人。换句话说。他们已从这经历中创造出一种关联性。他们从某些事当中得到了启示。这些事情以前也许是毫无意义的。也就是说 他们围绕着他们的经历创造出故事。这经历已不再是零散的 无关的片段。现在已经是一个连贯 完整的故事。现在我可以面对它了。仔细想想 人们通常最记得的是什么?他们最记得的是故事。为什么?你能记住一个故事。是因为它是一个整体。如果我现在给你们100个随机的词。那你得花很长时间才能记住。我们都清楚 我们都做过学术评估测试。如果我跟你讲一个故事。你多数都会记得。也许不是每个词都记得 但大概意思。因为我们能将它连贯起来 可以理解。它有关联性。我能理解 能记住。而不像一些零散的 不连贯的单词。而同样的道理。我们想感受到我们的生命的关联性。让生命变得有意义。Pennebaker说"我们模糊且无可预知的世界的产物之一。就是对不能完满成功而产生的焦虑。和无法理解对一些造成痛苦困扰的缘由。的简单解释"。 

我们自然地去寻找…。事件的意义和完满性 它让我们有了对生命的…。控制以及预知能力。这再次说明了疗法能起作用的原因。因为我们从经验中创造出关联性。这就是大屠杀造成的创伤后遗症远少于…。越南战争造成的创伤后遗症的原因。因为在大屠杀之后 士兵们说出了这次经历。他们围绕这次经历讲出一个故事 虽然悲惨。但还是会有关联性 和越战的经历不同。对于越战 他们只有一些零散的记忆。并没有准许自己为人 也没有这样的社会环境。或其他途径让他们尽情倾诉。来说出一个他们可以应付。可以接受的故事。 

Pennebaker的研究大部分是以一位我们。第一天提过的心理学家的理论为基础的。那就是Aaron Antonovsky。Aaron Antonovsky。我把他看作幸福心理学的奠基人之一。如果你记得 就会想起"健康本源学"。病理模型的替代。就是那个专门讲病理学和疾病的模型。Antonovsky说。我们要关注的是健康的始源。Saluto是健康 genesis是本源。他的具体做法如下。他是个社会学家 他说。生活是艰苦的 人们都在艰苦奋斗。无论愿意与否 我们都会遇到困难。生活中 我们遇到各种困难。恋爱中 上学时 工作中。都有困难与挫折的存在。总有时候生活是艰苦的 大家都知道。然而 有些人…。对困难和挫折…。处理得更好 他们…。尽管不是没遇到这些困难。就算常遇到困难和挫折。他们仍能过上充实 满足且幸福的生活。不是指那种…。精神病患者或死人的那种永恒的快乐。而是有高低起伏 但是更高层次的幸福。或者说是一个焦虑水平比较低的生活。 

他说我们要研究这些人。不该只研究有病的人。不管是身体健康有问题的病人。就是通常在医学中用作病理学模型的病人。还是心理方面的病人-精神分裂症病人。忧郁症病人 还是心理学中研究最多的焦虑症病人。他说我们该关注健康的人 并研究他们。正是这个模型…。让他在80年代时调查了一些风险人群。"是什么使得某些人。在艰苦的环境中取得成功?"。就是因为他的这个问题 改变了一切。于是在他的研究中 他做了同样的事情。找了一些人来做调查 问他们。"为什么他们会健康?他们健康来自哪里?他们和其他人有何不同?"。 

于是他发现了关联性的作用。那些人让生活保持关联性。也就是说。他说他研究并最终确定的因素有三个。第一:。理解能力-我能理解世界。世界对我很重要 我看得到 感觉到 理解到。世界 事件 困难 挫折。生活的高低起伏对我来说是宝贵经历。第二 管理能力。我可以处理得了 能承受得了。我能利用各种内部和外部资源处理事情。而不是孤立无援。这是一种能效 一种自信。能够处理突如其来的困难。这就是第二个因素。最后一个因素是…。如Aaron Antonovsky所说 能够产生关联性的。意义性。困难的出现不是毫无意义的。我和伴侣意见不合不是坏事。因为通过这事 我们更了解对方。我们会变得更亲密。我们在之间的冲突是有意义的。我已经从中学习到东西 并得到成长。这个错误是有原因的。这件事的发生不一定是好的。但可以学习如何在事情发生后用最好的方法解决。他定义了这三个因素。并被用作之后的精神健康研究来源。根据Pennebaker和Antonovsky的说法 关联感:。是"一种整体适应性。主要包含三个方面。对生活的感受性和信心:。(1)在生命中个体感受到。来自内外部环境的压力。是明确 具体 可预测的(即领悟性)。(2)个体感受到应对内外部环境压力。所需资源是充分的。可以利用的(即可控性)。(3)个体感受到来自内外部环境的压力具有挑战性。值得花时间和精力去应对"。 

看看这三点。对写日记的人来说。我肯定能在你们的日记中找到这三点。这就是日记所产生的巨大影响。当我们真的写下我们的经历。那些艰苦的经历。就必须要有关联感。我知道 我懂 这经历是很重要的。我现在可以面对它 承受它。我刚找到了一种处理它的方法。尽管只是将它写成日记。最后 它对我的人生来说 是有意义的。它现在变得有意义了 尽管之前意义不大。所以 如果你没有写日记的习惯。我强烈推荐你开始写日记。 

我来总结一下"改变"这一课:A B C三点因素。三点是相互关联。事实上 如果我们真的想有所改变。必须将这三点紧密连接起来。为什么?因为习惯就像洪水一样。如果我们只是开一个凹角或一条小缝。是不足以养成一个新习惯。因为洪水会将它冲走。我们需要的是结合A B和C。将这些变化融入我们生活中。它们是内部相连的 举一个例子。例如那些自尊心很弱的人。低自尊心 C 认知。他们会小瞧自己。自我评价低。当你瞧不起自己时。就会比较内向 感觉没自信。更别说有动力了 也因此而产生不良影响。产生不好的情感。因此很可能会变得碌碌无为。也就是说 B是碌碌无为。通过自我认知理论。这种行为进而影响我对碌碌无为的认知。我处理不了事情 我逃避事情。这会使自尊降低。进而使自我评价更低 就是C。这导致更低落的情感。等等 形成一个向下的螺旋。直到在某些情况下 我们自我放弃了。或用Martin Seligman的话来说 就是"习得性无助"。现在 想想那些自尊心强的人。他们很有自信。我相信自己 我认为我能做好。那样会导致高水平的激励。 

记得Marva Collins是怎么做的吗?我们讲过的关于信仰的模型。自我实现预言。能提高个人动机和导致强烈的情感。我充满动力。也反过来引发更多的行动 就是B 行为。因为我做得更多 我应对的就更多。我更多地把自己置于水平线上。那我对自己的积极评价就会更高些。因此也会产生更多积极情绪等等。呈现一个上升的螺旋。也就是Barbara Fredrickson所说的螺旋。他在"拓延-建构"中谈到。自我能效时所提到的螺旋。既然涉及到这方面 就很有必要介绍一下了。可以的话 三点我都想说说。从哪里说起都可以。我举个例子 例如一个有社交恐惧症的人。一个害怕社交活动 不善与人交际的人。治疗这种症状的一种方法…。我们先从A说起 就是影响 也就是情感。其中一方面是药物治疗。直接影响我们的情绪。另一方面是冥想。同样直接影响我们的情绪。有些人比较适合第一种 有的适合第二种。这取决于他的状况有多极端。这是A 影响。另一种治疗方法 行为。通过所谓的认知行为疗法来进行 曝光。逐步将自己曝光在产生这种恐惧的刺激物前。首先 先走出家门10码。先通过想象来曝光。之后付诸实践。因为我们的大脑分不清楚…。想象和现实的差异。逐渐走出更远。随着时间推移 曝光度越来越高。直到我可以去到一个商场。并再也没有一年前那种焦虑。这就是行为治疗法。很有效 曝光的方法。最后 在治疗过程中还能介入认知。来处理不理智的想法 心理困惑。那三个M "我有将事情放大吗。我有轻视了自己的一些才能吗?例如小瞧了自己在与人相处方面的成功。我有在编造事实 想象情景吗?本可以更现实地看待的事情 我有将它灾难化吗?这就是我的想法 认知。跟你们说说我克服某种困难的做法。你们知道 我天生就很容易焦虑。 

我很容易焦虑。容易吃惊 容易被惊吓 现在好一点了。但还是会容易吃惊。以前参加壁球比赛时非常紧张。在重要比赛时常常呼吸困难。但我面对的事实是。要在观众面前讲话 非常紧张和焦虑。我决定要…。克服这种心理。我面临的最大困难不是其他 而是焦虑。那我是怎么做的呢 我先从C 认知说起。我分析了我日记中的情况。利用学校咨询机构。学习并研究了关于三个M的知识。那三种不理智的想法。将事情扩大化 缩小化 编造情景 灾难化。通过认知 我得到了很大的帮助。但那远远不够。学习了C之后 我开始学习A 影响。 

影响是通过身体锻炼来实现的。直到今天 早上我还跑过步。我锻炼身体是因为…。它能大大降低我们的焦虑水平。可以说如果我不锻炼身体 我的口才…。一定不会像现在这么好。甚至在三四周前 我儿子David身体不适。周二早上我要先带他去看医生。没有时间锻炼 我那天讲课就感觉到了。我觉得更紧张 更焦虑 明显不一样。像我们开学时讲的一样。锻炼的效果基本和吃抗焦虑药效果一样。因此锻炼对我的情绪产生直接的影响。瑜伽 在我处理焦虑和…。降低焦虑水平上发挥了很大的作用。这是情感上的。 

另一样是音乐 放松的音乐。你们知道我喜欢什么音乐 就不重复了。我和Tammy结婚后 就搬到了一起住。她简直不相信我收藏什么样的CD。她说"你看起来不像这么容易焦虑的人"。因为我买的CD是…。"世界上最放松的经典音乐"。"放松音乐" "降低焦虑的音乐"。全是些安静的音乐 整个架子都是。但它们的确对我有帮助 让我得到放松。 

最后是行为 我介绍一下行为。通过曝光来治疗恐惧 先是做演讲。因为我想当老师。我先给壁球队演讲 给我的家人演讲。反正是在那些我感觉安心的地方。给任何愿意听的人演讲。从开始的少数人慢慢增加听众数量。通过想象来曝光。再强调一次。那种想象久而久之会让人变得更自信。另一样对我极具意义的事是。作为大学生 在课堂上公开发言。我不敢公开发言。当我决定那样做 我肯定。所有人都能听到我的心跳声。因为它就在我脑海中。然而 逐渐地。我慢慢地越来越敢于发言。通过自我知觉理论。情况变得越来越好。这些事情我一直在坚持做。我还在做瑜伽 定期做运动。我还有参加认知行为治疗。我一直有写日记 内容主要和上述三件事有关。我还会听放松音乐 很有用。改变并不容易 非常困难 需要很多时间 然而。但那并不代表 这过程本身没有乐趣。并不是说改变了就会快乐。这过程本身就非常值得。有时很困难 但非常值得。那旅程本身就跟目的地一样重要。关于改变的另一要点是。这话引用自一个人的作品。她研究精神分析学。并把这门学科提升到了积极层面 Karen Horney。 

我之前提到过她。她认为 神经症精神病。不可能完全治愈。她所说的神经症。是广义上的定义。举个例 如果我是个完美主义者。我心里会有一些完美主义思想。春假后 我们会谈这问题。如果我有焦虑倾向 我永不会成为…。达赖喇嘛 不可能 因为我永远都会焦虑。它随时可能被任何事激发。她说"没关系 那很正常。这是人之常情"我们要接受它。因为如果我们不接受它。我们就会一直感到沮丧。因为我们想完全改变。或是期待我们的伴侣完全改变。这非常困难 不切实际 需要时间 要循序渐进。我们也需要学习 可以学习享受这过程。19世纪40年代梭罗说过:"我没有看到过。更使人振奋的事实了。人类无疑是有能力来有意识地提高他自己的生命的。 

能画出某一张画。雕塑出某一个肖像 美化某几个对象 是很了不起的。但更加荣耀的事是能够塑造或画出。那种氛围与媒介来 从中能使我们发现。而且能使我们正当地有所为。能影响当代的本质的 是最高的艺术"。再强调一次 重要的是过程而非结果。结果会让人有短暂快乐。但很快就会回归到基础水平。是进行这些活动的这个过程…。事件A B C这些过程让人更快乐。而并非完美 并非最快乐 而是更快乐地生活。需要时间来雕刻出生活的雕像。切掉多余的石头。破除限制 打造美好的生活。这不仅是最高形式的艺术。我认为这还是高级的科学。那就是心理科学。 

现在我继续往下讲。等等 好了。我想探讨一个论题。关于整个改变…。关于整个改变过程的论题。那就是设定目标。现在 我问你们 这里有多少人…。如果你是这样的就举手。你是否想变得更有效率 更少拖延。是的话 请举手。一 二 三 四 五 六…好了。现在举起你的手。如果你希望自己没那么紧张 更冷静。无论是不是要期末考 好。一 二…好。如果你举过一次手。就该留下来。如果你两次都举手。就不仅要留下来 还要打起精神来听。因为我们接下来的两节课。要谈的是 目标的重要性。以及目标如何帮我们处理压力。如何帮我们处理拖延问题。我们如何变得更有效率 不是超级有效率。 

两周之后。你们不会变成上好油的机器。你们只会变成更快乐的人。思考很重要。因此一开始。我们会探讨目标设定 首先。理解目标设定的理论及实践。它的书本意义及社会意义。第二 我们怎样处理压力。下节课我们会探讨这个。最后 我想我们没时间谈这问题。但我还是做了课件。让你们自己看 我在书里有详尽说明的。从物质观念到幸福观念。基本上是关于我在我的书里。第七冥想:幸福革命的章节里谈到的。以及所发生的内部变革。我知道你们在放假前没多少时间了。但我还是准备了相关的材料。我们就从理论和实践开始吧。我现在想做的仅仅是说服你们。去设立目标 说服你们这是很重要的。你们要在作业上写出来。你们已经开始这么做。但我真的想让你们明白 作为一种生活方式。应该不断设定目标。不管是为工作或私生活。首先 我会试图说服你们。向你们展示关于设定目标及工作表现的研究。其次就是目标设定与幸福的研究。它是如何帮助我们赚取社会财富。以及终极财富。非常简单 设定目标的人一般…。掌控其他事都比较成功。为什么?一个重要的原因是 因为目标能使我们专注。通常来说。我们常四处奔波 不知道往哪个方向走。如果我们不知道方向。就不可能到达目的地 而专注能让我们找到方向。这能带来额外的资源。外部和内部资源 帮助我们达成目的。Abraham Maslow说:"专注于一个任务可以…。在个体及自然环境中提高组织效率"。目标的作用非常显著。我们设定目标时 或向自己灌输某种思想时。无论是私人的还是公开的。我们心中或身边都会有些什么变化。而且 目标对工作表现及幸福感有重大影响。因为目标会让我们更具适应力。还记得第二节课我们探讨过适应力吗?那些孩子成功的其中一个要素是。排除艰难的外部环境的影响。就是他们的适应性强。其中一个显著特征是。他们会设定目标。他们是以未来为导向的。不但思考过去 没人帮助他们学习。学习适应没人帮助的情况 专注于未来。 

尼采曾写道 只要我们有目标。一切皆有可能。只要我们有目标 一切皆有可能。我们更能克服困难。如果我们设定了目标 任务或是我们在意的事。以及我们想实现的事。目标让我们更成功。就跟积极信念发挥作用的原因一样。有目标就代表。我们相信我们会得到某样东西。Roger Bannister宣称。他会打破四分钟的纪录。爱迪生曾说过 到1879年12月31日。他会发明出电灯。设定目标有助于成功。因为我们的思维不喜欢。内部和外部不一致。而希望能相一致。如果我相信一个目标 对外宣告这个目标。那外部事物就会趋向于与这目标相一致。但正如我们探讨过的 并不会完全一致。但一定更有可能发生。这让我们更容易成功。 

这个背包是什么?想象一下 你去远足。你背着一个背包。你遇到一道墙 一个障碍物。你会怎么做?遇到那道墙时你可以有很多应对方法。那墙又高又长。其中一种做法是"太可惜了"然后调头走。也就是说 避开那道墙。另一种方法是 拿出锤子。试着把它敲碎。还有另一种方法是。把背包扔过墙去。把背包扔过墙去。为什么?因为必需品是发明创造的源泉。我需要背包来继续 需要我的背包。我需要它 而现在它在墙的另一边 我别无选择了。但要越过这道墙 要么把它敲碎。要么找一条路绕过它 从下方或是上方。突然间 我会想出几个办法。这些办法都是我之前没想到的。目标起的作用十分显著 而出于同一个原因。当你宣告一个目标 例如"我想买电脑"。那你就会发现到处都是电脑广告。你以前从没在这些地方看过电脑广告。或是说你想买某种车。一时间 你会发现到处是这种车。你以前从没看过的。为什么?因为我们一起创造现实。通过大胆地提出疑问。还记得一直在车上的那些孩子吗?你们没看到他们 直到我设下目标。让你们数出车上的孩子数量。然后孩子们就变得显而易见。就在你眼前。而之前你们就当他们不存在 这就是目标所起的作用。如果我宣布 我要越过这道墙。我的问题就是"要如何越过这道墙"。而不是去想"这是否有可能"。而是想"我要怎样越过它"。这个问题开启了无数机会。有很多是我之前没发现的机会。忽然间 车上的孩子就出现在我眼前。忽然间 墙上的一个洞出现在我眼前。忽然间 我看到了一把之前没看到的锤子。就在我旁边。必需品是发明创造的源泉。如果我们问对了问题 就能开启无数机会。 

另一件起作用的东西是 言语的力量。言语创造世界 要有光。我们在宗教中看过这典故 上帝说要有光 就有光。言语创造世界。"约翰福音"中写道:一开始先有言语。但不只在宗教里是如此 看看这伟大的国家。美国被宣布存在 言语有其力量。它们有某种意义 尤其是当言语对我们意义重大时。当我们宣布的目标对我们意义重大时。就更有可能实现。概念和构想之间的关联。不仅是语源的 而且还是形而上学的。这也是真实的。因为当我们宣告某事 保留某事。就更可能成真。基本上 言语能在我们的脑海里塑造一幅画面。尤其是我们想象目标时。它在我们脑海中创造一幅画面 而我们并没意识到。该想象和现实之间的差距。我们的思维想让两者统一 目标就是这样起作用。它们让我们更容易让两者统一。 

现在我念一段摘录。我在书里谈了很多相关的话题。你们可以看看来参考一下。但现在我要念一段Murray的话。他带领苏格兰探险队前往喜玛拉雅山。是个成功卓越的登山家。曾登上珠穆朗玛峰。他说。关于他的探险旅程 他写道:。"在所有创造的过程中 都有一个基本真理。对它的忽视使得无数优秀创意和计划夭折。那就是一个人一旦全身心投入。就会触动冥冥中的天意。 

很多帮助他实现目标的事。若非这样也不会发生。这样的决定会带来一系列事件。各种意料之外 对他有所帮助的事件。人和事物。这完全是他之前没有想到的。从歌德的诗文中我有深刻体会:。'无论你能做什么或梦想什么 去做吧'。勇敢带来智慧 魔力和力量"。它为什么会起作用 如何起作用 我们不确定 反正就有用。就像是如果你说要买电脑。就会看到很多机会。投入也是同样的道理。我们投入了 很多事就会接踵而来。我们开始发现外部资源。以及内部资源。会发现之前没看到的东西。因为我的问题变成了"我怎样能成功"。然后车上的孩子。或时钟的时间忽然变得清晰可见。我们有能力成功。比以前更有可能成功 言语创造世界。所以这与成功息息相关 书上有写。你们的一些阅读材料与之相关。从整体上设定目标 且从其他方面实现目标的人。更成功 无论是工作方面。还是个人生活方面 目标至关重要。他们不仅对"硬通货"重要。对"终极财富"也很重要。也就是幸福。被正确理解的目标。被正确理解的目标会让人幸福。这里我强调"被正确理解"。为什么?因为我们知道实现目标。实现目标本身。并不会带来幸福。没错 得到职位我会有短暂的幸福。但很快我就会回归基础水平。中了彩票或是赚大钱或…。升职 都会让我短暂地感到快乐。但不会带来长期的持续的幸福。我们知道那是短暂的快乐。因此我们知道实现目标不会带来幸福。那什么能带来幸福?充分理解目标的正确作用。要明白到。不是实现目标带来幸福。而是拥有目标让人幸福。 

两年前 我教积极心理学。一个非常重要的学者出了本书。关于幸福的书 他是达赖喇嘛的左右手。他就是翻译家Ricardo Matthieu。推出一本关于幸福的好书。我们在Williams James大楼展开了一场辩论。那是一次哈佛学生和全体教职员的全体活动。我们参与了辩论 题为"幸福:东方与西方"。我们很多意见不谋而合。你们知道 我经常冥想。我当然非常认同。佛教心理学的力量。然而 我们对目标的意见不一致。我们意见有分歧。因为在佛教方面 根据很多说法。不是全部 但很多说法认为。我们想实现的状态是身外无物。那个状态中 我们不会想要拥有身外之物。渴望的东西。也不会有现在的这个课程。我想或许那是个非常理想的状态。但我一直认为 那是不现实的。作为人类 我不认为。经过30年来每天八小时的冥想。我不相信我们能实现那种。身外无物的状态。设定目标是与外界关联的一个例子 因为。当我说想赢这场比赛 或我想这门课考试优秀。或我想在银行找份工作。这些就是确切的目标。让我与结果联系起来。如果我没与结果联系起来 我就不会在乎 它就不重要。我觉得这不仅对成功很重要。对幸福也很重要。因此我不能认同。佛教的说法。他们认为身外无物是种无欲无求的状态。然而我们有着相似的看法。专注于当下。被正确理解的目标的作用是。解放我们 让我们享受当下。这意味着什么?例如你去远足。你不知道自己要去哪。你没目标 心中没有方向。你就不大可能。不大可能享受旅程。因为每分钟你都在左顾右盼。不知道是否错过什么。如果你知道要去哪。你有方向感。你就被解放了。你就更可能享受旅程。才可以欣赏到路边的鲜花。 

想想你们的生活 有些时期里不知道何去何从。对你们很多人来说 此刻就是这样。你不知道明年会去哪。目前对大多数人来说 还是没关系。但一阵子后 你想知道要去哪里。因为当你专注于某事。你就会有更清晰的方向感。你更容易开心。这就能解释 为什么很多人退休后。变得不大开心。尽管他们多年来一直期待着退休。到他们真的退休后 就变得闷闷不乐。那些退休后更开心的人。为自己设定了目标。或是上一门课 或是学新知识。或是和亲人朋友相处更久。他们都有一个目标。而不是"让我们享受生活吧 顺其自然"。我们需要目标 结果 未来的导向。那样我们就能更享受此刻。目标让我们得到解放 才能享受当下。换言之。如果你仔细想想 目标是一种手段。它们是达到结果的一种手段。结果就是当下的体验。再次强调 目标本身。如我们所说的 并不能让我们感到更幸福。无论我们是否能实现这目标。我们会经历人生起伏。但如果我们想要基础水平 此刻就是了。当下 就是这过程。是我们正经历着的旅程 而不是结果。 

在书中一开始的地方。我谈及 冠军赛会让我开心。的确让我开心了四小时 但之后我又回到基础水平。获得终身职位的教授以为。这事会让他们余生都快乐。那个目标的实现。其实不然。他们回到了幸福的基础水平。无论他们是否实现了这个目标。关键是学习 享受这个过程。目标的其中一个作用是解放我们。那样我们就能享受这个过程。目标是获得结果的手段。这与许多目标理论背道而驰。这解释了为什么如此多的成功者不快乐。这也解释了为什么如此多的成功者。沉沦于药品及酒精。如今这样的人比以前更多。也许是如今我们了解得更多。他们常进心理治疗中心。那样进进出出。我们问"为什么会这样?"。表面上看那些人似乎已得到一切。似乎已得到一切。名声 财富 美人 他们想要的任何人。大部分人梦寐以求的生活。为什么他们最后还要进心理治疗中心?为什么会不开心? 

现在就来说说原因。多年来 他们不断努力向上。他们心怀梦想 希望成为名人。渴望得到别人的尊敬。渴望得到一切想要的东西或人。当他们梦想这些的时候 他们可能是不开心。但他们对自己说。"好吧 我现在不开心 但我成功后会很开心"。然后他们成功了。他们意识到并没有料想中的开心。他们以为那会让他们开心 但没有。那时真正的问题才到来。因为到那时他们才开始体会到无助。那时他们就会放弃。那时他们就变得…根据Hamburger的模型。他们会患上习得性无助。自我放弃和怀疑主义。因为大家都跟他们说。他们实现了这事就会快乐。而他们也对自己说过"一实现了这事 就会快乐"。但他们并不为此而更快乐。一开始 当刚开始有点名气。他们刚开始走上成功之路。他们常常梦想成功 当然就会快乐一点。但之后他们又回到基础水平。当回到基础水平 他们很压抑。很沮丧 失望。更甚的是 他们很害怕。因为不知道怎么办好 但在那之前。他们以一种希望支撑着 想着成功后。他们会很快乐 但他们并没有更快乐。他们很害怕 很迷茫。他们想寻找解决方法。通常就会脱离正常经验。脱离常规生活。那有什么解决方法呢? 常常就是药品和酒精。因为那让我们远离枯燥乏味的…。日常生活 无论是好的或坏的。这种错误的想法非常普遍 大部分人都如此。他们认为…这就是为什么。这么多人会经历中年危机。 

这不是唯一的原因 却是其中一个原因。因为有很多非常成功的人。他们成功后会说。"那现在怎么办?这就是它带来的快乐吗?"。是的 它就只能带来这么多快乐 而面临的难题是。去找出埋藏起来的幸福宝藏。它就在我们身边。幸福并不在于我们的地位。或银行存款的多少。它与我们的思维状态有关。就视乎我们如何看待自己的现状 以及我们专注于何事。想变得更快乐 当下更重要。引用我书中的话:。"幸福并不是登上山峰。也不是在山的四周没有方向地乱爬。幸福是向山峰攀登的过程"。它与设定目标有关。与设定一个终点有关。然后就释放自己 享受过程。有很多人和我谈过这问题。每次我都会经历同样的情况。"接下来我要做什么工作呢?"。"我该走哪一条路呢?"。当他们选了一条路 通常都会后悔。"我应该走另一条路"应该接这一份工作。不该去那里 我的回答通常是 这并不重要。对终极财富来说 并不重要。因为如果我接了一份应该做的工作。我会非常成功。当我非常成功 我会很开心。但那件事并不会让我开心。关键是设定目标。关键是自身的投入。对正在做的事的投入。就算那种投入每三天变一次也没关系。设定目标 全心投入才是最重要的。如果你选择某条路 却又后悔了。记住 这不重要的。走哪条路你都可以开心的 只要你投入了。因为当我们投入其中。就能享受这旅程。而我们投入于何事并不重要。当然必须是道德的事。不能伤害他人。因为那样的话 最终不但会伤害他人。也会伤害到我们自己。 

David Watson在"积极心理学手册"里说。"当代研究学者强调。追寻目标的过程。而不是目标的实现。才是实现幸福的关键"。该领域的主要研究学者。David Myers和Ed Diener说:"幸福感…。在理想情况下的消极体验中的增长。比在有价值的活动。以及实现目标的过程中的增长要少"。这有一首优美的诗。由Gwendolyn Brooks所写的:。"不为胜利而活 不为曲终而活。为美好过程而活"任何目标都并不重要。并不是所有目标都一样。就我们当下所处的境况来说。思考你的目标非常重要。没错 投入是最重要的。但有某些目标是比其他更好的。简单地说 自我和谐的目标。在你们的阅读材料里有写 是最根本的目标。和你的个人兴趣和价值观。还有你在乎的东西相融合。做对你来说重要的事。举个例 就读医学预科 是因为。你想要当医生 救死扶伤。学习经济学是因为市场让你着迷。你想做与之有关的工作。你对它充满了激情。参加学生组织。是因为你认同他们的使命。融合你的个人兴趣。和你的价值观 热诚。这些目标是你自由选择的。而不是由外界所强加的。不是由某个特定的人。或抽象意义上的社会所强加在你身上。不是因为义务或责任而做。而是去做你内心深处最在意的事。换句话说 这些是你想做的事。或是你感到 察觉到自己想做的事。而不是你不得不做的事。 

我的老师Ohad Kamin 在我大学毕业时。我不知道何去何从。而我的老师给我说了这话:。"这是一个50岁的人给你的建议"他当时50岁。他说"找出你能做的事。然后再从中找出你想做的事。把它们一一写下来。写张清单 长短都无所谓。接着从你想做的事中。找出你最想做的事。看着这些 从中找出…。你最最想做的事 然后就做那些事"。 

那是我得到过的最好的建议。非常简单 但非常重要。这需要花点时间 认真想想这些事。那我的自我和谐目标是什么呢?我的兴趣是什么?什么对我最具价值?我对什么有激情?我想做什么 而不是我得做什么。有时答案并不简单。有时答案并不中听。而其中的含义…。接下来的路可能没有最初的好走。因为它越来越崎岖不平。它让我去做一些困难的事情。别人或环境所造成的困难。可能不是最受欢迎的选择。我在书里谈到的一个问题是。 

我认为幸福的重要组成部分。是精神性。字典对精神性的定义是。其中一个定义是。对有重大意义的事情的感知。因此如果我发现一些极具意义 极其重要的东西。意义非凡的 我就对它有种精神性体验。那什么对我意义重大呢?什么对我重要?要记住。选择什么 往哪个方向走并不重要。如果我选择的东西是自我和谐的。和我个人的目标和使命相符合。我就会过上精神性生活。事实上 一个投资银行家。从事投资银行业的最合适理由。是因为他或她在乎这工作。因为他们认为这是重要的 因为他们乐在其中。因为他们喜欢和数字打交道。喜欢那种刺激感。没错 他们真的享受这工作。如果他们是因合适的理由而去做这份工作。那他们就会比一个…。不虔诚的僧人有更好的精神生活。 

当然 反之亦然。无论我们选择什么 自我和谐目标。我真的相信。世界会变得更美好。不仅是个别地方会变得更美好。如果人们追求他们的热情所在 他们的自我和谐目标。说比做容易。"实现自我和谐是非常困难的。要求精确的自我知觉能力。还有抵抗社会压力的能力。那种压力有时会把人迫往错误的方向"。我们要参与到这些活动当中。为什么?因为人生苦短 人生…抱歉。"人生短得完成不了我必须做的事。人生的长度仅够我完成我想做的事"。越早开始我们想做的事越好。这不代表你不能对自己说。"我最想的就是自己做生意。为了筹集资金 我会先当顾问。或是投资银行家两年"这么说没问题。非常好。就算你的兴趣不是一周工作80小时。处理些电子文档 也没关系。有时我们需要延迟满足。危险在于。我们的整个人生常处于这种延迟满足的状态。那就成了一条糟糕的路。这是我们要意识到的。我们需要记住的。我想为你们展示一个视频选段。选自我最喜欢的一部电影《死亡诗社》。Robin Williams谈及。定义我们想做的事的重要性。换言之 就是我们的自我和谐目标。(死亡诗社电影片段)。那么抓紧时间 做你想做的事吧。 

第13课-面对压力 

幸福课的同学们你们好 我们是Fallen Angels。这周六 3月15号 是我们在中心剧场的首次公演。我们将和哈佛Opportunes乐团共同出演。希望能在那时见到你们快乐的脸庞。周六晚八点。可以在哈佛票务中心买票。也可以找Opportunes团员 或者来找你最爱的天使。 

演唱"谁知道"。如果有人曾说三年以后。你早已离去。我会站起来将他们推出去。因为他们都是错的。我知道不是这样。因为你说过会陪我。直到永远 谁知道呢。亲爱的 我想你。亲爱的 谁知道。我手上有票 下课可以来找我。每张八美元 来找我吧。 

抓住眼前 及时行乐。让我们沿着这个理念再深入一点。探讨一下自我和谐。自我和谐有什么好处呢。自我和谐。有很多好处 其中第一点非常显而易见。你不需要成为一个火箭科学家。或者社会科学家 就能理解。制定自我和谐的目标可以让我们更快乐。因为我们在追求自己在意的事。更加增加了人生旅途的乐趣。第二 除了增加幸福感。当然也跟幸福感紧密相关的是。制定自我和谐的目标 整体上制定目标。但特别制定一些自我和谐的目标。能解决内在的心理冲突。比如说 它可以帮我们解决焦虑。解决疑惑 解决有关存在的问题。"我是谁 我在干什么 我为什么在这儿"。如果你思考这些 通常是在那种。我们不确定该走向何方。站在十字路口 前路迷雾重重的时候。这就是有关存在的问题出现的时候。这就是犹豫出现的时候。这就是不快乐出现的时候。如果我们知道我们该往哪儿走。就不会有那么多的内在心理冲突。我们可以从一个很有趣的角度来看它。也就是将它与人际间的冲突相对比。我们现在讲的是人的内在冲突 是心理内在的。焦虑 沮丧 通常会造成类似的后果。有很多关于人际间冲突的研究。我在讲到实践唯心主义。以及冲突的解决时简单讲过。面对冲突的办法。面对并解决冲突的最好办法。就是制定一个协调的目标。使得冲突的双方都参与其中。并内在地相互依赖。这���就能够解决人际���或组织间的冲突���这是Muzafer Sherif或Elliot Aronson所做的研究。同理 制定目标有助于解决内在冲突。人内在的心理冲突。因为它能让我们暂时忘却那些关于存在的。很重要 但通常很难的问题。尤其是不断出现时。它能使我们远离焦虑 远离沮丧。我们能集中精力于我们十分想做的事。同时 它还增加了成功的可能性。制定了自我和谐目标的人会更有动力。他们会更努力地工作。他们会全身心地投入到。他们所做的事中去。从长远来看。那些追求自己热爱事业的人都更容易取得成功。这似乎非常直白。显而易见 简直是常识。但我们常说 常识也并不是那么显而易见的。很大程度上说。这种在我们追求自己热爱的事业时。成功的可能性的增加 重新定义了"不劳无获"的公理。它将其重新定义为如下定理。"愉悦劳作则多获"。如果你要引用 我会否认我说过。如果你给别人看我说这句话的录像。我会说这是我的双胞胎兄弟。但确实是这样。如果我们热爱这项事业 就确实能把它做得更好。在实际生活中 同样也有很多关于这个的研究。你们读到了一些 我把某些研究编进了书里。有很多研究显示。当我们沉浸于自我和谐的目标时。就会更倾向于。继续追求其他自我和谐的目标。因为这样很好 我们更成功了。我们还不满足 这叫做自我强化。不仅是从生活目标的宏观角度来讲是这样。微观角度也是这样 这其实是种涓滴效应。制定了自我和谐目标的人其实。在非自我和谐的方面也做得更好了。打个比方。比如说你是一个大学生合唱团的成员。你很喜欢这件事。从自我和谐的角度来看 这是你所信仰的事。是你喜欢的事 是你感兴趣的事。是一件你正在做。因为你确实非常想做的事。换句话说 自我和谐的目标会对你生活的。其他方面产生积极影响。比如说一门你不那么喜欢的课。你去上课是因为你觉得你必须要去。而不是你真的非常想去。因此找到自我和谐的目标。其实可以帮助你提高成绩。虽然它们似乎毫不相关。因为总体来说 你更有动力了 你更兴奋了。你更感兴趣了 你用更积极的感情。来享受整个人生 这就是涓滴效应。这个效应可以想象为黑暗房间里的蜡烛。通常你不需要将点很多灯照亮整个屋子。有时只需一根蜡烛。就足以将光明遍布房间每一个角落。 

自我和谐的目标也是这样。它们具有涓滴效应。选择做我们想做的事同样对健康也有好处。这是心理学领域的重要研究之一。同样是由我们学校的Ellen Langer做的研究。她所做的就是去了一个养老院。将老人们随机分为两组。第一组所有需要都能得到满足。不管什么需要都能满足。他们想吃什么 就能吃什么。他们想在某些事情上寻求帮助 同样也能满足。所有事情都无需他们动手。很大程度上来说 这是大多数人梦想的养老院。而第二组就没有这么好的待遇了。他们不能衣来伸手饭来张口。时常都需要自己动手。比如说 他们必须自己浇花。他们必须自己规划每天做的事。服务没有第一组那么周到。他们需要什么的话 必须自己告诉职员。经常需要自己动手丰衣足食。同样 他们需要自己浇花。而研究的内容是。Langer创造了这两种不同的情境。然后一年半后再回来。一年半后 第二组。自己浇花的那一组。自己照顾自己的那一组。没有人管顾的那一组。跟第一组相比起来。他们没那么沮丧 他们更快乐。他们更有活力 更独立 也更健康。更关键的是。一年半后 他们活着的比例比另一组高出一半。唯一的区别就在于更健康 更开心。更高存活率的这组 有选择的权利。他们能做自己想做的事。他们会询问自己想做什么。而不是衣来伸手饭来张口。从某种角度来说 这太轻松了 不需要做什么选择。当我们能够选择时 当我们做自己想做的事时。是有益于我们的幸福 我们的成功。以及我们的健康的。尤其是能使我们更加长寿。小小的差异导致了巨大的区别。但今天的很多养老院依然是基于。"迎合老人所有的需要"。"满足他们所有需求"。"为他们提供方便"这样的理念上的 这其实不好。太过于轻松了。不管是我在书里提到的概念。"特权中的非特权"。一切过于轻松 我们失败得不够。抑或是老人院里 一切过于轻松。不需要做选择 无需挣扎 无需做决定。而且同样适用于我们之后会讲到的。压力的概念 以及它对于。培养耐性以及最终获得幸福的重要性。过于轻松不一定是件好事。最后我们看到压迫政权和民主之间也是这样。人们之所以在民主制度下更幸福的重要原因之一。记住 这是少数可以预测幸福的。外部条件之一。原因之一就在于 民主制度下人们有选择的权利。而在独裁政治下。通常人们会被告知"好 你去做会计。你去做工程。你去做运动员" 诸如此类。如果你有选择 那就是快乐的预示。让我们稍微换一下话题。我要讲一些与自我和谐非常相关的东西。但是是从不同的角度来讲的。我要讲讲行为价值观 缩写为VIA。行为价值观 VIA 可以说是积极心理学中。最有前景 也是。最重要的课题之一。它最先是由图中的Chris Peterson。密歇根大学教授 以及Martin Seligman共同提出的。他们想说明的是积极心理学。可以代替DSM DSM就是。《精神疾病诊断与统计手册》。其中包括已知所有心理疾病的分类。这是本非常厚的书 非常重要的文件 上面讲了。比如说 它列出了特定的标准。然后说如果一个人在某一段时间内符合。这十条标准中的七条 那他就有重性抑郁症。如果某人在另一段时间内符合四条标准中的三条。他就有精神分裂症。或抑郁症 或季节性心理疾病。或者我们已知的定义并分类好的。精神疾病。非常重要的文件。而Seligman和Peterson。在积极心理学问世之初就说。"让我们来造一个替代品 让我们替代掉。DSM 不再鉴别弱点。疾病 或者缺陷。相反来鉴别人的力量和长处"。然后他们定义了24种人格力量。全部囊括在此书中 并进行了解释和描述。他们不仅定义了这些人格力量。还创造了机制 方法。来测量这些人格力量。比方说一个网上测试 下周你们会做一个这样的测试。可以作为帮你鉴别你的。行为价值观的第一步。你的人格力量是什么 你的美德是什么。你擅长什么 你热爱什么。关于人格力量 他们在鉴别它们时。确保了他们所鉴别的是普遍的人格特点。换句话说 并不是特别挑选的例如。美国的白人学术男性。这是在全球范围进行的研究。跨文化以鉴别那些共有的特征。不管在美国 欧洲 亚洲的中国 日本。甚至包括肯尼亚的马萨伊部落。以及北格陵兰岛的因纽特人。这是个全球范围的研究课题。因为他们想。如果我们能找到所有文化都共有的东西。就能找到人类本性的东西。而不是后天养成的。他们承认这是可以随时间变化的。在测量时我们可能会变得更老练。但作为开始 人格力量和美德。行为价值观 手册和测试 是一个好的开始。你会从中受益良多。很快你们就会明白。这些特点的关键方面在于。它们具有道德价值。例如跑得快的能力。很显然有些人有这种能力。身体各部位动得更快。肌肉更强劲。这是种力量 但不是人格力量。因为它其中不含有道德成分。你可以将力量用于道德相关的事情。但就其本身来说。它并不算是一种道德特点。同理 一样东西要成为力量。就必须是用于道德方面的。比方说。幽默是一种人格力量。但如果幽默被用来伤害他人。它就不能算作一种力量。所以它必须是用于道德高尚的方面。行为价值观很大程度上是。关于自我和谐的过程的 意思是。当我们制定了目标和结果 有一个目的地。对 这当然是对整个过程有益的。它通过解放我们来帮助我们。享受现在 享受当下。但最终我们关注的是结果。有了行为价值观。我们就能直接关注这个过程。我如何能到达终点 我的目的地。是通过大量的学习吗。是通过我可能拥有的人格力量之一。对学习的热爱吗 还是通过领导他人呢。领导也是人格力量之一。是通过精神上的祈祷吗。如果信仰和敬畏是我的人格力量的话。它们能帮助我们享受。前往我们认为有价值的目的地的过程。换句话说。如果我们有自我和谐的目标和自我和谐的旅程。这就是快乐的所在。既包括当下的好处(过程)。也包括将来的好处(结果)。这就是为什么自我和谐的目标很重要。并不是说自我和谐的目标。对我们享受奋斗的过程没有帮助。而VIA也不能帮我们得到更好的结果。它其实可以。然而 当它们同时呈现时 我们更能够感受到。完整的快乐 以及成就感。很多研究显示如何找到我们的行为价值观。更重要的是如何应用它们。能够使我们得到更高层次的快乐和成功。这不仅仅是相关关系。而是因果关系。你们下周的练习就是使用它。最终可能带你们走向更大的成功。以及终极的目标。除了你们即将做的问卷之外。这里也有一些其他的标准来鉴别。你的个人人格力量是什么。第一个问题就是"这是真的我吗"。我是否觉得这真的是我自己 当我在团体里工作时。当我领导别人时 当我学新东西时。当我祈祷时 当我在看一部喜剧。或者讲笑话 听笑话时。我什么时候觉得最是我自己。十九世纪时 William James说。"我时常觉得定义一个人性格的最好办法。就是寻找特定的心理。或道德态度 它们出现时。他觉得自己最为积极和活跃。在这种时刻 他内心有一个声音说。'这就是真的我!'"。你的真我在什么时候出现呢 是你表现英勇的时候吗。是你全身心投入 认真地做某事的时候吗。是你作为一个审慎的人 你的计划。很好地计划了未来的时候吗 这也是一种力量。当你按照你的人格力量行为时。你会觉得充满活力和动力。这是发自内心的 很本能的。这是来自你的人性本能的。本质的内在观念。而不是说"所有力量都是我的。我可以培养所有这些力量" 对 没错。我们都拥有所有力量中的一些。然而 我们仍然有特定的偏向 它植根于。我们的基因。或者早期的经历。并不是说我们就不能或不应该培养其他力量。而是我们应该。更专注于这些对我们来说更本性的力量。因为这是我们能获取最多的地方。也就是说最终的结果。最后的成功 传统意义上的成功。当我们利用这些力量 个人的力量时。我们最能够成长 最能发展。学到的也最多。再问问自己 很快你们就会。看到列出各种力量的单子 问问自己。我什么时候最能感觉到自我。是我投入地工作时吗。是我惩恶扬善表现英勇时吗。这是我觉得最真实 最有活着的感觉的时候吗。是我学习新知识时吗 是我去博物馆时吗。是我祈祷时吗 是我被幽默的人环绕时吗。什么时候我最真实? 这是力量的列表。花一分钟看看。它们分为六个类别。共有24种人格力量。我只是让你们提前看看。因为你们的作业中。会拿到一份带描述的所有力量的列表。花点时间将它们看一遍。问自己 说到这些力量时 你是谁。 

好 接下来我们要做的。我们现在要做的是。我要向你们介绍两个练习。两个建立于VIA上的练习。建立于这个问卷之上。建立于你们对自己的个人人格力量的认知之上。你们要做的第一个练习是。是下周你们的作业中要做的。第二个练习我非常推荐大家做。但由于时间关系不能一起做了。 

第一个练习 基于。Peterson和他的同事所做的研究的结果。对于幸福和成功。具有显著效果和影响。它是关于建立能力的。关于找到人格力量。然后不断追求它们。再简要概括一下什么叫建立能力。当你在考虑以及。应用人格力量时 我们所知道的是。它可以使你有一个更快乐的 自我和谐的旅程。比如你发现做某事。并持续不断地学习新鲜事物。是我的人格力量 或者表达感恩。是我的人格力量。是我喜欢做的事 那我就会更快乐。换句话说。回到我们第二或第三节课上讲的模型。它可以帮助我们从零走向正。然而 这项研究发现的是。它不仅能使我们从零走向正。而且能使我们更好的处理负。这是直接的 这是间接的。怎么做到的呢 通过建立能力。如果你仔细想想 这正是积极心理学所要做的。正如我们已经讲过很多次了。是关于建立能力的。是增强我们的免疫系统。使我们适应性更强。或者用另一个类比 造一个强劲的马达。从而使我们更好的处理困难。处理艰难的逆境。因此培养VIA直接影响我们的奋斗过程。我们更乐在其中。从零到正 然后它还有别的作用。它还帮助我们建立能力。这样我们就能更好地面对消极 面对困难。面对可能出现的逆境。过程是这样的。第一步是找到你的人格力量。显而易见的。而找到力量的方法 又分为两步。首先 做问卷 做测试。这就是那个网站 我会将它发给你们。是用Powerpoint做的 做这个问卷 共240道题。可能花半个小时的时间。没有正确或错误的答案。不是像"你最好把敬畏列在前五。最好不要把审慎列为第一"之类的。顺便说一句 我把审慎列为第一。没有好或者不好 只是识别出你是谁。它们所有 我们都有24个。感激是排在我的第19位的。这是不是意味着我不喜欢感恩。我每天所进行的感恩对我来说。都是没有好处的呢 当然不是。但排得更靠前的是我对学习的热爱。所以做做这个网上测试。找到你的前十 他们推荐前五位。我推荐前八位到十二位。找到这8种或12种力量 读读它们的描述。它们是什么意思 有什么含义 说明了你的什么特质。然后问问你自己 这里面哪五个。或四个 或者也可以是六个 大约是五个。哪五个符合我们之前讲过的标准。也就是 什么是真正的我 或者哪些。最能让我感觉到充满活力 充满动力。或者哪些最能使我成长发展。从这8到12种力量中。找出4到6种符合这些标准的力量。然后从中任意选择一个。并应用它。也就是接下来一周七天的每一天。这会是你们下周的任务。来应用这些人格力量。这七天之后的时间。就看你自己了 希望你们能将其转化为一种例行公事。一种习惯 然后再应用下一项人格力量。或者还是同一种 但用不同的方式。给你们举一个个人例子。我第三次教积极心理学时。学期到了一半时。我开始觉得有点疲倦 有点精疲力竭。我之前已经教了两次。我知道自己会说什么 我听过自己以前说过。我觉得我失去了精力。失去了教学的动力。然后我就回到VIA。我说"这倒确实说得通"。因为我在审慎之后排第二位的是对学习的热爱。而当我反复教同一门课时。我就没怎么学新东西。确实我在工作时能学新东西。我经常在学习 那些来过我办公室的人就知道。我们说话时我会写下一些想法。但总体来说 在这门课上我教的是一样的东西。然后我就说"好吧 那我就要实践一下这个练习"。对我来说。这个练习就是继续追求对学习的热爱。即便我是在教同样的课。即使我教同一门课 我还是很忙。但我仍然每天抽一小时出来做点新鲜的事。阅读一些我没读过的东西。而两天之后就对我起了效果。我的能量回来了。它对其他领域产生了涓滴效应 例如我的教学。因为我阅读了新鲜事物。它帮助我更好地教学。不仅仅是带回了活力。而且给我的课堂带来了新的想法。有时我会告诉你们我刚读了这本书。我读到了这些内容。即使我很忙我也一直这样做。因为我知道这对于幸福。对于终极目标 对于成功都非常重要。所以不管是什么。不管你排名第一的力量是什么 都应用它直至它成为习惯。然后 这也将是你们练习的一部分。每天晚上描述你当天如何应用了你的力量。然后用它规划第二天。你明天要做什么 为什么。因为这能加强你的神经通路。请记住这一点。某种程度上说 VIA之于DSM就像PPEO之于PTSD。还记得吗 这可以作为一道GRE题。这个类比是PPEO-积极。PTSD-消极 VIA和DSM相比较也是同理。等价于。积极心理学以及传统精神病理学。这是之后的第二个练习。你可以将其应用于解决问题。因为在VIA间接地建立能力。帮助我们面对问题 困难和逆境的同时。我们也可以用它来影响这些困难。来直接的帮助我们解决问题。比方说 第一步 和前面一样。你首先通过做测验 看看这些力量是否符合标准。来找到你的人格力量。然后找到你的问题 不论是在个人生活。或者是在运动方面。或者是情感方面 或者工作方面。找出你想要解决的问题。然后问问自己如何能够应用其中一种。或几种力量来解决这个问题。举个例子。比如说我遇到的一个。我极力避免的问题。我害怕冒险 我不会不顾一切。因为我很焦虑 并且害怕。而对学习的热爱是我的一种力量。我所能做的就是用对学习的热爱来处理我的。恐惧 焦虑 和害怕。怎么做呢 通过学习任何有关。如何更好地解决焦虑和恐惧的知识。我也正是这样做的。阅读任何我能找到的书。和我认为了解这方面的人聊天。利用这个力量。将其应用于我的弱点。人际关系也一样。比如说我们的人际关系正面临一定的困难。而我的力量之一 比方说是真诚可靠。那我如何将我的可靠。用于我和我伙伴之间的关系。最终达到更高层次的亲密。长远来看 改善我们间的关系。我也正是这样做的。将我的力量。应用于产生更加亲密的关系。想想你们自己的例子。这不是作业。但我十分推荐你们试试。 

让我们来汇总一下这些概念。让我们看看你们现在或者将来。会做的决定。无论是明年 或是三年之后。或者20年 40年以后。你如何去做决定。什么是制定一个自我和谐的奋斗过程。以及目标的有效框架呢。其中一种方法。就是看看Amy Wrzesniewski的研究。我在之前的课上讲过。但还是值得反复再提。关于工作方向的研究。这不仅适用于工作。也适用于我们生活的其他方面。Wrzesniewski他们发现的是。人们对于工作的概念。可以分为三类。有些人认为他们的工作是工作 有的人认为是职业。还有一些人 但很不幸不是多数。希望在将来成为多数。有些人认为它是一种使命。这并不意味着。将工作视为使命的人。就不会时不时觉得这只是工作。但总体而言 他们认为他们的工作。是一种使命 而非职业或仅仅是工作。你可以用以下标准来区分出他们。他们的动力是什么 是什么让他们持续努力。什么推动或拉着他们 他们如何理解工作。他们如何诠释工作的地方。他们的期待是什么 最后。他们对这份工作的未来所寄望的是什么。那些将工作仅仅视为工作的人 他们的动力是。每个月末 或者每个周末的工资。工作是我不得不做的事 它是琐事。我别无选择 他们期待什么呢 没有什么。他们期待星期五 期待放假休息。期待学期结束 期待两年研究生结束。或者之类的。而将工作视为职业的。薪酬和晋升是他们的动力。不断地升职。工作就像赛跑 跑到最高处。这是好胜者常有的情况。他们的期望是 更多的特权 更强的力量 更多的钱。更多的特权 更强的力量 晋升 升职。不断地成就。他们寄望的是不断的升职。这也同样于其他领域 比如说。从医学预科到医学院 到最好的医院的实习。到主治医生 到部门负责人。再到医院院长。可以是从分析师 到合作人 到合伙人。也可以是从助手到。副教授 再到正教授。这种竞争适用于各个领域。而那些幸运地将工作视为使命的少数人。他们的动力来自于工作本身。是一种自我和谐的目标。是自我和谐的奋斗过程。他们怎么理解工作 他们认为这是种任务 是种使命。他们本就该做这个。做这件事比起所有其他事都更乐意。他们不认为这是琐事 能够参与其中。就是一种特权。他们的期望 也是他们的使命 是使世界更美好。而最后 他们寄望的是更多的工作。从而他们能实现自身。成就自身。引用军事名言 可以成为任何他们想成为的人。讲到这个。要在每时每刻都感觉到召唤或使命。以及自我和谐的旅程是不太可能的。就像不可能总是能够。体验到这么多的快乐。举个例子 教学就是我的使命。我感受到了召唤。来自于这件事本身 我做这件事。比所有其他事更乐意。同时。教学中也有很多我不喜欢的东西。我也确实偶尔会感觉到。"这是我的工作 我必须要做"的感觉。比如每次上课前备五遍课。第一遍非常有趣。第二遍第三遍就有点重复了。但我知道如果我要把课讲好。因为教学是我的使命。我必须要反复备课。我也能感受到。备一两遍课和备五遍的区别。所以这对我来说很重要。但这时我的工作。就是种琐事 我不得不做的事。然而 记住涓滴效应。由于整体来说 这是种使命。整体来说 这是自我和谐的目标及过程。所以即使备五遍课也没问题 也没那么糟。或者比方说你很喜欢一门课。假设是1504。再做个假设。你一周后就要期中考了。你可能不会享受其中每一步。比如把书看两遍。或者看三遍。但如果整体来说这件事是自我和谐的。对你来说很重要 你很喜欢。那么那样也就没那么糟糕 是可以接受的。尤其是当你发现在这个过程中。你建立了新的联系。理解了你觉得很重要的知识。 

关于使命同样要记住的一点是。它确实从某种程度上取决于。你工作的种类。但更取决于你对工作的理解。我在书里讲到的研究。是关于医院里的清洁工的。他们将他们的工作视为使命。为什么 因为他们不是将工作诠释成。"我必须清扫垃圾 打扫厕所"。虽然这是他们实际的工作 而是。"我在照顾这些病人 没有我的努力。没有我这些出色的工作 就会产生更大的污染。医院里会有更多人受害"。而这些人。这些将工作理解为使命的。清洁工会更愿意。和医院里的病人聊天。更愿意伸出援手。 

有趣的是 同一家医院里的很多医生。将他们的工作仅仅视为工作。你会更容易 同样没有好坏之分。你会更容易在工作中找到使命感。如果你在单位里是领导。而不是门卫 或者清洁工 会更容易感受到。但这并不代表。打扫医院的人就不能找到使命感。有很多这样的例子。Wrzensniewski在研究中给出的例子。就是比较这些清洁工和其他清洁工。比较使命和琐事 以及比较将工作。视为琐事的和将其视为使命的发型师。我相信你们都。被将理发视为使命的发型师剪过头发。我的意思是心理学家比理发师要多。或者工程师 有的觉得他们的工作是。"好吧 这是我不得不做的事 就这样吧"。有的则从更大的角度来考虑。我不是无足轻重的 我是一个很重要的项目的一部分。我们可以将所有对工作的理解。都归于这三类。想想投资银行。我知道你们中很多人都想去投资银行。一种理解投资银行的方式是。"我要去那儿 我要赚大把大把的钱。过上舒适的生活"。这是种很重要的动力。人们不应该完全抵制物质上的舒适。没有关系 然而仅仅这样只看到这个的人。"这能让我过上很好的生活。我要送我的孩子去最好的学校 这挺好的。我要住在漂亮的社区里。有一所漂亮的房子和一辆豪华的车 有很多的特权"。这是人性 这些需求都是正常的。但仅仅看到这些的人不会快乐。或者不会长期地快乐。但如果不仅看到投资银行的这些方面。还看到它创造岗位的机会 而它确实如此。JP Morgan和他之后的。所有投资银行家 创造了成千上万的岗位。在投资银行业出现前都不存在的岗位。而投资银行。也能润滑经济的齿轮。能够提供机会 使得你能够继续。不管从经济上。还是知识上 奉献于非营利组织。因为当今许多非营利组织最需要的。就是有效 健全 专业的财政政策。而这正是你作为投资银行家。或者顾问 所能学到的。有的人说"你这样看待投资银行。是在给自己找借口"。你并不是在找借口。你知道你所做的是什么吗 你只是看到了事实。真实的事情 不是人为制造出来的。但很多人看不到它们。就像很多人看不到公车上的孩子。而如果你看不见它 那么你就会认为。它不存在。对于很多投资银行家来说。他们最终筋疲力竭因为他们没有。没有看到这份工作的积极部分。如果将其当成使命就一定能看到的部分。许多工作。无论是在医院做清洁工作。或者是作为一个医生。或者是成为投资银行家。成为老师 志愿者 和尚 不管是什么。都能被理解为工作 也能被理解为职业。也可以被理解为使命。通常选择权都在于你 你怎么诠释它。Abraham Maslow说"没错。我们许多人回避了我们天生的职业。(使命 命运 天命 职责)。因此我们经常逃脱了一些本性。命运所指定(或者说暗示)的责任。即使有时是意外的。就像Jonah试图逃离他的命运 最终失败了一样"。你的命运是什么 使命是什么 天命是什么。听从暗示你天命的那个声音。听从指引你使命的召唤。它是什么 问题在于。我们没有花时间来思考这些问题。有时是因为我们害怕得到答案。万一你想去教书呢?而所有来自外界的压力。让你去做顾问怎么办。或者万一是投资银行呢。而来自朋友和同事的压力说。"你应该做点别的"。用你的哈佛文凭去找更好的工作时的压力。最好的工作就是听从你的内心。因为当我们追求自己热爱的事业时。当我们追求自我和谐的目标。和自我和谐的奋斗过程时 我们才能真正地活着。才能将世界变得更美好。甘地说过"你必须成为你希望看到的改变"。如果你想让世界更快乐 就要从自己做起。说起来容易做起来难。但这可能是你要问你自己的。最重要的问题。而答案是有可能改变的。今天可能是这样。两年半后可能就不一样了。二十二年半后可能又完全不一样。或者五十年后。但问这个问题 如果我们真正地。问自己这个问题 就有了可能性。能看到我们以前看不到的东西。打个比方 看到了公车上的孩子。我们的命运 天命 职责 使命。我觉得将这一点。概括得最好的是。两年前坐在你们现在坐的地方的一个学生。也是《商业中的女性》的创办人之一。Ambani Carter 她说"我们不应专注于生命中可以拥有的。而应想到那些我们生命中不可或缺的。少了什么你活不下去 你需要什么。 

我们的下一话题 关于目标。关于我们上次讲到的。我相信你们没有一个人能想到。这对于你和你身边的朋友来说。都是非常陌生的。就是…… 我没预料到会出现这样的情况。但很快我们就会讲到的 我保证。在我们进入那个非常重要的话题之前。我只是想制造气氛 这都是设计好的。在我们进入那个话题之前 讲讲制定目标的一些建议。非常基本的建议。首先 要把它们写下来。写下来就是做一种承诺。比起只是把它们说出来。最好是公开地把它们说出来。而不是只说给自己听 或者只是想想。但把它们写下来完全不同。很多研究都显示了制定计划。列出并写下目标的作用。其次 制定最早期限。很多人都会说"最后期限"。我的妻子Tammy说"我觉得制定最早期限。会更有效 更有帮助"。也就是我们希望完成这些目标的日期。为什么要制定最早期限呢 

因为目标。尤其是自我和谐的目标是可以激励我们的。换句话说 给我们注入活力。它们使我们的生活更快乐。更具有活力 因此要制定最早期限 制定目标 这非常重要。不是"我马上就做这些事"。而是"我在2009年7月1日前做好这些事"。加上日期 使它们细化。换言之 不是说我马上 或者很快就会。提升销量 而是。我会在2008年12月31日前使销量增加5%。具体的最早期限 以及具体的目标。同样 比如说我想健身。不是一般的"我想要更苗条健康"。而是"我要在2008年10月15日前。每周跑四次步 每次5英里" 明确 具体的目标。我们讲过。制定目标最重要的不是在于完成这些目标。不是说我们会因为完成就感到高兴。或者没完成就悲伤。其关键在于给予我们能量 给予我们动力 使我们自由。是通往最终目的的一种方式。而最终目的是过程本身。是那些最有助于这个过程的目标或者。我们所拥有的。不是所有 但一部分 完成了一半的目标。就像肯尼迪在1962年提出的目标。"我们要在这十年中登上月球"。而那时能够忍受。飞到外太空以及回到地球所产生的高温的。金属都还没发明出来。那时他们。都还没有能将人送往月球。并使其着陆的技术。然而他宣布了这个目标。"十年内我们要做到" 这给予了NASA动力。有些人觉得他疯了 但它确实给予了他们动力。而在1969年后。NASA动力严重减少。热情急剧减退 一直持续到80年代中期。因为他们没有类似的激励性的目标了。在个人水平上同样可以看到。我在书里也说道。当人们有了目标而没有实现这些目标。或者实现了目标却没有制定新目标时。他们就会感到精疲力竭 失去动力。要明确目标。 

Ellen Langer做了以下研究。聚集一组人。将他们介绍给成功的科学家和发明家。然后告诉他们。这些是世界知名的科学家。然后她让这些人。让这些学生评估这些成功人士。然后问"你觉得你有多大可能达到他们的水平"。然后他们给了这些人非常高的评价。尊重他们 敬仰他们。说自己永远达不到他们的水平。然后她又召集了另一群学生。带他们去见同样的人。同一批非常成功的科学家 发明家。一个举动改变世界的人。她还介绍了他们走向成功之路的。一些过程和特点。比如说经历的失败。不安 失望 坎坷。以及他们最终是怎样从这边跨越到另一边的。然后她让他们评价这些成功人士。他们同样给了这些人很高的评价。而当她问。"你觉得你有多大可能达到他们的水平"时 回答则是有可能。我们都知道 信仰是自我实现的预言。长期成功的最佳预言者。此外 是John Carlton做的研究。除了不断地问问题。重要的是相信你能做到。如果我们只看结果不看过程。我们比较不容易相信。我们能取得这样的结果。使成功破灭。Langer说"人们能想象到自己的每一步。但却觉得高峰是难以攀登的"。制定一个长期目标。然后将其拆分为短期或者中期的目标。短期目标。然后制定计划 基于这些计划建立习惯。一步一步来 将成功细化。就会更容易相信那是可能的。也就更容易确实到达那个目标。女士们先生们。你们等待已久的时刻。一些你们都不熟悉的东西。你们可能没有仔细想过。但是是一种…… 请让我说完 我会讲的。压力 我在课堂开始时问。"有谁想要少点压力的"。你们大多数人都举了手。压力是一种全球性的 不仅限于美国或西方社会 而是全球的通病。在这节课最后。这二十分钟里 我希望我们都像这样。如果你们没到21就去掉啤酒。处理压力最重要的一点。这是一种全球性的严重的流行病。我们讲到拖延症。我出现拖延症时会这样做。我去Google图片搜索这样的东西。然后告诉自己这是你工作的一部分。这很重要。这是关于认知重建的。在处理压力时 我们所要做的就是抽出点时间。我来和你们分享一下 我以前讲过的。摘自《新英格兰医学杂志》。是由我们学校的Richard Kadison写的一篇文章。他是本校心理咨询处的负责人。"在一项针对全国13500个大学生的调查中。近45%表示非常抑郁。以至于没法工作学习。94%表示对于他们做的事。感到了过大的压力"。45%。人们问我哈佛的课程情况。事实上那课程还挺有名的 他们说。"是因为在哈佛吗。是因为在哈佛压力比较大吗" 不是的。事实上 当Crimson做了类似的调查后。他们发现在哈佛是47%。所以其实没有区别。这是全国性的 不管是哪个大学。哪个学院。这非常令人吃惊 我震惊了。过去16年里我都在大学校园里。我知道情况是怎样 但我还是对这个数字很惊讶。而且这不是我们平时一天要经历十遍的。正常情绪起伏。而是会导致无法工作学习的抑郁。怎么解释这45%呢。通过看这94% 让我说得具体一点。可以看到 今天的学生要做的事太多了。当我还是Leverett宿舍的一个住校教师。因为我的博士学位是商学院的。我当时还是个预备商学教师。所以我当时做的事 你们很多人都知道。我看学生们的简历。履历 然后帮他们润色。使其更适用于他们找工作的。招聘及其他环节。令我非常惊讶的是 每年。几乎每年 简历都愈发令人印象深刻。字体更小 标题更大 页边距更窄。每一年 老一辈进入哈佛的人。他们会说。"放在今天我肯定进不了哈佛"。确实是这样 很不幸 但这是真的。当我看到这些简历时 感到印象非常深刻。我会说"哇" 直到我注意到学生们。为了这更小的字体和更窄的页边距所付出的代价。这代价就是过度紧张不安。太多要做的事 压力 焦虑。结果导致。有许多数据能支持这一点。结果导致了更大可能的抑郁。这种情况我们时常会在大学校园见到。会在整个国家各处见到。会在世界到处见到。我最近刚从中国回来。并没有很大的差别。我最近也去过澳大利亚 也与此非常相似。在过短的时间内希望完成过多的事。学更多的课 参加更多的学生社团。这是需要付出代价的 这就是她。心理学家 独一无二的Ellen DeGeneres。在谈论国家大事。我觉得有一天情景喜剧只需要30秒。因为我们只需要这么长的时间。我们的注意力就只能维持这么长时间。因为我们的注意力维持时间很短。我们都有注意力缺乏症。或者说ADD 或者OCD(强迫症) 或者这种三个字母的病。因为我们没有时间和。耐心来把整个病的名字说完。应该有一个这样的病 TBD-过于忙碌症。天气预报可能是新闻最快乐的部分了。因为你们就能不时听到些积极的东西。你们可以听到"今天天气不错"。或者"今天将会有个好天气"。听积极的东西总是很好的。因为我们总是听到那些消极的东西。我们外面的世界如此混乱。以至于都没办法注意到天气很好。我们的节奏太快以至于无法注意到。我们需要帮助来追赶这节奏。所以我们在这建一个咖啡店。这里建一个咖啡店 那里建一个咖啡店。最小的咖啡店叫"大咖啡"。我会要一杯咖啡伴一罐红牛 因为我很忙很忙!我患了TBD 我的瑜伽课要迟到了 快 快点! 

TBD应该被收进DSM里。我们为这个TBD付出了很高的代价。首先 心理健康。抑郁症最容易导致的后果之一 生理健康。许多医生估计约80%我们的生理疾病。是由于压力所致 使我们的免疫系统变弱。最后 当下 今年。旷工的最大原因 全球来说。是关于我们的心理的。不管是压力 或者情感 感觉 经历。过去从没有这样。如今则是因为心理的原因。我们付出的代价。不仅是效率 还有创造力。回想拓延-建构理论。当我们处于压力下时 更容易变得狭隘和约束。相比于在盒子外思考来说。好了 那我们该如何处理这种压力呢。该怎么应对呢 让我们应用积极心理学。因为这确实是我们希望。能够解决的问题。至少我个人。很早就想解决它了。传统的问题。传统的心理学家会问。"为什么这么多人压力过大"。这是非常重要的 非常好的问题。能够引出很多好的回答 但仍然不够。然后积极心理学出现了。而他们所做的是 同样问一些积极地问题。就像。我们在讲到关于危险人群的研究时说过。就像Marva Collins曾经问过一个不一样的问题。因为问题能打开思路。探索我们之前可能没有见过的东西。这样问题就变成"那些成功的。过着健康快乐生活的人是怎么做的"。换句话说 关注起作用的事物。也就是关注茎尖端的芽。因为虽然很多人都压力过大 但不是所有人都这样。有一些人能够很成功。同时过着健康和快乐的生活。他们是怎么做的 他们做了些什么 他们是这样做的。人们发现。他们有一些特别的特质 有两点。他们为自己建立例行公事。我们之前讲过例行公事的。第二件事。也就是我现在要详细讲的。他们所做的第二件事 除了建立例行公事。不管是很多的例行公事。还是寥寥几件 视乎个人需要和个人差异。除了建立例行公事。他们还特别地不仅为工作 还为恢复也建立了例行公事。让我来解释一下 解释一下。我现在要说的是。心理学家以前没有意识到的。因为它就像公车上的孩子一样隐藏着。或者被第一个问题掩盖着。只有问了第二个问题。他们才会注意到。他们注意到这些人将工作和恢复。例行公事化 本质上他们发现的是。这是我觉得研究压力时所得的最惊人的结果。那就是问题其实不在于压力。我们找错了地方。事实上他们发现压力对我们是有好处的。压力实际上培养了我们的忍耐性。力量 并且从长远来讲让我们更快乐。稍等片刻。这是什么意思呢 想想下面这个类比。想想运动方面。你发生了什么。你去健身房时实际上发生了什么呢。当你去健身房 你举重时。你对你的肌肉做了什么 你在给你的肌肉压力!事实上。你其实就在…… 你的细胞在偷偷流泪。请问这是坏事吗 你所做的事是坏事吗。完全不是 这是件好事。因为你举重 然后两天之后。再去举重。然后下周 你接着举重。长此以往。你举了一个月 两个月 四个 六个月。然后举了一年的重之后 你就会像我一样。或者也不会 所以压力其实对我们是有好处的。那真正的问题是什么呢。比如说压缩肌肉时 问题是从什么时候开始的呢。是你去举重 然后一分钟之后。你又去举重。然后不断举 不断举 不断不断地举。然后会发生的事就是。如果你不休息 不管是运动之间的休息。或者是休息几天 那么会发生什么呢。你就会受伤 到时你就会受伤。那时你的肌肉就会拉伤。那时你就会锻炼过度。这既有生理上 也有心理上的影响。这时候你就是锻炼过度了。换言之 问题不在于压力。不是生理上的问题 也不是心理上的问题。问题在于缺少恢复。那些既成功又快乐幸福的人。是会感受到压力的。然而他们很注重恢复。就好像在座的任何运动员都知道。你们会有调整期 在重要比赛前都会有休息。你们不会每天去举重。从心理层面上讲也是这样。如果我们不懂得休息。我们会付出代价 在生理层面上 我们会受伤。在心理层面上。我们会感觉到焦虑 最终可能导致抑郁。压力没有关系 它是好事。如果我们能恢复 它有时甚至是很令人兴奋的。我想和你们再分享一些。我之前提过的书里写的东西。《精力管理》。我非常推荐你们读这本书。也许在春假时你们可以读一读。Tony Schwartz和Jim Loehr想说的是。我们所要做的是将我们对生活的理解。从马拉松运动员变为短跑运动员。从不停地跑 跑 跑。变为短跑 恢复 短跑 恢复。这个想法 这个想法改变了我的人生。我曾经是个马拉松运动员 我是指在工作上。我每天工作14个小时。我热爱我的工作 非常努力。然后我决定结婚生子。我知道因此可能一切会变得不一样。我说"好吧 我必须妥协。我只能一天工作少于14个小时。如果我想和我的妻子以及孩子。保持良好的关系的话"。我某种程度上有点失望 但我有我优先的考虑。我做了这样的决定。然后我遇到了Tony Schwartz。他只用了几分钟就改变了我的观念。改变了我对我的人生应该是怎么样的理解。当他跟我说时 我立即意识到就应该是这样的。我开始应用它。而今天我可能没有一天工作14小时。那么有效率 但也十分接近了。而我显然更快乐了。我比还是个"马拉松运动员"时更有创造力。他的建议是这样的。他书里讲到这样一个东西。这是我推荐的一种习惯。用一到两个小时 也就是大约一个半小时。来"短跑" 专注地工作 精力非常集中。然后在那之后 "短跑"之后。在尽可能少分心的情况下。用十五分钟来恢复。心理学家发现我们能够"短跑"。能够专注 按照我们的生物钟。能持续一到两个小时 每个人可能不同。有些人是一个小时 有些人可以持续两小时。平均是一个半小时 "短跑"之后 休息十五分钟。可以是冥想。听你最喜欢的音乐。也可以是去健身。可以多于十五分钟 但至少要十五分钟。吃午饭 休息一下 但在午饭时。不收邮件 不接电话。那些只会增加焦虑。而是和朋友一起或独自完全放松。如果我们的一天是。短跑 恢复 短跑 恢复。我们就能持续。能够维持高水平的精力。做更多的工作 并且更快乐。我现在的生活就是如此。我的一天就是短跑 恢复 短跑 恢复。我能做更多的工作。我理想的一天就是一个半小时工作 十五分钟冥想。一个半小时 午餐 一个半小时 运动。一个半小时 回家和家人在一起。我一天净工作6个小时。但在这6个小时中。我能做比十小时还多的事 更不用说。我更快乐 并且更有创造性了。这样的关键在于维持灵活性。因为…… 对我来说理想的一天是这样的。然而总是有一些别的事。比如我经常出差 我要教课。在这种时候 我不在电脑旁。写东西或做研究。但是因为这些事对我来说很重要。所以我会做的是 比如说。我要十个一个半小时的时间段。例行的每周十个一个半小时时间段。有一些这种时间段是在飞机上 这没关系。有一些时间段我在家里或者在办公室里。但每周至少要有十个 通常多于十个。一个半小时的时间段 当我做到时 我很快乐。我感觉很好 我做了很多事 又比如瑜伽的习惯。维持灵活性。对我来说很困难的一件事就是。每天留出45分钟或一个小时的时间段。但瑜伽对我来说很重要。它使我感觉良好。我在放假回来之后会谈到关于它的研究。所以我所做的是 拆分瑜伽的时间段。现在那15分钟的间隔有时是15分钟的瑜伽。15分钟的日落瑜伽。然后下一个15分钟做另一种瑜伽练习。所以我将瑜伽拆分成每天两到三个时间段。又是例行公事的灵活性。这对我来说有用 非常有效。关键在于 我最后要讲讲这个。关键在于保证不同程度的恢复。第一种恢复 从最小的级别。15分钟的冥想 一小时健身。或者是午餐休息 随便什么。中间级别的恢复就是睡一个好觉。我们会讲到睡觉的重要性。睡眠是对幸福 对快乐 对创造力。以及对你的记忆力的重要投资。睡眠很重要 每周都要休息一天。即使上帝也需要一天的休息 我们的生活也是这样。 

最后就是假期 J.P.Morgan说过。"我可以一年工作九个月。但不能工作12个月" 娱乐带来创造力。记住这一点 请记住这一点。下周二我们会在这里进行期中考试。周四不用上课 记住假期就应该。好好地享受 祝你们好运 假期后见。 

第14课-过犹不及 

各位早上好。早上好 很高兴又见到大家。希望你们都度过了一个愉快的假期。恢复元气 释放压力。这也是今天的主题。我们接着上节课继续讲。上回我们提出 事实上。压力不是问题。这是压力研究最有趣 最令人惊讶。现在看来 也是最显著的发现。压力本身其实是件好事。就像在体育运动中。我们给肌肉以压力 肌肉就会发展 增长。我们心理上情感上给予自己压力 我们就会成长。变得更开朗 更坚强。从小在无菌环境里长大。从未受过外界任何压力。对人是不利的。比如影响机体的细菌。会使人体产生抗体。所以长在无菌环境中对我们是不利的。不仅指生理上婴儿的密闭环境。缺乏运动。还包括心理及情感。所以我常常说 我希望你们遭受更多的失败。因为艰难困苦。能给予我们极大的锻炼。结束"目标设定"的课程后。就会开始关于失败的课程。我们会讨论完美主义。也就是对待失败的态度。 

所以问题不是压力 而是缺乏休整。不仅仅在哈佛 我们的文化中。都对休息缺乏重视。大多数办公场所 都没有足够的地方进行休整。大多数学校 大学。所以压力转变为慢性压力。变成了长期焦虑。变成了抑郁症。就因为我们没有休整。我们还讲了不同层面的休整。快速回顾一下。微观层面上 休整可以是。比如 每冲刺90分钟 就休息15分钟。不要做马拉松运动员 要做短跑运动员。可以是15分钟冥想。一小时的午餐。给自己时间好好休整。可以是定期去健身房。听15分钟喜欢的音乐。与朋友聊天 等等。这是微观层面上的休整。上班时间进行休整的人……。河对岸的Teresa Amabile和Leslie Perlow等人。经过研究发现。如果工作场所能提供休整。能使员工效率更高 更有创意。并且长期来看 会更幸福。提高工作满意度。中度休整包括一夜好眠。每周至少休息一天。放松休息以便重新投入工作。最后还有假期 节假日 长假。 

如果我们片刻不停的运作。会发生什么。我们错过了身边和自身的风景。错过了获得快乐幸福。欣赏日常生活中美的机会。所以我们把一切都当做理所当然。因为我们没有花时间去欣赏体会。这有悖自然。你们能想象。一头狮子生活在现代世界?在麦肯锡或者投资银行上班。或者成为终身教授?这很反常 大家都知道狮子要打瞌睡。要追逐嬉戏 这才正常。还记得我们头几节课。怎么说的吗。人性中视野的局限性。服从权力。人的天性需要休整。所有动物都是。如果我们违背心理或生理需求。就会为此付出代价 无论是需要休整。需要身体锻炼。还是需要某种维生素或蛋白质。我们会为此付出代价 关键是。将这种本能的休整期引入我们的生活。关键就是集中注意。同时关注工作与休整。想一下我们节前提出的问题。处于相同的现代社会 拥有相似的雄心壮志。为什么有些人能获得成功?他们知道。何时聚焦工作 何时聚焦玩乐。套句老话 他们努力工作玩得开心。具体是什么意思呢?首先 要休整。他们明白多则劣 少则精。 

我们来看一下诺贝尔奖得主。以色列心理学家Daniel Kahneman。进行的一项研究。他曾获得诺贝尔经济学奖 后来成为了积极心理学家。他希望研究一天内。妇女的情感感情经历。他让一些妇女在经历了某些特定事件后。对此经历做出评估:。刚才的感觉怎么样。这些经历包括工作。购物。与亲密伴侣 孩子。共处的时光。与朋友共进午餐 等等。评估一天内她们的心情。结果令人惊讶。这些妇女并不特别享受。与孩子共处的时光。这一结果让Kahneman很吃惊。他和合作者进一步研究发现。这些妇女并不是不爱她们的孩子。她们很爱孩子。对其中大多数人来说 孩子是她们生活中。最有意义 最重要的一部分。但是 她们与孩子共处的经历。通常并不愉悦 这是快乐的第二个组成成分。有意义但不愉悦。为什么呢?他们进一步研究揭露了确切的原因。因为这些妇女与孩子共处时。并不是真正全身心与孩子在一起。可能同时还打电话。或者写邮件。思考上班的事 或者要做的家务。她们一心多用 并没有全身心与孩子一起。单独来看。她们可能和乐意与朋友讲电话。工作 思考工作。或者等下要做的事。这些活动单独发生可能很有趣。但是同时进行 就乐极生悲了。量影响质。听听这个类比。听听这个类比。想想你闭着眼睛 全神贯注地。听你最喜欢的音乐。我最喜欢的歌是。惠特尼休斯顿的《我会永远爱你》。谢谢。英雄所见略同。于是你听着惠特尼休斯顿的歌。或者其他你最喜欢的音乐。闭着眼睛 全神贯注。然后从一到十打个分数。绝对是满分十分。你深受感动 灵感迸发。于是又开始听第二喜欢的音乐。如果是我。会选贝多芬的《第九交响曲》。你全神贯注地听着。然后从一到十打分。没有《我会永远爱你》这么高。但也有9.5分了。接着 为了达到最大效果。你把两首歌同时播放。结果怎样? 19.5分吗。不 不是10分 连5分也没有 纯粹是噪音。这就是现代生活。我们进行各种活动 生活中有这么多美好的东西。但是又多过头了。这会怎样呢?这常常会导致内疚 沮丧。为什么 因为我对自己说 这怎么可能。我喜欢做这些事。生活中能有这么多美好人事物。我真是太幸运了。但我竟然不快乐。我肯定有什么不对劲。不 你没有什么不对劲。这正常得很 就好像你不可能。同时欣赏两首歌。即使它们分别都是你最喜欢的音乐。过犹不及 多则劣 少则精。 

这种情况常发生。两年前 我教授积极心理学时。就发生过。刚开学一个月。我很困惑 为什么我不快乐。教学工作很顺利。我很喜欢上课 同时还担任咨询顾问工作。在全国进行关于积极心理学。及以色列的讲座。写作出书。做各种我喜欢在乎 充满热情的事。和家人在一起。于是我退一步 发现我"过犹不及"了。一旦我减少活动量。幸福指数又重新提升。多则劣 少而精。两性关系研究者。Hendrick夫妇认为。"爱与性与压力成反比。如果我们能帮助人们简化生活。降低压力水平。那他们关系的将会大大改善。更重要的是 他们生活中的积极方面。也会相应得到提升"。它会影响我们生活的方方面面。无论是爱情生活 与孩子共处。工作 阅读。与朋友共处 写作。都可能造成过犹不及。所以如果我们把午餐作为休整。同事还打电话 写邮件。那就不是休整了 而是更多的压力。但如果我们只是享用午餐 享受美食。或者只是与爱的人呆在一起 那就是休整。专注是关键 工作也是如此。但是在当今社会。尤其是处于领导地位。你们很多人可能已经……或者将来即将体验。不可能避免多任务。关键就是减少任务。 

我们来统计一下 在座有多少人。从事需要集中注意力的工作时。无论是写作 阅读。或者其他什么需要专注的工作时。有多少人同时还开着邮箱的?很好 占大多数。几乎所有人都这样。我以前也这样 后来我读到了这项研究。当你开着邮箱。从事需要集中注意的工作时。等于智商减少10 整整10分。我不知道各位怎样 但我可不能少10点智商。作为比较 如果你一整晚通宵。36小时不睡。也等于智商减少10点。另一项比较: 如果你抽大麻。智商会减少4。所以如果你……面临选择。是要找乐子 还是降低智商 但千万别吸毒。希望你们没录到前面。伦敦大学进行过一项研究。上班时发短信的人。智商会降低10。与整晚通宵的人一样。抽大麻才4.4点 比这2倍还多。听着 如果你关邮箱2小时。不会发生什么大不了的事。什么也不会发生 除了这种情况:。你可能会收到另一条消息问:。服务器是不是当了?你没收到我的邮件?其实你收到邮件了。如果一两小时内不回复。对方就以为服务器挂了。除此以外 什么也不会发生。即使你不回复。即使把邮箱关个两三小时。同样 就算你不是每个电话都接 也不会怎么样。除非你是救护车司机。那还是开着电话吧。除此以外 都可以关了电话。少做点事 可以完成得更多。创造力提升 效率提高 工作满意度提升。因为我们分心时。就会过犹不及。 

我提到过很多技巧。比如记日记。表达感激 写感谢信。今天我们还会讲到生理锻炼。仁爱之举 等等。但是最关键不是要多做事。而是少做 或者说是明确。你真心想做的 然后坚定不移地执行。因为多则劣 少则精。人本主义心理学家Tim Kasser。做过一项关于时间充裕的研究。我们已经知道。物质充裕对于我们的幸福感影响不大。基本需求得到满足之外。几乎不会影响我们的幸福感。但是 的确会影响我们幸福感的。是Kasser所说的"时间充裕"。这是一种感觉 我们感到时间充裕。感到我们能充分享受手头所做的事。而不是竹篮打水 做无用功。东奔西跑 压力巨大。时间充裕的人。往往容易获得幸福感。这点我们可以详细介绍。即使现代生活。充斥着种种使人分心的诱惑。我用了"使人分心"这个词。虽然这些诱惑。对你来说可能美好精彩 意义非凡。但如果同时出现。很不幸 会让你分心。 

现代社会的运转常常与快乐对立。所以问题是。如何让我们的生活 充满乐趣同时又正常运转。类似于当下与未来的利益。我们知道要休整 重拾生活中的乐趣。听喜欢的音乐。而且每次只能一首。或者与朋友共进午餐。与室友在家一起享用晚餐。当我们一起享受时光时。不仅仅是快乐 而且有用。是正式休整。问题是如何调和快乐与有用?还记得积极心理学提出的问题吗。快乐 健康且成功的人士 是如何达成这一切的?这一问题一经抛出。就打开了所有可能性。出现了各种各样的回答。其中一种回答是 聚焦工作 聚焦休整。则工作效率更高 质量更佳。英语语言中最短的。但又人人害怕的单词是。"不"。无论是对寻求帮助的人说不。还是对眼前的机遇说不。因为我们对某些机遇说"不"。对某些提议说"不"时。其实是对自己说"是"。那如何选择对什么说"不"。对什么说"是"呢。很简单 问自己究竟想要什么。简化 宁缺毋滥。关键就是找到最适宜的简化程度。因为过于简化。就会走向另一个极端。和其他事物及诸多心理学显现一样。简化与效率是以曲线形式存在。我来解释一下。这张图表的X轴代表工作量。Y轴代表效率 创造力与快乐。两者的关系是这样的。如果工作量太大。就会发生TBD 即忙碌紊乱症。我们会不快乐 缺乏创造力 效率低下。精疲力竭。也许短时间能效率和创造力能维持一段时间。但这点代表精疲力竭。另一方面 如果我们什么也不做 那也不好。后果相似 我们会不快乐。没有效率 缺乏创新。这点代表极端拖延。我们马上会讲到。这点代表过度工作。我们要发掘。各自的最适宜点 每个人都不一样。有些人 图的位置可能靠这里。有些人可能更靠前。但是每个人都会有个最适宜点。这点代表了效率 创造力。以及快乐的理想状态。例如气体再少也能充满整个空间。也适用于时间管理方面。即使我们工作再少。它会占用我们所有时间。这样我们变得效率低下 缺乏创造力 不快乐。 

另一方面是……我常常引用JP Morgan的一句话。他可以在9个月能完成一年的工作。但不能花12个月 需要休息时间。你的理想值在哪里 这两者间你在哪。你得自己摸索 反复试验。期间会犯很多错误 没关系 我们就是这样学习的。本质上 我们关注的就是可持续增长。梭罗说过"要简单 简单 再简单。依我说 你要做的事应当是两三件。而不是成百上千件。数上半打 而不要数上百万。在这多变的文明生活的海洋里。云雾 风暴 流沙。和许许多多事情都得考虑。他要成功 就必须是台出色的计算器。简化 再简化" 这段话写于1840年代。那到了21世纪 我们还要如何简化呢。我们关注的是可持续增长。这一比喻借鉴自环境保护论。什么是可持续增长? 比如塞内加尔。想要进行可持续增长 该如何进行?这不是说他们停止增长。而是指他们想要增长 但以一种可持续的方式。也就说"对环境的损耗。要控制在可以人工帮助下自然恢复的范围内"。与我们所说的心理学概念。完全一致。我们提倡的不是。无需奋斗 无需努力的。无压生活。我们提倡的。可持续增长是指。我们努力奋斗 竭尽全力 给予自己一定压力。然后通过充分的休整得以恢复。这才是可持续。因为如果不恢复 那就无法持续。会出现焦虑 抑郁 疲惫。等等情绪。所以问问自己 我的生活需要简化什么。我的生活还需要做些什么。也就是说 该如何克服拖延心理。回到我们朋友那。因为可能有点太极端了。全国有70% 超过70%的大学生。抱怨拖延心理。不是别人拖延 而是他们自己拖延。另外30%矢口否认。超过70%。拖延与不快乐。生理免疫系统脆弱。压力过大等等都有关联 可能会导致抑郁。虽然关联程度不高。但是两者呈正相关。 

幸运的是 有许多关于拖延的研究。其中一项研究很有帮助。这项研究是由加拿大卡尔顿大学。拖延心理研究小组进行的。我浏览了……。你可以访问他们的网站 做得很不错。还有许多有价值的照片。其中一组照片。是来自他们年度的讨论拖延心理的会议。该会议意义重大。这些是会议的照片。所有人于第一天抵达。进行了聚餐。聚餐后 他们又吃了一顿。因为几小时后有人又饿了。第二天 他们结伴出游 四处走走。校园风景很美。然后 一起吃了点东西。然后 一顿接一顿 他们又吃了几顿。到了会议最后一天。他们觉得完成不少任务。实践了他们倡导的内容。于是拍了一张全家福作为结尾。事实上 他们对拖延。进行了精彩的重要的研究。以下是部分结果。如何克服拖延。首先 最重要的一点 很简单 容易实践且有效。他们称之为"5分钟起步"。他们发现所有拖延者都有一个共同点:。他们对行动的要求有所误解。全世界的拖延者都以为 要有所行动。必须先受到激励。我必须感受到共鸣 才能开始编写经济学10号课程的题库。我必须有所灵感。才能下笔写文学艺术论文。但事实并不是这样。并不是这样。第一步并不是改变态度。即受到激励 然后行动。恰恰相反。要先开始行动。行动会慢慢地影响我们的态度。因为通常行动开始后 我们会有惯性。或者有时候。在90分钟冲刺中。我们需要重新5分钟起步。我不知道你们是否留意到了。如果还没有 我想跟大家分享一下。现在我们应该互相非常了解了。或者你们非常了解我。如果你们不知道 我想你们应该知道。我非常非常热衷积极心理学。我喜欢积极心理学。但有时候 我早上起床。写关于积极心理学的文章。准备积极心理学讲座。或者阅读积极心理学文章 有时候我完全不在状态。这让我太太大吃一惊。那时我们刚结婚几年。我告诉她。"是的 有时候我都没心情起床"。她说"这怎么可能。你早上起床。打开电脑 开始工作"。是的 没错 因为我了解5分钟起步。有时候在90分钟时间段里。我不得不再做一次5分钟起步。因为我感到精力不支。所以要么花时间休整。要么我只是有拖延情绪。我说没事 就5分钟 坚持住就行。这5分钟往往会让产生一个向上的螺旋。不是通过思想 不是通过心灵。而是通过行动。5分钟起步可能是克服拖延。最有用的技巧了。换句话说 就是放手去做。奖励自己 你们都懂得生意是一点一滴累积起来的。你能想象没有奖励的生意吗。就像是自动回墨。你在从事某业务 想要激励自己。那就奖励自己。好了 我要工作3小时。然后在食堂吃一顿3小时大餐。作为奖励。或者我这周五要用功。这样周六就能出去玩了。没问题 我们做其他事也这样。但不行。我得受到激励 才能制作出题库。也许有时候是。但有时候我们并没有得到激励。奖励当然也很重要。公布出去 破釜沉舟。告诉大家你的目标。我在1995年就是这么做的。我第一次决定教授心理学专题讨论会。我当时在新加坡工作。我打电话给我的领导 Hugh Hang先生。我告诉他"Hugh 我想开一次专题讨论会。为时两天的讨论会 一天关于领导力。另一天关于追求卓越与成功"。他说"太好了。这会对员工帮助很大。你想什么时候进行?"。我说"7月1日" 当时是1995年1月。他说"好的"挂上了电话 我是打电话说的。我当时在这里 是打电话给他的。我一放下电话 就脱口而出"天啊。我干了什么?"我做过最正确的事。因为我公布了目标。破釜沉舟。不得不着手开工 无论我是否喜欢。这帮助我克服了拖延。因为我知道自己想开设专题讨论会。一旦对外公布。某种意义上 我就被迫去做了。后来我开设了讨论会。虽然不是1995年7月最棒的专题讨论会 但我办到了。我准备得很努力 团队方法。比如 坚持训练课程最好的方法之一。就是和其他人一起进行。可能不是每个人都适用。有些人。喜欢一个人进行训练。一个人听着随身听 或者独自在家进行。完全没问题 但是对大多数人来说。与别人一起进行会有所帮助。 

书面写下也很有用……等于建立了契约。把目标写下来 分阶段进行。一一击破。还��得Langer的研究吗 把目���写下来。列一张表 对多数人来说很有用。最后一点 允许��己娱乐消遣。准许自己为人。有时候拖延一下也未尝不可。允许自己休整。你不是机器 我们不是机器。我们需要了解能力的范畴。否则就是违背自然。自然就会报复我们。让我们付出代价 效率降低。失去创造力与幸福感 也就是我们终极货币。我给大家放一段视频。是我最喜欢的Ellen DeGeneres。这是一期有关拖延的节目的结尾部分。强烈推荐大家看看。任意音像店里都能找到。这期节目时长一小时 主题是拖延。这是结尾部分。唯一的。Ellen DeGeneres《此时此地》。我们想尽一切办法把事情挤在一起做。以节省时间。我不知道你们怎么样 但我时间不够了。我时间越来越少了。假设我们能省出一堆时间。空闲了出来。然后拿它干什么呢。什么也不做。什么也不做 意义不正是如此吗。但以后的一堆时间是无法保证的。我们能控制的只有此时此地。所以拖延的感觉才这么好。拖延不是问题。是解决方法 是全世界共同的表达。"停下 慢点 你太快了 听听音乐。 

什么 听音乐。(杜比兄弟《听音乐》)。我的意思是 如果你今晚离开这时。不记得我说的任何话。只要记住这句就行:现在就拖延。别拖延拖延 谢谢"。 

好了 我教的所有课程中。这节课这一主题对我来说意义最深远。因为从许多方面 我的转变 从完美主义到……。这一转变仍在进行。此时此刻正在发生。从完美主义转变成追求卓越。这一转变与我的快乐历程。紧密相连。我想给大家讲讲。我的一些个人经历。有些已经写在书里了。我想再扩展一下 不止壁球的故事。我11岁开始打壁球。放弃成为篮球明星。这一前途光明 充满激情的事业。那时。我意识到NBA的梦想。无法实现了。我转而开始打壁球。几乎每天都打。当时我的生活围绕着壁球转。每天早上5点起床 出门跑步。回到家 吃早餐 上学。放学后 我会去打壁球。学校就在壁球场附近 距离不远。与教练一起练习 在健身房锻炼 打比赛。一直到傍晚。才回家做作业 然后睡觉。好几年一直如此。那段时间很苦 很艰难 尤其是间歇训练。与教练一起锻炼。但是 这些训练强度与我心理上遇到的问题比。根本不算什么。我不断受到压力。而且几乎没有休整。我总觉得胃里有个结。感觉很真实 就好像有个球在那里。每次训练情况不好。与我给自己设定的标准产生距离时。这个结尤其明显 尤其难受。而距离几乎一直存在。比赛期间情况尤为严重。经常 我闯入了一场重要比赛的决赛。突然发挥失常 输了决赛。哪怕理论上能赢的比赛都输了。那个结没有消失。就算消失 也只是很短的一段时间。但我相信它会消失的。当时有一个信念 一个愿望。一个目标一直支撑着我。要赢得以色列全国锦标赛。我对自己说。"如果我赢了 我就会快乐 就会平静。一切付出都是值得的"。为什么 因为只有付出 才有收获。1986年 我闯入了以色列全国锦标赛决赛。但我输了。我身心交瘁。虽然我是有史以来最年轻的决赛选手。虽然我击败了排名高于我的选手。我一蹶不振好几个月。不过最终 我还是挺过来了 只有付出 才有收获。我继续刻苦训练。1987年 我在决赛中又碰到了同一个对手 这次我赢了。我欣喜若狂 从未如此快乐过。那一刻 我对自己说"付出终有回报。都是值得的。我会变得快乐"。我跟亲朋好友外出庆祝。庆祝会后4小时 我回到家。我回到家 睡觉前。想再回味一下。胜利的喜悦与快乐。我坐在床上 突然毫无预警地。完全相同的感觉 完全相同的情感。那个结又回来了。我放声大哭 不是4小时前。喜极而泣的泪水 而且痛苦无助的泪水。那个结又回来了。我想"也许这是高潮后的失落"。我指望结会再消失。但它没有。几周后。我意识到它究竟是什么 这一次我彻底搞清楚了。我明白了。我要成为世界冠军才能真正快乐。高中毕业后。离我入伍服役还有2年空余时间。我去了英格兰。住在伦敦。加入了位于Ealing大道的Stripes壁球俱乐部。当时世界冠军Jansher Khan常常在那训练。因为我对自己说"这是我的目标"。我每天都去俱乐部。看Jansher Khan训练 模仿他。无论他做什么 我都照做。他每天早上跑7英里 我也每天早上跑7英里。他去壁球场 去健身房锻炼。我都照做。他与其他选手进行无数场比赛。我也与其他选手进行无数场比赛。他有训练搭档。当时我水平还不够成为他的训练搭档。所以只是偶尔去一次。有时候 只有我们两人时 他会说。"来吧 Tal 我们对打看看" 于是我们会打一会儿。每次与他比赛后 我都会从头再来。更努力的打球 更努力的练习。因为要像他那样。成为世界冠军 我就必须像他一样训练。我训练越来越刻苦 表现也越来越好。六个月后。我的水平能够成为他的训练搭档了。一年后 我成为了他的常规搭档。我跟着他去了世界各地。只要他有比赛 我们都会进行训练。但是 伤痛开始出现 为什么。因为Jansher从三岁就开始锻炼。他的耐力体格 都是日积月累锻炼出来的。而我 要么全有要么全无。要么进行业余训练 要么完全不训练。要么像世界冠军一样训练。我的身体无法承受 开始出现伤痛。一开始 只是踝关节小伤 所以我休息了两天。如果膝盖有伤。那就休息三四天。恢复训练以后会有一点不舒服 不过不严重。我会继续训练 要么全有要么全无。要么不训练 要么像世界冠军一样训练。因为我想成为他那样。就这样坚持了4年 伤痛时有出现。越来越严重 直到我20 21岁时。医生告诉我"你面临一个选择。你可以继续训练 打职业壁球赛。但你的背部会非常危险。可能很快就需要动手术。要么就放弃职业壁球"。 

怀着沉重的心情 我放弃了职业壁球。放弃了我的梦想 申请入学 来到了这里。同样的模式又发生了。只不过以前是壁球。现在学术成了我生活的中心。开学才一周 那个结又回来了。因为每一篇论文我都力求完美。书上每一个词都要阅读并总结。每一次作业都要毫无偏差地完成。要么全有 要么全无 我不快乐。在我大二时的某一天。当时我23岁 我对自己说。"够了 我身在名校。有这么多出色的同学 优秀的导师。应该感到荣幸 但我一点也不快乐"。我觉得这不是外部问题。我必须更好地了解自己。于是。我转而研究哲学与心理学。希望了解快乐生活的意义。了解如何才能更快乐。为什么我不快乐。很快我意识到。我发现自己从事……。或者说遭受到某种普遍。且错误的观念折磨。完美主义。这就是今天。及下节课的主题。完美主义的概念。以及对完美主义的研究。人无完人。没有人会处于糟糕的极端。也没有人会处于卓越的极端。这点我等会儿会细讲。关键是 如何变得更快乐。随着我渐渐脱离完美主义 我变得更快乐了。完美主义并不是说要放弃志向。我有志向。甚至可能比16或23岁时志向更远大。而是说对人生历程。对走出的每一步 尤其是对失败。采取一种不同的方法与态度。所以我要谈的第一点。先定义以便理解。因为它常常被误解。我会谈谈完美主义的特征。我们下节课才会讲。特征……成为完美主义者代表了什么?然后我还会讲讲后果。以及如何保持成功与快乐的平衡。我们在目标设定课上问的问题。这一问题我们在课堂上会经常重复。不是为了放弃成功。放弃日常生活中的金钱。而是更好地享受我们的幸福。因为它是终极货币。然后我们理解完美主义的源头是什么。它来自哪里。因为如果我们了解了它是如何产生的。才更有可能克服它。然后 最重要的是 如何克服。如何帮助别人克服完美主义。不是让他们放弃志向 或者停止努力工作。也不是消除失败时的痛苦 这是不可避免的。而是获得一种更理性 更有益。更可控的处理失败的方法。 

好了 我们开始前 我需要一名志愿者。你们举手前 注意我的要求。我要一个很特别的志愿者。我需要一个很棒的艺术家。绘画非常出色。不要有压力 好吗 要一个画画很棒。且愿意上来展示的志愿者。要上台来需要一定勇气。有勇气并不代表不害怕。而是害怕但仍勇往直前。有人自愿吗 有一个志愿者。你上来好吗。你好 怎么称呼 Vincent。你是VS专业的吗。别告诉我你是学经济的 好的。那你是什么专业的?生物 很好。生物学也要画许多图表。所以我……。经济学也得画图标吧。所以也没关系。好了 准备好了吗。最后一次机会让你回到位子上。你确定要……。- 但我想战胜恐惧 - 好的 很好。好了 准备好了吗 Vincent?你确定 好。以下是我要你做的。我不知道是不是真的想这样对你。但是我也只能这样了。请在黑板的这边画……。请画……。准备好了吗 你确定?好的。请画一个圆。慢慢来 不急。可以擦了重画。拿出你的看家本事。画一个圆。我们还有半小时时间 不急。我们边冥想边等你。我不是真的说半小时。不过……好的 就当是练习时间。一个圆 Vincent 画得漂亮。还没完呢。等等。先别急着走。好的 我在这画一条线。这边要画其他东西。准备好了吗 请画另一个圆。不过这次画的时候。当作你三岁时候画的样子。3岁 慢慢来。想象一下。你三岁时候。画的圆是什么样子的。好的。于是你父母说"梵高重现人间"。能画一个你三岁时候画的圆吗。我知道在座各位才华横溢。但我们想要些真实……很好。不过还是一位三岁神童 好的。很好 现在请再帮我一个忙。再画一个圆。这次画的时候。把你当成一岁的孩子。一岁神童啊 不过是一岁孩子了。很好 现在Vincent 我……。当老师有一件是我不喜欢做。我喜欢教书。但有一件事我不喜欢 就是打分。我讨厌打分。但作为老师 不得不做。必须打分。所以我要给你的画打分了。- 抱歉 - 没关系。我看着这样东西 我看到的是。我看到的是 真的是一个圆。如果我在中间画一条线。差不多都是等距的。如果我在这儿画一个点……真的很不错。非常好 Vincent 画得好 我真为你骄傲。干得好 你及格了 没错。但是我往这看。我想找个合适的词来形容。哪个词?"灾难"也许?Vincent 拜托 这不是圆。还有这个?抱歉 抱歉。我说了 老师不好当。不过我想问你个问题 Vincent。非常重要的问题。如果没有经过成千上万给失败的圆。你能画出这个真正的圆吗。换句话说 没有这两步 你能达到这步吗。当然不能 成功没有捷径。非常感谢 Vincent。学会失败 从失败中学习 这是唯一的途径。 

我是一名心理学家 你们知道心理学家干什么吗。心理学家观察人。你们每次进教室我都会观察。非常仔细地观察。知道我看到了什么吗。这需要极高的洞察力 经验及智慧。知道每次你们进教室 我看到了什么吗。我看到你们所有人走路非常优雅。了不起。你们走进教室。抬起左腿。踝关节微微往上 然后优雅地。慢慢放下 抬起后脚的脚跟。走得非常优美 非常优雅。还装作没有注意到。一边走一边与朋友交谈。有些人还同时嚼口香糖。连爱因斯坦都办不到。然后你们又抬起后腿 如此优雅。稍稍弯曲膝盖 这一动作。同时挥动手臂 太美了。然后微微扭动臀部。像这样。你们有没有考虑过转行当模特?我是说真的 漂亮 然后继续走。边走边说话 微笑。假装这一系列动作都是信手拈来。漂亮 你们值得表扬。但是问题来了 如果没有成千上万次的摔倒。你们能走到这里。走得如此优雅吗。连Vincent都不行。这是不可能的 学会失败 从失败中学习。我们小时候都知道。小时候。我们摔倒了 疼的话可能会哭鼻子。但是马上又站起来了。摔倒了 还会笑。来看个例子 有人在学走路。\[视频\]。去妈妈那。去她那儿 去找妈妈。太棒了。Sophie。能走回爸爸那儿吗 哎呀。把手给我。慢点 慢慢来。好紧张。去妈妈那儿 找妈妈去。找妈妈去 找妈妈去。慢慢来 宝贝 慢慢来 好姑娘。慢点。会走路了 激动吗。学习的喜悦。这种喜悦上哪儿去了 去哪儿了。它消失了 因为我们到了一定年纪。开始意识到别人在看我们。开始产生种种想法。"我不要试 万一摔倒怎么办"。或者"我不要约她出去 万一她拒绝怎么办"。"哦 不不 我不要演这出戏。万一没人看懂怎么办"。"哦 我不要分享经历。万一有人不喜欢怎么办"。又或者"大家都这么聪明 我还不献丑了"。我们开始逃避而不是应对。这会影响我们的自尊 信心 乐观精神。长期的快乐指数。约会时一切必须很完美。非常完美。忘了我们第一次学吃饭是什么样子。学习失败 从失败中学习。这不是第一次约会的实在建议。恐怕行不通。但记得我们是如何学习的。别无他法 成长没有捷径 学习也没有捷径。乐观 快乐 成功更没有捷径。我以前说过这个吗 这是我的私人祷文。我常常一遍又一遍对自己诵读。尤其是遇到失败。或者需要克服恐惧。出现"也许我不该做"这类想法时。因为万一……万一什么?万一最坏的情况发生了会怎么样:我学到了。最坏的情况 很伤心。但我会复原。Rina事件后我也恢复了。我能应付任何事。汉字"危机"。由两部分组成:第一部分指危险。是上面那个字。下面那个字代表机遇。有失败的危险。但有危险的地方。就能学习 成长。 

Elbert Hubbard说过"一个人能犯的最大错误。就是害怕犯错"。我想给大家举个例子 有一个人犯过错。失败了一次又一次。我甚至想给他授予。失败大王称号。原因如下。22岁时 刚找到新工作 就失业了。不是辞职 是失业。23岁时 他决定投身政治。因为其他工作他都不在行。但是也没有成功。于是继续回去经商 又没有成功。27岁时 压力太大。他崩溃了。精神崩溃。但又重新站起来了。7年后 他34岁时。竞选国会议员 名落孙山。没有成功。39岁时 他还没有学乖。又一次竞选议员失败。他说"让我试试更高层的"。46岁时 真失败啊 还没学乖。没关系 学会失败 从失败中学习。你都46岁了 适可而止吧。到了47岁 他试图竞选副总统。还没学乖 又失败了。到了50岁 他试图竞选参议员。他几乎想放弃了 但最终没有。到了51岁。他成了美国第16任总统。大概是这个国家历史上最有影响力总统。他谈起这段经历时说。"失败让人痛苦"当然 他的确有成功过。我简化了他的生平 失败让人痛苦。他不享受失败 但还是挺过来了。因为和历史上其他成功者一样。他懂得。学习别无他法 成长别无他法。甘地 他自传的副标题是。"我的对于真理的实践经历"。他谈到了经历 尝试与失败。尝试 失败与成功。再次失败 再次尝试。你们可能听说过他在南非的故事。他在南非进行的事业与之后在印度的非常相似。在南非 他经历了惨痛的失败。但到了印度重新尝试。经过无数次的失败。成功把祖国领向了独立。他比其他人更懂得失败的重要性。不涵盖自由犯错的自由。是不值得拥有的。他指的自由犯错。是指国家层面上的。人民能自主自由。同时也是个人层面上的。 

我最喜欢的作者乔治艾略特。她原名玛丽安娜艾凡斯 后来改为乔治艾略特。因为在当时。19世纪中晚期。妇女很难出版。乔治艾略特说过。"推动世界前进的重任。并不是由完人来进行的"或完美女子。读她的文字你会发现。是我读过的最接近完美的。无论是《米德尔马契》还是《丹尼尔 德龙达》。是我最喜欢的书。放手去做 放手行动 准备失败。我给大家举个例子。录像的这位主角懂得自己的不完美。允许自己失败。一位电视主持人。\[周六夜现场\]。没错。Bennett Brauer又有话要说了。没想到管事的还能让我回来。还以为他们会把我的节目。换成某个曲奇店模型。也许我是不"正常"。我不"上镜" 衣着"不合身"。不够"催泪" 从没和"女人上过床"。不知道"怎么勾搭女人"。看来我不"符合要求" 我不讲"卫生"。我上厕所"屁股不擦干净" 不够"有型"。没有"魅力"没有"自尊"。没有"自己的牙刷" "喜欢抠血痂"。我"够不到身体的各个部位"。睡觉时汗如雨下。不过我想"掌权的"。会继续给我发薪水。观众开始起身拿遥控器。调回新闻评论。他们不会"吓坏孩子"。不会"吃自己的头皮屑"。不会"用高中用的指南针。抠粉刺"。谢谢 Kevin。女士们先生们 感谢Bennet Brauer。谢谢。运动员Bay Brooth。曾保持本垒打次数最多的纪录。记录保持了多年。有多少人知道他还有另一项记录。5年来 他也是三振出局第一。这种情况很常见。最成功的运动员竟然也是。失败最多的运动员。一旦你准备击球。得做好失败的准备。又或者是他们当中最伟大的。在乔丹学会飞翔之前。他先学会了走路。他学会走路之前 先学会了如何爬行。他也失败过许多次。来听听他是怎么说的 唯一的迈克尔乔丹。我的职业生涯投篮失误超过9000次。输了大约300场比赛。有26次。我被委以重任投制胜一球 但失手了。我这一生一次又一次失败。所以我能获得成功。乔丹 我喜欢乔丹。人们记得让他声名大噪的制胜一球。那是在北卡罗来纳州 他最后一投压哨获胜。但你们是否记得他有20次。你们是否知道他有26次压哨失手。他失败了一次又一次 最终获得胜利。成功别无他法。爱迪生说过"我从失败中走向成功"。他失败了一万多次 才最终发明了电灯泡。但他没有把它们简单地视为失败。而是成功展示了哪些方法是不可行的。 

美体小铺的创始人Anita Roddick 去年刚过世。取得了伟大的成就。从诸多方面改变了商业世界。参与了多项慈善事业。也是失败了许多次后才成功的。没有其他方法。河对岸哈佛商学院教授。她是心理学系的硕士研究生。当时我还在读本科。我还记得当时她在写学位论文。我们的导师是同一个人 我和……。我俩的导师都是Richard Hackman。她当时的研究是有关医院的 目前的研究也与医院有关。她想研究如何才能消除。或至少降低医院的失败率 为什么?因为医疗事故一年会损失几百万乃是几十亿美元。更重要的是。许多人因为医疗事故而丧失。所以她想研究这一现象。并寻找解决之道。她的导师是Richard Hackman。他当时已对有效群组研究了数十年。参加过心理学501号课程的同学应该认识他。如果还没有参加过 不妨列为第一选择。他对团队和小组进行了多年研究。他发现在某些条件下。如果人们置身于某个组织。更有可能表现优异。某些条件。所以Amy Edmondson的论文想要证明。如果医院采取Richard Hackman的建议。使用Richard Hackman所说的条件。比如给予团队明确可行的目标。建立有效的团队结构。及时的组织支援 高效的训练。当一个团队具备了这些条件。就会表现得更出色。这也是Hackman十几年来的研究结果。而且会减少医疗事故。这是项非常重要的研究 于是她着手研究了。花费了很长时间。进行实地调查 在医院研究。经过很长一段时间。在William James大楼15楼。我还记得她得到结果的那天。把数据输入了统计软件 按下了确认。两组统计数据。有明显差异。使用Hackman条件的小组。以及不使用Hackman条件的小组。找到了 她欣喜若狂 几年的工作成果。她跑去了Richard的办公室 对他说。"Richard 我得到结果了 在这里"。他说"让我看看" 他看了一眼。说"Amy 太惊人了 结果令人惊讶。这不仅仅是模凌两可的边界了 太棒了"。他仔细阅读起来。"你确定是这个结果?"。她说"是啊 结果意义重大"。他又问"你确定?有没有重新输入数据再测试一遍?"。"有啊"她工作非常有条理 非常努力。她说"都核对过了"。于是他说"结果的确意义重大。但事实却恰恰相反"。也就是说使用Hackman条件的医院。反而出了更多医疗事故。这简直给几十年的研究一个大大的耳光。不仅是针对医疗事故。其他很多方面也是如此。于是他说"结果就是结果 现实就是现实。我们得回到现场 分析原因。理解这一现象。于是她从头开始 精疲力竭 衰弱无力。不理解怎么会这样 又回到了医院。很快就发现了问题出在哪儿。这些事故更多的医院 或医院小组。其实并没有发生更多医疗事故。而是报告了更多的事故。事实上 其他小组虽然表面上事故少了。其实发生了更多的医疗事故。但他们没有汇报。她是怎么发现的呢。因为有些事故是无法掩盖的。比如 很难掩盖一具尸体。所以这些大事故发生时。这些小组不得不汇报。并不是他们事故减少了。而是增多了 只不过没有汇报。于是她改变了研究的重心。转向心理安全概念。心理安全是一种感觉 经历 空间。让你觉得可以在办公地点 在组织内。安全地讨论错误与过失。让你觉得摔倒了也能安全地站起来。有心理安全的组织。会表现地更好。当时是1999年 她与全世界各国研究人员。进行了大量研究。证明了有心理安全的团队组织。他们的雇员不会害怕告密。也不会害怕摔倒 或犯错。这些组织更为成功。它们是学习型组织。雇员不断地学习成长。从错误中学习。下一次就能做得更好。 

意义重大 影响深远的研究。无论是领导者。还是家长……你们中很多以后会成为家长。你们会不会为自己的孩子营造心理安全。让他们毫无约束地告诉你。他们做的不好 失败的事。因为如果他们愿意跟你说。就不容易进入害怕失败的完美主义。而会更快乐 更成功。来听听关于Thomas Watson的故事。他是IBM的创建者 他至今仍旧。每天不断地改造它。所以IBM能发展至今。IBM刚成立 还在生产打字机时。有一名员工。犯了战略性错误。给IBM损失了一百万美元。当时对于IBM来说这是一大笔钱。于是他犯错后第二天。去找了创始人Thomas Watson。给了他一封信。Thomas Watson打开信读了读。问那名员工 "这是什么"。他说"先生 这是我的辞职信"。"为什么辞职"。"先生 我刚犯了个错。给您的公司造成了一百万美元损失。我不想……您对我一直很好。我不想让您烦恼。是否要炒了我。所以我主动辞职"。Thomas把信撕了 说"炒了你?我刚为您的教育投入了一百万美元。现在我要炒了你?"。无论这则故事是否100%真是 还是部分真实。在IBM公司广为流传。在IBM公司营造了良好的氛围。告诉大家"你必须尝试"。世界最成功的公司之一 强生公司。也有类似的故事。Johnson将军。他与兄弟创建了该公司。他们俩都叫Johnson 所以叫强生公司。Jim Burke是强生公司的新人。由于一个错误给公司造成了巨大损失。简朴硬朗的Johnson将军。一早会见了他。Jim Burke走进他的办公室 可想而知胆战心惊。他走进Johnson将军的办公室。Johnson将军从桌边。走到他面前 说。"恭喜 Jim Burke 恭喜"。先生 我刚犯了个错。"是的 我知道 这是学习的唯一方法。现在犯得错越多 将来越有可能成功。恭喜"。Jim Burke非常高兴 正准备离开。Johnson将军又说"还有一件事 Jim。如果你再犯同样的错 我会炒了你"。这则关于失败的重要性的故事。在强生公司流传开了。而Jim Burke。成为了强生公司最成功的CEO。Thomas Watson 马上就结束了 再给我一分钟。 

"如果你想提高成功率。就要将失败率翻倍"。我们已经讲过这项研究。天才的起源。历史上最成功的艺术家。最成功的科学家。也是失败最多的 经历过最多的失败。爱迪生 达芬奇。米开朗基罗 居里夫人。最成功的。科学家与艺术家 经历过最多的失败。还有 Klam。Jane Klam对节食成功。或成功开始运动的人士进行研究。她发现最成功的。有毅力坚持下来的人。都平均经历过5次失败。也就是说他们减肥5次 都没有成功。到了第五次 或者第六次。才成功并维持住的。运动也是如此。学会失败 从失败中学习。 

周四见。 

第15课-完美主义 

大家好 上节课我们学到了休整。时不时地放下工作 放松一下自己。我想邀请你们参加一项传统并且。有历史渊源的休整节日 犹太安息日。这周五的晚上。哈佛的Howard Hill和Hervat将邀请大家。来Mac中心享用犹太安息日大餐。欢迎来到Mac中心的露天平台。和你的朋友们一起享用免费大餐。我们的Tal Ben-Shahar教授将会发表讲话。中心将于六点半开门 晚餐七点开始。如果想参加请迅速在Shiva1000.org上回复。所有人都欢迎前来。 

大家好。我还被要求替女子垒球队通知大家。她们原本是要来这里通知大家的。但是她们现在正在进行比赛 一场被推迟了的比赛。这周六和周日都有比赛。还有男子棒球队。所以请去为他们欢呼 还有一个通知。他们听不见 抱歉 好的。好的 就是5月21号 九点十五分开始决赛。希望你们能过去看看。是周一 好的。 

每年只要我一要讲完美主义的课。就有点坏事要发生。第一年。我记得我是把电脑忘在了家里。第二年 投影仪又坏了。而这一年 我忘了带电源线了。把电源线忘在了家里。所以希望这点电能支持一个学期。如果不行的话我们就只能面对不完美了。但是其实也没关系的。这肯定是潜意识里的问题。因为这也太巧合了。每次都有这些事情发生。 

的确是这样。那么上次我们讲到哪了。上次我们讲到你们是怎么学会走路的。我们讲到你们是怎么学会画圆圈的。都是通过不断地失败 如爱迪生所说。"我通过不断失败以至成功"。我们从那些最顶尖的运动员身上看到。不管是Babe Ruth还是乔丹。我们也从最顶尖的生意人身上看到。不管是IBM创始人Thomas Watson还是。Anita Roddick 美体小铺的创始人。我们从艺术家和科学家身上看到。最成功的人也是失败最多的人。而他们就是那些。深刻明白失败的价值与重要性的人。我向你们保证 他们没人喜欢失败。失败让人受伤 让人极度失望。让人沮丧 也经常使人尴尬。但是 那些最为成功的人 都会。认识到失败和从失败中学习的价值。有一本非常好的书。是由我的一位导师最近完成的。他在商学院教了几年书。他是一名在。南加州大学任职的教授 Warren Bennis。Warren Bennis写了一本书。名叫做《极客与怪杰》。在这本书里。他将非常年轻的顶多30岁出头的。年轻成功人士和老一代的。真正有所成就的划时代领袖人物。他们多在七十岁 八十岁甚至九十岁以上。他将两组人进行了对比。极客 指那些年轻的 怪杰 指年长的。而他发现 有一些。非常有趣同时也很有意义的差异。存在于年轻的二十多到三十出头的成功人士。和年长的成功领袖人物之间。其中最明显的差异就是。工作与生活的平衡。这对于年轻的成功人士来说非常重要。非常重要 他们自己也经常谈论这个。但是对于年老一代的成功人士那里。这几乎完全就是一个外来观念。这是什么意思啊 为什么会这样呢。因为他们总是处在工作中。在那个样本中所有人都是男性。而女性则待在家里 负责生活的部分。所以说他们的工作与生活的平衡是。丈夫负责工作 妻子负责照顾生活。在年轻成功人士的样本里 男女都有。那他们自然会谈论到工作与生活的平衡。两代人之间还有其它的不同。但是也有一个相同点。这一点对于两代人来说都是一样的。那就是有至少一次。一般来说要多于一次的 显著的失败。Warren Bennis称其为"考验" 一次灾难。无论是重要的竞争失败 或者失去了工作。还是受到了屈辱 失去了亲人。一些严重的考验 一次失败的经历。这是两代人都共同拥有的。他们都将失败看做一个关键点。一个生活的转折点。一些帮助造就今天的他们的因素。再说一遍 我不是说已经发生的是好事。而是他们能够好好利用已经发生的事。每个人都会经历考验。都会面临困难和阻碍。但是那些极度成功的人。如Warren Bennis的书里面写到的那些人。和一般人的不同之处在于。他们能够好好地利用已经发生的事情。他们将这些事情看成机遇。一种学习的经历 一块垫脚石。这就是失败对于适应力与完善人格的作用。 

我们继续来定义一下完美主义吧。因为现在有很多的定义。我们经常提及这个词。但我要说的完美主义是什么。我的定义是。一种充斥在我们生活中的对失败的失能性恐惧。尤其是在我们最在意的方面。注意"失能性"这个词 不只是对失败的恐惧。好吧 我也不认识哪怕一个不害怕失败。也不会因为失败感到尴尬或沮丧的人。这是很自然的。这就是人性 不管我们喜欢不喜欢。但对失败的失能性畏惧是。一种让我们在面对问题的时候。裹足不前的的畏惧。而我要在这个定义里强调的另一点。就是"尤其是我们最在意的方面"。你们也明白。我在玩大富翁的时候可不是完美主义者。这也不是因为……。这的确很有竞争性。但是对于我来说胜败并不重要。但是完美主义确是我生活的重心。比如说 就如同我提到过的。壁球对我而言非常重要。学术对我而言也非常重要。和别人的关系对我来说也非常重要。在这些方面 我就体会到了完美主义。也是在这些方面 我必须并且一直在努力。在这些方面 让我详细说明一下。这是我们在面对人生的旅途的时候。也是我们从A点到B点的过程中。采取的一种方式 一种认知与情感的基模。重点在旅途上面。一个追求卓越的人仍可能和完美主义者。一样 或者比完美主义者更有野心。两者之间的不同。在于他们面对旅途时的方式。我来举个例子 首先是一个完美主义者。他正站在A点。他想到B点去。什么是最完美最有效率的方式呢。就是直的路线。这就是路线 这就完美主义者在面对。旅途的时候抱有的认知与感情的基模。怎么从起点到终点。一个追求卓越的人。和完美主义者一样雄心勃勃。也想从A点到B点去。没什么不一样 不同之处在哪里。对于这段旅途 认知基模上的不同。追求卓越的人明白。当他进步的时候 不可避免地会遭遇失败。一个追求卓越的人明白。有时候她不一定能拿到她想要的分数。但她会吸取教训。一个也明白在恋爱关系之中。根本就没有完美的恋爱关系。所以 他会犯错。她也会犯错。但他们都会不断吸取教训。然后他们的感情不断地变得更加坚固。完美主义者明白。他需要不断地失败。五次 十次 有时候一万次。就如同我们了解的爱迪生的故事那样。一样地野心勃勃 但是基模有所不同。我必须画出许多不好的圆圈。在你们画出Vincent上次画出的那个圆以前。 

现在的问题是 哪个才更现实。拥有这样的基模是否现实。或者说这个才比较现实。我想这是很明显的。要达到成功可没有直线的捷径。要达成一段快乐的恋情也没有捷径。要发明电灯泡也没有捷径。学好心理学也没有。想成为好的父母或者朋友也没有。我们会犯错。而且 自然的规律我们必须遵守。追求卓越就是要克服困难。是啊 我也希望我能直接从A点到B点。我希望我能做到 我希望我不必经历失败。我不喜欢不断地失败与打壁球输掉。我当然也不会喜欢和我妻子意见不合。我甚至宁愿没有万有引力定律。然后我就可以到处飞或者到处飘。这是现实 而这就是幻想了。当我们心中存在着这样的基模的时候。我们就在让自己反抗自然规律。就如我们如果不接受万有引力定律的话。我们的生活就会充满不快与沮丧。我们活着 那么多人也活着 哈佛有不少人。不只是哈佛 所有地方都有 时常沮丧。因为在他们心里 他们一直有这样的基模。再说一遍 这并不是说我们该享受失败。但当我们有这样的基模的时候。同时又不享受失败。我们就会因为难以接受事实而沮丧。拥有这样的基模也会有后果的。我就想讨论一下这样的后果。怀有拥有这样的基模的人的一些特征。根本就没有完美的完美主义者。就如我说的 我们都存在于一个闭区间内。确实有一些人。确实有很多人更接近这个区间的这端。我想和你们分享的这些特征 都是那些。已经被记载于文学作品之内的。有很多人都拥有这样的特征。你们大家或多或少都具有一些这样的特征。而不是别的特征。这也没关系的。所有的这些特征 除了都已被心理学著作。详述以外 我也都有过体会。或者现在仍有所体会。所以这些特征对我来说具有一定的意义。它们本身也非常常见。 

第一个是 自卫性。在争论和讨论中 完美主义者都有自卫性。为什么 因为批评在定义上就是一种。对于完美的直线方式的偏离。这就是不完美 是一种挑刺。就是我没做得完美的事情。我们都不喜欢偏离我们所持基模的东西。记住了 我们的心是不喜欢内在与外在。之间的脱节的。我们需要的是平衡 一致。我们不喜欢横生枝节。而如果我们心中抱有的是完美主义基模。而批评又是对其的偏离。那我们就会变得有自卫性。这是我多次有所感受的。并且也是我一直在努力解决的。附带一提 或者说是正好相反。一个追求卓越的人是心胸开阔的。他会欢迎时不时的建议与批评。这不是说他或者她喜欢这样。而是他或她明白这是必要的。这是成长与发展的重要部分。完美主义者只注意到空着的那半个杯子。为什么 因为完美主义者为失败所困扰。被对直线完美方式的偏离所困扰。而不管是什么 困扰我们的。就是我们所关注的。他或她一直关注未能达成的东西。关注潜在的可能失败或者是真正的失败。追求卓越的人。则会专注于已经满的那半个杯子。为什么 因为他或她既享受达成目标。又学会享受过程中的每一步 那就是旅途。即便是失败也能是机遇。 

完美主义者会过度一般化 夸大化问题。非黑即白 我要么是完美的 要么一无是处。我们等一下就能看到。这样的"全部或没有"的对立于现实的方法。造成的后果。我们又谈到了 明白现实是什么样的。明白事物的过程是怎么样的。对于完美主义者来说 是没有自我接受的。对于追求卓越的人来说。能够接受现实中存在的各种弯路。而没有一步登天的捷径。能够接受个人的失败。能够将自己视作一个整体予以接受。对完美主义者来说 只有一种方式是合适的。并且是固定不变的。没有任何提升的空间 也没有丝毫偏离。更没有问题和错误。只有一种方式 而不是更为动态性的。更有弹性与自发性的过程与旅途。在行为中自我证实。完美主义者恐惧失败 恐惧来自于内心的。自己将自己视为失败者的失败。也恐惧他人将自己视为失败者。就是为了维持那种完美的幻觉。而不是将失败视为反馈 不是说要享受失败。我还没遇到过享受失败的人。而是要将失败视为一种反馈和成长的机会。完美主义者只想着怎么到达这里。他们只关注这个 只针对目的。直来直去的。别的事情根本不重要。如果它不能让我达到目的的话。但对于追求卓越的人来说。旅途和终点都是成功的一部分。这样的基模上的差异。会导致很多的后果。 

我在这里列举一些。第一个 就是完美主义者只能感受到。最多也只是 暂时性的轻松。是的。他们会感受到持续的压力 但当他们。达到目的之后 他们感到了轻松。这就是典型的逐利者。 

我来给你们一个这样的逐利者的例子。一个人 我们姑且称他为X先生。去上学了 上小学。在那之前。他一直很享受学前的生活。而他上学以后就开始有压力了。因为他明白 他必须很努力地读书。并取得非常好的成绩来上一个好初中。因为他父母想让他进的那间初中。竞争很激烈 非常不好进。他确实进了一间不错的小学。但是现在他要为初中而努力了。所以他有了压力 他并不享受小学生活。但他确实感受过轻松 在假期到来的时候。在全家去度假 或者考试结束的时候。在他可以不用担心考试和朋友玩的时候。他经过了小学阶段 却不怎么喜欢。他感受过轻松 却丝毫没有感受过。在他还在幼儿园的时候感受过的那种。对于学习的热情与喜爱。他后来进入了梦想中的初中。那所他一直想要进入的初中。他做到了。他只在前两周里感觉非常高兴。因为他要开始考虑。他梦想中的高中了。为什么 因为那所高中。是进入全国最好的那些大学的垫脚石。他也很想进入一间全国最好的大学。所以他在初中又很努力地学习。他并不怎么享受这样努力的过程。但是他成功了 他进入了梦想中的高中。他很兴奋 从来没有这么高兴过。因为他已经是全国最出色的孩子中的一员。进这所高中不容易 但是他做到了。然后他又开始学习。他又只开心了一周 压力又来了。因为这所高中里面竞争非常激烈。他必须加入两个运动队。还有三个学生组织 因为他必须。充实他的简历好让自己进入梦想中的大学。他过得很挣扎 一点意思也没有。但是他对自己说"这只是暂时的。痛苦是短暂的 等我进入了全国最优秀的。大学就会有回报的" 然后他付诸行动。在4月2号 一封信来了。信的信封很大 他打开信知道自己成功了。他又感到非常地兴奋。甚至比进入高中时还高兴。然后他说 "现在 我终于能放手。好好地放松一下了。因为我已经进了这间大学。我的人生从此稳定下来了"。接下来他十分享受高中的最后一年。还有假期以及进入大学的第一周。他很兴奋 这位X先生。但是在第一周过去后 压力又出现了。因为很快就会到期中考试。而很快又会出现激烈的竞争。并且每个人都和他一样努力。因此他也必须非常地努力。因为他想得到他的梦想中的工作。他在大一暑假的时候得到了理想的实习机会。但是压力又继续出现了。他感受过轻松 也感受过愉快的时光。但是这些都是在考试结束以后。或者假期即将到来之前。或者是在假期期间。但是每次都一样 压力又出现了。他自己也不明白为什么他不开心。因为这一切都是他想要的。他非常想要这些 多于别的一切。他说"好的 只要我得到这份工作就好了" 为什么。因为那份工作能让我进入最好的商学院。这能让我在一千六百个。一同毕业的人中脱颖而出。所以他继续努力 继续充实他的简历。他是三个学生组织的成员 担任两个的部长。是两个大学运动预备队员 一个正式队员。还要加上每学期五门甚至六门课。为什么 因为这很困难 竞争很激烈。这就是适者生存 不付出就没有收获。大学最后一年到来了 他又有压力了。但是他得到了那份工作 他很高兴。这就是他想要的。这就是他所预想的。现在他很高兴。然后8月3号到了。是他正式开始工作的第一天。他上班去。他周围的所有人都是最高等学府的。毕业生 都是最优秀学校的学生。他觉得已经功成名就 却只高兴了两周。因为在两周以后。他又感受到了非常大的压力。比大学里面的压力还要大。他没有时间再像大学里面一样在饭堂里。悠闲地吃晚饭。他每周工作八十到九十小时 一点也不享受。但是他会做到的 他会做得很好的。他会得到一封很好的进商学院的推荐信。在从商学院毕业以后。找到一个真正的理想工作并享受生活。然后他真的进了最高的商业学院。也找到了他最好的工作。工作上 作为一个员工 他很高兴。事实上他甚至不敢相信自己的好运。他赚钱赚得很多。那三年前看起来难以忍受的助学贷款。现在一下子就还清了。他感觉很棒。他有了一间很不错的公寓。他成了员工 他很兴奋。他很高兴 "我终于做到了"。但是过了几个月 压力又回来了。因为他只是一个员工。他非常非常想成为合伙人。但当他成为合伙人以后。当他获得了"终身任期"以后。他的生活就可以安顿下来了。他继续努力工作 奋力拼搏。他不怎么喜欢他的工作。只要再努力多一点 再忍受多一点痛苦。然后就会有真正的回报了。在艰苦努力了五年以后 他又做到了。他成为了合伙人 事实上。也是那个公司历史上最年轻的合伙人。他很兴奋。他休了一个长假 然后回来工作。斗志满满。因为他现在是所有者。是这个著名的公司的所有者之一。现在他也已经成家了。他买下了市郊的一座大房子 非常贵。但是他轻易买下了 他也买了一辆豪华车。他还有了一个司机 他感觉真的功成名就了。他感觉良好地过了三个月 压力又来了。因为他只是一个初级合伙人。他能成为高级合伙人吗 这可能吗。好吧 只有努力工作才可能。因为没多少人能爬到金字塔那么高的位置。越来越难了 但是他会成功的。他挣扎着 忍受着 一点意思也没有。但是他成功了 没有付出 就没有回报。这一次 他在Hamptons区买了一栋房子。用来度假和休息。还有一辆新车 更大 也更快。然后他说 我终于 终于 能够放松了。然后他开始放松自己 过了两个半月。压力又回来了 因为他只是个高级合伙人。但是常务董事只有一个。想得到那个位置是非常困难的。但是他决心要当上常务董事。他一直是明星级人物 没理由止步不前。这对他来说没什么乐趣可言。他没有花多少时间和他爱的人一起。而因为他心中的焦虑 他必须证明自己。足足花了七年时间。在这七年内 他在物质上已经很富足了。只是愈发地缺乏时间。他不断地奋斗 最后终于成功了。董事会发表了任命声明。他登上了华尔街日报的头版新闻。X先生成为了这件传奇公司的常务董事。他的朋友们祝贺他。他感觉万分自豪 他很快乐。但是随后焦虑回来了。有一天他走进办公室里。坐在椅子上 脚搭在桌子上。他已经成了老板 成了掌门人。他回头就可以看见中央公园。那景色实在太美了。他真的是 而不是修辞地说。他真的是站在了世界之巅。然后有人敲门了。他被敲门声吓了一跳。因为一般来人了他的秘书会向他通报。访客不会就这么过来敲门。毕竟他是站在世界之巅的人。于是他走过去开了门。来人是董事会主席。他们俩互相打了个招呼。然后董事会主席说 "今年的盈利记录。非常之棒 不可思议 您做得太好了。我们都为您而骄傲 感谢您的存在。但是 您应该退休了 您已经72岁了"。这就是逐利者的生活 

最佳情况下。当然这只是大概的描述。我只是着重描述了几个特殊的阶段。这也不是要么全部要么没有。但是 总的来说 这就是逐利者的生活。只关注从A点到达B点。然后轻松半个到一个月。然后……B点又变成了新的A点。然后 就像一直不断向上爬的老鼠。你们必须现在就问自己 不现在问那还要等多久。你们必须现在就问自己。"我到底想要什么样的生活"。我给的是一个生意场上的例子。其实对于其他领域也一样。不管你是想成为一个医生。是因为能否上到最好的学校而有压力。是想要最好的公寓 最好的实习机会。最想要好的 最高的职位等等。或者你是打算为非盈利机构工作。无论你是留在学术界 还是成为律师。都一样 都是在不同领域重复X先生的故事。这样的情况适用于任何地方。问题就是 "我们想要什么样的生活"。要记住其中最重要的。一个追求卓越的人。不会放弃自己的雄心。但是 一个追求卓越的人。也不会放弃整个旅途。我们很快也会讲到。追求卓越的人。不仅仅不会放弃整个旅途。或者结果。他们常常会获得更高的成就。我们很快就会讲到。追求卓越的人。并不只关注短时间的放松。还关注长久的满足。是的 生活中会有起起伏伏。但是 也能够享受旅途中的每一天。在大学里的经历。和朋友们美美地吃了一餐的经历。阅读名家的作品的经历。上有趣的课 参加有趣的课外活动的经历。并且重视这样的经历 而不是忽略它们。不是把这些看做普通的。为了成功的最终目的。 

我必须经历的事情。你或许会非常渴望成功。但是你所在的却是此时 此地。不是别处。你们也知道 我从来没有 或许你们有。我从来没有遇见过完美的人。我认为这样的人是不存在的。换句话说 如果根本就不存在完美的人。那么对每个人来说 失败都是不可避免的。为失败所困惑的完美主义者。只是专注于失败 而追求卓越的人。则会发现路途上每一步都有成功的机会。甚至在失败中也有这样的机会。完美主义者常常将时间浪费在。必须读清楚每个字。"我要做到最好 拿满分"。要么全部要么没有。要么我交上去一份完美的论文。要么我就干脆不交论文。要么我就拿到A 要么我就一无是处。要么全部要么没有 有时候这是合适的。有时候追求卓越的人也会持有这种态度。比如。如果你是一个医生 你想做一场完美的手术。你可不会说 "好吧 没关系啊。我切对了八成应该切的东西"。没错 在有些场合你会需要完美主义。但是在我们生活中的其他方面 都不必要。甚至会伤害我们 

完美主义者往往倾向混乱。我是指多方面的混乱。比如进食问题。我来给你们提供一个个人的例子吧。在我还在打职业壁球的时候。我必须控制我的食谱 要吃得健康吃得好。但还是产生了问题。我的生活中有一项重大挑战。这挑战就是我的母亲个完美的大厨。她做的的蛋糕尤其诱人。于是就有了这样的我在长大的时候。常常遇到的状况 痛苦的童年。我在壁球训练过后回家。饿得要命地打开冰箱。然后那里面有一大块蛋糕。不是一整个 但差不多四分之一个。非常大的一块 实在是非常诱人。我根据以往的经验知道 这很好吃。我看着这块蛋糕 然后关上了冰箱门。因为我不能吃。当时我正在在为一个重要的巡回赛训练。但是在三分钟以后。我回到冰箱那看蛋糕还在不在。为什么 你们中认识我兄弟的都明白。我兄弟块头比我大多了。他对我母亲的厨艺的喜爱一点不在我之下。所以我去确定蛋糕还在那。确实还在。再过了十分钟 我必须再去检查一遍。因为我兄弟在我视线内消失了几分钟。我以为他到冰箱那去了。所以我又打开了冰箱一次 蛋糕还在。然后我感到一阵轻松。大约五分钟以后我又回到冰箱那里。我打开冰箱门 两分钟以后蛋糕就不见了。要么全部要么没有 要么我不碰要么我吃完。这就是完美主义者持有的基模。不会说"好吧 我就吃一块" 要么不吃要么吃完。所以这就是完美主义者的基模。我要么成为超级模特 要么超重。要么全部要么没有 这是很有破坏力的。这会对我们产生各个层次的严重伤害。相对于健康的方式而言。好的 我就吃一小块。好的 我重了几磅 这算大事吗 我是人。我不是机器 也不是芭比娃娃 更不是Ken。我说错话了。这很伤自尊 为什么 因为。Nathaniel Branden写过很多关于自尊的文字。我们在未来三周内将会更多地讲到他。 

自尊的第一个基础就是自我接受。要记得完美主义者是不接受自我的。第二个伤害自尊的原因是。完美主义者总会遇到长期且不可避免的失败。因为根本就没有完美的人。而如果我一直失败或者视自己为失败者。那我还会有较高的自尊心吗 当然不会。另一个完美主义者容易自尊心受伤的原因。是他们更不愿意去尝试。而如果我不愿意尝试的话 其后果就是。更加低的自尊心。还记得自我知觉理论吗。相对的是持续的自我提升。不是一条直线直上的 而是旋转向上的。完美主义伤害人际关系。 

回想过去 我从这个角度来说。我很难捉摸 我基于完美主义。在人际关系中所犯的错误的类型。为什么 第一 自卫性。如果我一直是处于自卫性的位置。如果我不能接受或者处理批评。那我就很难和别人形成亲密的关系。这样的情况下怎么可能建立亲密关系呢。 

另一个完美主义伤害人际关系的原因是。我们经常。我们对世界的看法和对自己的看法一样。如果我是完美主义者那我期望的就是完美。对我来说是一条直线 对他人亦是如此。不论是对于我的伴侣。还是对我的孩子和朋友。现在我们……。我以前也说过了 没有完美的人。我们第一次见面的时候可能感觉是完美的。在度蜜月的时候 也是完美的。他或她是完美的。但是突然 我们开始发现对方的缺点了。那不是我一开始以为可以在一起的那人。当然 如果我期待的是芭比娃娃或Ken的话。完美的 但是对方可是人。如果我从对方身上期待的是完美。我肯定会失望的。这也会引起沮丧。对自己的伴侣不认可。也常常会伤害到人与人之间的关系。相对于持续成长的关系来说。这是我们会在谈论人际关系的时候谈论到的。理想的人际关系。不是一个没有失败。没有不和的关系。理想的人际关系在包含众多积极因素同时。也会包含不和 失望和争吵。那才是健康的人际关系。这让人际关系随时间变得更加稳固。 

关键是其中的度。你在人际关系中包含的。积极因素与消极因素各是多少。这里的关键不是只有积极而没有消极的因素。这样的关系是不健康的。这样的关系。有过多的抑制与压抑。完美主义会导致焦虑和压力。总会有对于失败的恐惧。而不是兴奋。这就是Peter Senge所说的"创造性张力"。但是表现如何。那句已经镌刻在我们心中的。"没有付出就没有收获"的真言呢。我是说我们很清楚地都知道。我们必须要努力才能成功。那难道我不应成为完美主义者吗。而人们不愿放弃这种方式的一个原因是。他们认为这是最快的方法。是最为有效的通往成功的方法。但是最后事情往往都不是那样的。长久来看 追求卓越的人往往。会取得更好的成就。在这方面有很多研究进行了阐释。我来举出一些成果 一些原因。现在说说少一点付出 多一点收获。首先 以终极货币的形式。用幸福来进行衡量。追求卓越的人。远比完美主义者要快乐。这很好 也很棒。但也不仅仅是在终极货币价值方面。在硬通货等物质方面也是同样的。不管是工作上的成功 还是体育上的成功。还是处理人际关系上的成功。为什么 其中有很多原因。首先 追求卓越的人。享受更加可持续的成长。记住了。这个我从环保方面引用的词 可持续增长。不是要创造一个贫瘠的环境。不是关于回到以狩猎与采集为。主要生活方式的时代去的。是关于以现代生活方式生活。是关于享受进步与现代性 与此同时。不从环境中索取过多。让发展得以持续下去。放到个人层面上也是一样的。完美主义者不会享受持续性的成长。因为他们不允许休整和偏离。所有东西都是不变的 如同一台机器。至于追求卓越的人。有所偏离 有所中断 也有休整。相比起来更加可持续 如果我受伤了。我的壁球生涯就结束了 因为要么全部要么没有。要么我就完全不训练。要么我就像世界冠军Jansher Khan那样训练。同时追求卓越的人所拥有的。固有动力也比完美主义者高很多。当我们有固有动力的时候。我们就能更好地在一段时间内持续地。付出努力。如果所有的动力都是固有不会减退的。那就能上升到另一个层次。获得嘉奖 或者能得到升迁。丘吉尔说完美主义让人瘫痪。当我们被失败所困扰的时候。我们就更不会付诸行动。完美主义是事情耽搁的最主要原因之一。对失败的失能性畏惧。因为只要我们不行动 我们就不会失败。就如同我之前说的 追求卓越的人的自尊。相比起来更高。而且如Nathaniel Branden所说 自我知觉。是终点 而信念变成自我实现预言。如果我相信我自己 如果我认为我能成功。如果我的过去有跌倒之后。又爬起来的经历。那我就更可能成功。我以前讲过这个吗 好的。因为这很重要 要成功没有别的方式。你们读到了"运气的因素"。英国的Wiseman教授极为优秀的作品。他所说的是 比别人有更多运气的原因。似乎是一件很神秘的事情。但其实也是可以用科学方法解释的。其中一个有更好运气的方法。便是尝试新事情 小事情。即使是你课后去食堂的路上。找一条和你每天所走的那条路。有些许不同的路线。而她说 这些小事就可以改变我们的生活。这些小小的对于直线方式的偏离实际上。能带来更多的幸运。因为我们开始看见以前没见过的东西。完美主义者 只有一条路。没有偏离 没有进步 没有自发性。这些时常都让我们变得不那么"幸运"。说到创造力 Simonton说 那些历史上。最成功 最有创造力的科学家和艺术家。都是那些失败得最多的。他们都不是完美主义者。他们都追求卓越 非常有雄心。都明白要成功没有别的方法。John Updike 一位非常有创造力的作家。说过"完美主义是创造力的敌人"。 

我们会谈到另一个作家。他对于我们如何克服这种完美主义。有非常好的建议 他叫Samuel Coleridge。80/20规则。这条规则 了解这条规则并将其应用在。时间管理上改变了我在哈佛的学习经历。那帕累托法则是什么呢。帕累托是100年前的意大利经济学家。他得出了这样的法则。并将其命名为帕累托法则。这法则表示 在大多数社会中。百分之二十的人拥有百分之八十的财富。他们也将这条法则应用于商业组织。那就是你百分之二十的客户。给你带来了百分之八十的收入。这也被用于经济学的许多领域。而到最近 被用于时间管理方面。比如说 在我们百分之二十的时间里。我们可以完成百分之八十的工作。而当我明白这一点的时候。它改变了我对学术的态度。为什么 因为我意识到。我不必将那些材料完全读完。再说一遍 这点别学我。我交论文前不必确认每一个t都写好了。也不必确认每一个i上面都点了点。我开始遵循更多的愉悦原则 或者说。快乐原则 于是我说。"好的 这才是我感兴趣的东西"。比如说 我不知道这门课还有没有。文学与艺术学院课程 C14 "英雄"。还能报名吗 这是我本科时候最喜欢的课程。而且这课程还不算太难。我完全可以不怎么努力也得好成绩。但是我在这科的期末论文上花了50个小时。因为我写的主题是我非常关心的。但是我没有得A 对我来说毫无道理。但我仍然在上面下了大工夫。因为这实在是太有意思了 至于其它课程。我只投入了能让我得到足够成绩的时间。不是要么全部要么没有。不会说要么我就得A 要么我就不学习。另一门课。我不知道这门课还有没有 还是文学与艺术。Maria Tatar教的"神话故事" 她现在上别的课了。我本科时看见她第一眼就爱上了她。我真不敢相信我就这么说出来了。那么 她拒绝了 其实我没有问她。也是在那门课上 不算太难的课。在许多层次上来说非常享受的课。又一次 我花了许多时间在这篇。最终成为一篇以自尊为主题的。专题论文之上。所以再一次问自己到底什么才是重要的。除了要么没有要么全部外 我还在乎什么。我的成绩确实下降了一点点。但是只是很少一点点。因为我在应用这条20/80法则。将百分之二十的注意力放到重要的东西上。然后把百分之八十的工作做好。而我在学术的其它领域。却获得了更多的成功。因为我有了更多的时间来打壁球。来享受壁球练习。我有了更多时间来和别人交流。有时间去坐在法学院里吃一次五点开始的。持续两个半小时的晚餐。就是在这样的基础上实现的。另一个对于80/20的应用。我们每人在一天中都有不同的。最有工作效率的时间。实际上 在一天中百分之二十的时间内。我们完成了百分之八十的工作。这取决于我们是早起的鸟还是夜猫子。有的人早上6点或7点就起来了。然后他们立刻就可以工作了。我也是这样的 我很早起床 对我很简单。我可以直接开电脑工作或去给孩子换尿布。很早的早上 直接去工作 没问题。但是到了晚上9点10点 就完全不行了。有人却是截然相反的。他们可以熬夜到凌晨2点 3点 4点。那才是他们最有干劲的时候。这和我们内在的生物节奏有关。和我们的生理节奏有关。因为有人。基本上没有人的周期是正好24小时的。大多数人都是23到25小时之间。对于周期是23小时的人来说。他们到了晚上就会非常疲劳。但是他们早上却会精神抖擞地起来。那些周期是25小时的人。早上需要多睡一会。就是因为那些人我才将这课调到11点半。我知道是很早但是感谢你们的到来。我上大学的时候 大多数人都学到很晚。我的室友就学到很晚 从晚上10点开始。然后学到12点甚至凌晨2点。这才是他们最有效率的时候 可我不是。我宁愿早上早点起来。我在晚上会很累 根本无法集中精神。我听说了这样的效率时段后 我改变了作息。我早早睡觉。在我的室友开始学习的时候 我上床睡觉。然后在我的室友还在睡的时候起床。然后在早上早早地把工作完成。做完以后去上11点的课。在早上的那3个小时内。我完成了更多东西 更有效率更有创造性。比我在晚上熬夜有效率得多了。我刚才说到的那两样东西。真的改变了我在这里的经历。在更短的时间内学到更多。不是完美主义的方式 是追求卓越的方式。 

大家都了解过"流动"了。Mihaly Csikszentmihalyi谈论过流动。是我们最良好的感受 最出色的表现。也就是"没有付出就没有收获"。我们什么时候感受到流动。我们焦虑的时候感受不到。就如完美主义者那样 我们无聊时感受不到。只有我们乐观愉快 兴奋的时候感受得到。追求卓越的人。更容易感觉到流动。而那些一直畏惧失败的人。一直为失败所困扰的人。一直在设想如果我失败了怎么办的人。我来谈谈完美主义的根源吧。因为一旦我们明白 就如我上次说的。一旦我们明白它是怎么来的。我们就更能克服它。完美主义的根源。 

第一个 最重要的因素就是社会影响。我们都不是天生的完美主义者。你们在录影带中看到的孩子 他们天生。比我们更能享受学习的过程。我们畏惧失败 他们却不断跌掉了又爬起。所以是社会的影响导致了。我们对于失败的畏惧 确切地说。是这样一种在我们心中烙下的。并且几乎从出生开始就不断被加深的基模。那就是 成果才是真正重要的。当我们得到了某个成果的时候。比如学会走路 做得好。在我第一年期末的时候得了A 真不错。只有我们的成果受到了奖励。过了没多久 我们就开始内化这种基模。并且我们也开始相信我们接受它。当我们达成了某种成就的时候。这样的事在我们的生活中一直延续着。我们什么时候会得到奖金。在4月3号吗 当然不是 是在12月31号。或者说是圣诞节之前 年末。今年大家都干得不错 很好。我们什么时候会得到分数。还被别人轻拍后背表示做得不错。是在学期末。或是在考试后 我们完成了某些事的时候。不会在旅途中得到。换句话说 整个旅途都没有受到奖励。于是我们也就开始认为它是无关的。是不重要的 只是到达终点的一种方式。只有非常少的教师 家长 组织。还有学校会奖励旅途 奖励旅途中的愉快。当然 也有很少的人会奖励失败。可这是旅途中不可避免的。所以我们觉得这不好 这一点也不好。我们必须尽可能地追求直线。一种直接的基模。我们内化它的时候也付出了代价。这就是我们出生所在的社会环境。我们成长所在的社会环境。而这样的社会环境很难以改变。我也知道 我家里也有孩子。这太难以改变了。因为我自己也时常对孩子说并且专注于。"Shirell学会走路了。做得好 你学会走路了"。而不是奖励其中所付出的努力。我们等会会讲到 

奖励付出的努力。甚至奖励失败 应对与整个过程。非常难。这当然也和准许为人有联系。准许看到人类本性的限制。而不是理想化 孤立地。分离地看待生活。因为这不可能。如果不是准许为人 而是要求完美的话。压力是持久存在的。存在于媒体中 存在于工作中。存在于教育机构中。这无处不在 我们也为之付出了代价。 

现在 我来给你们看几个例子。在那前先给你们介绍一下。你们都知道 学术研究赚不了多少钱。进入学术圈的人 大多都不是为了钱。有很多学者。我也不例外 必须额外赚取收入。我也有额外的收入。��我希望和你们分享一下我��外做的事。我的眼镜��饰了我有腹肌��事实。那么 他是真实���吗。这间房间里有谁是长这样的吗。我也没指望你们来回答。但我的意思是 他是真实的吗 部分真实的。这照片修过不少地方。但是没错 确实有人长是这样的。但是我们在杂志会看到什么。我是说 如果你们进一家杂志店里面去。你们也知道这种杂志的 这些就是。封面上的人物。十分完美 完美的存在。甚至更加地。我们的Ken还有芭比娃娃都更加完美。我在谷歌图片上找了好久 来看看吧。"科学的进步"。"近代艺术的范例"。就像今天的Cosmo杂志一样 不是么 看这个。"性爱女神的秘密"。我是说 这甚至不是"性爱人类的秘密"。那还不够好。看看我们为自己定下的这些标准吧。我怎么才能变成那样 这里 我没看到。"百分之三十的……"我应该好好读一读。但这不是我想指的。其实我想指这个"如何快速进步"。要多快有多快 对吧 时间不多。"永远不长痘痘"我都37岁了还有痘痘。他们怎么做到的 又是不现实的模型。1997年我还在新加坡住的时候。我想带Nathaniel Branden去新加坡。他被认为是自尊研究的领头人。他的妻子Devers Branden也在自尊。与自我成长方面花了很多功夫。所以我希望能让他们来新加坡。所以我请他们来为我工作的公司。做咨询。我在寻找其他途径。因为其他途径所要筹集的经费比较多。而我的一个好朋友。Pat Lee那时在为美体小铺工作。她说"这实在是太好了。因为美体小铺还有Anita Roddick。我之前提过她 对于女性自尊也很感兴趣。那为什么我们不能让Devers和Nathaniel。还有美体小铺联合起来。在新加坡搞一次关于女性自尊的活动呢。我认为这主意非常不错。我和Anita Roddick见了面。然后我们共同赞助了Nathaniel Branden。后来这次活动非常成功。本来有700个座位 可来了1000多人。我们得在场外放置等离子大屏幕。还有 这次活动对很多人产生了激励。并激发了一场运动 或者说是小规模运动。在新加坡的 女性自尊的运动。作为结果。Nathaniel Branden写了本书 叫"女性的自尊"。因为他被我们在研讨会上所讲述的。那些故事给深深吸引了。不管怎么说 Anita Roddick在女性和自尊方面。做了不少工作。其中一个项目 她在1997年策划的。我觉得你们大多都太年轻了 可能不知道。但是所有美体小铺的分店都有一个。叫做Ruby的洋娃娃 用来取代芭比娃娃。然后那次活动是这样的。抱歉。"有三十亿女人长得不像超级模特。只有8个才像"。或者"媒体们。在我们家乡 文化与生理的多样才是标准。那里就是地球"。这次活动非常成功 也很重要。因为我们今天文化中的。模型就是这样的 要么全部要么没有。这些照片是一同被找到的。我没把它们特意放在一起。要么你大吃大喝 要么就不吃不喝。我们从电影中得到了什么。谁能达到《壮志凌云》里面的标准。汤姆克鲁斯和凯莉麦吉利斯。他们拥有的那种完美的爱情。但这是我们所有的模型。等我们谈到恋爱关系的时候会讲到。谁又能度过《爱你九周半》那样的九周半呢。更不要说过一辈子了 那电影里的激情。都是不可能的标准 还有那些自助的书。比如《思考致富》。和这个没什么关系 只要思考 就能致富。这本书卖了几百万本。这里还有一本书。这本书或许不错。我得承认我还没有读过。但先看看 《小步迈向幸福 现在》。我们又迫不及待了 说的是专注于目标。而不是专注于努力 旅途 人格。还有所需的改变。还有别的。 

还有别的导致完美主义的因素。这是一个你们已经读到过的研究结果。这是由Carol Dweck完成的出色研究。她现在在斯坦福大学工作。这就是她的研究成果。因为。她所描述的是 不是所有的赞美都是好的。对一个孩子说 你很棒 你很出色。不吝赞美之词。你太聪明了 你真是惊人 我的小爱因斯坦。这也不总是好的。长期来看这可能也是有害的。这也会导致完美主义的基模。 

那么我和你们分享一下她的研究成果吧。我想你们已经读过了。她所做的就是找到一些十岁的孩子。然后把他们随机分成两组。第一组孩子都做了道题。他们都每个人独立完成了。到最后 对每一个完成的孩子。她都说"你真聪明伶俐"。当然 孩子们都感觉不错。第二组做的是同一道题 做完了 做的不错。结束后 她说。"你真努力 你很认真"。这随机分成的两组孩子。一组是"聪明伶俐" 一组是"努力认真"。 

然后她开始第二部分的研究。第二部分。两组孩子要选两道题。他们被告知其中一道很简单。他们可以很好地完成。另一道非常之困难。但是他们能够从中学到许多。那组被称赞。聪明伶俐的孩子里。五成孩子选了简单的题目。五成孩子选择了可以学到很多的难题。那组被称赞努力认真的孩子里。九成孩子选择了。他们能学到很多的难题。这就是研究的第二部分。 

第三部分。这次她让孩子们做。一道非常难的题。这题基本上是无法解答的。她想看看两组孩子的反应。被告知他们很聪明。很伶俐的那一组。他们没有坚持多久。并且非常沮丧。并且很快就开始放弃了。与此相反 被告知努力认真的。那一组孩子。他们更能坚持并且享受解题的过程。即使到最后他们都没能解开这道题。但是他们享受这个过程 并且更加努力。看看这微小的操纵带来的结果。简简单单的一个句子"你真聪明"。"你真努力"。一句话带来了这么明显的不同。用她的话来说 "重视努力让孩子拥有。一个他自己能掌控的变量。这能让他们认为自己能掌控自己的成功。强调自然的天赋。让孩子们无法掌控成功。这不会给孩子提供面对失败的方法。事实上当你仅仅注重于天赋的时候。你就是在制造完美主义的基模。而不是专注于。旅途当中 努力的基模。她继续说道。"如果你称赞孩子的智力 当他们失败时。他们会认为自己不再聪明。然后失去对于眼前工作的兴趣。相反 那些被称赞努力的孩子。在困难面前不会气馁 甚至更有动力。这两种方式造成了多么明显的不同。被称赞智力的人会产生这样的基模。要么我很聪明 要么我就不聪明。我很畏惧不聪明。所以我就选择容易一点的任务吧。当我无法做到的时候 这就是攻击。是对于我的基模的侮辱 没错吧。这是对于直线的偏离。 

相对的 如果努力才是最重要的。"好的 让我试试看吧"。"我学到了很多 这真有意思"。因为 一个是"既定思维模式"如Dweck所说。而另一个。是种可拓展的 时刻改变和发展的思维模式。一个是 畏惧失败 因为我想聪明。被人称赞聪明很有意思。我不想威胁到这个基模。相对的是 努力的 重视过程的。当追求卓越的基模。成为惯常做法。 

那么我们如何克服完美主义呢。如果我们确实有这个问题怎么办。要记住 根本就没有完美的完美主义者。也没有完美的追求卓越者。你们必须选择你能快乐的生活方式。选择你认为能够更加快乐的生活方式。不会有现成的答案摆在你面前。对你而言可能只是有趣的学术理论。或者是非常个人化的原因。 

那么我们该怎么更多地采用卓越的基模呢。首先 是自我认知 在于自我了解。对我来说 明白 "我太过于有自卫性了。我不想再这样了"是一个大突破。因为我知道这会伤害亲密的人际关系。并且从那时候我开始在这上面努力。并且有所进步。并且在将来仍将继续提升我的人际关系。或者"看看我有多么畏惧失败。我有多么害怕听到个不字。看看我有多么害怕被拒绝。看看我有多害怕应对与尝试"。就应该意识到它。如果你想让你的网球技术进步。你必须首先明白你想要进步的是什么。这就是第一步 说起来容易做起来难。我来给你们介绍一个简短的例子。是关于一个不完全了解自己的弱点的人的。 

视频:威尔和格蕾丝 第三季第十集。 

你好 Jackie Brown。为什么没有咖啡了。和你没有老婆和三个孩子的原因相同。上帝决定的。该死的 这可是间办公室 应该有咖啡的。嗨 嗨。为什么不把你的怒气留在卧室里呢。为什么你不去见你的咖啡小爱人呢。你的拿铁小情人 疯子先生。他辞职了 好吧。这是什么意思 你在审问我吗。难道想喝咖啡时 就不能好好的喝一杯热咖啡吗。还是说我们都住在"茶之国度"里。只要提到"咖啡"就是犯罪。好了 神经质 冷静一点。别让我冷静一点 我很好。好吧 或许你想照照镜子。因为你看起来有点疯 好吗。我的天啊。Karen 我到底是怎么了。好吧 我不是专家。但我觉得你是犯瘾了。别把眼泪滴在台布上。犯瘾 不 不。我可没上瘾 多谢你了。- 好的 好的 - 我没事 我没事。好了 好了 都结束了。现在听我说 \[Karen拿起药片\]。首先你要做的就是承认自己的问题。\[Karen把药片放进酒里搅拌\]。因为如果你连症状都不能发觉。\[Karen把药片吞了\]。那你就真的糟糕了。\[Karen用酒送服药片\]。你怎么对我这么好呢。Karen 我办不到 我真的上瘾了。那么甜心 如果有作用的话 我就和你一起吧。我也戒咖啡了。是啊 这不简单 但是每当我早上起床时。我只要习惯直接喝掉我的百利甜酒就好。那仍然会是起床最好的一刻。好的Karen 我们一起能做到的。让我再喝一杯。不行。半杯。不行。你喝一杯然后我舔你的舌头怎么样。很诱人 但是不行 过来吧。Karen 小羊们什么时候才能不叫啊。 

不管是要应对上瘾问题。还是对于生活中别的问题 完美主义。还是要改善什么 第一步就是要有意识。第二步是 还有五分钟就完。那些看着钟的同学 那钟快了五分钟。或是四分钟 还有四分钟就完 专注于对努力的嘉奖。Carol Dweck向我们展示的。就是当我们专注于努力的时候。我们就能够改变那种根深蒂固的基模。即便这种基模已经存在了多年。所以对我们自己 或者对别人。我们应专注于旅途。专注 并不时地嘉奖自己。甚至是嘉奖自己的失败 自己的尝试。在几个小时内 如Carol Dweck所见。告诉人们他们的心是可以拓展的。告诉他们神经可塑性 并非已经定型。我们仍可以作出改变。积极的接受。Karen Horney。我认为她是幸福学创始人之一。她在神经症领域做了很多工作。她研究所得的其中一点是神经症。它从来不会消失。它总是我们的一部分。它会变得更加可控制 但是总是存在。而极度的完美主义也是一种神经症。所以关键就是接受它。我总会有完美主义倾向 但是这没关系。但是 在以前。我有着极度的完美主义。现在我更加接近追求卓越的极致。这是持续一生的 是不断进行的一个过程。这是不断进行的一个过程。所以现在的关键就是接受。它一直都是我们生活的一部分这个事实。然后问 "好的 那我该怎么办"。换句话说就是采取行动。行为。所以举例来说 就是怎么应对 把自己置于线上。我最初所做的。在我留意到我的完美主义与自卫性的时候。我会撇开我的做法并且请求批评。我会问别人"给我点反馈吧"。当我收到负面的反馈的时候。和我的直线有所偏离的时候。我就阻止自己回击。你们也知道 对于完美主义者来说。最好的防守就是进攻。慢慢地我就习惯了 我没那么有自卫性了。我得以创造更高层次的亲密关系。或者我开始尝试别的领域。那里我可能会遭到否定。让我以一个故事来结尾吧。 

她的名字叫Brittaney 她是一个展示女郎。我们在大学第一周时认识 我对她一见钟情。我们几乎整个第一周都在一起。到那一周的周末。为了克服我的完美主义。我约她出来 她拒绝了。这让我很吃惊 很受伤 但是她确实拒绝了。在大学第二年 我们又见面了。我再次约她出来 她又拒绝了。第三年 在年末的时候 我和另一个人约会了。所以Brittaney就出局了。然后我们又见面了 在我大四的时候。我们谈了些有趣的话题。那时候我已经开始学习心理学了。我也可以读懂她的身体语言。她经常这么弄她的头发。然后我知道她确实很喜欢我。然后我约她出来 她拒绝了。然后我就还留在这里和你们在一起。这就是个很重要的学习过程。学会失败 或败于学习。 

我们下周见。 

第16课-享受过程 

今天有人有消息宣布吗 没有。好的 早上好。 

我们今天要把完美主义讲完。然后开始讲精神肉体。这个学期对我来说挺不错的。没有什么东西。让我后悔做过。不过下次我还想挑战点别的。教这门课时 我现在知道了。应该更早讲精神肉体的。因为这个话题十分重要。它是如此根本而且基础 因此当我意识到。在这个学期很早的时候就意识到。我应该早点讲。我那时就开始停止介绍身体锻炼。开始讲"念"的概念。从较低一点的层次来介绍它。今天 在一两个月之后。我们要再讲讲精神和肉体的联系。身体的健康。首先我们先把完美主义讲完。 

上节课我们讲到了两个模型。完美主义的基模。一条从A到B的直线。以及追求卓越的基模。更像是蜿蜒曲折的螺旋上升。有些人昨天在办公时间来问我。"区别到底在哪里"。我们可以用无意义追逐目标的人来类比。还记得那个72岁的人吗。他达到了很高的地位。管理着世界的中心。在承受了一生的压力。不停地追逐到达B点。而不是享受当下 享受旅途之后被告知。退休的时候到了。 

我问你们一个简单的问题。"你想要怎样的生活"。那么另一种模型是怎样的呢 积极的模型是怎样的。 

积极的模型从外表上看来似乎一模一样。可能也是一个充满雄心壮志的人。极度勤奋刻苦的人。可能是个投资银行家 或者是个医生。或者律师 老师 或者在流浪者之家工作。从外表上看可能完全一样。但其内在则是迥然相异。怎么个迥然相异法呢。区别在于这个追求卓越的人。也许同样充满雄心壮志 有同样的目标。也可能达到了同样的目标。然而这个人同时享受这个旅程。这就是区别所在。是的 他或她会进入名校。进入他们非常想进的学校。在学校里他们非常勤奋刻苦。但同时他们也知道 他们几乎可以。不是完全 但几乎可以。享受他们在学校的所有时光 同样的起伏坎坷。同样的兴衰变迁 但他们享受这个过程。然后获得他们想要的工作。或者他们想要的工作之一 然后继续努力工作。充满雄心壮志 每周工作80个小时。然而这个人同样享受整个过程。然后他们享受下一个旅途 和别人合作。成为别人的搭档 当上营销总监之类的。唯一的区别就在于他们享受这个过程。 

我还可以举出一些实例。其中之一就是我的前室友 他是个投资银行家。我们一起上大学 也是96届的。毕业之后 由于其经济学的专业。给一个著名的对冲基金工作。之后他就回到哈佛 获得法律博士和工商管理硕士的双学位。学业非常优秀。之后就获得了所有他想要的机会。他又选择了另一家对冲基金。之后他又自立门户 和一些朋友一起开了一家公司。当我和他谈起投资银行时。我对此了解很少。我对投资银行表现得很兴奋。因为他的兴奋能感染我。他每天都期待着工作。他工作的时间也很长。他就是一个追求卓越的例子。虽然从表面上看。就和我之前描述的形象。一模一样。因此不是要我们失去雄心壮志。也不是变得没有竞争力。如果那能让你快乐的话。这只是让你们学会享受过程。朝着你认为有意义的目的地。所迈出的步子才是最重要的。 

而问题在于 "我们如何克服完美主义。这个解不开的结呢"。我们如何克服活在未来中的感觉。因为害怕失败而害怕当下的感觉。我讲了一些方法。 

首先是认识它。认识到我想改变什么。我想保持什么。还记得Ellen Langer的研究吗。关于更确切地了解。我到底想改变什么。因为长久以来。我都是个完美主义者。因为总是想着完美 总是想着成功。我的潜意识让我没法摆脱完美主义。因为我会认为那就意味着。把宝宝和洗澡水一起冲走了 也就是成功被冲走了。但现在我更确切地了解了完美主义。它就是一种对失败的无法抑制的恐惧。我明白了我仍然可以充满抱负。而无需放弃成功。同时享受这个过程 这是更好的理解。这样我就能继续前进。因此第一步就是认识。 

在我们获得了认识之后。就是Dweck关于工作回报的概念。记住Carol Dweck能够在短短几小时内。就改变学生和成人的思维模式。有时候。这些思维模式在他们非常小的时候就已经奠定了。思维模式是固定的 还是可塑的。Carol Dweck通过教他们可塑性。神经的可塑性 我们的大脑是时刻在变化的。通过教他们关注过程 关注所付出的努力。而不是结果 或者最后的成就。从而改变他们的思维模式。我们自己也可以关注我们付出的努力。关注过程本身 不管它是失败还是成功。主动地去接受。它将永远是我们的一部分。完美主义者经常会做的一件事。当他们意识到自己是完美主义者时。这让他们很难受。他们就用完美主义的方式来克服完美主义。也就是说。我会成为一个完美的追求卓越的人。这不会起作用的 只是更进一步的完美主义。但主动的接受意味着。"好吧 我仍旧是个完美主义者。我可能永远都会有这方面的性格。这没关系 没有关系"。佛教徒会说的一句话是 通常情况下。我们面临的最大挑战。Karen Horney会称之为"神经官能症"。不管是完美主义 对失败的恐惧。还是时不时出现的愤怒 焦虑。不管我们的问题是什么。佛教徒很巧妙地重新定义了它。"我们要将其视为一种工具 一种成长的工具。一种了解自己的方式。使我们深入挖掘自己 然后得以成长"。要这样看它 这是个成长的机会。了解自我 反省自身的机会。能够更设身处地为别人着想的机会 无论它是什么。将其视为一种工具。接受它 然后行动。有助教能查查看它到底怎么了吗。谢谢 John。 

下一步 在接受它之后 我们就去改变它。我们要通过建立行为来改变它。这里的关键在于行为。我上节课讲过的一件我做过的事。就是向别人寻求反馈。寻求批评。一开始我要控制自己不去回答。不去顶嘴反驳。为什么 因为批评是对直线箭头的偏离。偏离那个基模。但我逐渐习惯了 这真的帮到了我。我现在更能够包容批评。然而。我有时仍会想回答 想顶嘴反驳。但总体来说 我通过改变态度的行为变得更好。还有一点 使自己更经常处于水平线上。包括我跟你们讲的关于Brittany的故事。连续说了几次不。但没关系 她说不。我还在这 我还活着。屡战屡败 屡败屡战。 

下一步 我们不一定要外在地表现出行为。我们也可以在内心中模拟。你的思想是无法分辨。现实和想象的区别的。因此我可以想象自己 幻想自己。就像一个追求卓越的人一样。比如说在上课之前。我想象的一件事就是。假想我自己站在台上 感觉很轻松。冷静 并且很包容 换句话说。想象追求卓越的人会用的方式。而非总是害怕失败的完美主义者。你的大脑是不会知道有什么区别的。你的大脑需要一致性。因此我就能那样去做。或者更容易那样去做 因为我先幻想了。同时冥想也会有所帮助。我们可能会在周四讲到冥想。因为冥想能使我们处于冷静的状态。能够接受事物 是一种存在 而非一种行为。当我们作为一种存在时 我们就能活在当下。我们就能享受过程。而不是总想着目的地。以及如果我达不到这个目标会怎么样。周四我们会更深入地探讨。 

这是我从Samuel Coleridge那里学来的方法。一个英国的哲学家。活跃于18世纪末 19世纪初。影响了爱默生的哲学家之一。Coleridge本身是个完美主义者。后来丘吉尔说完美主义。"就是瘫痪"。他在写作时就经历了这样的"瘫痪"。写作是他生命中最重要的一件事。但他就是没法写好。因为他很害怕写不出那终极的文章。或者终极的诗。所以他所做的是 他说 "在我生命的结尾。我会写出我的巨著。在那之前 一切都只是草稿"。这让他得到解放。因为他不再焦虑了。他再也不用想着写巨著。而同时 他写出了无数优美的诗篇。有些是英语中最美的文字。以及许多非常强有力的。具有影响的 结构精妙的文章。而他们都只是"草稿"。这样就没有压力了 这给了我很大的帮助。当我想到这个时 我说 "好吧。看看完美主义。对我影响最深的一个领域吧。因为这个领域对我很重要 就是教学"。我就对自己说 "好吧" 这是大约十年前的事了。我说"二十年后 我会在名列前茅的。学院中任教。能开很棒的学术报告和研讨会。在那之前 一切都只是草稿。在那之前 一切都只是准备" 这真的起作用了。我现在知道这是个心理游戏。我知道我是在骗自己。因为12年后 等目标要到了的时候。20年到了的时候。我会再定一个20年后的计划。来解放我自己 这样我才能享受当下。这样我才能享受过程。而不是执着于要变得完美。毫无半点瑕疵 这是永远无法达到的。 

因此制定一个终极目标可以让你解放。这是铂金法则。黄金法则就是。"己所不欲勿施于人"。或者说"己所欲方施于人"。这是当今很多道德体系和宗教信仰的基础。铂金法则就是取自黄金法则。再对其进行小小的修饰。它说的是。"人所不欲勿施于己"。或者说"人所欲方施于己"。如果你的一个好朋友 你非常在乎的一个人。或者家庭成员失败了 没做好。你会怎样对待这个人 你会排斥这个人吗。"你考试只得了个B"。或者"你竟然没赢得比赛"。你会这样对待他们吗 还是你会拥抱他们。你们会因此少爱他们一点吗 当然不会。那么为什么我们要将这些不现实的。毫无共鸣的标准强加给我们自己呢。达赖喇嘛。当他刚开始大量接触西方文化时。被一样东西困扰了。就是我们用的一个词 同情。同情这个词在藏语中叫"tsewe"。拼为"t-s-e-w-e"。这个词在藏语中的意思。既是同情他人 也是同情自己。所以他完全被英语中的概念搞糊涂了。或者说在大多数西方文化 不仅是英语文化中。当我们说到同情时。我们是指对别人的同情。他说"你们对自己没有同情。又怎么能对别人有同情呢"。自我是基础。铂金法则。就是说对我们自己也要有同情。对待自己不能不同于对待别人。要有同样的标准。要像接受别人 我们所爱的人的失败一样 去接受。我们自己的失败。 

为什么不呢 我们是如何帮助别人的。因为通常我们处于一段感情中时。和完美主义者有感情的人。或者我们有个朋友是完美主义者的话。我们如何去帮助别人。首先要搞清楚的一件重要的事就是。帮助别人是非常困难的。处理完美主义的问题。通常都要从内部着手。他们必须要渴望改变。因为这是需要时间的。不是立竿见影的。我从研究中 或者个人经历中了解到这一点。但我们还是能去帮助他人 即使知道这很困难。首先最重要的事就是树立榜样。如果我能改变。成为一个追求卓越的人。享受过程 即使失败也要庆祝。即使偶尔挫折也要庆祝。那我就做出了榜样。还记得那个实验吧。人们更愿意照着你所做的去做 而不是你说的话。第二件事就是分享这个故事。也就是这门课。上两节课所讲的内容。基本上是关于我个人如何从。完美主义者变为追求卓越者的故事。讲故事 互相分享。再次强调分享和讲的区别。分享 两者皆有收获 讲就只是单行道。最后还有一种帮助别人的方式。回忆一下Carol Dweck的研究。工作回报 过程的回报。很容易陷入对结果的回报之中。也很容易忘记回报过程。我在抚养我的小孩时就觉得很困难。在感情之中也很难做到。作为上司或者朋友当然也很难。但一定要注意回报过程。要指出来"这是个不错的尝试"。或者"这真是太有趣了" 注重过程。 

我想通过做一个。对至今为止所讲内容的小小总结来结束这个话题。这是个技巧 一个很简单的技巧。但我发现是个很有用的技巧。因为它能够整合我们目前为止所见过的。所有内容。我将这个技巧应用于各种困境中。当我在面对人生中很困难的一段时期 可能是很长的一段时期。也可能是很困难的一天时。当我感受到焦虑或压力。或者深深的失望时。这是除了三个M之外的另一种选择。还记得三个M吗 放大 减小。以及编造 发明。三个P则是另一种选择。 

第一个P就是准许自己为人。如果我经历了困难。我遇到了挫折。第一步就是给自己许可证。也就是说接受这样的感情。接受这个困境 接受现实。它已经发生了。我对此什么也做不了。能够改变的只有我对它的诠释。 

这时我就会进入下一个层次。到下一步。也就是重建 将情况诠释为积极的。这就是积极者。这其中有什么闪光点。其中有什么成长的机会。所以如果我经历了困境。也许可以用佛教的眼光来看待。"这是成长的工具。这可以帮助我更好地了解自己 了解他人。成为一个更好的心理学家 成为一个更好的商人。诸如此类。再次发掘出失败中的闪光点 失败中的机会。最后。我们没怎么提到的一个非常重要的技巧。就粗略提到过的 就是分心。将注意力转向别处。这不一定是件好事。我们下节课会讲到 有时执着于。分析每种情感 感觉和想法 是有害的。反复思考并不一定有帮助。有时最好的方法。就是当消极的想法。或者负面的情绪出现时。将我们的注意力转到别处去 比如说。听听音乐 跑个步。和别人聊聊这件事或者别的事。这和逃避是很不一样的。因为这并不是说 "好吧。我一辈子不要再管这种事了"。这只是说如果某种情绪或想法反复出现。我永远都在徒劳地抗争。把石头推上去 然后又被推下来。上上下下。有时更好 更有用的方法是。"好吧 我不要再反复想了 这没有意义。这个神经通路在我脑海中太强烈了。不会有任何出路的" 然后说 "让我继续前进吧。让我去听首好歌。让我去跑个步 好让我能够真正忘记它"。很经常地。跑步还能使我从全新的角度。来看待这件事。我们会更深入地探讨这个分心。因为差异非常微小 并不是逃避。我们下节课。周四讲到处理沮丧。或者焦虑时 会用到冥想。用念来分心。但那也是一种积极对待的方式。 

最后 第三个P是指换个角度。这真的很重要吗。Richard Carlson写了一本很好的自助书。"别为小事伤神 一切皆是小事"。我不完全同意后半句。"一切皆是小事" 但很显然。许多生活中我们为之伤神的事都是小事。要不为小事伤神。就要问一个简单的问题 它在一年后会不会有影响。或者更久……。从长远来看 它真的很重要吗。它真的值得我去担心 焦虑。为之沮丧 为之焦躁吗。它在一年之后真的会有影响吗。通常情况下都不会。从另一方面说 有什么是真的重要的吗。 

几年前 有个朋友问了我一个非常简单的问题。他说。我们之前从来没聊过这个的 他是我很好的朋友。但也没聊过 "你能告诉我心理成熟是什么吗。通过你的研究。你能想明白心理成熟到底是什么吗"。就在这一霎那。就在当时 后来我又对这个定义做了些修饰。我当时这么说 心理成熟。就是根据自己的意愿转换角度的能力。我觉得这差不多是对的。基于这些关于转换角度的能力的研究。为什么 这种转换角度的能力是什么。这就是我们希望沉浸在当下的。一种能力。也正是我们退一步。时间和空间上都退一步 问这样的问题。"它真的有影响吗 一年之后呢。或者更长远来看呢 它真的很重要吗"。同时随心所欲地回来。能够灵活地将思绪拉回来。沉浸于聆听一首美妙的歌。或者和我们所爱的人一起。说起来容易 做起来难。但逐渐我们可以通过冥想来训练我们的思想。通过对事物进行认知重建。我们可以训练我们的思想。使其能够按照我们的意愿来转换。 

我来给你们举个应用三个P的例子吧。这是刚开学时发生在我身上的事。大约开学后两个星期。我要带我的女儿Shirell去托儿所。我就要迟到了 迟了很多。你们知道我上课前一般都会去健身。我意识到我没时间去健身了。我甚至几乎没时间复习备课笔记。我在上课前都会复习两遍。于是我很沮丧 很有压力。然后我说 "就用用三个P吧"。第一个P 许可 接受。我对自己说 "你压力很大 你很沮丧。这些都没关系。即使你是教积极心理学的 也没关系"。我说的第二件事就是尊重现实。这是学期刚开始的时候。很多事情都碰在一起。我们几个星期前刚从以色列回来。要倒时差 还有小孩要照顾。尊重现实 现实就是这样。然后我转向第二个P 积极性。有什么好处呢 第一个好处就是。我要更仔细地想想我能简化的地方。我可以说不的地方。做少一点而非做多一点 这是很显然的一个好处。因为它让我去思考。第二个好处就是当时我想。"我现在又有一个可以在课堂上讲的故事了。一个应用三个P的例子"。分心 很简单。我旁边就有最好的分心工具。Shirell 然后是转换角度。一年之后这会有影响吗。我在上课前没有锻炼。我没有复习两遍备课笔记。而是只复习了一遍。所以我上课就不能发挥95%。只能发挥90%。这真的很重要吗 不重要。但同时我对自己说。"我仍然希望能享受当下"。于是我和Shirell呆在一起。陪她久一点的时间。在托儿所陪她久一点 我们一起玩。然后我更放松 更平静地回家了。将三个P应用于实际发生的情况。准许自己为人 接受它。关注积极的一面。最后就是转换角度的能力。我什么时候要退一步问自己。"我现在需要为此伤神吗。我什么时候想沉浸于当下"。想想这个技巧 非常简单。这个技巧并不等于可以一劳永逸。你现在学到了这个技巧。我知道三个P。之后我的人生中 它都能帮助我。它并不是这样。它更像是一种药丸。你要经常定期服用。也就是说每当你经历一次困境。就应用三个P。你应用的越多 生效就越快。正如我所说的 三个P。我几乎就是自动过一遍这个过程。许可 积极 转换角度 现在已经快了很多。因为我应用它 不断地练习它。所以你要不断地用它 多试几次。 

我们总结了很多。课上讲过的内容。一年半前。我有幸被Martin Seligman邀请。到他家去玩了一整天。我们坐在他漂亮的玫瑰园中。就是他经常写的那个玫瑰园。坐在那儿聊积极心理学。它的过去 它的现在 以及更重要的是 它的未来。我们讲到了教育 讲到他在他的课上做了什么。我在我的课上做了什么。然后对话进行到一半时。他突然停住说。"Tal 你知道当今心理学的问题是什么吗。不仅仅是积极心理学 而是当今的心理学"。我说"是什么" 然后马上拿出我的笔记本。他说"当今心理学的问题在于。我们关注的大多是脖子以上。而真正发生在我们身上的问题都在脖子以下"。我们关注的大多是脖子以上。而真正发生在我们身上的问题都在脖子以下。听到这番话对我很重要。尤其是从Martin Seligman。那里听到 就更为重要了。 

首先 Martin Seligman总是走在前面的。他引领了认知心理学。在它还未盛行。行为主义还是主流的时候。因此他的研究是超前于时代的。他开创了积极心理学。或者说将其整合为一门学科领域。Martin Seligman。除了他是一个有远见并且。很重要的心理学家之外。他还是认知心理学的创始人之一。认知疗法是怎样的 就是关于脖子以上的。按照他的说法 我们过分关注脖子以上。我们需要关注脖子以下。这就意味着他已经想了很多。这意味着他确实研究了这种现象。确实了解了我们需要一个大的转型。而他十分正确。这将是下两节课的话题。 

实际上我要跟你们讲的是一种"灵药"。只提供给上1504课的学生。旁听的也算。你们也可以得到"灵药"。"灵药"将给你们带来这样的好处。"灵药"能使你们感觉很好 非常好。你的自尊将显著提升。你会更加自信。更加自信你在社会上的地位。对自己更加自信。你会得到更高的自尊。我们都知道自尊跟许许多多好东西相关联。我们两周后会讲到。你会觉得更平静 生活更有重心。有了"灵药" 你甚至能变得更聪明。效果是显著的。你们开始有点担心了。它还会使你更吸引人。你会觉得自己更吸引人。你也会散发出更大的魅力。这个"灵药"已经被测试了很多遍了。我说了。这都是研究成果 不是异想天开。我是在向你们推销真的东西。真正有效的东西。我还要说这药没有任何消极的副作用。相反 它还有无数积极的作用。我数都数不过来。不仅如此 它还是绝对合法的。我知道你们有点担心 但你们会接受的。你们能处理好。不需要担心。如果有人来找你 很担心你怎么。用了这个药 突然就变了个人。你可以让他们来找我。最后 我不会收你们的钱。我本该收钱的。但我不是个好商人。所以我接下来的两节课会免费发放给你们。这就是"灵药"。事实上它不仅仅是种药 是一杯鸡尾酒。是一杯药做的鸡尾酒。因为它们似乎要很多东西组合才能发挥最好的作用。这就是那个鸡尾酒。我希望你们都在听。因为这很重要。 

灵药就是 半小时的锻炼 每周四次。 

鸡尾酒的第二个成分。至少十五分钟的意念锻炼 每周六到七次。 

鸡尾酒的第三个成分就是。每24小时睡8个小时。 

最后一个成分就是 每天12个拥抱。最后一个可以过量。你想加多少就加多少。但至少每天要有12个拥抱。如果你服用这个药。你们可以拥抱 没关系。 

这可是幸福课 但不要在课堂上睡觉。如果你服用这个药 这个鸡尾酒 我保证。我保证以上提到的很多好处 就算不是全部。说不定还会比那更多 都会实现。 

你们可以看到只有今天。我们才开始认识到精神和。肉体之间的联系。 

有一个人 第一个真正认识到的人。我们接下来的两节课会讲到的 就是Jon Kabat-Zinn。以及他的著作 《全灾难人生》。摘自《希腊人佐巴》的一个短语。Jon Kabat-Zinn说"也许行为治疗。最基础的进展就是认识到。我们不能再认为健康。仅仅是身体或者精神的一个独立属性。因为精神和肉体是内在联系的"。这可是否定了。千百年来西方的观念。笛卡尔的二元论。"意识和物质是两种绝对不同的实体"。这也渗透到了医学之中。我们"修"一个人就像修机器一样 修他的身体。他的精神则是完全没有关系的。笛卡尔也不是凭空创造出的理论。我们可以在柏拉图的作品中看到笛卡尔二元论的种子。之后亚里士多德试图反对。重新建立精神和肉体之间的联系。但最终还是二元论成为主流。由十七世纪笛卡尔倡导。然后就有了Jon Kabat-Zinn 有了Herbert Benson。有了Ellen Langer 有了Tara Bennett Goleman。这些心理学家 医生了解这种联系。还有John Sarno 纽约大学的一个医学教授。纽约大学医学院的。他们告诉我们通过我们的想法。我们可以解决大多数背痛的问题。或者腕管综合症。显示出精神和肉体之间的联系。或者是安慰剂的作用 非常惊人。还记得我们说它能起到和催吐剂相反的作用。以及我们如何用精神和肉体。互相形成一个良性循环。精神帮助肉体 肉体帮助精神。 

我想从身体锻炼开始讲起。鸡尾酒的第一个成分。灵药的第一个成分。当今社会怎么了。当今社会普遍缺乏运动。现在越来越多人。享有无需进行辛苦体力劳动的"特权"。我用了引号 因为这是特权中的。非特权的表现。在过去 我们要去追一只鹿才能有饭吃。或者逃脱一只狮子 这样才不会被吃掉。我们要采摘果子作甜点。我们现在呢 去家里的甜点桌子就行了。或者叫份比萨。在过去 我们要想取暖就得收集木材。而现在呢 就这样。打开暖气 然后为此付出代价。我长大之后。当我在以色列长大时 我记得……。以色列是个内陆国家。下午两点到四点是午休时间 非常安静。孩子们都不到街上去。但四点整 有时甚至是3:59时。每天都会从街上传来一声叫声。叫着"Tal 踢足球"。然后叫其他的小孩 "踢足球"或者"捉迷藏"。或者"抓人游戏"。这是我们在午休时间之后从下午四点。一直到晚上九点睡觉 还有妈妈喊吃晚饭时做的事。而今天在以色列呢。不仅仅是以色列 全世界 下午四点是怎样的。不是下楼踢足球或跑来跑去。而是继续坐在电脑前。继续看电视。结果就导致了。我们更习惯于坐着。 

我们付出了非常高昂的代价。当然是从身体上来说。但除此之外还有心理上的代价。这不是巧合。抑郁的程度已经增加了十倍。这不是巧合 从20世纪60年代算起。这不是巧合。当今抑郁症的平均发生年龄刚好只有15岁。而在20世纪60年代 抑郁症的平均发生年龄是30岁。这是因为我们停止了锻炼。我们停止发挥力量。这不是巧合。注意力缺陷多动症的病例指数上升。更不用说 我之后也会说到。宅的生活方式的后果。锻炼不是种奢侈 这是种需要。就像氧气那样的需要。就像维生素那样的需要。就像蛋白质和碳水化合物那样的需要。我们不是。不是生来就要坐在电脑面前一整天的。也不是一整天聊天的。我们是要在一天结束时。打完猎采完果子之后干这些事的。不管是上帝决定 还是进化决定 还是两者共同决定的。我们都有对锻炼的需求。如果我们不能满足某种需求 无论是对维生素的需求。对蛋白质的需求 还是对锻炼的需求。我们都要付出代价。既包括生理上的代价 也包括心理上的代价。 

塞内加 两千年前的希腊哲学家说。"我们的祖先和我们过着一样的生活。他们靠自己的双手。来获得食物 驻扎在大地上。还未受到黄金与宝石的诱惑"。这是塞内加 一个很有思想的哲学家说的。讲到他那个时代存在的。缺乏锻炼的现象。现在这种现象严重了多少呢 今天在我们这个时代。一两百年前连皇家都过不上的生活。方便的 奢侈的生活。这是特权中的非特权。我们要为此付出代价。 

让我们来看看心理学家是怎么说的。Ellen DeGeneres针砭时事。 

"此时此刻"。 

我们很懒 做什么事都一个按钮搞定。做什么事都不用体力了。即使是开车库门。我们曾经要下车 打开车库门。现在就只用按一个按钮。还有车窗。这太过分了。我不想按按钮 我只想呼吸新鲜空气。我们现在还在用这个手势。如果想让某人把车窗摇下来的话。我们还在用 虽然已经没人知道是什么意思。因为如果我们这样做就会像个傻子。我们很懒 我们以前有薄荷糖。现在有洗牙糖。它们帮我们把舌头上的东西都溶掉。真是懒啊 我们不能吸吮了吗。放上来吧 我累了 我今天好辛苦。 

我和你们分享几个。关于这种生活方式的影响的研究。是由Michael Babyak。以及他在杜克大学医学院的同事共同完成的。这是个很重要的研究。做了很多很多的实验。模仿这个 并显示出这有多现实。锻炼有多重要 不仅对于抑郁。还有其他方面。 

Michael Babyak做的实验是。找了156个抑郁症患者。这些人有各种各样的症状。包括失眠 饮食不规律 无精打采。不愿动弹 情绪低落。其中很多人都有自杀的想法或倾向。诸如此类。这些人身体都非常不好。他将他们带到实验室。随机分为三组。第一组是锻炼组。第二组服药。第三组既服药又锻炼。他们所服用的药是抗抑郁药 左洛复。继百忧解之后最普遍的抗抑郁药。非常有效。很多研究 以安慰剂为对照 证明左洛复是有效的。是一种选择性5-羟色胺再摄取抑制剂。他们所做的运动。是半个小时中等难度的。有氧运动。可能是慢跑或者竞走 也可能是游泳。不会太难 中等难度的运动。一周三次 每次半小时。然后跟踪调查这三组人。锻炼组 服药组。锻炼加服药组 跟踪调查了四个月。这是得到的结果。你们预期会是什么结果。这156个病人 或者每组52个病人中。有多少治好了抑郁症。他发现这三组中的每一组。四个月后都有超过60%的人情况有所好转。他们不再出现。抑郁症的症状。或者大部分症状。他们不能再按照美国精神病诊断标准。被诊断为抑郁症。在各组之间没有显著的区别。只有一个区别 那个只锻炼的组。需要更长一点的时间。来治好抑郁症。在服药的情况下 一般要花上。约一到两个星期 差不多十天来治好。60%受试者的抑郁症。对于仅仅锻炼的组 他们花了差不多一个月。但在第一个月后 一旦他们好一些之后。就没有什么区别了。三个组都出现了好转。注意很重要的一点。这并不代表我们就要废除抗抑郁药。或者抗焦躁的药物 或者治疗多动症的利他林。完全不是。他要说的是。我们首先要问的是锻炼。或者缺乏锻炼是不是其潜在原因。 

我们为什么不废除药物呢。如果锻炼和服药的效果如此类似的话。我们为什么不废除服药呢 原因在于。比如说 让我画一个圆。假设。这是所有抑郁的人的集合。我们知道左洛复。或者说抗抑郁症帮了他们60%的人。我们知道锻炼同样帮了60%的人。但不是同一个60%。在定义上有重叠。因为多于50%。它们帮助的不是同一群人。有的人仅仅受了服药的好处。有的人仅仅受了锻炼的好处。还有的人两者好处皆有。所以不能说 废除服药吧。它很重要 它确实有用。但结果同时显示 "试试锻炼吧。因为我有可能是得益于锻炼而不是服药。或者得益于两者。研究的关键在于。我们在看这些研究时 想搞清楚的是。复发率 

实验结束半年后。我们不再让他们服药。不再强迫被试者锻炼。每周至少三次 半年后如何了呢。这是他们发现的结果。仅仅服药组的复发率只有。60%多好转的人中有38%人回头。又有了抑郁症的症状 38% 超过了1/3。这个既服药又锻炼的组。复发率为31%。刚好比三分之一少一点 在这些60%好转的人中。仅仅锻炼的组的复发率。他们可以选择继续锻炼也可以不锻炼。完全看他们自己。仅仅锻炼的组的复发率 为9%。十分令人惊讶的结果。和2000年之前的实验结果非常相似。追溯到1984年的精神抑郁症。精神抑郁症通常持续时间较长。但没一般抑郁症那么急性。是一种深层的悲伤的情绪。比抑郁症持续时间长很多。但锻炼对它也有帮助。 

在我第一次看到这个研究时。现在已经有无数这样的研究了。在不同的领域重复这样的现象。我等下会讲到。我说 "哇 太不可思议了"。不锻炼就像服用抗忧郁药 效果很像。锻炼就像服用抗忧郁药物。但我再仔细想想。就发现并不是这样。不是锻炼就像服用抗忧郁药。而是不锻炼就像服用抗忧郁药。这不仅仅是语言上的区别。这不仅仅是语言上的区别。为什么 因为我们不是生来为了坐着的。我们不适合长时间坐在电脑前。或者教室里。我们本来应该去逐鹿 或者逃离狮子的追捕。而当我们无法满足这种需求时 正如之前讲的。我们无法满足需求 就要付出代价。 

比方说上帝或者基因决定的幸福程度。基本的幸福程度有这么高。我们固有的水平就这么高。如果我们不锻炼。我们就把这个基本水平降到这里。如果我们的正常水平在这里。当我们不锻炼时。就会把它降到这里。然而我们是可以达到这样的幸福程度的。即使有兴衰变迁 也可能是螺旋式上升的。而我们现在是经历了。同样的起伏 只不过是在更低的水平上。我们努力挣扎着到达上帝决定。或基因决定的幸福程度。我不是个临床心理学家 我不是心理治疗师。但是。如果我是的话 我治疗病人的一个前提。就是体育锻炼。或者至少。我对他们第一个要求就是要进行身体锻炼。因为他们通常都是。没有体育锻炼的动力。但那会是我的首要要求。要么是治疗的前提条件。要么就是首要的要求。为什么 因为我不想和自然作对。我不想和自然作对。所以我要治疗的那个人必须达到上帝或者。基因决定的幸福程度。我想和自然合作。自然所要求的必须遵守。自然的命令之一。就是我们要活跃地运动。 

我在晚餐时就讲了这个。我和网球教练Dave Fish 以及他妻子。Bonnie Mosland 一个临床心理学家共进晚餐。然后她跟我说 "对"。她已经治疗了很多年病人了。"我就是这样的。我治疗的前提条件就是你必须进行身体锻炼"。确实有心理学家。能够理解锻炼的重要性。为什么要和自然作对呢。当你可以在这个水平开始的时候。还有其他好处 都是基于这个研究的。你们在阅读中可能读过一些。锻炼还有很多其他好处。 

首先是心理上的好处 自尊。从很多层面上说 首先是身体上的自尊。我们自我感觉更良好。也包括整体的自尊 我们感觉更良好。不管是整体还是特征性的自尊都得到提升。换句话说 不但锻炼之后我们马上就能感觉到。而且长时间内我们都能感觉到。如果我们保持规律的锻炼计划。所以整体和特征性的自尊都会得到提升。焦虑和压力。是最容易被锻炼所影响的事之一。Tammy能够看出来。我妻子能够看出来我48小时内有没有锻炼。我回到家时。她会说"你最近一次锻炼是什么时候"。为什么 因为她有感觉。她能察觉到。是不是有点小小的变化 或者稍微有点压力。正如我在之前的课上讲过的。我有焦虑的倾向。锻炼就是我的药 我的灵药。至少要保证每隔一天锻炼一次 每隔一天。这对无数临床的病症都非常有效。包括精神分裂症。它并不是自己起作用。而是和治疗以及药物协同起作用。但它增加了成功的可能性。如果你去锻炼的话。还有注意力缺陷多动症 不一定足以治愈 但通常十分有效。那些坚持常规锻炼的人。更容易克服注意力缺陷多动症。为什么呢 因为这对我们的大脑有无数的影响。对大脑有无数的影响。比方说我们释放的那个化学物质。当我们形成新的神经通路时。还记得神经形成或者神经可塑性吗。正是这个形成新的。神经通路时会释放的化学物质 就能在我们锻炼时产生。换句话说 我们去跑个步之后。这时我们的大脑就最适合于。形成新的神经通路。所以如果你在早上去锻炼 那么一整天。你就能更好地吸收你阅读的东西。或者你听到的东西。锻炼之后你会更有创造力。如果你已经锻炼了一段时间的话。同时它能帮助我们释放。恰好够量的这种化学物质 不多也不少。为什么 因为这是自然 自然比我们更懂我们需要什么。我们释放这刚好足量的化学物质。就达到了我们的最好状态。这又是一种方式 我之前提到过的。我每堂课之前都去跑步 不是次次都能去跑。我喜欢在外面跑 但不是每次都可以。所以我有时会跳家里的蹦床。我有一个小的。我在上面跳 戴着脉冲计。不应该告诉你们这个吗 

好吧。我能感觉到如果我不这样做。在上课之前不做锻炼。比如说我去旅行。刚好赶在上课前回来 没时间去锻炼。我能感觉到我阅读时。我的记忆保留率 集中程度。我之前说过 我有轻微的注意力缺陷多动症。锻炼是我的药。身体上的好处。我现在跟你们讲的不是什么新内容。但你要这样来想它。我们都有一个基因决定的体重水平。适合我们的体重水平。当我们锻炼时 我们在那儿。这样就是健康的 是我们应该达到的。当我们不锻炼时 体重就会超过这个。就像降低幸福水平一样。换句话说 锻炼的作用。就是将我们带回我们本来应有的水平。这也就解释了为什么大多数节食。如果不和锻炼一起进行 就很容易失败的原因。因为我们的身体会拼命想要回到原来的水平。我们本该有的水平。如果我们不锻炼 就会挣扎得更辛苦。 

慢性疾病 不管是糖尿病。还是心脏病 抑或是癌症。数量是非常惊人的。糖尿病 有些糖尿病原本只有45岁以上的人。才会获得。现在小学生都会得这种糖尿病。其原因很大程度上就是。因为营养问题以及缺乏锻炼。经常锻炼的人。更不容易得慢性疾病。患糖尿病的可能性大大降低。减少了50%以上患心脏病的风险。以及得癌症的可能性。可以降低50%的可能性。它是灵药。经常锻炼的人免疫系统更强。我要给你们打好提前量。 

最后 如果你现在还没被说服。希望这个能说服你。(更好的性行为) 不管男人还是女人。都能增强性欲 对性的渴望。能够增加高潮的可能性。经常锻炼的人性生活会更美好。很多研究都证实了这一点。它适用于任何年龄。不是任何年龄 21岁以上。但是关于锻炼要记住的很重要的一点是。它不是万灵药。它不是万能的。记住不是所有人都能从锻炼中获得好处。对于有的人来说 服药就是。很适合和很重要的方式。然而在很多情况下 研究和调查都证实了。在大多数情况下 锻炼是有帮助的。就算不是它本身的作用。即使是和别的东西协同作用的 它还是有帮助。为什么要和自然对抗呢。自然所要求的必须遵守。 

还有很重要的一点在于恢复。我给你们读读。Dienstbier和Zillig在《积极心理学手册》中。对研究者们这样写道。"重复的过于密集的训练 以至于没法及时恢复。可能会导致运动员感觉到'泄气'。典型的症状为。更高的焦虑水平。更高的神经紧张度。更高的儿茶酚胺和皮质激素水平"。他们想说的是。有时事情是过犹不及的。尤其是完美主义者要明白这一点。越多不一定代表越好。说到锻炼的话。确实会过度锻炼 有些过度训练的症状。会和训练不足的症状十分相似。高水平的焦虑 高水平的泄气。失去动力。最终也可能导致抑郁。 

这是完美主义的一种症状。我在做运动员时就经历过这样的事。过度训练 恢复不足 并为此付出了代价。身体上自然是受了伤 而且心理上也受了伤。我推荐每周有一到四天的休息时间。理想情况下你想每周锻炼五到六次。我认为至少要保证每两天一次。这样每周就是三到四天。至少每两天一次。理想状态下你们可能想更多一点。每次半小时 也可以增加到一小时。关键是要听从你身体的需要。如果你的身体很痛苦。超出了正常的不舒适范围 赶快停止。如果肌肉开始酸痛 停下来。宁可少也不要过度。还有一种聆听你身体的方法。如果你没那么敏感的话。就是戴个心率监控器。不是很贵 但非常非常有用。这样我就知道我的锻炼。会保持在最高心率的70%左右。我最高心率的70%大概是128。于是我就在120到135之间锻炼。偶尔我又会加强一点。达到90%。我的差不多是165。很容易发现你们的心跳。通常在最大心率是220减去你的年龄。所以如果你20岁 最大心率就是200。我37岁 所以就是183。差不多吧。然后你就保持在70%的水平。65%到75%的水平。来获得锻炼的最温和的效果。记住数量确实会影响质量。我们再次画个图表。我们经常看到的线性曲线图。在图的这边。是心理和生理的好处。这里是运动量。那么图画出来就是这样 太少了不好。我们要达到基本的幸福水平才行。太多了也不好。这时我们就会感到泄气和焦虑。我们想要达到的是差不多到最适水平。而这个最适水平也在于我们能投入多少时间。在理想情况下。我每周锻炼六次 每次一个小时。在理想情况下是这样 但我还想做很多别的事。所以我每周至少锻炼四次 每次半小时。这对我来说就足够了 要搞清楚什么对你是足够的。 

恢复也很重要。我们从心理学层面上讲过恢复。恢复很重要。我简要地和你们分享一下Derek Clayton的故事吧。Derek Clayton是60年代早期的一个长跑运动员。世界上最好的长跑运动员之一。最好的之一 并不是最好的。他是跑马拉松的。很多人说 他之所以不是最好的。是因为从身体上讲 他不太适合跑马拉松。他身高六尺二 1米88。并且他的最大心率。最大血液溶氧量不是很高。没有其他世界级选手那么高。但他的优势在于。他是最勤奋的一个。他每周跑一百多英里。远远多于一百多英里 非常努力。抱歉 我写下来了的。对 每周一百多英里。于是他成为了世界最好的长跑运动员之一。但从来没成为过真正的第一。然后发生什么了呢 由于他的过度刻苦勤奋。他受了很严重的伤。一整个月不能跑步。他当时正在准备非常重要的一项马拉松赛事。这个马拉松在日本举行。一整个月他都不能训练。走路都走不了。在比赛一周之前 他开始可以慢慢地跑。他说他会去日本跑的。他被邀请去参加了 他是世界最好的长跑运动员之一。他说他会将其作为一次训练。作为下一次马拉松比赛训练的一部分。于是他去参加了这个马拉松。一个月没有训练。结果提前八分钟打破他的个人记录。成为历史上第一个仅用了。两小时十分钟跑完马拉松的人。两小时9分36秒。1967年。他继续努力训练 两年之后再次训练过度。又再次受伤。又一次他没法跑步。然后他去参加了比利时马拉松比赛。又打破了他的个人记录 也打破了世界记录。两小时8分33秒。世界纪录 保持了12年的世界纪录。这时教练和运动员就开始明白。适可而止的重要性 适当休息的重要性。恢复的重要性。有时过犹不及。很相似的故事 Joan Benoit。最终赢得了1984年洛杉矶马拉松。在受伤了 被迫休息之后。还有丹麦足球队。另一个例子 我想应该是1991年。最终赢得了欧冠。之前都没在比赛中。在最后一刻才加入。因为当时南斯拉夫正搞分裂。于是他们作为了南斯拉夫的替补。在预选赛淘汰了之后。把丹麦队又找过来。把他们从沙滩上带来打冠军赛。因为他们恢复了 振作了 于是他们准备好了。恢复在运动中很重要。在心理层面同样很重要。锻炼是很好的 是一种灵药。鸡尾酒的第一个成分。我觉得算是最重要的成分。然而很多人 不仅仅在哈佛。世界上很多人都不锻炼。 

为了将定期锻炼引入我们的生活。我们要跨越怎样的障碍。 

首先 这是痛苦的。锻炼不是件舒服的事。比如说到达70%最大心率的锻炼。并不会很痛 但显然没有坐在电视机前。或者玩任天堂舒服。那我们该怎么办。我们要做的第一件事就是把目标分割 各个攻破。尤其是完美主义者会做的一件事就是。"我要开始锻炼了。为了保持健康 我每天要跑六英里。所以我一开始就要每天跑六英里"。很快会怎样呢。第二天你起床就会全身酸痛。但你是个完美主义者。你很坚强 一分耕耘一分收获。于是你又出去跑步。然后会怎样呢 你第二天更疼。但你咬牙忍着疼痛。几个星期之后你就会受伤。但即使你不受伤 在你脑海中。关于锻炼的概念是这样的。"锻炼就等于痛苦"。我们的意识是拒绝痛苦的 它总是极力避免痛苦。比如说你去度了个假 中断了你的锻炼计划。中断了一个星期或者两个星期。然后你回来了。想重新开始锻炼就变得更难。为什么 因为你的思想 你的潜意识会说。"我不想再经受这个了"。这会阻止你锻炼 使你瘫痪。这通常就是拖延症发生的时候。因为我们不想经历这种痛苦。 

一种更健康的方式是。"我先从走路开始。走十分钟 二十分钟的路。坚持一到两个星期 然后逐渐增加。宁可少也不要过度。因为那样锻炼。如果我逐渐增加 就不是种痛苦。它逐渐会成瘾。从积极的角度来看。我们想去做它 我们渴望它。所以要慢慢来 同时要应用让你分心的事物。当我在蹦床上跳时。在外面跑是不会有问题的。但在蹦床上跳时 我就要用一些分心的事物。我觉得蹦床太无聊了。有的人可能会去骑一小时自行车。那样也没问题。我更喜欢冥想 这很好。 

但很多人需要让他分心的事物 不管是音乐。还是电视 还是小孩子在周围跑来跑去。但总要找些分散你注意力的事。社会支持。长期锻炼成功的最好预测方法就是。我是否是和其他人一起做的。有没有给我社会支持。再次强调这不是对每个人都有用的。对有些人来说更好的方式是。他们自己去做。有可能的情况是。尤其是内向的人 更喜欢自己去做。闭上眼睛 听着音乐。或者听着心跳。心跳的声音。他们会想要独自一个人的时间 这是没问题的。但对大多数人来说 社会支持是有益的 

还有个研究。由Wing和Jerry 两名心理学家完成的。他们所做的是让人参与一个为期四月的项目。包括行为改变 包括饮食改变。包括锻炼上的改变 为期四月。分为两种情况 一种是有社会的支持。和你的朋友或者家人一起做。或者没有社会支持。那些没有社会支持 自己完成的人。有76%的人完成了这个为期四月的项目。有24%的人还保持到了六个月。76%保持了四个月。只有24% 不到四分之一的人 保持到了六个月。能够保持行为上的 锻炼上的 以及饮食上的改变。有他人支持的一组。95%完成了四个月的项目。95%而不是76%。同时67%的人维持了这个计划。保持到了六个月以后。也就是成功的可能性大大地增加了。再给你们看一个社会支持的例子。录像。这不是我想说的意思。但社会支持确实是有帮助的。所以如果你还没开始 那就从今天开始吧。 

周四见。\[掌声\]。 

第17课-运动与冥想 

大家可能已经听说了。美国防癌协会生命接力募捐活动。定于今周五晚在Gordon田径场举行。本次活动旨在引起公众关注。并为防癌宣传 研究 治疗筹集资金。我们随时欢迎大家的参与。大家可以在线注册,为参赛者捐款。或者出席观看赛事 了解活动。本次活动为大家准备了啤酒游戏。无伴奏合唱 乐队表演 接力赛小姐选举。还有电影和游戏。最后大家都会沿着跑道走一圈。藉此缅怀被癌症夺去生命的人。为癌症生还者庆祝 并展望未来。请大家查看邮件。浏览我们的Foocbook小组 了解更多。希望到时能见到大家,谢谢。 

大家早上好。我们接着讲上节课没讲完的内容。开始讲之前 上节课后有同学来找我。问我有没有其他的运动选择。我上次讲的研究 希望大家也去看看。就是Callaghan等人写的那篇文章。里面侧重讲的是有氧运动。我之前也提过。做这运动的目的是缓和高强度运动。做这种运动 你会达到最大心率的70%。大约在最大心率的65%到75%之间。大家搜索心率运动。就能马上得到想查的各种心率表。这是其中一种运动形式。有些同学问。"举重训练行不行 间歇训练行不行?"。快跑 休息 再快跑 再休息。这种是划艇队经常做的训练。运动员经常做的训练。这些运动行不行?我上节课没有专门提出。是因为目前对其他的运动还没有定论。我可以告诉大家。目前在研究中发现的一些趋势。目前看来。如果你只能做一种运动。那么最佳选择。时间最花得其所的是有氧运动。理想地每周三四次半小时的有氧运动。这是最低要求。但是。如果你还能做举重训练以补充。有氧运动 那也很好。一周两三次举动训练非常好。尤其是年纪大了以后。举重训练更加有用。但在年轻时 举重训练也很好。但是 它不能代替有氧运动。 

研究发现它改善心理健康的效果。不如有氧运动。另外一种运动 在经过研究后…。我要重申 这些都是初步结果。有人对这种运动做过初步研究。最后结果。因为样本不足 暂时无法下确切结论。但我们可以确切地说有氧运动。产生的效果能媲美。一些强效精神病药物。这个结论是确切的 有大量研究来支持。包括Babyak2000年以来的研究 以及之前的。另一种开始被研究的运动。得出可喜的结果 那就是间歇运动。间歇运动就是 以跑步为例子。达到最大心率的90%到95%。如果你20岁 你最大心率就是200。计算公式是220减去年龄。所以你跑步的心率要达到180到190。强度非常高。不能持续很长时间。但如果能持续30秒到一分钟。然后休息。让你的心率下降到最大心率的60%。也就是120 然后重复这个间歇运动。有初步研究表明这种运动。能产生显著效果。通常效果甚至比单纯做有氧运动更大。而且持续时间更长。所以 再说回来。我没有在课堂上正式介绍这种运动。但很多同学问起 所以现在介绍一下。因为我只想给大家介绍。那些有确切结论的运动。John Ratey也谈过这种运动。��等一下会讲John Ratey。他是哈佛医学院的精神���生。可以说他是一个前沿的倡导者。在研究运动和精神健康这个领域。他本身…虽然目前的结果是初步的。但他一周会做两次间歇运动。除了一周做六天运动以外。其中两天还会做全速短跑 休息。全速短跑 休息。研究表明即使没有做够30分钟。但还能得到同等的效果。同等的甚至更好的心理效果。我一周也会做两次间歇运动。所以 举个例子。今天早上我在蹦床上跳了30分钟。在这30分钟里 我还做了4个短跑。没错。你们不相信我会跳蹦床 是吧?我下次拍下来 还是不要了。4次短跑 我戴着心率监测仪。快速跑一段 我的心率大约是165。165到170之间 然后休息 下降到110。少于我最大心率的60%。如此重复四次。这30分钟里其余的时间。我只做缓和的有氧运动。所以我也做间歇运动。举重训练就比较少做 因为我做瑜伽时。就有这种加强体能的效果。但最重要的一点是。我想说 间歇运动。特别是对于完美主义者来说。很容易导致训练过度。大家最好不要达到最大心率。并且持续一段长时间 即使你做得来。更多并不意味着更好。再说一次 除非你是备战奥运。这种情况下有时你就要达到。100%的最高心率。想练出最健康的身体 达到90%…。最大心率的95%就行了。所以戴上心率监测仪就很有用。因为它能告诉你 运动是否强度太大。或者强度不足。总之 换换花样有助于运动。你可以跳舞 这也是好运动。你可以 偶尔去健身室健身。偶尔打篮球。划艇 花样越多越好。只要保证 至少对我们来说 一周四次。一次30分钟的有氧运动。 

上节课我们讲到克服障碍。其中一些障碍包括。伴随运动而来的疼痛 使运动更难了。即使你只达到最大心率的70%。运动引起不适。克服不适的最简单方法就是进入状态。因为通常我们跑步进入一种状态后。我们就感受不到不适了。但必须开了头才行。找别人来支持你克服困难 慢慢适应。用一些东西分散注意力 例如电视 MP3。能分散注意力就行了。另一个障碍。很不幸 这个障碍在校园很普遍。那就是我们没有时间。大家知道 在高中。我经常去高中演讲。我注意到的其中一件事是 通常。很多高中要求学生参加团队活动。他们除了体育课做运动以外。还要参加团队活动。所以他们运动得更多 虽然还是不够。但是我们读大学第一件放弃的事。尤其是那些不参加团队活动的学生。我们放弃的第一件事是运动。 

我没有时间 我有很多事要做。大学是一个新世界。在考试期间 大家也首先就放弃运动。因为我压力太大了。我有太多事要做了。其实 最不能放弃的是运动。最不能放弃的是运动。为什么?因为它是一项投资。虽然你失去了45分钟。你知道。运动30分钟 之后洗个澡 我推荐大家这么做。虽然你失去了45分钟。但你获得了更多 这时间花得其所。为什么?因为做运动 记忆力提高。所以你更能充分利用学习时间。你的创造力提高了 精神更集中。不管你是否有多动症 有就更要运动了。或者你有患多动症的趋势。即使对于没有多动症的人来说。运动也能大大提高注意力集中。活力显著提高 所以这时间花得其所。大家还记得我讲过。壁球赛季末我停止运动的事吗?突然我觉得我有了很多时间。但我完成的工作反而少了。第一因为我的生活规律打乱了。第二因为我注意力没那么集中了。我创造力下降了 活力下降了。最不能放弃的是运动。尤其是在考试期间这种压力大的时候。克服时间不足这个问题。不能靠自律。我们之前已经讲过了。我们看过那个研究 人的自律是有限的。我们要自己形成规律。例如一周中有三个早上做运动。如果下午晚上才有空 那也行。形成一个时间规律。 

另一个阻碍定期运动的障碍。是一个有趣但经常被忽视的障碍。这个障碍非常强大 潜意识障碍。Nathaniel Branden谈过这点。我在我的书中也略谈过 引用了他的话。他谈到自尊。自尊隐含了一个概念:我们值得拥有幸福。值得拥有幸福。当你问别人"你值得拥有幸福吗?"。大部人会说"值得" 但在潜意识中。通常很多人都会说"不值得"。可能因为脑袋里有一把声音说"不值得"。因为过去老师 教育工作者 父母 重要的大人。社会大众都要我们向不现实的榜样看齐。所以通常在我们潜意识里有一把声音说。"我不值得拥有幸福"。这是一个障碍 使我们无法更快乐。我们知道运动的其中一个作用就是。可以说它是让我们变幸福的最强效药物。我知道运动使人幸福 使人感觉良好。但如果我的潜意识认为。我不值得拥有这些感觉 不值得幸福。它可能阻止我运动。为什么?因为 大家回忆一下。意识不喜欢。潜意识与外界不一致。如果我潜意识认为"不应该快乐"。我就会听从我的潜意识。我就不会运动。因为我运动的话 就会与潜意识不一致。我会比潜意识认为的更幸福。在这个问题上有过很多研究。例如 心理学家Bill Swann。讲过自我确认理论。这是我们意识里一股很强大的力量。我们意识里有自我提高。我们想在别人和自己眼里表现出好的一面。但还有一股反作用力 就是自我确认。我们想确认对自己的了解是否属实。如果我们认为自己不值得幸福。通常我们就不会做让自己幸福的事。在其他方面也存在 例如在恋情中。如果我认为我不值得拥有好伴侣。不值得在恋情中得到幸福。通常我就会在潜意识里。破坏这段恋情。积极寻找恋情中不顺利的地方。潜意识中寻找方法伤害这段恋情。所以我们需要注意这些潜意识力量。通常它能影响我们会不会去做运动。我们该拿它怎么办?最好的办法就是运动。再一次说回来 把它规律化。不要分析太多 尽管去运动就行了。你知道一周三四次运动对你有益 尽管去做。再一次说到 即想即做法。潜意识会告诉你。"不要做 你没时间 还有别的事要做。你明天再做 你下周再开始定期运动"。为了达到目的 潜意识会很聪明地。用尽各种方法 几乎能使你的行动听从潜意识。当你出去运动时 你就让自己看到。"是的 我值得拥有幸福"。通常潜意识就会听从这个想法。所以尽管去运动。第二个是认识 这是认知层面的。认识运动的重要性。大家回忆Martin Seligman说的"大部分人都专注于脖子以上"。心理学家是这样。学术机构里的学生是这样。尤其是在这样的学校里。大过专注于思考和认知。你们也正因此才能考上哈佛 才能成功。但专注于脖子以下 也同样重要。事实上 要获得幸福 脑袋的作用。是有限的 所以我要告诉大家。如果我是心理医生 我治人的第一件事就是。要求他们运动。因为我不想违抗自然。我想借助自然的力量来治人。很大程度上 运动。是心理治疗中的"无名英雄"。它在很多方面都发挥了作用。John Ratey 刚才提过他。哈佛医学院教授 精神病医生 "某种程度上。运动可以说是精神病医生的理想药物。exercise can be thought of as a psychiatrist’s dream treatment.。它能对付焦虑症 恐慌症 普通的压力。压力和抑郁症有莫大关联。运动能释放神经递质。去甲肾上腺素 血清素 多巴胺。这些物质与重要精神类药物相似。做一轮运动就像吃了一点百忧解。一点利他林 针对性很强"。正确的分量 正确的身体部位。没有副作用。也有副作用 积极的副作用。身体 精神 和情绪的副作用。 

John Ratey在他的一本很棒的书中呼吁。这本书今年一月份刚刚出版。他呼吁发起运动革命。对于这个呼吁。他用了大量科学证据来证明它的重要性。我给大家讲他谈到的一个方面。他谈到各种年龄层。孩子 成年人 还有老人。运动对每个年龄层都有积极作用 从0到120岁。我给大家讲几个研究和发现。这些研究是在学校推广运动。有几间学校实行把运动。作为一门必修课程。例如 这个方法已经有几间学校效仿了。每天早上 做45分钟的运动。通常是跑步 有时划船 骑自行车 爬山。新泽西州有一间学校。甚至在运动场安装了一面攀爬墙。一些有氧运动。有时换换花样 做间歇运动。每天早上在上课前 45分钟运动。有些学校安排在上午以后。大部分学校安排在早上。这里就是其中一些学校得到的成效。那些推行运动的学校。肥胖水平下降 这是意料之内的。例如在伊利诺斯州的Naperville区。那里的学校推广运动。这种做法很有趣 因为大部分学校。昨天我在Horace Mann演讲。那里的老师告诉我。纽约很多学校。都取消运动训练了。而不是提倡做更多运动。运动训练和艺术课都取消了。为什么?为了考高分 专注于脖子以上。这样很可悲 他们因此付出了代价。Naperville的学校的肥胖水平下降。或者说伊利诺斯州的Naperville区。肥胖水平下降。在学生中 从30%降到3%。这不仅对学生目前的生活有影响。他们身体更舒服了 心情更快乐了。对他们以后的生活也有影响。他们不再那么易于。患上慢性疾病 例如癌症 心脏病。从很多方面来看都是双赢的。患糖尿病的可能性也降低了。这是肥胖症方面的效果。他们在Naperville区还发现了。分数大大的提高了。其实美国学生通常。在国际测试中表现并不是很出色。八年级考的那些数学和科学考试。但Naperville是一个例外。他们的数学成绩全世界排16名。科学成绩全世界排第一名。有些人会反驳说。Naperville是一个富人区 确实如此。那其他地方呢?一些很穷的内陆城市也推行了运动。 

例如。宾夕法尼亚州Titusville 很多人都低于。平均工资远远低于这个州的平均工资。在推行运动之前 学校的教学成绩。远远低于州的平均水平。宾夕法尼亚州的平均水平。推行运动一年后。成绩比州平均水平高17%。当年学校唯一的变化。就是推行了运动训练。其他地区的学校也发现同样现象。同时还发现在Titusville那所学校。他们发现 例如打架斗殴。因为推行运动训练 暴力完全消失。而在爱荷华州一间学校也发现。在推行运动训练后。一年内纪律问题从225宗降到95宗。在另一间堪萨斯州学校里 也是内陆城市。纪律问题下降了67%。这些都是在推行了运动训练以后。暴力变少了 运动使我们平静。使我们的身体处于自然状态。这种状态更为平静 而不是苦闷愤怒。人们需要释放那股怒气。很不幸的是 通常。人们都通过打架斗殴来释放。用运动将它转换成健康的活力更加有益身心。 

我给大家看一段我最喜欢的一部电影。《律政俏佳人》。在这个镜头里。一个女人刚刚被控谋杀亲夫。我们的法学院校友瑞茜·威瑟斯彭为她辩护。糟了 我找找看 谢谢。视频 律政 找到了。 

(律政俏佳人片段)。我们要为Brooke Windham辩护 她的有钱丈夫。被发现死在他们的Beacon Hill大宅里。挖金女?这样想也不奇怪 那老头都60了。但她自己也很有钱。拥有一间庞大的健身公司。电视导购节目上还卖她的健身录像。你说的人是Brooke Taylor?娘姓是Taylor 你认识她?她是Delta Nu社团的 但她和我不是同期入会。她比我早四年毕业。我以前上过她在洛杉矶运动俱乐部的班。她很厉害。怎么个厉害法?她可以帮你一节课减掉三磅。她绝对是个天才。很可能她也绝对是个杀人犯。我觉得Brooke不会杀人。运动能产生内啡肽。内啡肽能使人心情快乐。快乐的人是不会杀自己丈夫的。绝对不会。好了 对不起。大家知道在书里我谈过幸福革命。根据很多研究。幸福革命的基础。一定伴随着运动革命而来。为什么?因为如果我们不运动 就是违抗自然。所以幸福革命的基础。必须始于脖子以下的身体。 

我们接着讲第二个身心疗法。就是冥想。冥想是什么?我练瑜伽时第一次接触冥想。我当时打壁球弄伤了背。有人说练瑜伽有效果 我就开始练了。瑜伽不但治好了我的背 还有很多益处。太多益处了。所以我现在跟Debbie Cohen学当瑜伽老师。将来会当上一名瑜伽老师。如果她让我通过 应该很快就能当了。还有其他形式的意念冥想 静坐冥想。太极 气功。还有把注意力放在呼吸上的冥想。有把注意力放在同情心上的冥想。这些冥想都有一个共同点。我们来提其中几个。首先 它们都专注于一件事物上。可以是动作 姿势。呼吸。同情心 爱 善心。可以是火焰。可以是…但它都专注于…。可以是祈祷 反正就是一件事物上。通常大部分的冥想。基础都是深呼吸。就像婴儿呼吸一样。用腹部呼吸 把气一直吸进腹部。然后慢慢地轻轻地呼出来。深呼吸是很多冥想的基础。不管是太极 瑜伽 还是静坐冥想。最后 冥想没有好坏之分。从某个方面讲 冥想没有目的。在你冥想时 没有目的。你不能强求自己。不会说"我刚刚做了一次很棒的冥想"。或者"我觉悟了"。 

其实 我来给大家讲一个故事。关于我是如何通过冥想觉悟的。去年 我以前经常出行。我们那时住在以色列。我平均每个月来美国几天。当我来到这里时。我通常都是演讲 还会借机静修。我会做很多瑜伽。比我在以色列做得要多。我做很多冥想 两场演讲之间有很多闲余时间。我经常利用这些时间静修。我记得有一次回到以色列。我给大家讲一下当时的情景。在以色列 我们住在特拉维夫外的一个城市。叫Ramat Gan。我们的房子在一栋20层的建筑里。我们住在10楼。景色美不胜收。以色列是一个很小的国家。我可以看到以色列的两边 一边是地中海。另一边能看到萨米尔山区。非常美丽的景色。我喜欢每天早上…。因为我是一个早起的人。通常比家人起得都要早。我在客厅坐下。透过玻璃窗看外面。我们有一个玻璃窗。因为楼层很高。我们用的是强化玻璃 有些窗是开不了的。我透过那扇没打开的窗看外面。我经常这样看。我记得有一次从美国回来后。我看窗外 突然。所有颜色都变得鲜艳多了。整个世界似乎都变得更明亮了。我心里想"我肯定觉悟了"。因为我在书上看过有人说。当你觉悟时 一切事物都变得明亮。世界变得比原来更美了。我坐在沙发上。欣赏着鲜艳的颜色。美景 明亮的世界。我心想"经过这么多年"。我冥想了十年了。"经过这么多年 我终于觉悟了"。我的行为举止开始变了。因为觉悟了的人。不会到处告诉别人他们觉悟了。这节课是一个例外。但…我没有告诉别人。我甚至没有告诉Tammy我觉悟了。第二天早上 我步履轻盈了。我醒来又开始我每天例行的看景。看着远处的山 我想。"世界太美丽了 我觉悟了"。这种情况持续了几天。有一天 Tammy和我坐在休息室里。我们在聊天。她看着窗外 我也看窗外。她说"你去了美国的这段时间。我终于能把窗户擦干净了"。我说"什么?"。她说"我一直都够不着窗外。但我买了一个机器。一个有两边的机器 有一块磁铁的。我终于买到这个机器 把窗擦干净了"。我的觉悟原来是一场空欢喜。也许真的有觉悟这回事 我也不知道。但冥想的关键并不是达到这种境界。看到一切都变明亮了。要想这样 把窗户擦干净可能更快。但觉悟不是冥想的目的。冥想是专注于现状 此时此刻。与此时此刻的存在作挣扎。当我们思想涣散了。当我们对呼吸分心了 集中回来就行了。这就是冥想 分心 集中。分心 集中。这样我们就能把思想锻炼得更专注。更留神。 

我给大家讲几个对冥想的研究。对了 一旦你开始做冥想或瑜伽。你会摸索出适合自己的方式。我有多动症 很难静下来坐着冥想。做瑜伽对我更有用。因为我练瑜伽时很认真。瑜伽能把我的注意力集中在姿势上。对其他人来说 静坐冥想更适合。静坐呼吸。有些人会想着某一件事物。很多人会祈祷。只要适合你就行了。你甚至可以同时做几件事。你可以一边祈祷一边做瑜伽。我来给大家讲几个研究。关于冥想的严肃研究。一开始是研究冥想界中的佼佼者。那些练习冥想很多年的冥想者。练习了很久的人。这些研究者包括Richie Davidson。任职于威斯康星大学。伍斯特大学的Jon Kabat-Zinn。像Herbert Benson这样的人。加州大学的Paul Ackerman 还有其他人。他们想研究冥想时。他们说"我们来研究最好的冥想者"。所以他们联系了达赖喇嘛 说。"你能推荐一些冥想了。很久的认真的冥想者吗。我们想研究他们的大脑"。他们找到了 把这些人找来。他们研究的第一样东西是 大家应该记得。左前额皮层和右前额皮层的比例。大家记得 我们之前谈过。快乐的人通常更活跃的地方。是左前额皮层。不快乐的人更活跃的地方。是右前额皮层。所以这个比例很重要。它是测量快乐的一个"客观"手段。他们得出的一个发现是。两者之间存在很大关联。在扫描器下看到的大脑。和我们对自己是否快乐的自我报告。当然不是100%的关联 但也不低 很高。这表明自我报告是很可靠的。总之 他们想研究这些冥想者的大脑。他们再比较普通大众。我们都在一个钟形曲线内。快乐的人更靠前。左侧皮层更活跃。不快乐的人更靠后。右侧皮层更活跃。然后他们找来这些冥想者。测出他们左前额皮层和左前额皮层的比例。这是他们的研究发现。这是正常的钟形曲线。冥想者在这里。这是2001年的研究 他们找了很多冥想者。他们几乎偏离了这个图。表明快乐的程度很高。他们很容易感染积极情绪。对痛苦情绪的抵抗力很强。这是一个很惊人的结果。这些结果之前没人发现过。直到他们研究了这些冥想者。另一个研究方面就是。他们想看看另一个重要的指标。衡量健康和冷静的惊吓反应。当你听到一声巨响 你会吓一跳。这个惊吓反应很重要。因为你越容易受到惊吓。通常说明你焦虑程度更高。至少容易产生焦虑。大家都有一定的惊吓反应。即使是军队里的神枪手。他们每天都射枪。他们要完全集中注意力。当枪走火时 他们也会受到惊吓。他们会微微地抖一下 这也是惊吓反应。要压抑惊吓反应是不可能的。或者说当时的人是这样想的。Paul Ackerman找来这些冥想者。叫他们保持绝对镇定和冷静。他吓他们。他们一点反应都没有 有记录以来第一次。有人能够抑压自己的惊吓反应。Goleman writes在他的书中这样写道。破坏性情绪 "一个人惊吓反应越大。那个人就越可能。出现消极情绪。Oser的表现有着有趣的含意。表明他的情绪有着惊人的镇定"。这种镇定还会传染。如果有人想和这些喇嘛争吵。他们会很难生起气来。即使这个话题很容易让他们生气。例如科学家争论上帝是否存在。他们说"我生不起气来"。因为平静是会传染的 就是快乐会传染一样。有些同学可能不百分百了解。什么是惊吓反应。对于不了解惊吓反应的同学 我准备了一个短片。(惊吓反应视频)。面具吓不了他。你想知道什么吓到他了吗?是爸爸。好了 现在大家明白了。他们研究这些冥想者时。基本上他们方法就是应用。我们之前讲过的。生长锥数据 或者Maslow讲过的。生长锥数据 研究最好的。因为如果你想推动知识发展。如果你想更好地应用可行理论。你不会研究普通老师 你会研究Marva Colins。你不会研究普通的冥想者。自以为觉悟了的冥想者 你研究那些。冥想了很多年的冥想者。其中一个冥想者Oser喇嘛说。"冥想本身有一些与众不同的特质。但这与进行冥想的人无关。重要的一点是 冥想并非触不可及的。只要你有足够的决心来练习它"。也就是说 冥想并不专属于少数几个人。那些每天8小时地冥想了30年的人。大家都可以冥想。因为这个过程对所有人来说都很简单。Daniel Goleman说"从神经科学的角度看。这个研究的重点。并不是为了说明。Oser或者其他厉害的冥想者。本身有什么过人之处。而是为了扩展这个领域对于人体潜力的假设。其中一些关键假设已经开始扩展了。部分原因是神经科学。对大脑可塑性作出革命性猜想"。 

大家记住 神经塑性 神经形成。这些概念 我和。像Richie Davidson这样的科学家认为。大脑是可塑的 可以改变的。改变的方法之一就是冥想。这种观点非常好。所以我们可以下决心把它应用于生活中。但这课室里有多少人会认真地说。"听起来很好。我想我的左前额皮层更活跃。所以我要去喜马拉雅山。每天8小时地冥想30年"。有多少人真的准备这么做?有多少人每天有两小时做冥想? 

这个研究的第二个问题是 两者之间只是关联。但并没有因果关系。也许Oser喇嘛天生就是这样。所以他才对冥想产生兴趣。也许Matthieu Ricard…他写了一本关于幸福的书。他是达赖喇嘛的翻译 一个右撇子。也许他天生就是这样。左前额皮层比右前额皮层活跃得多。对了 他的比例完全超出钟形曲线。 

除了因果关系 还有一个主要问题 时间。我们有时间吗。我们有耐心经常长时间冥想吗?我发现当我开始冥想。开始做瑜伽时 我真的能看到益处。我说"我每天做半小时。就能收到成效了。如果我进行一次长时间静修呢?"。所以我想进行十天的静修。我跟Tammy。"我想去静修十天"。Tammy很了解我。她说"不如你试试…。你不用要么不做 要么就做到足。不如你试试先静修一天。然后再进行为期十天的静修?"。我说"这是个好主意"。我开始为期一天的静修。离这里不远 就在剑桥。我踏上这段静修之行 非常激动。那天是星期六 我盼了一整周。我当时想"通过这一天的静修。我就踏上觉悟之路了。等我做完这一天的静修。等我向Tammy证明这对我有很大益处时。她就会看到我静修后有多平静。她就会说'好的 去静修十天吧'"。所以我去了静修一天。 

那一天的静修 我们进去看到一个男人。长着胡子 坐在一班学生前。显得非常平静。我想"静修完今天 我就能像他那样了。除了没他的胡子" 我们坐下。他说"我们先从静坐冥想开始"。我们都坐下。我们都有坐垫 我需要几个坐垫。我们坐在那里 我把手放在这里。他说"好了 深呼吸 从鼻孔吸进去。然后呼出来"。我们跟着做 我坐在那里。我有多运症 我的思想开始涣散。他说"如果思想涣散了 集中回来"。我们坐了…。我觉得很长时间。我坚持住 我想"你思想涣散了 没关系。集中回来" 我继续坐在那里。我们坐了45分钟 最后他说。"慢慢地轻轻地睁开眼睛 站起来"。他说"接下来。经过45分钟后 我们开始第二部分。第一部分是静坐冥想。接下来我们要行走冥想"。他解释了一下。然后我们围成一个圈 跟着他走。我们走得很慢 很认真。把注意力集中到每一步上 把腿抬起。他说"集中到动作上。专注于此时此刻 如果你思想涣散了"。我经常这样 "再集中回行走上。继续深呼吸。就像刚才静坐冥想一样"。所以我们绕着房间 走了很久。继续走 绕着房间走。我的思想继续涣散 我感到无聊了。我开始四处看别人。他们都绕着圈子走着。然后我看了老师坐的地方。老师旁边有一张纸。我想"真有趣 不知道上面写了什么呢"。所以我继续走 又走到那张纸那里。我想"是时间安排表 真有趣"。我不够时间看清上面写什么。我想"下一圈我就有盼头了"。所以我继续走 我看到…。我看到我们刚上完的课 45分钟静坐冥想。后面就是45分钟的行走冥想。接下来是什么呢?我继续走。我有盼头了 所以很高兴。我往下看 45分钟静坐冥想?然后 45分钟行走冥想?又是静坐和行走。然后是午餐 午餐后面是什么呢?45分钟静坐冥想?然后是行走…天啊。然后是行走冥想?于是我心里想。"太恐怖了 再见 天啊"。我走出去了 我受不了了 所以…。我跑回家了 

说实话我感到很尴尬。所以我想"不如我在外面逛一天"。回家时装出一副觉悟的样子。但我没有这样做 我打给Tammy。我告诉她"你想和我一起吃早餐吗?"。我们一起吃了 她明白这是人之常情。我觉得自己无法静修十天。我觉得自己无法静修30年。我的问题是 在这种情况下。那些不想 或者无法长时间冥想的人。冥想怎么帮得了他们?别误会我的意思。我推荐大家进行静修。有些人去了静修以后。变得不同了。我见过 我有很多朋友去过。再说回来 静修一天还是十天…。有些人静修了长达三个月。他们都取得很好成果。但我坚持不来。 

我们该怎么做呢?Jon Kabat-Zinn在他的研究中。告诉了我们冥想如何应用于。那些没有一整天闲余时间 每天只能抽45分钟的人。甚至如何每天抽15分钟冥想或者做瑜伽。或者其他形式的意念练习。他做了两件事。第一件。他表明了冥想不到30年的人从中得到什么益处。第二件同样重要 他表明了因果关系的存在。他的一个研究中有两组人。两组。两组都想做冥想。他们都有这个兴趣。他让其中一组等候。他说。你们将在三个月后参加这个冥想班。是四个月 四个月后。你们将参加这个报满了的冥想班。第二组开始这个班的冥想练习。他们的练习就是每天冥想45分钟。周末时他们会聚到一起。学习一些技巧。但在工作日时 他们在家里。每天冥想45分钟。然后他拿他们跟参考组对比。参考组就是那个等待参加冥想班的组。因为他希望实验对象都是。想做冥想的人。一组已经开始冥想 另一组迟点再开始。然后他测量焦虑程度 对比参考组。那个冥想了八周的人的焦虑程度。平均每天45分钟 只做了八周。焦虑程度大大降低。然后他研究他们的快乐程度。积极的情绪 他不只关注消极程绪。从消极到积极的都关注。他们更快乐了 心情更好了。仅仅八周 但关键还在后头。他说。"这还不够 自我报告有可能产生安慰剂效应"。这个想法很好。他要不受安慰剂效应影响的证据。然后他有了这个发现。 

他研究他们的左右前额皮层比例。仅仅八周之后。那些冥想的人。就出现显著的变化。这是一个很惊人的结果。大家记住 1998年之前。科学家都深信大脑是不可塑的。过了三岁以后就不会改变。突然仅仅八周的练习。这些冥想的人。每天有规律地冥想45分钟的人。改变了他们大脑的机能。使得他们更容易感染积极情绪。对痛苦情绪的抵抗力更强。 

我们哈佛医学院的Herbert Benson。发现每天做15到20分钟的冥想。做了一段时间后 就能大大地改善健康。情绪健康和身体健康都改善了。因为Jon Kabat-Zinn做了另一个研究。他给两组人都注射了…。那组等候冥想的参考组和进行冥想的那组。他给他们注射了感冒细菌。他想测的是。他们体内的抗体会有什么反应。他发现冥想者的免疫反应。比那些没有冥想的人强了很多。那些在等候冥想的人。只需八周的冥想 每天45分钟。我们知道。 

你们知道当我们更平静时。我们对疾病的抵抗力就会增强。我们焦虑时 就更容易患上疾病。所以冥想能加强。心理免疫系统 增强心理抵抗力。同时还能改善身体免疫系统。 

意志训练应用于很多领域。成为很多领域中的一种治疗方法。今天就更是如此了。配合认知疗法。配合药物 冥想 或者单纯冥想。研究已经表明这种方法非常有效。我推荐一本书。如果你想对这个领域有一个深刻的初步认识。我推荐大家看《改善情绪的正念疗法》。这本书是讲。意念如何能帮我们克服严重抑郁。中度抑郁 或者焦虑。还能改善其他心理疾病。 

例如这本书中讲到的一件事。我们知道。很不幸 一旦一个人患了严重抑郁。复发的可能。不管是一周后 还是一年后。比从来没患过严重抑郁的人要高很多。为什么?回忆一下我们之前讲过的内容。我们讲过大脑是如何形成神经通路的。一旦神经通路形成。因为抑郁症的神经通路很强烈。通常一点相对不严重的事发生了。都能让这条神经通路活跃过来。如果我们有一段时间情绪不好 哪怕是一点小事。马上大脑就会寻找最大的那条神经通路。形象一点说 最大的那条河流。它就会找到那条通路。所以因为一点很小的原因 严重抑郁症复发了。复发了两三次后。这条神经通道更强烈了 复发可能性更高了。他发现 当他们进行正念冥想。或者正念疗法 再配以认知行为疗法。就能减少严重抑郁症复发的可能。那些患有 或曾经患过严重抑郁症的人。能减少50%的复发可能。这是一个纵向研究 这种方法真的很有用。还能改善中度以下的的抑郁症。帮助处理悲伤情绪。 

这种方法是怎么起作用的?我简单地…再说一次 我很推荐这本书。光用这本书就可以开一门课程。这本书还介绍了一个很棒的八周疗法。 

第一步是察觉。并接受身体信号。这是什么意思?当我们感受到一种情绪时。通常会伴有一个身体反应。例如像快乐这样的积极情绪 可能会忐忑。可能只是会感到上半身豁然开朗。痛苦情绪也有身体反应。例如。我个人来说 当我感到焦虑时 这里。我能感受到它在这里。或者这里 我以前讲过 就像胃打了个结一样。感受到抑郁甚至悲伤的人。会感觉到它在某一身体部位 可能在胃。可能在肩膀。一个情绪。总是伴有身体反应的。研究者认为 以及研究发现表明。当我们开始感受到痛苦情绪时。我们马上就要开始运用意念。开始找出这个身体反应在哪里。而不要钻角头尖。"为什么我会抑郁""怎么回事"。"发生什么事了""又犯了""千万别又犯了"。我们不要马上想到这些。而是马上找出这个情绪的身体反应。想到身体。我开始感到有压力 好的。把注意力集中到这里 接受这个结。不要尝试去消除它 这点我们稍后会谈。只需接受它的存在 看着它想。"真有趣 很大的一个结"。"不 很小 在变大了"。而不要想。"我想它消失 它消失了 真有趣"。这不是这一步的目的。这一步的目的只需察觉到它的存在。只需察觉到。这种情绪引起的身体反应。 

为什么这样能行得通?它能行得通是因为它形成了一条新的神经通路。在原来的神经通路下 当我们感到不适时。脑袋就会发现不适 苦苦思索它。通常就会走回原有的通路里。这些原来的通路。是和我们的抑郁相连的。和我们之前的消极情绪相连 并且得到增强。我们现在要做的。不是走回我们脑袋里这些增强了的通路上。并且再一次增强它们。而是形成一条新的通路 痛了?我们进入身体。那条新通路。这个方法管用是因为当我们形成新通路时。它通往我们身体的自然治愈能力。我们患上的大部分疾病。不是全部 但大部分病身体都能自行治愈。大部分切伤的伤口。你不用管它。事实上 你不应该处理小伤口。你不用弄它 看它 摸它。让自然的力量来治愈它。它会自己愈合的 消化也一样。你不用想着消化。身体有自己的内在智慧 内在医生。同样道理。当你不再集中注意力想。"我该怎么消除它 怎么回事。太难受了 我不想这么难受。我希望快点好起来。为什么我会这么难受 因为什么原因"。有时候这样想它 分析它。把它写下来很重要 但不要钻牛角尖。或者跟人谈谈 但不要钻牛角尖。最有效的。在大部分情况下 就是马上想到身体的感觉。察觉它 接受它 不要尝试去治好它。让它留在那里 观察它。我们身体内的医生就会治好它。 

要熟练运用这种方法 关键是练习。练习。一次又一次地练习。就像我们在课堂上讲到的其他方法一样。没有捷径。我们怎么练习?首先 这是书中的八周疗法中的一部分。我现在也在练 首先是身体扫描。了解你的身体。躺下来 思考 观察。先从你的脚开始 然后了解你的膝盖。了解你胯部的感觉。了解你胃的感觉。每一个身体部位都是。了解整个身体。多练习 多了解。这是冷状态下的练习。在模拟状态下的练习。当我们在热状态下。当我们真正感受到那种痛苦情绪时。我们就准备好用这种方法 我们知道怎么用。练习时 如果我们精神分散了 再集中起来。这是练习。练习并不表示30分钟都要完全集中注意力。练习时 精神分散了 再集中。精神分散了 再集中。然后在现实生活中也要练习。所以当我感到压力时 我想"好的。这种情绪引起什么身体反应?"。现在我们谈了情绪。谈了身体。你们也可以用认知重建来练习。 

怎样练习?举个例子。例如我想到今晚我要做的一个讲座。我感到很紧张。很多观众 重要的观众。每次想到这 我第一个反应就是紧张。我感受到这里的肾上腺素 很难受。感到我胃打了一个结。我怎么办?每次想到这讲座 我就会想。"好的 这个身体反应在哪里?"。另一个方法就是用认知重建它。不要觉得它是压力 觉得它是荣幸。我刚才说过。我要跟一个很大的客户谈。我为这次演讲感到很大压力。每次我脑袋想到这压力时。马上我跟自己说"不 这是荣幸。我能有这样的机会做讲座真是太棒了"。重申 这不同于逃避。不同于抑制。有时候只需要接受它。明白这是人之常情。但是 苦苦思索这种情绪上是没用的。只会加强这条神经通路。这种反应 或者专注于身体。或者专注于认知重建。都很有用。记得 当我们想改变时。我们要想到尽可能多的可能性。想到情感 行为 认知的方法。这只是其中两个有用的办法。我经常用这种方法 或者专注于身体。或者专注于认知重建。非常有用。有数据证明它多有用。 

我们也来做一次简单的正念冥想。运用我讲过的一些方法。和我们以前讲过的一些内容。大家尽可能舒服地坐在座位上。后背挺直。如果背靠着椅子更舒服的话 就靠吧。把脖子尽量伸长。上半身形成一条直线。把脚尽量舒服地放在地板上。如果觉得交叉着腿不舒服的 就放下来。可以把你的手放在腿上。也可以合起来 你觉得舒服就行了。如果你觉得闭上眼睛舒服的 就闭上。不管现在你的注意力在哪里。都把它集中在你的呼吸上。深深地吸进腹部。然后慢慢地轻轻地呼出来。深吸一口气 慢慢地轻轻地呼出来。重复几次。如果你精神分散了。慢慢把它集中回呼吸和练习上就行了。一边呼吸 一边转移你的注意力。转移到你的身体上。你的脚 小腿 腿 手腕 腹部 背。扫描你全身 你的胸膛 脖子 头。找出你身体感到微微紧张的地方。比其他部位更紧张的地方。没有其他部位那么放松的地方。不管是哪个身体部位。把你的注意力转移到上面 观察不适。观察这股紧张感。观察它 接受它。继续一边深呼吸。一边观察这个身体部位。看看这多有趣。你那个部位有这样一个感觉。每呼吸一次 就接受它。你的感觉。没有对错之分。它只是一种感觉。继续深呼吸。吸进那个部位 再从那里呼出来。只需观察它 接受它 与它同在。继续与你那个身体部位同在。一边呼吸 一边感受这种感觉。不管是什么感觉 感受它。然后轻轻地把注意力从那个部位移开。轻轻地。因为你已经接受它了 你要回到呼吸上。用你的呼吸洗涤整个身体。从你的灵魂开始。从下到上 一直到你的头。每次呼吸 平静加深了。接受加深了 存在加深了。体验 感受这股轻松感。因接受而带来的轻松。自然状态下的轻松。再深吸一口气。慢慢地轻轻地呼出来。呼完下口气 睁开你的眼睛。如果你旁边的同学睡着了。轻轻地推他们一下 或者抱一下他们。 

正念冥想 正念疗法。真的能改变我们的思维。我们练习这种正念冥想时。我们的注意力从作为转移到了不作为上。什么意思?通常当我们感到有压力时。或者当我们头脑陷入悲伤情绪时。我们第一个反应就是 怎么解决它。我怎么解决它?我们通过想其他办法来解决它。或者想办法让自己好受点。或者分析我们面对的问题。为什么?因为我们这种解决问题的思维管用。因为有解决问题的思维 我们才有科学。才有技术和进步。因为解决问题 分析问题的思维。你们才考进哈佛。你们能考进来不是因为你们是成功的冥想家。你们能考进 是因为你们考试成绩好。你们能解答SAT上的问题。这是好事 这是重要的技能。但通常它有反作用。尤其是用它来处理思维和心理问题时。我们会不由自主应用它 我们想解决问题。但通常会适得其反。同样会有反作用的。是我们想干预体内的。消化过程时。消化不受我们控制。如果我们想左右它 看清它 分析它 促进它。我们只会伤害它。为什么?因为我们体内的主宰 体内的医生。体内的智慧最擅长处理某些事。不是所有事 但是某一些事。例如消化 还有通常的 不是所有的。通常的精神问题 痛苦情绪。所以我们要把注意力。从解决这种情绪转移到与它同在。感受这种痛苦的身体反应。在他们的书中 在牛津任职的William说。"想消除抑郁是通常的解决办法。想解决不对劲的地方 只会让我们陷得更深。苦苦思索是问题症结之一 不是解决方法"。这里强调苦苦思索。它并不等同于…我们之前谈过。它并不等同于写日记 它是没有益处的。把你面对的问题写进日记里。是有益于你的。和亲密朋友谈谈 是有益于你的。但如果钻斗角尖。一次又一次地思索这个问题。苦苦思索它 想解决它 通常会使它恶化。如果我们只是感受这股情绪。感受身体反应 反而会更好。大家试试 看哪种方法适合自己。也许适合你的方法是。找个时间写半小时。然后冥想10分钟。也许光是写更适合你。每个人的方法都不同。我给大家讲这些。只是希望教会大家另一个心理工具。例如我们可以借助认知重建。把威胁看成挑战。把威胁看成荣幸。把失败。看成一次学习机会。我们也能情绪重建。不要对情绪苦苦思索。而是去感受它。"不再尝试无视和消除身体不适。而是抱着关怀之心去关注它。我们就能真真正正地改变我们的感受"。 

今天我教大家最后一个方法。这个方法很重要。我非常推荐 你们所有人都要学 没有例外。我非常推荐的方法是呼吸。呼吸很重要。事实上 它是唯一。所有正念冥想的共通点。你可以把它应用于生活中。当然我所说的呼吸是正确呼吸。你可以应用到你的生活中。即使你没有一天冥想15分钟 或45分钟。我所说的呼吸是深呼吸。因为现实情况是。在现代世界 我们经常感到紧张焦虑。因此我们的呼吸变得很浅。因为呼吸很浅。当我们呼吸浅时。我们就变得更紧张焦虑。我们就会陷入一个恶性循环。一个"或战或逃反应"的恶性循环。我们平常就是这样处理的。庆幸的是我们可以扭转这种循环。我们可以马上扭转它。因为我们还知道。平静和健康促进我们深呼吸。深呼吸又能反过来促进平静和健康。我们可以扭转这种"或战或逃"的反应。进入Herbert Benson所说的。放松反应 只需三个深呼吸的时间。我们可以把这三个深呼吸的方法。有计划地应用在一天的生活中。上课前 下课后。早上醒来时。或者上床睡觉前。三个深呼吸。有计划地按排到一天的时间点上 能改变你的人生。Andrew Weil 我们医学院的毕业生。现在任职于亚利桑那州大学 在他很棒的CD中。他这样说呼吸。"如果我只能说一条健康生活的建议。答案很简单 学习正确呼吸"。这意味着 像婴儿一样呼吸。吸进腹部。我说过 这节课不会教什么新知识了。你们都知道怎么呼吸。你们一出生就知道怎么呼吸。现在我们要做的只是去掉多余的牵绊。求助于你的自然智慧 自然医生。 

在一本很棒的书里。Alia Crum的父亲 她和Ellen Langer写了这篇文章。她父亲Thomas Crum提了几个很好的建议。如何扭转"或逃或战"反应。他的其中一个建议。就是每次遇到红灯。把它当作一个深呼吸的机会。不管是走路 还是遇到红灯。或者在车里。以前…我用过这种方法。以前每次遇到红灯时 都很烦躁。尤其是刚刚从绿灯转成红灯时。现在每次遇到红灯。我都把它当作一个机会 深深呼吸。在生活中学会深呼吸。一天几次。能产生深远影响。我们下周见。在此之前 大家深呼吸。掌声。 

第18课-睡眠、内向与外向、爱情、触摸 

大家好 我叫Jenny Spria。我在这向你们介绍。一个我发起的团体。这个团体叫做哈佛舞蹈实验。我们的初衷是。想把哈佛所有不同的舞蹈团体都聚在一起。让他们为零起点初学者提供完备场所。所以我们的想法是。让那些不懂跳舞。却对学习舞蹈有兴趣的人。能够从别的同学那尝试各种。哈佛现有的舞蹈风格。并且有一个非正式的空间让大家聚到一块。一起练习 享受乐趣。所以如果你觉得自己可能感兴趣的话。我们想核算感兴趣的同学的人数。看看我们能否把这件事做成。请加入Facebook群。哈佛舞蹈实验或者发邮件到hdancexp@gmail.com。谢谢。掌声。 

嗨 早上好。下面我们开始今天的内容。今天 我们将讲完身心疗法。我们将会讲到睡眠。讲到触摸。然后我们将继续到一个。本人绝对很喜欢的主题 就是爱情。我们会谈到人际关系。 

这个是我们上次课讲完的。我们讲了身心的联系。我们讲了运动的重要性。我们讲了正念冥想的重要性。它是如何实实在在地转化 改变我们大脑的形态。使得前额皮质的左半边。相对于右半边要更为活跃。接着我们讲了呼吸的重要性和如何呼吸。如果我们能在一天中有计划地进行深呼吸。就能真正的改变我们的生活。不管是等红灯时。还是早上醒来的第一件事。还是我们进入教室时或是就现在。深呼吸 深深的腹式呼吸。让我们从或战或逃反应改变为。放松反应。记住我们在本课程一开始。就说过的一句话 那就是。我不会教给你们很多新的东西。这就是一个很明显的例子。你们都知道怎么深深地 正确地呼吸。你们从出生那天。或出生那刻起就知道了。但多年以来。随着压力的不断升级并缺乏恢复。多年来承受着逐渐增加的压力。逐渐加快的现代生活速度。我们给自己施加了限制。我们一直承受着压力。却没人教过我们如何恢复。我们现在需要做的是逐渐去除这些限制。回到我们生来就喜欢的深度呼吸。当我们这么做时 我们扭转了或战或逃反应。并创造了一个。由平常的幸福和深呼吸组成的生长螺旋。它会带来更多平常的幸福和更多的深呼吸。 

下一个我想谈论的主题。大家不要在课堂上应用 回去再应用 这主题是睡眠。下面是一些关于睡眠的数据。在爱迪生发明电灯泡之前。人们每24小时就平均睡10小时。每天10小时。大多是在夜里睡觉 但有时也在白天睡觉。现今平均每人在工作日期间。有6.9小时的睡眠时间。周末则是7个半小时。这是全国的平均数据。至于大学生 数据则有些不同。有多少人…这样我们来举手表决。你们除了我之外有多少人。一晚上能睡足8小时的?好的。肯定不止这些人 没举手的可能睡着了。全国平均数据是:18岁到29岁的人中。大约有4分之一的人有8小时的睡眠时间。75%的人没有足够的睡眠。有些人需要7小时的睡眠。有些人需要9小时 有些人16小时 不。平均需要睡眠时间是8小时。我们的睡眠时间都就在这个数上下波动。很少有人需要很少的睡眠。也很少有人真的需要很多的睡眠。那我们如何知道自己需要多长的睡眠呢?你什么时候自然醒?我们什么时候睡觉 我们什么时候疲劳?我们可以这样估计睡眠时间 大约是8小时左右。现在有些人会说 "好吧。对我来说 要睡8小时的话 我一天要有30小时才行。没有足够的时间去…那是一天的三分之一啊。太多了 我腾不出那么多时间"。你可以腾出那么多时间 如果你把它视作一种投资的话。如果你把它当做一件优先要做的事的话。你知道。就像是商人去谈生意说。"在这笔交易上我负担不起10万美元"。你要是白送出去自然负担不起。但是你会收到每年20%投资的回报。这就不一样了。也许我可以负担得起10万。如果投资是有回报率的话。睡觉也是同样的道理。这些8小时的投资有很大的回报。许多研究都表明了睡眠的重要性。和我们把它优先考虑的原因。所以让我简要地列出来几点。首先 显著增强身体的免疫系统。我是说 我们都知道当我们睡得很少。我们会。或者我们在考试周结束后更容易不舒服。这当然也和压力有关。但睡眠本身。充足的睡眠增强了我们的免疫系统。精力水平。有很多理论认为。睡5个小时比7个小时好。由于某些循环。是的 这些循环存在。但普遍来说 在一定程度上睡多一些更好。如果我们一天睡眠时间超过10个或者11个小时。有时可能是因为生活上碰到困难。有时可能是因为抑郁。抑郁会导致失眠或者另一个极端。睡得太多。但正如我所说 如果你一天睡10个小时。这可能意味着你需要一天睡10个小时。而不是说你抑郁了。而不是说你有什么毛病。不过看一看。看看你整个生命中。你平均需要多少睡眠时间?在某些程度上说越多的睡眠通常越好。我们讲过身体动运 当我们不运动时。我们的体重就会超出基准水平。超出天生的或者基因决定的 我们的自然状态。这就是为什么越来越多不配合锻炼的。节食减肥会失败。对于睡眠也是同样的道理。如果我们的睡眠不充足 我们会超出我们的基准体重。实际上是相关的 不只是相关 而是因果关系。缺乏睡眠导致体重增加。如果我们尝试节食 我们又在和自己的天性作斗争。仅仅是充足的睡眠时间。就能有效地改善我们的体重控制。驾驶技术。每年有平均10万起车祸。是由缺乏睡眠引起的。在美国一年有4万人受伤 1500人死亡。因为缺乏睡眠。受害人中有的人是没睡好的司机。有的是被没睡好的司机害的。在工作场所 日复一日。由于缺乏睡眠引起的事故。估计造成了1000亿美元的经济损失。认知功能 无论是创造力。生产力还是记忆力 都会在我们不睡时受损。这就是为什么睡眠是如此重要的一项投资。你知道我们经常说。"让我再熬两个小时学习"。然而实际上你会完成更多的学习。你会把学习材料记得更好 更有创造性。如果你得到这额外两小时的睡眠的话。投资的良好回报。 

经常我们能在婴儿身上学到很多关于自己的东西。为什么?因为婴儿不压抑情绪。当婴儿没有得到充足的睡眠会怎样?他们会变得暴躁 会哭 会很痛苦 会很焦虑。我是说我们知道。我们学会了不要看起来像我们看到的那个婴儿一样。这一刻哭 下一刻笑。在我们讲为人之许可时讲过。所以我们压制这些情绪 但我们依旧有这些情绪。同样 如果我们没有得到充足的睡眠。我们的导火索就变短了。那时候我们就更容易大发脾气。我们就会感到焦虑。生理水平上变得不健康。当然心理上也不健康。 

睡眠对抑郁的影响更大 原因有二。第一个原因是。缺乏睡眠容易导致抑郁是因为。在生理水平来说 正如你看到的婴儿。他累了的时候就变得暴躁 我们在累了的时候也暴躁。这只是纯生理层面上对睡眠的需要。第二个原因是。有一点点微妙也有一点点有趣。晚上。我们的大脑要处理很多。我们白天经历的事情。它经常解决我们白天经历过的。未解决的问题 这就是为什么。你知道 你带着一个数学问题入睡。经常你会在早上醒来时得到解答。但不只是数学问题在晚上得到解决。还有其他人际问题。内心里的问题要在整个晚上解决。比如说梦。通常我们晚上做得比较早的梦。都是比较不愉快的梦。我们在夜里晚些时候做的梦更容易是愉快的梦。为什么?因为晚上的第一部分。是我们解决问题的时候。我们在解决问题 有些是有意识的。有些问题则没有意识。我们解决了一些问题之后。我们的梦就变得更为愉快了。这并不意味着你在上午11点半就不会做噩梦。它的意思是。你睡眠初期做噩梦的可能性更高。当我们不给自己机会去。"完成梦想"-。不可能在一夜睡眠中完全解决问题。不可能我们醒来的时候问题解决了。这个问题还在。不管你意没意识到。问题还没解决 我们要为此付出代价。随着时间过去 我们有许多未解决的问题。特别是当它们被压抑或抑制的时候。那时候我则更有可能变得抑郁。所以为了生理的睡眠需要睡觉。同样也为了心理的睡眠需要睡觉。现在大家都谈论美容觉。为什么?因为那是被证明的 是显而易见的。如果一个人24 36小时没睡过 你能看得出。眼睛周围是有黑眼圈的 黑眼圈可以掩藏起来。但你仍然可以看出一个人是否衰弱。睡觉不只是为了美容。还是为了幸福。更是为了智力。因为我们知道没有睡觉的人。比如说24小时没睡 24小时后。他们的智商降低10个百分点甚至更甚。所以毫无疑问睡觉能保持美丽。不仅如此 睡觉还能保持幸福和智力。在每个层面上睡觉都是个好投资。 

下面是些非常简要的睡眠小贴士。第一件事 一天差不多睡8小时。找到适合你的睡眠时间。你可以晚上睡7小时 午后1小时。或者晚上6小时 午后2小时 但总的来说 我们一天需要8小时。如果你一天没有8小时睡眠的话。你生命里有些时候确实没法做到。比如考试周。那时候睡8小时觉可能有些困难。或者其他压力时期 或者你有小孩的时候。那要怎么办?小睡。小睡实际上被证明极其有效。白天打个瞌睡。比如睡个20分钟。虽没有晚上多睡两小时的效果好。但也比白天不睡。要强得多。所以小睡也是非常好的投资。你知道白天有效地小睡个15 20分钟。在一定程度上 虽不及夜晚充足的睡眠。但在一定程度上恢复了某些认知。和情绪能力。许多人入睡有困难。是因为他们吃得太晚了或者锻炼得太晚了。你知道 血液依然流得很快。心脏依然跳得很快 他们很精神。所以不要在睡前吃太多…。有些人是可以边吃着边睡着的。但对于某些人。吃了太多 是很难入睡的。所以注意点自己的饮食。注意下自己的锻炼安排 不要在太晚的时候锻炼。还有患有失眠症的人。确实有很大数量的人有失眠症。这些人不要强自己所难。我想这个教室里没有哪个人。没遇过偶尔入睡困难的情况。在考SAT前 在考试前 在约会前。这是很正常的。你知道 在过去。总的来说 我没有入睡难的问题。我10点上床 然后就不省人事了。但有时 在困难时期 压力时期。我遇到问题的时候 最初的时候我告诉自己。"好吧 现在睡觉 现在睡觉 睡觉" 对吧?就像是"别焦虑 别焦虑"一样。然后你有了幻觉 它阻碍你入眠。不要强自己所难。现在我入眠有困难 我就对自己说。"嗯 这是个思考和反省的机会"。常常我会想得烦了就睡着了。不要强自己所难。 

最后 你内在的节律是什么?如今因为我们有了电灯。因为我们住在室内而非暴露的小屋。和与大自然相联系。直接相联系的洞穴。我们把自己从大自然中隔离了出来。我们也为此付出代价。不论是因为我们不需要狩猎和收割。因而锻炼不足而付出代价。还是因为我们不注意。我们不再认识。留意或关心自身的内在节律而付出代价。你晚上需要多少时间?你是夜猫子还是早起鸟?你需要晚上10点睡觉吗?或者你需要早上5点睡觉吗?什么时间适合你?试着去创造一个和你天性需求一致的生活。这并非总是能做到。如果你去上班 在大学里这相对来说还是可能的。你的时间非常有弹性。因为你知道你可以…这堂课是11点半上的。这堂课也被拍了录像。所以如果你想睡到中午 那也没问题。你可以下了课从头看 所以你可以自由安排时间。在你将来工作的时候 或者在你上中学的时候。中学是7点半或者8点上课 你不得不到。还是尽可能地知道。认识到自己的内在节律。比如你需要多少睡眠时间。和你什么时候需要睡觉。 

关于睡眠有很多研究。很多研讨会和课程。只研究睡眠需求。让我来总结一下William Dement的研究工作。他是斯坦福大学的教授。在他醒着的时候做了很多关于睡眠的研究。"剥夺睡眠对健康。和幸福的影响已被研究证明。睡眠剥夺会使认知能力和生理机能。受到损害。对心情的影响则更甚。夜晚睡眠不足的人。易于感觉不快乐 更紧张 身体虚弱。和精神和身体上愈发疲劳。充足的睡眠让我们感觉更好 更快乐。更有精力和活力"。再一次我认为。教室里没有哪个人是不知道这些的。再说一次我们没有教任何新的东西。我只是让提醒你们已经知道的东西。 

让我继续下一个主题 和我最爱的主题有关。那就是触摸。美国人普遍 再一次 这是个普遍的平均数据。普遍来说属于世界上触觉最迟钝的人。我们花了很多时间沉浸于我们其他的感觉中。去好餐厅。把我们自己和男朋友用香水喷得香香的。但我们触摸的不够。正像睡眠 像锻炼。像人际关系一样 它是身体的需要。它是人际交往中重要的一部分。在这一领域中最出色的研究工作是Tiffany Field做的。她本来是按摩治疗师 看到了按摩的功效。就去研究了这个课题。拿到她的博士学位 现在是迈阿密大学的教授。她的研究。及其他人的研究显示了。在人与人的接触中触摸的重要性。例如 对于身体健康来说 不论是。我们的免疫系统因为触摸而大大增强。还因为缺少足够的触摸而严重地降低。触摸有助伤口愈合。不需要专业的按摩师。不需要专业的理疗师。简简单单的触摸就可以了。儿童在受到触摸时成长得更好。我们一会会谈到一些研究。一些叹为观止的研究。它关系到身体健康 也关系到精神健康。再一次。儿童的认知发展受到触摸的影响。在心境障碍。饮食失调方面的精神健康常常与缺乏触摸。或触摸不足有关。由缺乏触摸引起的抑郁和焦虑。可以通过加强触摸得到克服。希望你们现在确信了:触摸能改善性生活。William Johnson 抱歉William Masters和Virginia Johnson。两位最重要的性治疗师及研究者。他们的研究表明性功能障碍。大部分性功能障碍。实际上 有70%到95%是可以。简单地在触摸的帮助下得到解决的。所以他们所认为的实际上是。前戏不应该只是达到目的的手段。他们认为前戏本身也可以是目的。触摸起到的作用本质上来说。正是幸福心理学的内容。性 不 那个…。我刚才是不是说得很大声?好吧 我想是的 好的。所以幸福心理学的目标就是。记住 让我们从消极达到零点。再让我们从零点到达幸福。所以他们在研究中表明了。触摸有助性功能障碍。好的 让我们运行起来。让我们到达这里 但他们也认为。触摸能活跃性生活。所以它不只是解决问题。它还让我们从零点变成活跃。Martin Seligman有本书。是关于普遍心理干预的。哪些起到作用的 哪些没有。正如我们在课上讲的。大部分干预没有作用。这点我们可以从剑桥萨默维尔研究中得出。一个5年的干预实际上长期看来。起到了负面的作用。我们看到那么多自助。提供权宜之计的书。它们不起作用。他研究了 Williams Masters和Johnson的研究。通过综合分析发现。70%到95%的时间里他们的方法是有用的。再一次 这几乎在社交心理学中。或社会科学干预中前所未闻 但它起效了。 

触摸是一个需要 如我之前所说 正如…。触摸是一种需要 正如身体锻炼是一种需要。正如吃饭睡觉是需要一样。 

这有个关于医院早产婴儿病发的故事。他们发现。早产婴儿病房有一个区域的。成功水平要远远高于。同病房其他区域。换言之。就是婴儿的身体好得更快。他们更健康了 在后续对他们的长期跟踪里。他们实际上显示出了更好的认知能力。和生理发育。他们对此感到很奇怪 他们就想 我是说。"可能是因为空气 可能是因为在医院的位置"。因为其他所有东西都是一样的 他们吃一样的食物。受到一样的治疗。有一位医生就特别想弄清楚。这个早产婴儿病房的这个区域。有什么对他们的健康那么有帮助。于是他观察了那个地方。一天晚上。他坐在一个看不见的地方 听到了一个声音。他看了看 不想被人看到他。一个护士进来了 一个他认识的护士。他在这家医院很长时间了。这个护士走进病房 抱起一个早产婴儿。这是违反医院规定的。因为他们不能被人触摸 他们必须呆在保育箱里。她抱起婴儿。轻轻抚摸他 轻轻地触摸他。跟他讲话 然后把他放下。再走到下一个育儿箱 抱起婴儿。非常温柔地轻抚他 把他放下。她把病房那个区域所有的婴儿都抱了。她在她值班时晚上这么晚做这些。是因为这是违反规定的。然后她走出去。这件事衍生了一堆。对婴儿的触摸的重要性的研究。特别是对早产婴儿。于是Tiffany Fields做了这个研究。她让早产婴儿在医院里。一天做45分钟按摩。非常轻的按摩 他们很幼小。他们发现。这些婴儿在医院里。增加了47%的体重。一年后 已经出院回家了。他们身体上。认知能力上和运动技能上有了更好的发展。只是因为被更多地触摸了。我们需要触摸 这是先天本能的需求。实际上 有关于触摸剥夺的研究。但并不是在人类身上进行的研究。是在猴子身上进行的研究。Harry Harlow。非常著名的心理学家 他研究的是一群猴子。他把这些猴子从它们的母亲身边带走。不给他们任何触摸。其他需求都满足。他发现。这些猴子身体上的成长不如其他猴子。他们的认知发育受到了损害。他们表现出自闭症的行为 例如。他们会坐下 抬起脚然后摇晃。你们看过这场景吗?你知道就像在 那个达斯汀·霍夫曼演的电影?"雨人"。在那部电影里 他就坐下然后这样摇晃。许多猴子表现出了完全一样的行为。这种摇晃 自闭症的行为。残忍的研究 无论是在蚂蚁还是猴子身上。自然也会在人身上做。不幸的是 这研究在人身上做过。这是人类实施过的最大的 悲剧的。自然实验之一。这个实验是在齐奥塞斯库统治下的罗马尼亚完成的。那个残暴的独裁者统治着这个国家直到1989年。他做的一件事就是把许多孩子从父母身边带走。特别是那些父母与之政见不同的。那些父母不是"有文化"的共产党员的。所以他把15万个孩子带到收养所养育。远离他们的父母。现在这些收容所 因为那是个非常贫穷的国家。没有钱提供看护 足够的看护。没有钱请足够的工作人员。这些孩子的身体需求得到满足。有食物 水 还能洗澡。但他们并没有得到爱。不是因为那些看护不愿意给他们爱。而是她们没有那个时间。在1989年 战争爆发 齐奥塞斯库被驱逐。这些孩子被从这些收容所里释放。心理学家去收容所里。他们看到的是这个悲剧的自然实验的后果。在身体成长方面。这些孩子中只有3%到10%的人。与他们同年龄段孩子的体型是相当的。认知力发展方面 水平远远低于100的平均智商。运动发育方面。同样也由于缺乏触摸而严重受损。许多孩子抱着腿坐着。不停地摇晃。触摸剥夺的影响。 

现在触摸是很重要的 再一次 让我们到达这里。我们在零点 我们很好。问题是我们如何才能向前?触摸也可以把我们带到那。不只是好 还要非常好。触摸的重要性。Virginia Satir 一位心理学家如是说。"我们一天需要4个拥抱才能存活。我们一天需要8个拥抱才能维持。我们一天需要12个拥抱才能成长"。她是以诗歌的形式说出来的。尚没有实质上的研究表明要多少个拥抱…。很好 你今天还差11个拥抱。不过…很好 很好。 

不过还有一个关于拥抱的研究。这是Jane Clipman做的研究。属于一个心理干预 很简单。一天至少5个拥抱。面对面的拥抱 不是和你的情侣…。就是拥抱 拥抱不同的人 朋友。也可以是你的情侣。但这些明确地是和性无关的拥抱。然后她发现4周后。这一组的人 与对照组不同。对照组的人要写出他们每天读了多少小时的书。对照组的幸福水平明显没有提高。每天拥抱5次的那组却显著提高。她谈论起这个实验时说。对某些人来说 拥抱要困难得多。特别是男人。但即使这样 参与这组的男人 大部分还是拥抱了。他们可以在运动场上拥抱。他们���以找机会或者想方设法地一���拥抱5次。而这些这么做了的人真的变得更快乐了。我本来想把这作为本周课后作业。我现在不布置了 但我推荐你们去做。强烈推荐和建议。从现在开始。在你以后的日子里 至少5个拥抱 12个更好。 

拥抱的好处就是它是双赢的。因为当我们拥抱某人时 我们也被拥抱了。当我们触摸某人时 我们也被触摸了。就像幸福 当我们分享它的时候 它就越来越多。这是个双赢 正正得正的游戏。关于拥抱和触摸有个重要的事要记住。那就是要尊重对方。因为不是每个人被触摸时都会像我们一样感觉良好的。或者我们被触摸可能感觉会不舒服。那没有关系 有许多原因。你知道有人可能比较有教养。有人可能只是身体上不习惯 人和人是不同的。有些人更喜欢被触摸。那没关系 这是自然的。所以我们需要尊重别人的界线。也要让别人尊重我们的。不过总的来说 触摸是很重要的 触摸很好。我们需要变成触觉更敏锐的人。同样 一个触觉更敏锐的社会。 

让我来给你们看个例子。就是一些人是如何比别人更喜欢拥抱的。好了 再一次 这是我们的药方。非常简单。一周至少4次 一次至少30分钟的身体锻炼。正念冥想。如果你方便的话 每天10或15分钟。如果不方便 至少。全天也要有安排地做几个深呼吸。晚上8小时的睡眠 一天至少5个 最好12个拥抱。这可能是你听说过的。最有效力的心理学药方了。它能让我们自然健康状态。我们天生的或者基因决定的水平 幸福水平。 

让我们现在继续到 另一个主题。下面是我今天想对你们说的。音乐:Let's Talk About Love 席琳·迪翁。让我们来谈谈爱。让我们来谈谈我们。让我们来谈谈生活。让我们来谈谈信任。让我们来谈谈爱。好了。那么咱们来谈谈爱 被谈论得最多的话题。是让人有感触的话题 到处都在谈论。从餐厅到卧室 到教室 现在就是了。然而它也是最受误解的话题。人际关系 亲近的 亲密的关系。无论是和爱人 还是和家人。还是和亲密的灵魂伴侣 朋友。这些爱都是头号的幸福预言者。我们今天的重点会是。下节课也是 情侣关系。但是 我们仅仅重点讨论情侣关系。因为在很多方面 它是具有代表性的。它包括了其他关系中的许多东西。我们所讲的观点。也可以应用到亲密的友情。家庭关系中去。一开始我给你们看一个。电影"偷穿高跟鞋"中的片段。当卡梅隆·迪亚兹向她在电影中。结婚的姐姐。读了一首优美的诗。把声音开大 谢谢。电影片段。此刻非常神圣的时刻。小姐?这不在婚礼的程序中。因为这是个惊喜。我经常给Rose惊喜。通常 她讨厌那些惊喜。我想 我希望她喜欢这个。这是E.E.Cummings的一首诗。献给你。"我将你的心带在身上。用我的心将它妥善包藏。天长日久也不会遗忘。无论我前往何方 都有你伴我身旁。即便我单独成事 那也是出于爱人 你的力量。面对命运我从不恐慌。只因你就是我命运的方向。世间万物于我皆如浮云 只因你在我眼中就是天地四方。这秘密无人知晓 在我心底埋藏。它是根之根 芽之芽 天之天。都是生命之树所生长。这大树高于心灵的企望 也高于头脑的想象。是造化的奇迹 能够隔离参商。我将你的心带在身上。用我的心将它妥善包藏"。 

那是什么?根之根 芽之芽?能够隔离参商?让我们试着来理解它。只是比我们已知的要多那么一点。再一次 人际关系是一种天性的需求。没有人是孤岛 没有人能脱离人际关系。而生存 而成长。让我们再看一遍。我们几个月前讲过的关于极度幸福的人的研究。还记得在晚餐时。Martin Seligman选择了这10%的最幸福组。最拔尖的 并研究了他们。他们的发现之一 如果你还记得的话。是他们和别人一样也经历了困难。他们有时会经历焦虑 紧张 抑郁 沮丧。但是 他们和其他人的区别。那些不是最幸福的人。是他们恢复得更快。换言之 他们有更强大的免疫系统。为何他们有更强大的免疫系统的原因之一。他们对自己经历表现与他人不同的原因之一。是因为他们有…。这是他们和其余人的不同之处。他们有很强的人际关系。无论是和情侣 还是和灵魂伴侣。还是和家人 或者以上这些都有。这是这组人与人不同的一个特点。他们有亲近 亲密的关系。这就大有不同了。为什么?两个原因。首先 因为当你快乐时。你会和你亲近的 真正在意的人。分享快乐 那就增加了。放大了你的快乐 同样也是他们的快乐。这是个双赢。当你经历艰难困苦时。拥有亲密的关系能帮助你克服困难。所以再一次 无论是从消极到零点。还是从零点到幸福 亲近的亲密关系。显著地有助于变得幸福。但是人际关系中的关键是认识自己。就是知道自己的需要。因为我们每个人需要的人际关系都不一样。没有统一的药方。不可能说 一天5个拥抱然后你就更幸福了。5个密友 你知道 一个或两个爱人还有…。没有这样的药方。是因人而异的。 

我还想回顾一下千层面定律。还记得幸福书里的千层面定律。那就是我们都需要…。比方说 我很爱我的家人。家人对我来说是世界上最重要的。但是 那并不意味着我愿意一天。花8个或者10个小时和家人在一起。你知道我还需要独处的时间配额。我需要练瑜伽 我需要锻炼 写作 教学。那并不意味着我的家人对我来说就不重要了。或者说我喜欢写作 我确实喜欢。我一天能写3个小时 不能再长了。为什么那是千层面定律…。如果你们还记得 千层面是我最爱的食物。我每次回家 我妈妈总给我做千层面。但那并不是说我需要每天整天都吃千层面。你知道一周吃5顿对我来说就够了。或者一周吃一顿。但是你需要找到自己的配额 人际关系也是一样。人际关系也是一样。问问自己 每天和别人相处多少小时。我就觉得愉快 觉得和人相处是一种享受。决定这个合适时间的因素。就是看你是内向还是外向。我们都有一个。相对于最优水平而言的心理基准水平。让我来画个图。这是所谓的最优激励水平。在认知学上说。激励的最优水平。这是 比如说是我们的最优激励水平。我们生来就在这附近 有些人高 有些人低。极少有人正好在这条线上。内向的人有更高的激励水平。更高的先天激励水平。这是内向的人。相对于最优水平而言。外向的人有较低的激励水平。现在我们都身体上 或者精神上。认识上并身体上。寻求体验的最优激励水平。因为那是最方便的。那时候我们在最佳状态。那时候我们心情更愉快。内向的人的天生自然状态。是高于最优激励水平的。所以他们实际上需要降低刺激水平。所以他们喜欢自己单独呆着。与之相反 外向的人有着较低的激励水平。这个水平没有他们达到最优水平时那私愉快。于是他们出门去参加派对 那里有很多刺激。有很多事情发生。有很多为了达到自己最优水平的人。这就是为什么普遍来说…。酒精起什么作用?酒精降低我们的激励水平。这就是为什么内向的人常常需要。在他们跳舞前喝杯酒或喝瓶酒。因为当着。其他人的面跳舞会刺激水平上升。所以酒精的作用是降低他们的刺激水平。然后跳舞就容易接受了。与之相反 外向的人需要咖啡。咖啡是保持清醒的刺激物。因为咖啡让他们上升到自己的最优激励水平。再一次 我给了你们一个梗概。再一次 自Issac后数十年里有很多。关于内向和外向的研究。顺便说下 这不是意味着我们不能…。用Brian Little的话说 我们的举止不能脱离本性。就是说一个内向的人可以举止像外向的人。我站在这里 我是个内向的人 顺便说下这是心…。我们生来就是这个状态 这是天生的。我生来就是内向的。但当我站在教室前。唱着优美的歌曲 我的行为像一个外向的人。但是当我们的举止脱离本性时。明白这样通常会让我们衰弱这点很重要。它会带走我们的精力。这对我们来说很难 因为我们在假装 所以如果我在这。当我上课时 我会变得更兴奋。离我的最优激励水平更远。这会消耗精力。这就是为什么内向的人在表现得像外向的人之后。无论是参加派对还是讲课。都需要Brian Little所说的恢复龛。恢复龛 一个恢复精力的地方。在那个地方。人们可以和人一对一相处。也可以给自己一点时间独处 恢复。记住 这个问题不是压力。这个问题不是压力。而是我们没有恢复。内向的人需要恢复。有事是靠自己 有时需要一对一。比如说我的生活 恢复的形式的坐办公室。让我一连4个小时面对学生更容易。这是一种恢复的形式。如果我只是上课 只是站在听众前面。那最终会导致过度工作 会导致我毫无生气。外向的人也是一样。如果你是个外向的人。有时你需要做点内向的人的事情。比如说你需要为了考试学习。你是可以和很多人一起学习备考。但可能效果并没有那么好。所以你自己一个人或者和另一个人一起学习。而在那之后你需要的就是一个恢复龛。因为对一个外向的人来说。远离其最优激励水平是消耗精力的。你需要一个恢复龛。恢复龛在哪?我需要和朋友出去参加派对。这对他们来说就是一个恢复龛。所以再一次 认识你自己。知道你自己人际关系的需求是什么。不是说你的行为不能脱离本性。你可以。但同时你必须要有一个正式的恢复。一个与你本性相对应的恢复龛。不管你是内向的还是外向的。简而言之 你需要的人际关系数量。取决于你个人的 独特的需要。你想和别人在一起相处的时间。因人而异。这里无分好坏。一些人喜欢每天和朋友相处10小时。那些特别外向的人。其他人则喜欢一对一 亲密的时间。每天就需要2到3小时。这里没有什么好或不好。并不意味着你对别人的爱就冷淡了。也不意味着。你不需要亲密的关系如果你是前一种或者后一种的人。两种人都需要亲密的关系。和灵魂伴侣 情侣 家人 都可以。都是一种需要。 

David Myers说:"几乎没有什么能比。一段和好友之间亲密的 悉心培育的 公平的。亲近的 长达一生的相伴更能预言幸福了。关于异性关系所做的研究要比。同性关系多。但是 过去几年 特别是经过。华盛顿大学John Gottman的研究工作。他可以说是这一领域的领先专家。他在同性关系上做了更多的研究。他们的发现…。非常相似的结果 但略有不同。我们会讲讲这些有趣的…。因为它们是我们没见到过的不同。可能今天不讲 但下节课会讲。所以当我谈到关系时。我说的是关系。而那位David Myers说的是长达一生的相伴。浪漫的关系。同性或者异性。关于关系有趣的事情是。每个人都能从中获益。但如我之前一次提到的那样 男人获益更多。为什么他们获益更多?因为首先 通常。男人有可以分享事情的人。而女人 通常 更可能…。和特定的女性朋友谈论特定的话题。分享快乐和悲伤的女性朋友。但是 两种性别的人不管是异性恋还是同性恋。都能从亲密的关系中获益。 

我来放一个前段的视频片段。关于什么是婚姻的。"人人都爱雷蒙德"的片段。这就是婚姻。你早上一睁眼 她在那。你夜晚回家 她在那。你吃饭 她在那。你…你睡觉 她在那。我知道这听起来很恐怖。但是不是 不是。如果你… 如果你找到了合适的人。这样结婚就很好 很好。真的真的很好。我想要一分钟时间来反驳。 

那么我们来看看各种关系。首先 我们来看看婚外恋。看看到底今天。长期的关系出了什么问题。婚姻或是其他长期关系。我们看到可不是非常好看的景象。就婚姻而言。如今三分之二的婚姻都以离婚收场。这并不意味着剩下那三分之一就过得很好。因为经常 经常。人们在一起是因为责任感 习惯。因为没得选了。因为他们认为这就是他们的命运。现在 我当然不是要去论证。所有婚姻或关系都不应该破裂。当然有。不是每一段关系都是合适的 我们常常会犯错。但我要表达的是。很多结束了的关系。或者许多变成。用梭罗的话说"安静的绝望"的关系。毫无生气 成了一种惯性。许多这样的关系。要么不应该终结 要么实际上可以很好。然而问题是。我们看到 当然有三分之二的人以离婚收场。但还有三分之一的人 常常。无法长期维持爱情。我还不知道在婚礼上。或者在别的结婚仪式上。会有哪对夫妇说。"这只是暂时的 我们将…你懂的。携手2年 直到死亡把我们分开 但也不见得"。你知道当人们步入一段关系中时。他们的想法 他们的希望。是延续他们在关系最初时。拥有的那份爱与热情。这就是关系如何开始的 这是爱的承诺。但是通常 很显然。事与愿违。一个主要原因。一个被许多性治疗师和心理学家引用的原因。再一次 你不需要研究它。因为这是如此地不证自明。一个原因是因为爱情。特别是欲望和热情会随着时间磨灭。因为我们知道新鲜事物产生更高的激励水平。或者用康奈尔大学心理学家Daryl Bem的话说。"新奇的总是引人性致的"。如果是新鲜事物 它总是更有趣 更让人着迷。这就解释了为什么这么多关系都破裂了。但问题是…这必然是我们本性的一部分吗?是也不是。是。新奇的事物更能刺激我们 我们适应变化。这必然是我们本性的一部分。记住 我们是会改变的探测者。当新事物出现时。我们变得更加受激励 不论是从认知 心理学。还是身体 生理学上说。我们随着时间变化适应着 不论好与坏。所以如果不是以好结果为目标的事件。如果不是找情侣这样的积极的经历。这种能力是很好的。可惜不是。问题是 我们该怎么办?在我们开始讲"我们怎么办"之前。我先跟你们分享。一个我做的研究。希望它能尽快发表在。一个有名的学术期刊上。但我研究的问题 是个非常重要的问题。就是谁是世界上最帅的男人和最靓的女人?说得科学点 谁是商场里最美的人?我把这个问题提给了我一个大二的班。我当时教一个大二班的心理学。我们有了很长时间的辩论。但最后 我们得到一个科学的结论。没有看过PPT的同学告诉我。你们认为谁被选为了最美的男人?你得了A。还有人想要A吗?不。谁被选为了世上最美的男人?比尔·克林顿?不是。谁?乔治·克鲁尼?很接近但还不对。(布拉德·皮特!) 答对了。这张图是他站在特洛伊城前。告诉海伦他去去就来。谁是这世上被大二班科学地证明是。最美的女人?你!对!另一个A。她是谁?(安吉丽娜·朱莉) 安吉丽娜·朱莉?不是。(哈莉·贝瑞)哈莉·贝瑞。这张图是她跟阿德里安·布罗迪说。他要是再敢接近她一步 她就杀了他。现在想象一下 想象一下。想象你找到了你的布拉德·皮特或者哈莉·贝瑞。你理想中的男人或女人。他们不仅美得冒泡。而且特别感性 聪明。老成 圣洁 慷慨 仁慈。一部电影挣1000万美元。而且最重要的是。他们爱你绝不少于你爱他们。想象一下如果这发生了的话。你遇到了你的布拉德·皮特或者任何一位你的致命情人。或者你遇到了哈莉·贝瑞或者女超人或者女蜘蛛侠。或其他真命天女 想象一下。你看到他们 你跟他们讲话。然后你爱上了他们 他们爱上了你。然后你们结婚了。你婚后的第一晚 或者婚前第一晚。你们选。你回到家。没有盔甲会阻挡你。没有枪会把你赶走。你们成小时 成天 成周 成月地激情不休。你甚至不来上幸福课。感觉太棒了。想象一下 想象一下。然后当然银幕落下。你们永远幸福快乐精力充沛地生活在一起。对吗?也许。现在想象一下 相爱5年。和这样一个感性 智慧 性感 富有的人。相爱5年。你参加了一个心理学实验。这个实验是。他们把你挂在各种各样电极上。各种各样生理学的 认知学的测量工具上。然后他们测量你的生理兴奋水平。然后你的布拉德·皮特或者哈莉·贝瑞。或者谁走进了这个房间。看看你的生理兴奋水平如何。然后他们离开。再然后另一个有适当魅力的人。不是布拉德·皮特和哈莉·贝瑞。但是个有相当魅力的人走进这个房间。你什么时候身体上 本能上。你什么时候身体上更兴奋?总得来说 不是总是发生。但大多数情况是。你会在那个生人进来时。身体上更兴奋。为什么?因为新奇的总是引人性致的。因为新鲜的事物有刺激。现在我们在这笑。其实在一定程度上也解释了这个统计数据。这就是为什么这么多关系破裂了。尽管它们一开始抱着最美好的意愿。最真诚的意愿 但结果它们深陷泥潭。这是部分的原因。所以这是不是个坏消息?不尽然。它不坏也不好 就是这样。就像我们说万有引力定律。万有引力定律 不好吗?不 没什么不好。好吗?也没什么好的 它就是这样而已。就像我们说痛苦情绪 它们不好吗?不。它们好吗?不 它们只是如此。如果你想过幸福的生活。如果我们想有良好健康的关系。我们需要做的是首先接受本性。这不是什么好或坏消息。只是真实的消息或者不是消息 因为这是我们都知道的。问题是我们要怎么做?我们要怎么遵循本性才能驾驭它?我今节课和下次课要讲的。是我们该如何对待关系。鉴于新鲜事物带来兴奋的事实。鉴于生理兴奋水平会升高的事实。顺便说一下这对男女一样适用。当新人出现时兴奋会升高。鉴于此。我们怎样才能创造一个活跃 进步。超过零点到达幸福的关系。这是可能的。幸福心理学可以帮助我们做到。在接受现实前的第一步是。理解真爱的真正含义。因为许多人 许多人会问这样一个问题。真爱在现实中存在吗?是的 它存在于小说中 存在于电影里。但这些电影荧幕上的角色。实际上是比生活中的夸张了。他们说着完美的台词 他们创造着完美的爱情。我要怎么才能做到。我要怎么才能去经历他们所经历的?真爱真的存在吗?我能做到拥有。英格丽·褒曼和亨弗莱·鲍嘉那样的爱情吗?我能到达4万英尺的高度吗?我能拥有梅格·瑞恩和汤姆·汉克斯。那样的心灵感应吗?我会不会有可能和英国首相一夜风流?或者和休·格兰特?可能吗?会是真的吗?你知道自助书帮不了多少。我最爱的自助书作者之一 Leo Buscaglia。他也是南加大的教授。非常棒的作者。富于同情心的热情演讲者 帮助了我很多。这是我读的第一本。关于生活 爱情和学习的自助书。关于爱情和关系他如是写道。"完美的爱情确实是稀有 因为作为一个爱人。需要你一直拥有智者的敏锐。儿童的灵活性 艺术家的感性。哲学家的领悟。圣人的包容。学者的宽容和笃定者的刚毅"。多美的文字。如此美 如此严苛的条件。谁能做到这些?有谁吗?你瞧 完美的爱情是不存在的。去期望自己 自己的情侣。自己的关系这么完美 注定会失败。我不希望Tammy对她的伴侣有这样的期望。因为我能向你们保证我会辜负她的。根据这些标准我辜负了她。我对她也是一样 这是不可能的 是不现实的。当我们的期望高于生活时。当我们的期望是完美的台词 或完美的爱情时。我们就在为自己的失败做着准备。但我们读到的都是完美。我们看到的都是完美。我们开始把这些。这条我们关系的直线。完美又不真实。但是尽管完美的爱情不存在 真爱却存在。真爱存在于不完美的人中。什么是真爱?幸福心理学又怎么帮你识别它?首先 问这一领域里。最经常被问及的问题。其次 不仅关注什么起了作用。还要关注什么在这些少数真的很好。没有破裂 没有走到尽头的 没有到达安静的绝望状态。或者弗洛伊德所说的"惬意的麻木"的。毫无生气的关系中。什么起了最大的作用。让我们从他们身上学学 向最佳情侣学学。 

所以我们需要做的第一件事就是重新组织一下这个问题。还记得我们在讲冒险的青春时也是这么做的。我们对那些快乐 健康 成功的人。也是这么做的。讲几何形状时我们也是这么做的。因为问题引起了探索。问题定义了现实。传统心理学问过。夫妻咨询师 这一领域内的研究者问过。为什么这么多长期的关系失败了?为什么这么多人在一起一年两年或者5年之后。结束了他们的关系?为什么许多人。在一起后就不再幸福了?他们得到的答案之一就是。因为新鲜事物刺激我们。因为寻求刺激是人类本性的一部分。新奇的是引人性致的。那是一个答案 并且是个正确的答案。也是一个我们要考虑的重要因素。这是现实的一部分 不管我们喜欢或不喜欢。但那个问题还不够。就像当我们问风险人群时。或者当我问你们几何形状时。类似的问题一样并不足够。当我们只问这个问题时。我们实际上忽略了一部分现实。非常重要的一部分现实。这一部分现实能引导持续 良好。激情的关系。幸福心理学所问的问题是。"什么让一些关系发展良好。并随着时间越来越好?"。因为这样的关系确实存在。我有幸看到了这样的关系。我跟你们讲过的我祖母和。我祖父之间的关系。他们结婚53年。我们有一些拍他们一起的照片。在照片里 我祖母 她是家里的女家长。非常强势的女人 一直也非常幽默。她坐下来。我祖父挨着她 把他的手搭在她肩头。看着她。就像16岁相爱的小年青。他们结婚53年。当然也有起起落落。有过纷争和失望。他们也不完美。但他们有一段发展良好的关系 充满激情的关系。这些关系有什么特别?什么让它们如此成功?重要的问题是。"我们能学到什么并应用到我们的关系里。如果我们想。维持一段激情的关系?"。所以让我们学习那些成功的范例。不只是学习成功的关系。还要学习最好的 我们能研究并学习的。最好的关系。下面是一些我们能从。这些关系上学到的东西。我要从研究者们那借鉴。首先是John Gottman 我之前提到过的。然后是David Schnarch。David Schnarch和我之前提到的John Gottman。在异性关系和同性关系方面。做了一个非常好的研究。下面是一些我会讲到的他们的发现。"首先 当我想出怎么预测离婚时。我以为我找到了拯救婚姻的钥匙。但和我之前的许多专家一样 我错了。我没能破解拯救婚姻的密码。直到我开始分析在幸福的婚姻里什么是对的"。他如是写道 如果我没记错的话 是在…1999年。2001年的时候他的研究更进了一步。我们下周会讲到这个研究。他研究了同性关系。他学到的东西多于。他研究异性关系促进因素时所学到的。总之关键是要去研究成功的关系。不管这个关系是什么 不管它的本质是什么。当他研究了这些 一切就大为不同了。他可以预测。他可以以94%的准确度预测离婚。94% 那是个社会科学领域前所未闻的数字。非常卓越 他可以做到。但这没能帮他创造更好的长期关系。只有当他开始问这个问题。"什么让这些关系发展良好。并随着时间越来越好?"。and grow stronger over time?" –。就是那时他继续研究。并且能为更好的关系提供挽救方法。我要借鉴的第二个人是David Schnarch。下面是他在他那本非常棒的书。《激情婚姻》中写道的。"橘皮组织和性潜能是高度联系的"。他研究最成功的关系的结果表明。他们多年来享受一起的时光。而且他们这辈子最棒的性爱。并不是在他们18岁或25岁或35岁时。事实上是在50多岁和60多岁时。所以你们还有很多可以期望的。这不是说。18岁时的性爱就不好了。只是在50和60多岁时会更好。这就是他在他研究成功关系时发现的。但这样的关系只占少数。很可惜。因为现今大部分人到了50岁和60岁的年纪时。他们已经到了第2段或第3段婚姻。或者长期关系中了。但常常是越到后面才越精彩。就是那时生理兴奋水平。偏离 远离了合适的水平。据说可能在男人24岁 女人35岁时。更为强烈。我不知道他们从哪得到这些数据的。但我们实际上有潜在可能达到高峰 注意这个用词。我们有潜在可能在50多岁和60多岁时到达高峰。为什么?我们之后会理解。下面是我们下次课。将要讨论的内容。我们将分析增长的统计数据。我们将研究最好的关系。就像我们研究最好的老师Marva Collins一样。就像我们研究最好的冥想者Lama Oser等人一样。我们将研究最好的关系。并提问我们能从中学到什么并应用到自己的生活中。再讲一分钟 一分钟。我们将讨论4个不同的话题。第一个 维持恋情需要努力。如果你想在工作上有好的发展 你需要努力工作。如果你想在关系上有良好发展 你需要投资。第二 我们会讲到怎样投资。最好的关系。创造了高水平的亲密关系。让彼此之间有更深层次的了解。这些是良好的关系。这些是保持激情的关系。并且随着时间过去 感情升温。最好的关系并不能。免于冲突。我们会讲到其涵义及其状态。最后我们在上幸福心理学的课。这是关于欣赏积极事物的课。因为当我们欣赏好的事物时 好的事物会青睐。我们及他人。我们周四再见。掌声。 

第19课-如何让爱情天长地久 

大家好 下午好 早上好。今天我们继续谈论爱情。讲之前先说一下 一位叫Nadia的同学。你的钥匙链落在这教室了。就在我这里。课后请来我这里取。 

那我们来讲讲爱情吧。上节课我们留下的问题多过答案。问题是…很难的问题。"我们怎样维持爱情。怎样在热恋期后维持激情"。因为即便我们觅得自己的布拉德皮特。或海利贝瑞 也不能确保激情长存。其实真正能确定的是。我们和那个人历经的生理上的兴奋。会慢慢淡去。这在一定程度上解释了几个数据。三分之二的婚姻以离婚告终。那些即便以后没有分开。也是出于省事。出于责任。而不是因为。曾经在一起时的那种激情。那种在恋情一开始时存在的。带着想要与之共度终生的。彼此承诺与对方长相厮守的美好夙愿。若问"幸福心理学能为我们做什么?"。这是对另一个问题的提问。另一个被问了很多年的问题。"为什么这么多恋情失败的例子"?这问题很重要 很有必要问。这个问题涵盖面远远不够。但这个问题犯了一个错误。它有一个缺点 只问了消极的问题。当我们研究问题少年。时间管理与压力时 也犯了这样的错误。因为当我们只问了消极的问题时。我们就会看不到巴士上的孩子。这是一个比喻。我们就会看不到眼前真实的颜色。如果我们仅关注一个问题。"一共有多少几何形状?" 问题决定现实。很多时候 如果我们不问某个问题。某些答案就不会出来。所以我们来问这个问题。不仅问"怎样让恋情维系下去"。还要问"怎样让恋情美好下去"。如David Schnarch和John Gottman两位心理学家。在他们写的大作中。在他们的研究中 谈过这些方法特征。让我们来了解一下。 

第一个方法是努力。努力是世界上守得最好的秘诀 是成功的秘诀。是个人成功的秘诀 是感情成功的秘诀。成功没有捷径。若想让一段恋情一直美好。我们就得为之努力。这看起来似乎是不言而明的普遍常识。但在这个班上 我们经常体会到。正如伏尔泰所说"常识没那么寻常"。为什么要用常识来维持一段恋情时。却发现这样的常识不是人人都有?这是因为。很多人对恋情有着错误的期望。或者确切的说 对美好的恋情有着错误的期望。大多数人认为拥有一段美好的。至死不渝的恋情的关键在于。寻找到真命天子。这确实重要。但是错在了将重点放在寻找上面。我们也套用一下Carol Dweck研究中的。固定心态和可塑心态的对比。人们有一种会破坏恋情的寻找心态。我来为大家解释一下。先回到春假前我们讲过的知识。Carol Dweck证明了。当人们获得游戏测试成绩后。我们赞赏他们"你真聪明"。这时他们就有了固定心态。相反 如果我们赞他们"你真努力"。他们就有了可塑心态 这两种心态都会产生影响。因为固定心态。即赞赏"你真聪明"帮不上什么。却在他们进入下一轮测试时。当测试变难 威胁到对自己的自我定位。"若做不出来。便说明我不聪明了" 这时他们便更容易放弃。他们更易被失败威胁到。他们也无法享受过程。他们变成了完美主义者。或者说更倾向于变成完美主义者。这就是固定心态。你真聪明 你的智商真高。而可塑心态 "你真努力"。我会努力做的。即便我没成功 没做好。我也照样能学习 能够更享受过程。这就是可塑心态。现在将这个理论运用到恋情上来。如果我认为拥有一段至死不渝的。幸福的恋情的最关键所在。寻找到真命天子。这里的寻找心态就相当于。Dweck提出的固定心态。假设我现在就有这种寻找心态。认为这是最重要的事。此时若我们的恋情经历一段艰难岁月。接下来会怎样呢?我会开始思索。"不对 我肯定是还未找到真命天子。我肯定弄错了 找错了对象"。这种心态常会让情况恶化。可能会导致恶性循环。这就是寻找心态。因为这种心态是固定的 我找到了白马王子。我找到了自己的女神 我的皮特我的贝瑞。问题是现实总是残酷的。我们无法脱离现实。现实证明完美的恋情根本不存在。记住Leo Buscaglia曾说过。"完美的爱情少之又少"。不 不是少 而是没有。当我们意识到。若我们有了寻找心态 这种固定心态。它会威胁到我们的自我认定。一有矛盾发生我们就开始琢磨。"估计我还没找到真命天子"。若我们反过来抱着培养心态 这种心态是可塑的。它与我们的努力相关 "好的。我们的感情碰到了困难。我们在经历困境。但不怕 因为我们在努力改善。我们在努力解决问题"。这就像那些被赞扬努力的学生 "好的。虽然我现在解决不了问题。但又怎样呢?我在努力。我在尝试 我甚至在享受这过程"。看这类比多么相似 Dweck研究的心态。和我们用在恋情上的心态。如此之重要 两种心态 结果相差甚远。很多时候我们是可以转变这心态的。不是马上可以变 但也不用太多时间。要知道Carol Dweck能靠一个。几小时的实验就将人们的心态转变。我们也能在对待恋情上转变心态。只要我们真正接受并内化一个事实。那就是在一段恋情中 找到真命天子。找到合适的人 当然重要。但培养一段感情更重要。这个一会再讲。但是问题接踵而来。为什么人们会有寻找心态呢?如果培养心态更健康。为什么人们没有培养心态呢?主要原因之一在于电影 现在的电影很精彩。电影很棒 向我们展现一切皆有可能。亚里士多德曾对小说发表如下看法。小说远重于历史。因为历史只是白描。而小说却将一切描述到了极致。所以说电影是很美好的。它让我们看到了可能性。尤其是浪漫的电影。这里的浪漫不是指爱情上的浪漫。是指18 19世纪的浪漫主义。向我们展示事物可以有多美好。潜在的人性 我们与世界的关系。这是件很美妙的事。但是电影有一点没做好。电影聚焦于找到真命天子。于是这就巩固加强了寻找心态。因为大多数电影。爱情故事总是这样的 有挣扎。有争吵 有意见相左。途中挑战不断。到最后。Smith夫妇走到了一起 从此幸福生活。荧幕落下 他们走向夕阳。你们知道这画面有什么问题吗?问题就在于 爱情才刚开始 电影就结束了。荧幕落下后才要开始真正的努力。在刚开始时 要恋情顺利是很容易的。蜜月期。或者说一开始的一两年。但是这之后。当最初的肉体上 生理本能淡去。当我们开始意识到。我们的伴侣并不完美时会怎样?会怎样?这就是爱情开始的时候了。这也是真爱得以培养并形成之际。我并不是贬损寻找。真命天子的重要性。这当然重要。通常找到王子前要亲无数只青蛙。但是对于一份美好而持久的恋情。培养更重要。这才是努力开始的地方。在日落之后 困难与险阻浮现。与寻找心态紧紧相连的。是真命天子只有一个这样的观念。我所需要做的就是。穷尽一生找到那个真命天子。如果得去耶鲁大学找。我也会愿意去。就像是莎翁悲剧的现代版。不论需要做什么 我都愿意做。只要能找到真命天子。即便我在波士顿 他们在西雅图。我总会找到真命天子。我不信这套 不现实。这不是信念的问题。我的意思是 很多人。觉得找到了真命天子。也许确实是找到了。却因为不幸的原因。或是为了正确的原因 没能继续下去。他们之后反而找到更多爱情。这世界上有60亿人口。真命天子绝非仅一个。那么是什么让恋情独特的呢?不是找到真命天子。而是与你选择的恋人培养恋情。通过共同的努力 共同在一起。共同度过 彼此奉献。这是创造一段恋情的方式。一段独特的恋情。让两个"我"变成一个"我们"。再次重申 这不能在一夜间完成。甚至不能在一两年里完成 需要时间。事实上感情没有完成一说 这是一个过程。它无关恋情的成功。而是恋情的延续。 

恋情的培养不仅靠交换戒指。或是结婚誓言。而更要靠共同参与。这为什么很重要呢?我们讲一下心理学其他领域的研究。我在其他场合曾提过几次。Muzafer Sherif在20世纪50年代。完成的伟大研究之一 在Muzafer Sherif之前。我之前说过了 人们有这样一个假说。当要解决人际冲突。或者群体间冲突时。人们认为此时的解决方法就是。让冲突双方聚在一起就行了。这个主流假说被称为"接触假说"。让冲突双方相接触。无论是族群冲突。还是个人间冲突。让他们聚在一起 冲突就能解决。这一理论主要由Gordon Allport提出。在20世纪30年代我们的心理学系。Muzafer Sherif出现后 指出这不行。事实上 仅让冲突双方聚首。让冲突双方接触。通常反而会激化矛盾 而非解决矛盾。我讲这个时 用阿以冲突作为例子。那里的人们聚在一起。希望能实现两国间的和平交往。但反而挑起了更多的战争 仅接触是不够的。不仅不适用于国家间冲突。不适用于族群间冲突。也不适用于夫妻间冲突。一段恋情中 发生冲突不可避免。这世上没有完美的恋情。即便有了完美恋情。也会有冲突的完美恋情。对此马上会为大家讲解到。我们的恋情会有冲突。不是在第一年里出现。就是在两三年后出现。而至于重大冲突 这里不是指。类似于"你怎么不放下马桶盖"。"把马桶盖掀起来"的冲突。当然马桶盖冲突也会变得很严重。但我指的是其他重大冲突。当这样的重大冲突出现时。一段关系中的双方只是聚在一起。相互接触。这不足以解决冲突。很多情况下 反而会由小冲突。转化为大冲突 冲突逐渐升级。Sherif对此有什么建议?我们研究出什么方法了?解决冲突的要点之一。在于确立一个超然目标。当两个民族或国家共同努力。彼此互助。无论在公事上还是恋情上。例如共同抚养孩子 只要共同努力。不管属于同一个政治阵营。还是相对立的两个政治阵营 都可以。只要共同努力。彼此支持对方 当你们共同努力。共同参与时。才是你们可能解决冲突之时。我们很快就会看到这样的冲突。反而还能让恋情变得更加巩固。但情侣间需要共同的有意义的目标。再说一遍 可以是一起要个孩子。共同抚养。不是"我们有个孩子。由父亲或母亲抚养。通常会是母亲"这样。这样不算是共同努力了。孩子必须处于一种微妙的位置。用专业术语来说 孩子应是合资企业。用这个术语来比喻生孩子是不是很有爱?只要他们不带来任何坏账。对他们的投资会带来很好的回报。非常适合的经济学术语。琢磨琢磨一下就往上套。共同做事 共同抚育孩子。或是共同参与竞选。或在工作上给对方帮助 但要共同为之。John Gottman说过"最稳固的婚姻里。丈夫与妻子彼此间深度融合。他们不只是一起生活。他们还支持彼此的愿望与抱负。为他们的生活融入共同的目标"。这不是说他们要什么都一起做。我的意思是。他或她可以在工作上有自己的目标。而他们并不一起工作 这完全没问题。但是必须要有他们共同做的事。换言之 我们需要的是主动的爱情。我第一次理解主动爱情的含义。还是通过Shahar给我讲的一个故事。他是我在以色列最好的朋友之一。我们过去是同班同学。不过他很早就结婚了。他二十出头就结了婚 很快有了儿子。他是一个非常善于自省。很诚实很正直的人。他告诉我 刚生儿子Noam时。当他的母亲与妻子带着Noam从医院回到家。Shahar并未感受到对Noam的爱。儿子很可爱 很招人喜欢。但他对儿子没有自己想象中。那么深爱。更别提他对此有多愧疚了。幸运的是这没有持续很久。因为自从Shahar开始照顾儿子。比如为他换尿布。大半夜起来给他喂奶。带他散步 给他擦洗身体。爱就这样培养出来了。平生第一次他感到了。父亲对儿子的爱。这感情不是自然而然便有的。对有些人是天生的 尤其是女性。但不是女生都是这样 总体来说高于男人。我们可以等着这感觉自然发生。也可以主动参与其中。表达我们的爱。还记得自我知觉理论吗?我们总是以自己的行为为基础。形成自我的理论及信条。我们过去曾讲过的例子。如果我想象自己会去约会某人。即便她们说不 我也感到了自信。无论对方答不答应 我的自尊都会上升。若我想象自己能去演戏。我会感到自己是有勇气的。如果我感到自己。如果我照顾自己 我会感到自己。是自尊自爱 关心自己或关心她的人。这对孩子也是一样的。若我想象自己能照顾孩子。根据自我知觉理论 我会得出。"我一定很爱那孩子"的结论。这爱会随着时间而加深。无论是对孩子还是对伴侣。爱情开始时也许是自发感受到的。也许是一见钟情。尽管如此 若我们不主动维持爱情。长此以往 由于自我知觉理论。那份最初的激情会淡去。因为我们没在恋情中有所投入。自我知觉理论会告诉我们。"这恋情对我一定不怎么重要"。爱情慢慢淡去 更不必提。激情被时间冲刷掉的生理原因了。如果没有主动爱情 则恋情无以为继。我们怎样维持爱情?保持主动?恋情老规矩 我们谈过老规矩。我跟你们讲过我和我妻子的老规矩。至少每周两次。与全家共进晚餐。一般每周会有三四次。但我们确有这些老规矩。因为在现代社会 若没有这些老规矩。用Steven Covey的话说是 重要却非紧急。重要却非紧急的活动就会被冷落。即便我们知道出去约会很重要。但不如刚接到的工作来电紧急。所以我不得不放弃约会 留下来加班 然而若有老规矩。不管遇到什么事都坚持老规矩 感情就得以维持。 

第二种维持爱情重要的途径是。做出改变。从想要被认可 到想要被了解。这个建议来自David Schnarch。 

出自他的书"充满激情的婚姻"。就是这个简单的建议。真正地改变了我与人们的关系。包括我与妻子的恋情。我与朋友的友情。与家人的亲情 甚至于师生情。这点在以前提过 再讲深一点。因为这对维持激情至关重要。对于维持友情 维持快乐。维持个人内心的 以及人与人间的快乐。这意味着什么呢? 

什么叫被了解而非被认可呢?首先 David Schnarch及其他心理学家。表明了 如果我们想在蜜月期后 在一见钟情后。仍然维持着激情。我们就得形成深层次的亲密关系。那些接受我们研究调查的伴侣。他们已共同生活了几十年。却依然有着充满激情的恋情。很不幸 这样的伴侣少之又少。但确实存在。他们的一生都在培养 并一直培养。而非花一生时间寻找亲密关系。你们知道怎样培养亲密关系吗?通过逐渐了解对方。逐渐加深了解。通过理解对方。通过像了解自己一样了解对方。这意味着 我们若决心让对方了解自己。我们就必须打开心扉。我们必须吐露心声。这意味着我们得彼此分享。不只是那些在首次约会分享的。精彩美妙的事 当然也包括这些。但是除此以外。还包括一些让我们不怎么自豪的事。也许是我们的缺点。也许是我们耻于提及的。让我们觉得不舒服。不为人知的事情。这些事不会在首次约会上讲出来。这也不该在首次约会讲。很多情况下 甚至不会在五年十年内讲。只有经过长年累月的信任加增。逐渐坦露出来。有时我们坦露的是。自己都不知道的。被我们压抑了很久的事。但这需要时间。这也是要回到David Schnarch关于。脂肪团和性能力的言论上。这通常发生在10年20年30年后。这也是为什么到了五六十岁时。甚至会有比18 20 30岁时。更强的性能力。倒不是说18 20 30岁时不好。而是会比年轻时更好 为什么?因为有了更深层的亲密联系。若我们理解了充满激情的关系的本质。这段关系就有很多东西值得我们期待。这不能仅靠"我们在一起。我找到了梦中情人 我们会幸福一生"。不是这样的 这有时是有风险的。意味着除了坦露我们内心深处的需求。我们最深的幻想。我们最厉害的长处 还有我们的缺点。我们的不安及痛苦。要坦露 要敞开心扉 就要做到这些。这意味着。一段健康的恋情 需要我们表达自己。而非长期抑制自己。这就是被了解与被认可的区别。被了解是要表达自己。被认可则是要取悦别人。这样做是有缺点。敞开心扉表达自己风险更高。因为若她不喜欢这样的我 怎么办?如果他深入了解我后 不喜欢我怎么办?但是要知道当我们表达自己时。我们更可能收获一段美好的恋情。但是没什么可以保证一定就有好恋情。这无法保证的。然而若我们一味抑制自己 失败是肯定的。原因如下。首先 当我们逐渐敞开心扉。也许短时间内没有效果 但是。随着时间的流逝 人们会被我们吸引。特别是我的伴侣会逐渐更喜欢我。因为我们有了深层次的亲密关系。即便有时他们对于刚了解的。关于我们的事不是很喜欢或赞赏。但久而久之 真诚的人。会吸引到其他人。而且 这种方法也适用于。人际关系领域。它适用于友情。适用于提高领导能力。像Peter Drucker阐述的那样。最伟大的领导不是最有魅力的。也不是最迷人的那些人。而是极其诚实正直的人。这样的领导。他的属下会长久追随他。无论是团体间的人际关系。还是个人之间的私交关系。只要我们坦诚 就更可能被容纳接受。第二件应该时刻谨记的是。如果我给人留下了一个很好的印象。使得这个人 或者这群人 都很喜欢我。这奏效了 我成功了。可这是真的吗?他们喜欢的是谁呢?他们喜欢的是我还是那个形象。是我塑造的那个戴着面具的人吗?换言之 他们喜欢的不是真正的我。也许看起来是 但那不是真的。这喜欢是假象 换言之 我失败了。而且这种假象也维持不了太久。所以说坦诚相待更可能持久。而当我们的目标是给人留下一个好印象时。这种情形持久的可能性为零。这并不是要你在第一次约会。就放弃尝试给对方留下深刻印象。我是说我还没见过谁没尝试过。在首次甚至第十次约会 乃至第十次约会。给对方留下深刻印象 这是可以的。然而 共同生活最重要的动因是什么?就像我们前面说的那样 敞开心扉。我们还需要积极努力。去了解我们的伴侣 去了解他们爱喝的红酒。喜欢的花。他们喜欢哪里被咯吱 哪里喜欢被触碰。他们怕什么 渴望什么。什么时候该给他们一些独处的空间。什么时候该和他们沟通。什么时候该触碰他们。所有这些都需要时间。这不可能在首次约会时实现。也不会在第一年里实现。不会在头10年或30年全部实现。这意味着我们要用一生的时间。去了解对方 营造一生的美好恋情。我的祖父祖母。是我在感情方面的榜样。他们在一起50年后 还在继续了解对方。他们的感情很好。他们的感情还没圆满 而是正在圆满。David Schnarch说过。"亲密就是让你自己被真正的了解。即使是你或你伴侣不喜欢的 也要被了解。但这并不代表坦露的都是缺点。有时也会是一直没发现的优点。大部分方法。都专注于让你的伴侣 在你坦露时。认同和接受你。但你不能寄希望于此 如果你心存这样的意图。那么你坦露心声时必然有所掣肘。因你只会说那些让你的伴侣认可的话。打破僵局需要亲密关系。这基于认可你自己"。大多数婚姻咨询师 性治疗师认为。一段恋情中最重要的。一段健康长久的恋情中最重要的是。伴侣间彼此认可。这很重要。我当然想被对方认可。我当然认可我的伴侣。然而这不是健康恋情的基础。而这是很多治疗师。很多夫妻以及我曾犯过的错误。没有认识到被了解才是基础。无论是在一段恋情中。还是与朋友的友情中。当你坦露心声时。你是冒着风险的。而且很多时候。这会引发冲突 破坏感情。然而通常 几率不是特别高。但通常 这只是短暂的破坏。另一方面。若你长期坦诚 敞开胸怀。这会引领你走向长久的感情。对恋情是有积极效果的。一方面来看 你坦诚时。肯定会觉得不舒服。舒服的话也就不难了。也就不算真正的坦诚。短期来看这是很难的。但从长远看 它能助长感情。关键在于你想要什么?要短暂的舒服感觉?你被认可了 你很棒。我来让你看看我有多优秀。因为被认可的感觉很好。还是你想要一段长久而美好的。友情或是恋情?这是个问题。接下来的话我几个月前就讲过。现在我想再重申一次。我是怎样改变与学生的关系的。怎样转变我的教学方式。因为最初教学时 我的主要目标是。不管讲得是否清楚明白 当时确实很清楚。但潜意识里最主要是想被认可。我真的很想让我的学生喜欢我。那是我的主要目标。慢慢的 看了很多人的著作。尤其是David Schnarch的书。还有Parker Palmer的"教学的勇气"。我的目标变为了被了解。这是我现在的咒语了。每天上讲台前。我天天想着并念着我的咒语。其实这就写在我的笔记本第一页。这些是我的笔记 我的小抄。第一行字就是"被了解" 每天都是。被了解而非被认可。这改变了我与学生的关系。以及教学方式 原因有下。首先,老是想着。"我得让给他们觉得我很厉害 无论是对于我的伴侣。还是一千个学生" 压力太大了。这样很难 带给我太多压力了。而若是站在这里说。"我想让我的学生更了解我"就简单多了。这也意味着要了解。那个我非常喜欢和关心的领域。我想要他们的了解 这让我觉得轻松。当我想被Tammy了解 而不是让她有个好印象。让她认可我时 我觉得和她在一起。轻松多了。这是第一个要点。长期想取悦别人的缺点是。若我们带着面具 且不断不断地试着。让自己被认可 看起来很完美。以便让学生喜欢我认可我。但这样做的同时也伤害到了学生。因为若我擅于演戏伪装。扮演一个完美老师的角色。我就变成一个很差劲的榜样。久而久之会伤害到学生。因为那个完美的我并不存在。如果有人想向这个完美的形象看齐。无论是感情中还是在个人问题上。都不可避免地通往完美主义。也是通往失败与不幸。除非我是想被了解。也就是说我的缺点 我的失败之处。不管是在剑桥还是在哈佛的失败。或是我的焦虑 你们会见到真实的我。或是犯错误 你们见到的是真实的我。这样的我是一个更好的老师。相较于那个完美形象 更平易近人。好过那个被认可的我。除此以外 这样的我也更快乐。因为我在这里与你们分享。我在分享自己的一部分内心世界。我与大家分享的。是我非常喜欢和关心的。被了解而非被认可。说起容易做来难。但是利大于弊 尤其是长远来看。好处远远大于弊处。 

第三点是要允许恋情中出现冲突。关于健康恋情的一个误解是。健康恋情没有冲突 没有争吵。五星级的恋情不会发生冲突。这是一个很严重的误解。John Gottman曾对此做过很多研究。他得出的结论之一是。不存在所谓唯一的标准恋情。这也意味着每段恋情看起来都不同。无论外人看来 还是自己看。有一些恋情美好而安宁。有一些恋情则美好却激烈。两种都可持续下去。恋情没有统一的正确形式。恋情可以有无数种模式及不同之处。但健康恋情都有一个共同之处。Gottman发现的一个共同之处。那就是在所有的恋情中。无论是激烈的还是非常平和安宁的。或是两者兼之的恋情 都有争执。而且平均来看 只是平均来看。因为有些人可能是10:1。有些人可能是3:1。但是平均来看。每5次积极互动就会有一次争执。这并不是让你回家去和伴侣说。"我们已经有了7次积极互动了。现在该吵架了"。绝不是这样 但这是他研究发现的。当他研究了成百上千的和睦夫妻后。差不多这是。积极互动与争执冲突的平均比率。这个数据启发到我。因为过去只要Tammy和我发生争吵。我们也会争吵 我就会说"这是怎么了?"。而不是说"这很正常 不用担心。看看该怎么解决这冲突。从中学到些什么 怎样改进"。这并不是说我喜欢冲突。冲突不好 有时很伤感情。但同时要理解到这是正常的。就像我们允许自己。有个人的不足之处一样。我们要允许我们的关系。存在不足 这是很重要的。那些实现了5:1比率的人们。基本上都有一段美好恋情。而那些低于此比率 如100比1的。这样的感情并不能长久地美好下去。那些冲突多于积极互动的伴侣也维持不下去。完全不吵不好 经常吵也不好。冲突的重要性在于它使我们免疫。先来看自然或生理上的免疫。若把一个出生的婴儿放入消过毒的。储氧箱 而不把他抱出来。这个孩子会怎样呢?比如说两三年甚至十年后。孩子长大了 离开洁净的环境。回到真实的世界生活 他会怎样呢?他很有可能会生病。且很可能是重病。为什么?因为在无菌环境里生长。就生理上说是不健康的。因为每次你生病。或是每次你暴露于污染与细菌中。你的身体都会因为产生抗体而变强。这对恋情也是同样的道理。如果你的恋情从不发生冲突。久而久之 恋情无法得到巩固加强。你会变得脆弱 当冲突无可避免时。恋情就会难以维持。所以说冲突使我们免疫 这很重要。再谈一下冲突的种类与形式。因为不是每种冲突都有益。Gottman说过关键不在于。消除负面的东西。而是要强调巩固积极的东西。所以对于那些不成功的恋情。那些经常发生争吵的恋情。像这样的恋情。强调积极方面。而若是处于这种无冲突恋情。很可能你们在压抑 压抑感受 分歧。换言之 我们没被了解 我们需要改进。坦诚 敞开心扉 彼此分享。这里有几种强调积极方面。大家记得那本"更幸福"的书。我讲过的关于幸福助力器。有一些细小的事情。可以是几分钟也可以是几小时的事。能让我们快乐 也很有意义。就像我们借助幸福助力器。滴水穿石地提高我们一辈子的幸福感。我们也可以借助爱情助力器。 

有一个关于爱因斯坦与Ludwig van der Rohe的故事。Ludwig van der Rohe是量子物理学的创始人之一。Ludwig van der Rohe写信给爱因斯坦。说"我想你进入量子物理。这个神奇而美妙的新领域。我需要您的支持"。爱因斯坦当时颇有声望。他回了一封信。信里对Ludwig van der Rohe说。"我对量子物理不感兴趣。因为我对细节上的东西不感兴趣。我感兴趣的是上帝在创造世界时。他在想着什么。我对细节没兴趣。我只关心上帝创造世界时的。他在想着什么"。Ludwig van der Rohe回信说。"上帝就在细节中"。爱情也是这样 爱情就在细节中。爱情不在一周或一个月的。环游世界中。爱情不在五克拉的钻戒里。这些东西是很美妙。也确实激发了幸福感。有时甚至激发了爱情。但它们不能维持一段美好的感情。能够长期维持幸福恋情的是细节。是那些微小的事情。是每天的生活常规 触碰 凝视 共进晚餐。迷你爱情助力器也能维持。这是我从Peter Fraenkel学来的。他是纽约Ackerman家庭服务中心的。他讲的是一天里的。30秒钟的幸福点。无论是一个激吻。或只是一个单纯的拥抱。或是发给伴侣的一条短信。告诉他们你有多爱他们多想他们这类小事。他们发现。正是这些小事产生重大影响。当然 大事也能产生影响。我不是贬低大事 但那只能导致激变。而维持幸福需要长久的努力。需要奉献。感情的维系主要是靠。注重这些日常琐碎的小事。我们怎么做到呢?我们表现出我们的关心。我们问"你今天过得怎样?"。或"讲讲你刚做了什么"。或"你在想什么"。或"你不太高兴啊 我能做什么"。表现出关心 画爱情地图。去了解对方 他喜欢什么。不喜欢什么 有多喜欢。就是那些小事 比如触碰。微笑 送花。记得重要的日子。现在这些简单多了。我们有电脑提醒我们。就我个人经验来讲。称赞 称赞对方。不要以为一切都是理所应当的。记得我们讲过感激 两个意思。其中之一就是称赞某人 "你看来棒极了"。这太好了 谢谢你做这些。谢谢你想到我。带有感激的赞赏是最好的感激。当我们不懂得感激与赞赏时。美好会贬值 马克吐温说过。"一个好的赞赏能让我高兴两个月"。不要只理解了他的字面意思。 

给你们讲个故事。这是我的岳父与岳母的故事。我的岳父毕业于哈佛法学院。很聪明很有才华 他叫Amie。娶了一位可爱的女人Rachel Rahel(希伯来语发音)。有一次他们约会。Rachel穿了件很美的裙子。他们去参加一个派对 整晚都在那里。派对上人很多 那是一个美妙的夜晚。派对结束后他们回到家。Rachel问Amie。"Amie 今晚整个晚上。我得到了无数赞美 说我很漂亮。说我裙子很美 很容光焕发。大家都赞美我 却惟独你没有"。要知道Amie是哈佛法学毕业生。非常聪明 反应灵敏。他是以色列的一流律师 他说道。"Rahel 你还记得几个月前。我告诉你 你有多美丽动人吗?"。她说"我当然记得"。他说"直到另行通知前 你都那么美丽动人"。这句话几乎能过关了。问题是Rachel也是个律师。同样聪明同样成功。她对她说。"Amie 直到另行通知前 你都睡沙发"。所以不要等别人开口 你才赞美他们。主动赞美 又不花钱。然而虽不花钱。但它的价值却是不可估量的。因为它令我们获益甚多。赞美与被赞美的人都获益。最能预测一段治疗关系是否成功。最能预测一个治疗师。心理学家 社工是否成功的 是移情。你有多理解多同情。多认同你的客户。在恋情中也一样。我真的在聆听我的伴侣吗?我真的在看我的伴侣吗?我真的对我伴侣做的事。表现出真诚的关心吗?他们好吗?我真的想要了解他们吗?因为如果不是这样。要维持长久美好的恋情是很难的。最后 性在长久美好恋情中很重要。性不一定要在婚前发生。或者等到彼此承诺厮守一生才发生性关系。这厮守可以是结婚也可以不是。可以是婚姻 在现实生活中也经常是婚姻。是不是结婚都行。然而 对于长期恋情的成功延续。生理方面也很重要。极少的情况下 也存在。也存在极少的恋情。是没有生理方面的因素的。生理因素是区分爱与深厚友情的。一个显著特征。 

爱情 准确地说 性的至高点使爱具体化。使爱具体化。这也是为什么我们会说做爱。它是感情 是抽象的爱的具体化。生理方面很重要的一点是。健康的交流 就像别的感情一样。因为我们经常坦露自己 分享自己。在卧室远多过在其他地方。无论是肉体上 还是情感上 还是认知上。那是我们敞开心怀的地方。那是我们真正赤裸相对的地方。交流是很重要的。因为人与人的期望值是不同的。这也没什么统一的规定。一周5次性生活 最好的恋情。三次 不错 一次 合格。不 有很多美妙的恋情。一天一次或一个月二次 对不起 或者一星期一次。有一些美妙的恋情。或是一天三次。不 对此没有统一的规定。不知道我刚才说的哪句好笑。不过我课后会看一下的。因为我相信佛洛依德现在一定也在看着。最重要的是交流。因为它是冲突的根源。是冲突的潜在根源。并不必一定如此。来看一段短视频。注意力从我这里转移开一下。这是伍迪艾伦的影片。\[电影:安妮霍尔(1977)-屏幕上字幕\]。我记得对上一次开心的时候。是在布鲁克林的那天。我们再也不敞怀大笑了。我一直情绪低落 欲求不满了。你们多久做爱一次?性生活频繁吗?很少 一周三次吧。经常做 一周三次吧。前几天晚上 Alvy想做爱。前几天晚上她不想和我睡 你知道吗?我不知道。如果是六个月前 为让他开心我会做。我什么都试过了。放轻音乐 浅红色的灯光。但是自从我们来这里讨论后。我觉得我有权顾及自己的感受。我觉得你听到这个会高兴。因为我真的能坚持自我了。最讽刺的是。我出钱让她看心理医生 她有改善了。我却更糟糕了。我不知道。我觉得好内疚 因为Alvy出钱让我来看。所以不和他上床 让我内疚。但要是真的和他上床。感觉像是我不顾及自己的感受。我没有进展 我赢不了。有时我觉得我该找个女人住。 

"很少 一周三次"。"经常做 一周三次" 不同看法。沟通很重要 

不是所有的冲突都是好的。冲突有积极的有消极的。对这个问题的有用研究。其实主要来自组织行为。确切的说 组织行为研究表明。健康的组织内或组织外冲突。是认知冲突 而非情感冲突。认知冲突针对人的行为。或思想 观念 对这些提出质疑。不健康的冲突 注重于人 情感。他们自身。而当我攻击某人自身。那个人本身及情感时 这是不健康的。无论是组织 还是夫妻。与此同时。当针对观念 想法 行为时。我可以不赞同 可以发生争执。举例来说。这是一个针对人自身的例子。你不体谅别人。这不是针对行为的 而是针对人的。而针对行为的是。你用完马桶后把坐垫放下来好吗?这两句话可能是由同一件事引发的。就是你没把马桶坐垫放下来。一个反应是 "你不体谅别人"。这就是针对人和针对行为的例子。另一个例子 "你这个懒鬼。答应了要扔垃圾的。我不能信任你了" 这是针对个人了。我觉得是这样的。与之相对的是"你说过要扔垃圾却没有做到。让我回家就看到乱糟糟 很恼火"。一样的事。不同的态度。这两种评价带来的后果也不同。除了质疑行为而非个人以外…抱歉。尽量避免恶意 侮辱及蔑视。John Gottman在其研究中多次提及。他说自己预言离婚有94%的准确度。就像我上次提到的那样。在社会科学中从没有过这么高的准确度。他判断的根据是。是否在争吵和冲突中看到…。他让夫妻描述他们的冲突。通常他们在他的观察下开始争执。当出现恶意及蔑视。通常是坏的预示 但不总是导致离婚。绝对不是 但确实是坏的预示。这不仅是在感情能否长期维持的坏预示。这还是在享受。现有恋情方面的坏预示。所以有必要避免这种侮辱的发生。避免针对人身 认可本人。尽量赞赏对方 仅对其行为。或是其想法观念不苟同。还有很关键的一点是 私下才争吵 意思是说。本来恋情中的争吵就很不好。若是再引来尴尬羞愧。就更糟糕了。所以夫妻在有外人时 吼叫对方。或者有朋友在场时 轻视对方。这是非常具有破坏力的 极其危险的。所以让争吵私下化。能潜在的让私密关系好起来。 

最后 我们能从同性恋情中学到很多。像我上次提到的那样。John Gottman研究了很多同性恋情侣。他发现了一件很有趣的事情。他们争吵起来很独特。有趣的地方在于。若是两个女人争吵 那没什么新奇的。有意思的是 这也发生在男同性恋间。因为女人通常比男人擅于争执。也确实有生理上的原因。这是有生理上的原因。举例来说 男人受到言论攻击时。或觉得别人在攻击 威胁 或不赞同他们时。他们其实会有更强烈的生理反应。这也是为什么男人通常吵着吵着就不吵了。这不仅是因为他们是男人。这是因为有生理上的差异。而女人 举例来说。被批评时 虽不会喜欢。但相对于男人…这是普遍来说。并不是指所有男女。普遍来说 男人对此更紧张。他们会想要保护自己。而方法之一就是不跟你吵了。逃避 而��是应对。而女人对争吵更适应。��至是生理上的更适应。我��是说心理所反映��生理感受。而是说单纯的生理��面上。女人更善于处理应对。这种争执的弊端及不适。所以首先要知道的是。无论你是否是男人 你要明白你不该逃避。我知道我不舒服 这是天生的 要接受。接着你便能处理 而非避免。而女人 要明白。她们的伴侣与自己的感触不同。但这里有意思的是 在同性恋情中。男女同性恋 都能更好应对分歧。他们怎么做呢? 第一。更积极 更幽默 更多爱意与触碰。无论是男的还是女的同性恋者。这对解决分歧很有帮助。能够缓解紧张的气氛。男女同性恋者都是。人身攻击的情况较少。所以即使有了严重分歧。我理解这是感情成长的一部分。记住这点很重要。这一刻我们的确不喜欢。这是严重的争执。但我能学到什么?我们要学什么?我们怎样成长?因为我明白 冲突是正常 不可避免的。这是我们恋情中的一部分。同性恋情侣都能做到。异性恋情侣没理由做不到。只要这么想想就行了。最后 要多关心对方。冷静下来 理解对方的感受。无论男女。要设身处地理解对方 让对方平静下来。通常 就在争执中。说一些"我很抱歉" 或是一个拥抱。或"我很受伤也很爱你"。这些方法都能让对方平静下来。减缓冲突带来的压力。因为冲突不是好事。对同性恋情侣当然也不易。只是他们能应对得好一些。并不只是同性情侣才懂得处理冲突。 

我们讲过 黄金法则。己所不欲 勿施于人。我们讲过白金法则。人所不欲 勿施于自。这是我们讲完美主义 和允许自己有不足时讲过的。然后是钛规则 你不会对泛泛之交做的事。也千万不要对你亲密的人做。有多大的可能你会对。陌生人或刚认识的人大吼?鄙视他们 侮辱他们?挑衅别人?你对陌生人做过几次这种事?他们一定是对你做了极糟糕的事。才让你表示出如此恶意轻视。对他们大发雷霆。然而平均下来 我们常常。虽然不是人人都经常这样。但是确有很多人这样对亲近的人。无论是家人还是亲密的朋友。还是爱人。没有理由这样做。没错 是因为我们与他们更亲近。因为和他们一起更舒适 但是为什么呢?为什么对外人。比对我们最爱的人反而要好?为什么对所爱之人。比对不相干的人要差?这无理可循 可以有争执。会有争执 且争执很重要。问题是你怎样将其保持在认知行为上。而非情感的 感情的 蔑视的层面?这会导致极大的不同。 

我们讲了这么多内容。在一段恋情中 最重要的。是培养一份深厚的友情。John Gottman说"我的方法的核心是一个简单的真理。即幸福的婚姻基于深厚的友情之上。这句话意味尊重对方。享受与对方的相处。这些情侣能深入了解对方。他们熟知对方的喜恶。性格怪癖 理想与梦想。他们对彼此有着持久的尊敬。不仅在大的问题上表达喜爱之情。也在日常琐碎间表达出来"。 

爱情就在于细节。在于事无巨细的了解。在于分享与被了解。这就是健康的恋情。这就是你如何让恋情维持激情数年。这就是你如何持续享受激情。性关系 虽然在生理方面。一开始会很自然地倾向于新鲜体验。"新鲜的才是激情的"。但不一定就要这样。当然可以是这样 这很自然。可以用双方的幻想来体验一下新鲜。但是对于长久的恋情。充满激情的恋情 最根本的是要被了解而非被认可。 

最后 关于健康恋情的第四点。我现在先提一点。然后在下周再讲完 积极认知。在一段恋情中 一段健康的恋情中。伴侣一定要做优点感知者。他们要赞赏对方。要记住若不懂赞赏对方 美好会贬值。如果我们不感激这段恋情中顺利的地方。过了蜜月期后。恋情会急转直下。不幸的是 这种情形常发生。因为我们开始把好的东西视为理所当然。而当我们这么做时 恋情就会消逝。若我们将自己的潜力看做理所应当。那潜力会衰减乃至消亡。所以说优点感知 对一段恋情很重要。欣赏伴侣的自律。而不是把自律看成固执。欣赏伴侣的幽默感。而不是把幽默看成耍滑头。把注意力放在优点上 放在顺利的地方上。这些优点就会加强。然而成功的幸福伴侣做得更好。他们不仅是优点感知者。他们还有Sandra Murray所说的积极错觉。换言之。他们认为伴侣要比别人眼中还要好。Sandra Murray做的实验如下。她让情侣们评价另一半的优点。然后她再去叫那些。与他们相熟的人们 评价他们的优点。她发现那些准确地评价对方的情侣。也就是与朋友家人的评价差不多的。他们的感情还算说得过去。他们在一起过得凑和。那些对伴侣优点评价得比别人还差的。也就是他们对彼此的评价不如。外人所认为的那么优秀 强大等。不管是来自朋友还是家人的评价。这些恋情一般持续不了很久。而那些对彼此的评价。要高于其朋友 或家人。给予的评价的情侣。这些情侣的恋情最可能。持续且美好下去。这部分人 Sandra Murray将其称为。因为他们对彼此的评价与外界不符。Sandra Murray将其称为"积极错觉"。或按Brad Little的说法是。他们有着"虚幻的光辉"。认为彼此远比实际上要好。我部分同意Sandra Murray的观点。我同意她说的 对伴侣有积极的评价。是很重要的。但我不同意她的是。对于错觉这一词汇的运用。因为我认为这不是错觉。事实上 这是真实的。因为这成为一个自我实现的预言。在这里我引进一个新的术语。除了优点感知 还有优点创造。看到不存在的优点 或是。别人没看到的优点并让其成为现实。想想教育界里的一个例子。Marva Collins脱离实际了吗?Marva Collins对她的学生有积极错觉了吗?完全不是。我认为 是其他老师脱离切实了。因为他们没有看到学生身上的潜力。Marva Collins非常切合实际。她关注于积极方面 并创造优点。Abraham Maslow说过"爱不仅能感知潜力。而且还让其转换为现实"。 

给你们讲个Tammy的小故事。如我所说 Tammy是个完美的优点感知者。事实上 她是个优点创造者。这事发生在Tammy怀孕8个月的时候。胎儿很大了 你们都见过David了 大宝宝。虎父无犬子。所以在怀孕8个月时。Tammy和我有过一段交谈。你们谈过很久恋爱的人就会明白这谈话是怎样的了。她说"Tal 他变得越来越沉了。这真的对我很难 我需要你更多帮助。你需要在家帮我一点小忙"。我说"好的"。我感到很窘迫 觉得自己很不体贴。我说"抱歉 我能做什么?"。她说"你可以今天下午去帮我买东西。我再也拿不动那些大包小包了"。于是我去了超市 把冰箱塞满了。不仅是买回家。还把它们放进冰箱。因为我觉得很内疚。Tammy看到后说。"太感谢你做的这一切了"。我对Tammy说。"不 我很抱歉要你开口我才做。我早该做了"。她说"不 我就是爱你这一点。你如此体贴 肯仔细聆听我"。于是我又把盘子洗了。 

创造优点 而非单纯感知优点。而不是专注于不顺利的地方。周二就不来给你们上课了。我有一整周来不了 过逾越节。但你们周二还要来上课。Shaun带来的关于幽默的精彩演讲。下节课见。 

第20课-幸福与幽默 

早上好 很高兴能给大家讲课。Tal走了 所以今天早上我来代课。给大家上"黑魔法防御术"。我很肯定 在这种情况下。我可以套用美剧"办公室"里Dwight说的一句话。我是幸福课的助理教授。而不只是幸福课教授的助理。很多人会叫我宿舍长TF Shawn 或者Shawn。或者Kirkland的Imzar 或者幸福先生。也有人叫我 那个像蹩脚罗马战神的家伙。但你们很多人都不知道。我也有温和的一面。 

小时候。我给我妹妹的手臂弄了三处骨折。我们在玩打仗 我们在客厅里到处跑。像士兵一样 从坐垫上跳过去。打着打着 就像我后来…。写给我父母的检讨说的那样。我看到眼镜蛇指挥官乱枪扫射我妹妹。于是我不顾自己的安危。我转过身 向我妹妹直冲过去。推着她躲开那些射向我们的隐形子弹。一直把她推到砖墙边。这时我爸爸走进客厅。我站在她身边 说"我是英雄 我救了她"。我真的成英雄了吗?没有 这样算不上英雄。你知道真正的英雄是谁?真正的英雄是那些每天醒过来。穿上平凡的衣服 上平凡的班。接到市长的求助电话。穿上制服去抗击罪恶。这些才是真正的英雄。你知道谁是英雄吗? 电视剧"英雄"里的Hiro。你知道这句话从哪里来吗?"办公室" 很好。我们有一个同学看喜剧 很好 谢谢。 

我之所以跟你们说这个故事 是因为那天下午。我妹妹和我 在我们去了急诊室回来后。我们在双层床上铺玩 我们又玩打仗了。我们还没学到教训。这一仗是我的特种部队 对抗她的小马。突然在这场地缘政治冲突中。我妹妹玩得很兴奋 往床外靠得太出了。结果。她突然从上铺上消失了。我往双层床外一看 "怎么今天这么倒霉"。我看到我妹妹躺在地板上…。她着地时 用手和膝盖痛苦地撑在地板上。我看到她在痛苦 折磨。和不公中 正要大声哭出来。我父母已经交代过我。要和我妹妹玩得尽量安全 安静。因为我刚刚弄断了她的手臂。他们会怪谁。如果那天再出事的是我妹妹。而不是他们最爱的那个孩子。那个他们希望长大后能当捉鬼敢死队员的孩子。那个弄断她手臂的孩子?当然是我。所以我说了。我那个抓狂的7年级学生脑袋能想到的唯一一句话。我说"Amy 先别哭 你看到你刚才怎么着地的吗?没有人类能那样着地 你是独角兽"。当然这是骗她的。因为我知道在这个世界上。我妹妹最想要的就是让全世界的人都明白。她其实是一头独角兽。你可以从我妹妹的脸上看到她内心的挣扎。她的大脑想把精力用来关注她刚刚。带着骨折的手臂从双层床上掉下去而受到的。痛苦 折磨 和不公 但在另一方面。她的大脑又想把精力用来看看这个世界。关注她的新身份:一头独角兽。后者胜出了 我妹妹脸上露出一个微笑。她大笑起来 爬回双层床上。带着一头小独角兽的优雅。 

我们从中发现。5到7岁这个脆弱的年龄。带有一些特征。这些特征将会成为幸福心理学革命的核心。这个我们已经讲过了。我们讲过我们的大脑就像一个单一处理器。能够有意识地选择把精力放在。痛苦和折磨上 或者把精力用来。用乐观。和正念的眼光 来看这个世界。所以我觉得幽默也正是这样。我认为幽默本身就是一种选择。有意地去选择 怎样来看待世界。是用痛苦 乐观…。是痛苦 不是乐观 用痛苦 折磨 不公的眼光来看这个世界。还是用一种。适应性的强有力的眼光来看。 

这个其实是我今天新买的。因为旧的那个用不了。因为我没有打开电源。今天我把幽默定义为一种正念的眼光。通过幽默 我们可以用。正念的眼光看待世界。就像乐观主义那样。这时一个人会提高。对一种情况的各种可能的意识。今天我们会讲。对幸福的很多种不同的定义。以及我们看待幸福的不同角度。及其对我们的影响。在讲之前 我想说说。我们为什么要讲这些。 

首先 Tal知道他今天不能来上课。他三月时告诉过我很多次 他没办法。星期二下午4点20分以后上课。我想他的意思是因为逾越节的关系。所以我同意上一节关于幽默的课 因为我觉得。它能把这门课布置的各种阅读资料。以及课程大纲 很好地衔接起来。如果你不知道这门课有布置阅读。和课程大纲的话。下课后来找我。所以我很肯定。Tal这周和上周末都不会来学校。因为他很想去看"武当派"乐队。Gavin Degraw终于让它团聚了。你们看到昨天Crimson的文章吗。讲武当派的一个成员。一个不知名的武当派成员 在音乐会上。拿出他的麦克风 让观众们。让他们帮他唱完歌词。结果全场鸦雀无声。他见状说。"看来你们都没有自己以为的那么聪明"。当一个不知名的武当派成员。嘲笑你时 你就知道你跌到谷底了。 

我今天很高兴来上课 这是有几个原因的。首先 我认为幽默。是我们将要讲到的幸福心理学。最重要的一部分 因为。它在我们说的每件事上都占了重要地位。如果大家了解我的话 大家应该知道我本科在哈佛读。我学英语和宗教 我当时上神学院。学习基督教和佛教伦理。现在我转学心理学。还在Kirkland宿舍外23号滩。开了一间幸福心理咨询公司。我还是Kirkland宿舍的校内指导员 传奇人物。和国家英雄。我是重要人物。我有很多皮订书。我的房间一股红木家具的味道 很像有钱人。看起来很像一间宿舍。我跟你们说这些 是因为我找不到话说了。这个宗教 心理 英语。和哲学的综合知识将会在今节课。发挥重要作用。 

我简单介绍一下今天会讲的内容。我们会讲幽默心理。我们会讲幽默。对心理和生理的益处。我们会讲幽默的积极社会影响。你们自己培养幽默感的实用方法。以及幽默最强大的功效。我认为那就是幽默的心理治疗价值。通过改变我们的认知心态 用一种完全不同的。方法看待世界的能力。很多同学都知道。大部分期末试卷考题都是我出的。而且我评卷看得很慢。这两个因素结合起来 意味着。期末考试大部分内容可能会出自这堂课。所以我一定会指出这节课。有哪些内容是你们一定要写下来的。而且今早的课一定会有大量的笔记。我告诉大家…我知道你们坐在那里。看着我的幻灯片想。"我前三个孩子都会取命为Shawn。不管是男孩还是女孩。不知道是什么。驱使Shawn学心理学的"。我会告诉大家 好让你们能专心听课。也可以作为今天的一个案例。我爱上心理学。是当我爱上照顾我的保姆时。不止一个保姆。我的保姆都是我爸爸心理班的学生。他在贝尔大学教心理学。所以我的保姆都是那个班上的。我爱上了她们全部。有一天我发现。这种约会没有我希望的那么顺利。因为只有我父母付钱时 我们才能约会。所以我决定自己来解决这个问题。我决定我能和这些女孩。约会的唯一方法。就是进入她们的世界。这招我是从"海的女儿"里的小美人鱼学来的。所以我决定当我爸爸班上的一个心理实验的。自愿者。他很高兴我会去他的班。所以当他们班上。准备了脑波机那天 他请我去他班上。他们会把人连接上那台机器。我觉得这个好方法能让我给那些女孩。留下深刻印象。因为没什么能比头上涂了电极润滑油。还连着各种各样的电线更吸引人了。那天和今天一样热。总之我去到班上 我在笑着。因为她们都朝我笑。我觉得这个约会进展得还不错。突然 我爸爸因为太激动了。他忘了做一件很简单的事。那是很久以前的事了 那台机器很原始。他忘了给脑波机接地线。所以我马上就遭到…。在我们的约会上 我被电击了。我当时这样叫"把它关掉"。大家都大笑起来。我和这些女人的未来恋情也都泡汤了。我爸爸笑到流眼泪。他笑都连帮都帮不了我。他能关掉机器。但他没有。他可能觉得这样很好笑。也许残酷而不同寻常的惩罚是很好笑。爸爸 不如下次用水刑吧 谢谢你。我很喜欢再参加你的实验。总之 我很尴尬 很生气。我一个一个都拔掉所有电极。当时大家都在笑我。我从教室边上昂首挺胸地走出。当我走到门前时 幸好。我转过身说。"爸爸 谢谢你毁了我和这些女朋友发展的机会"。然后摔门而去。这是真的 那些女孩我一个都没追到。 

刚才那是怎么回事?那几个听到这个故事大笑的同学。这是什么…真奇怪。这个教室的音效设备真好。你总是能听到笑声。太厉害了 就像这个教室能吞掉我的笑声一样。刚才那是怎么回事?我想很有趣的一件事是。你们之所以听到这个故事会笑。因为我说的不是一个悲剧。而是一个充满爱的虐童故事。我们之所以笑 是因为这一刻。我们都做好了笑的准备。或者说 我们周围都是人。我们笑 很可能是因为我们有一大班人。群体对我们有社会影响 对吧?如果你想隐藏感情 请尽管。拉住坐在你旁边的同学的手。但有趣的是 当你笑时。笑声很怪异。维基百科把幽默定义为。"一种有节奏的发声的不由自主呼吸动作"。这也是我对我的舞姿的定义。但当你笑时 就像现在。你身体有15块面部肌肉 在疯狂地运动。包括你的大颧肌。它能同时拉起你的嘴唇。非常酷 我不知道它的原理是什么。会厌软骨开始盖住你的喉。这样就收缩了你的呼吸。所以基本上你是在窒息。这就是为什么笑的时候很难呼吸。别怪我引你们笑。血压降低 同时。血管里血液的流量增大。腹部肌肉伸缩。呼吸 脸部 腿和后背的肌肉都会伸缩。这是我做过最强烈的健身运动。这一切发生得太神奇了。很显然。幽默和笑声会对我们产生情感的影响。很明显它有社会影响。如果你现在心情好。你可能会笑得更多。如果你现在心情不好。你可能听到我的笑话 会忍不住大笑。所以这当中的几个作用因素 是社会环境。教室 还有笑话本身。我的身体因此而表现出变化。 

问题是 你能研究幽默吗。就像我们问的那个问题"你能研究幸福吗?"一样。研究幽默有什么用?就像Tal讲过的那样。现在的学术期刊上。消极研究与积极研究的比例是。17比1 或者21比1。昨晚我写这节课的讲义时。我在eResources上查东西。我发现消极研究与幽默研究的比例是97比3。例如 研究抑郁症的有125000篇文章。而研究幽默的只有4943篇。搜索Steven Colbert时 结果为零。这其中有一个重要原因。因为一种叫医学院症候群的现象。我们知道。我们的学习对我们的身体是绝对有影响的。所以如果你知道医学院症候群。你就会知道。当医生对世界上各种疾病的知识。越来越丰富时。他们会突然发现 自己好像什么病都患上了。我妹夫从医学院打电话给我。他说"Shawn" 我妹夫名字叫Bobo。这是题外话了。Bobo从医学院打电话给我。他说"Shawn 我患麻风了"。我不知道怎么安慰他。因为他刚刚才过完更年期。 

我们研究世界的视角 真的能改变。我们看待世界的视角。如果我们不研究幽默。我们就不会了解社会互动的。一个重大方面。也无法了解世界是怎么运行的。那为什么我们对幽默的研究这么少?首先 幽默是非常难定义的。每次你想定义幽默。它就从你的指间溜走。每次你想…大家觉得好笑的东西都不一样。例如我今天想放的视频。你们可能会觉得一点都不好笑。你们觉得好笑的地方 客观来说一点都不好笑。我想让我父母看"太坏了"。因为我觉得那部电影很好笑。但就算他们心里觉得好笑。他们也不会笑出来。一部分是…。有些人听到严肃的事会笑。或者说本应该很严肃的事 例如"绯闻女孩"。Kirkland宿舍的人。kirkland宿舍的女同学 要用枕头打她们。威胁把她们登在广告栏上 才能让她们笑出来。所以我为什么说…研究幽默的难点之一就是定义。我们很难定义幽默是什么。因为当你一定义它。它就从指间溜走了。当你想向别人解释。向你交友网eHarmony上的对象解释。为什么你比她们想象中的更幽默时。你就知道。不管怎样看 你都走错方向了。我们从中发现。一旦我们试图去定义某些东西。一旦我们试图定义是什么让一些事情变得幽默时。这些事情马上就不幽默了。这是一个死青蛙问题。死青蛙问题就是说。你越是解剖一只青蛙 它死得就越快。笑话也如此。因此。这使得幽默很难研究。即使我们能研究它。假设我们能研究它。我们知道它对我们的身体和社交有益处。但如果有些人天生就是幽默的。有些人天生就是不幽默的。就算研究它也帮不了我们什么。如果你天生就幽默。我们认为确实有些人是天生幽默的。如果你认为幽默是天生的。这样就知道它有什么益处 对我们也没什么帮助。因为我们都得不到这些益处。这跟幸福心理是一样的道理 对吧?在研究积极 幸福。乐观的人是 也有同样的问题 他们太莫名其妙了。但我们可以看到。我们研究幸福心理的一个担心是。如果幸福 积极 和乐观都是由基因决定的。那么研究我们在这个班上所讲的内容。还有什么意义?幽默也是如此。如果我们得不到这些益处。只会给幽默的人更多理由来嘲笑我们。我想我们今天能克服所有这些障碍。我知道幸福心理领域已经出版了。一些非常棒的研究。还有专门研究幽默的研究。这些研究能让我们。用全新的角度来研究幽默这个人类现象。并且给我们带来各种各样的益处。所以我想告诉大家 我们有…。为了让我们能够像研究幸福心理一样研究幽默。我们要看看。心理学的人文传统。所以我们要立足于。幽默的哲学。我们会讲到研究幽默的三大巨人。我们还会讲两个理论。我觉得这两个理论 没有什么高低之分。真正高明的是我的解释。我稍后就会讲给大家听。 

这三大巨人。研究幽默这个心理学领域的三大巨人。就是弗洛伊德 伯格森 和我。我们先来讲弗洛伊德。弗洛伊德他在很多课题上 都作出过无懈可击的。符合逻辑的假设 而且这些假设都将经受得起时间的考验。他写了一本书叫。《笑话及其与潜意识的关系》。《笑话及其与潜意识的关系》。如果你还没有看过 对于写一本关于幽默的书的人来说。这本书是自《利末记》以来最不好笑的笑话集。在这本书里 他描述了一种典范意识。在我们内心有一个本我 这个本我包括了大家都有的。性欲 生命力 和冲动。我没有 因为我读过神学院。但你们都有。本我之外 是自我 这个自我会把本我视为。本我冲动和社会之间一道可渗透的屏障。最外面的是超我。超我是加在我们身上的。道德约束和社会约束。例如 如果我的本我冲动。说我很生气 我想揍你。我的超我 那个道德约束就会阻止我打你。同样 如果我想和某个人上床。我的超我说这样不行。超我。这个社会道德约束会阻止我和别人上床。这可能是唯一能阻止我上床的东西了。我的超我。我们发现。弗洛伊德说的是 我们都会受挫。我们都会受挫 因为我们都有本我冲动。但我们没办法把这些冲动释放出来。所以我们就会被压抑。你越压抑。最终就会爆发。所以压抑会带来严重后果。他认为 幽默是一种社会可接受的。用来疏导这些欲望的信封。这是一种社会可接受的。把本我冲动释放出来的方法 这就解释了。为什么我们很多幽默 本质上都是与性和攻击有关的。我们可以举很多例子来说明。把幽默视为一个心理保险库。这就解释了。为什么你们会在首次约会时说一些隐晦的黄段子 对吧?在我们的体格特征得到社会接受前。这也解释了"办公室"里Michael Scott说的那个。"她是这样说的"笑话。他不管什么事都说"她是这样说的"。他有些笑话很隐晦。很难领会 但我每次听到都笑而不语 心满意足。这也解释了我们为什么喜欢拿政治人物来开玩笑。很多时候 政治人物不能…。因为我们无法通过性行为。来发泄这种挫折感。所以我们就会通过幽默来表达对政治的不满。通过语言上的幽默。我们能够允许冲动释放出来。因为这些话。不会破坏我们大部分的社会结构。我们等一下会再讲到这一点。弗洛伊德理论的缺陷之一。正常来说 言语伤不了人。它是社会可接受的。Steven Colbert说六块石头就能砸碎我的骨头。但言语永远伤不了我。除非你拿本字典砸我。写下这句 这句很重要 写下来。 

亨利·伯格森是幽默研究领域的第二大巨人。亨利·伯格森 我肯定没念错。因为我用法语口音再加上我的德州口间念的。这个名字的发音…念法语可以省念很多音。真的话? 这句你们都觉得好笑? 好的。他认为幽默是一种社会纠正器 他认为…。就像以前上过这门课的Abraham Maslow说过。人是有一个发展轨迹的。终点是自我实现 或者说实现我们的潜力。伯格森认为 幽默就是轨迹上的一个点。在一个点上 当我们偏离这条轨迹时。当我们做一些阻碍我们自我实现的事时。幽默就会马上。纠正我们。例如 喜剧演员Lewis Black这样说那些。参加哈利波特魁地奇夏令营。骑着个扫把 玩飞盘的孩子。我念不对 跟玩伴玩飞盘。他其实是拐了个弯来骂这些孩子。他说 这群怪小孩。现在就该开始为毕业舞会找不到舞伴发愁了。他的意思是 当我们做了一些不利于适应的事时。例如去参加哈利波特魁地奇夏令营。或者花14美元去Om喝杯酒。或者打心底里喜欢经济计量学或者地主理论。或者在香港俱乐部和餐厅订餐。在这些情况下。我们笑这些人 是为了纠正他们。阻止他们偏离这条发展轨迹。这样他们就能继续迈向自我实现的目标。如果有人打电话来 如果我女朋友打来…。我还希望大家能给点掌声。我就知道大家都不觉得奇怪我有女朋友 至少没人笑。如果一个女朋友打给电话给她男朋友。留了一段不雅的留言。意外地这段留言。留到她男朋友父亲的电话上了。我们都会大笑 为了纠正这样的错误。另一个我在Kirkland宿舍听说的笑话是 有一个人。在万圣节那天穿了一件。用Cosmo杂志的文章和胶带做成的裙子。在大冷天走出去。滑了一跤 屁股掉到冰面上 裙子撕破了。这个故事有太多值得我们笑的地方。但我们会笑的原因之一是。我们有时都会偏离发展轨迹。我记得柏拉图说过。杂志裙子不能帮我们更快地达到自我实现。 

我们已经讲了两个内容了。我们讲了幽默是一种纠正器。我们讲了幽默是一种心理释放。我个人觉得在研究幽默作为社会纠正器。这个领域中。有一个非常聪明的人 他就是三巨人之一。这是我自己评的 他就是Dale Sturtevant。Dale Sturtevant是一个。你们绝对需要在笔记本里记下来的人。很幸运 多亏有了神奇的Youtube。你们能看到Dale Sturtevant是怎么研究心理学的。我是Dale Sturtevant。我六岁起就养狗。没什么能比训练小狗给我带来。更多的快乐和挫败感。养一只小狗。纠正它不正确的行为尤其让人恼火。像你一样。我试过所有的办法 骂到声音嘶哑。不给东西它们吃 把它们锁在柜子里连续几天。或者毫不留情地打它们。但在我第三次被气得心脏病发。和上了一个法庭强制的控制愤怒心理治疗后。我学会了把我的愤怒。转化为一个有效非暴力的训练手段。这个手段叫"鄙视你的狗。如何用讽刺和口头羞辱来训练你的小狗"。你们看。狗要比我们想象中的更聪明。它们知道自己什么时候被人嘲笑。当它们不听话时。一句到位的讽剌 或者一句伤人的话。就能达到一根打狗棒达不到的效果。不管你的小狗有什么行为问题。我保证可以帮你纠正。例如 跳到沙发上。不 Humphrey 别起来。你应该躺在上面休息 毕竟。你在一份要求高压力大的工作中忙了一整天。不 慢着 那应该是我才对。我想起来了 我才是要工作的那个人。你只是一条整天在家里躺着。流了满地口水的狗。在家里大小便。"Walter 多亏你"帮忙"装饰了这张米白色新沙发。大家都说你拉在上面的那堆东西。给了它一个完美的"点缀"。做得好 对了 Milton Berle打过电话来。他想要回他的膀胱控制器 这就对了 真乖"。挑食。"对 Margaret 你想吃的是顶级排骨。事情是这样的:Palm餐厅不接受订位。我没试过打去Morton's餐馆订位。因为我知道他们换了新大厨。所以现在 你就吃Alpo狗粮吧 好吗?我知道这不是你的第一选择 但别忘了。你只是一条狗!"。 

这个现代卓别林让我笑得流眼泪。为什么我们看这段视频?我们为什么要学习这两个理论?它们对我们有什么帮助?其实它们帮不了我们 它们都是错的。它们错的原因是 他们忽略了一个。幽默的特征。我们认为这个特征是最重要的 我的解释也很重要。你们不用马上全部记下来。假装PPT上的东西是慢慢地出来的。我认为幽默就像一种正念视角。我们透过这个视角来看待世界。我们研究这点的一个角度是。我们可以把幽默视为一种认知心态。我们通过它看待我们周围发生的事。要研究这个问题 我们可以问。为什么一些别人认为不好笑 或者很悲剧的事情。有些人会从中看到幽默?为什么有些人觉得好笑的东西。别人会觉得不好笑?同样道理 为什么。乐观的人能在一件事中看到好的一面。而不悲不乐的人 或者悲观的人。就看不到好的一面呢?要明白这点…这个麦克风真好用。谢谢。我也觉得这个一个好方法。除了这些以外 我们还要…。为了要明白这个问题。我们要讲一下打破社会规则。我相信我们有一些不成文的社会规则。不停地指导着我们的生活 但我们不会讨论它们。 

我教过社会心理学。在教社会心理学时。我们在班上做过一件事。我们把同学分成两人一组。叫他们去打破社会规则。在那一周里 他们会做很多事。他们会走进哈佛其中一间宿舍的电梯里。然后慢慢地在电梯里转过身。面朝着身后的同学。或者在电梯里躺下来。看看那些同学会不会说什么。或者去星巴克 买杯咖啡时跟人讨价还价。说你在家里做的话要便宜很多。或者上Coop网站。用一本书来买东西。或者去商店里问人要免费样品。那些没有免费样品的商品 例如Felipe's店的玉米煎饼。我还知道有个同学走进他的运动员更衣室里。问遍更衣室里的每个人 问他们。愿不愿意帮他擦背 因为他太累了。还有一个女同学 在女生更衣室里。看到有人在浴室里洗澡。她就只进那个浴室洗澡。还有一个同学…我想说明的是 有很多事情可以打破社会规则。例如走进浴室。隔着浴室门和别人聊起来。这个非常好玩。你们应该试试看 "你好 隔壁的同学"。然后两个人聊起来。 

最好笑的笑话…声音真的变大了。你们能听到我吗?太棒了 早就该这样了。现在我后面有很多东西了。我讲到哪里了?课堂上可能不应该说这种话。你们知道…我记不住笑话的。有人在网站上评选出一个最好笑的笑话。他们看了网站上的笑话 真的找出了有科学根据的。最好笑的笑话。有一个美国人去打猎。我不会讲笑话。他…。这个人找电话给接线员 他说"接线员"。接线员说"这是911报警中心" 所以说我不会讲笑话。他当时在宾夕法尼亚州的Scranton市 那天是星期二。总之 他打电话给接线员说"接线员。我刚才打猎时 我好像打中我的朋友了。他好像死了"。接线员说"好的 冷静 先生。我们要你做的第一件事是。确保你的朋友已经死了"。他说"好的"。经过一阵平静的停顿后 他放下电话。几秒钟后听到枪声。砰!他拿起电话说"好的 现在他死了 接下来怎么办?"。这是美国最好笑的笑话 有点意外。欧洲最好笑的笑话是…有点不适合在课堂上讲。有两只雪貂坐在一间酒吧里。一只雪貂中另一只说。"我昨晚和你妈妈睡了"。另一个雪貂什么都没说。那只雪貂见它什么都没说 就说。"我昨晚和你妈妈睡了"。第二只雪貂还是什么都没说。最后它…。另一个雪貂 雪貂A说"我昨晚和你妈睡了"。雪貂B说"爸爸 回家吧 你醉了"。掌声。 

打破社会规则的一个有趣之处是。它不仅影响了其他人。影响了我们对于整天。围在我们身边的社会条文的意识。当我们打破社会规则时。还使我们意识到了自己的行为。有些同学。做这个实验时大笑不止。他们连实验都做不下去了。有些人在尴尬中发现幽默。因此 他们能够享受这个实验。把自己的快乐建筑在别人的痛苦之上。我们从中发现。我们有一个心态。决定什么是好笑的 什么是不好笑的。如果这真的是由心态决定 那我们就可以控制它。 

今天我们就讲几个。控制这个心态的方法。我们这个学期之前讲过。Wiseman对幸运配方的研究。大家回忆一下 幸运配方是指。他研究 并创立了一间运气学校。教人们怎么去觉得自己更幸运。给人们同一个情景。有些人会觉得幸运。有些人会觉得不幸。例如 如果你走进一间银行。那里有100个人在等。假设 我们在…。我们不在Annenberg 我们在Sanders。有人拿着一支枪走进来。开了一次枪 打中了我的手臂。这是走运还是倒霉?根据Wiseman所说。这取决于我用什么反事实去看待这件事了。如果我拿这件事跟。我平时进戏院的情况相比。那这件事肯定是倒霉了 对吧?这样看 非常不幸。这个房间里还有很多人。这个人可以打中很多人。我还是读过神学院的。我肯定比这个房间里至少一个人要好。肯定至少有一个人比我更活该中枪。可为什么偏偏打中我?但从另一方面看。我可以用另一个反事实去跟这件事比较。看看这件事是走运还是倒霉。这个反事实可以是"他有可能打中我的头"。我喜欢这个乐观。幸运的观点 因为我总是选择。能造成最多伤害和流血的反事实 对吧?或者说 我有可能被打中头部和心脏。流血过多而死。这样…我坏了大家的心情了。每天都这样子。通常都是上课20分钟就这样。所以这取决于我用什么反事实。来决定我觉得什么好笑 什么不好笑。幽默也是同样道理。环境非常重要。 

如果我说一个事实。一件毫无感情的事 如"我弄断趾甲了"。这一点都不好笑。但如果我说"我在上课时弄断了趾甲"。这样说就有点好笑了。不是很好笑 但有点好笑了。"酒精能使人醉" 这是一个毫无感情的事实。"酒精能使人醉"。在上一节讲幽默的课前说这句话就有点好笑了。所以我们可以从中看出。你看待环境的视角。你分析世界的视角 能直接影响。当你一听到某件事时 会不会觉得它。好笑。在我的课堂上 我们有两样东西叫。α因素和β因素。α因素就是指客观的现实约束。我们都有同样的α因素。我们都坐在同一个教室 有同样的灯光。但我们的β因素。是对现实的主观感受 是不同的。这间教室里每个人都有不同的。感受。有些人觉得无聊 有些觉得饿。有些人觉得兴奋 有些觉得高兴 我觉得很紧张。我们从中看出。他们看待世界的视角。绝对会影响你如何看待α因素。到目前为止 我们讲过的幸福心理学的内容…。我讲过在什么情况下。我们的β因素是可塑的。我们可以改变我们的β因素。因此能大大地改变。我们的适应行为。我认为幽默是一个很强大的工具。我们可以利用它通过改变我们看待环境的视角。来改变我们的β因素 让它适应我们。 

在哈佛医学院有一个。俄罗斯方块效应。俄罗斯方块效应是…多少同学玩过俄罗斯方块?很多。俄罗斯方块是…还没玩过的同学。在这个游戏里 有不同大小的形状。从上面掉下来。你旋转这些形状。把这些形状排成一整排。如果一排排满了。这一排就会消失。然后你继续这样一直排到顶。他们在试验中发现…。他们找大学生来玩电子游戏。在医学院 他们发现。学生连续玩了四五个小时的俄罗斯方块后。他们是有钱拿的 非常好的一个交易。玩了以后 他们回到宿舍。继续他们平常的生活。有主观报告说 这些学生。当他们去到超市时。他们开始重新排列货架上的面包。让它们排成直直的一排。或者他们看到路上的车时 会想。"我要把这辆车放到这里排在一排。这样来交朋友"。或者他们会…有一个男同学。看着体育画报的泳装写真日历号。上面有。很多模特在同一张日历上。他说了一句话 "我希望有一个摆Z姿势的女人。她能放进这里 把这一排排满"。很明显他把整个世界都看成…。甚至包括一张泳装写真日历。他都看成能够改变形状 排出直直的一排。每个输入的信息都有脑波接收。我们的大脑接收了太多的信息。大脑就形成了这个模式。 

我们从中发现了一样东西。我称之为认知残象。认知残象就是。如果你盯着太阳看一段时间 笨人做笨事。结果就是你暂时性地烧伤了你的视网膜。你看东西时 就会看到一个蓝色或绿色的点。你看到哪里 它就跟到哪里。我在Kirkland宿舍看到太多照相机闪光灯。就会出现这种情况。出现这种情况后 很多时候都会这样子。你的视网膜有一个残象。在认知层面上也有同样的现象。我们的大脑能保存一个残象 在俄罗斯方块效应里。学生会有一个认知残象。这个残象会停留 他们看这个世界时。就会把世界上的事物看成。能够旋转变形 排进一条直线里。幽默的人也是这样。当人们看到某件事…。例如Jon Stewart的编剧 或者"Steven Colbert报告"的编剧。当他们看报纸时。当他们看到一件不悲不喜的事。或者一件悲剧时 他们就开始让它变形。他们在脑海里把这件事旋转变形。结果他们就会不停地寻找模式。他们能从一件事上找到无限可能性。而不是只看到一张报纸。他们能从报纸上看到幽默。 

你们能发现。认知残象有时也会带来坏处。你会在一个思维模式中出不来。总是找幽默的东西。这样有时会有利于适应 但如果你是和人吵架。你唯一能做的就是开玩笑 这是非常不利于适应的。所以把幽默作为改变我们视角的东西还有一件事要注意的。在我们讲幽默的益处前 我们先来讲一个这个。当我们笑某件事时。Peter Berger认为 我们会暂时性地。暂时地中止了现实。我们能看到一些有别于现实的东西。它让我们知道。我们眼前的现实 是可塑的。它不是唯一一种可能的现实。现状是可以改变的。所以幽默可以使我们变得更正念。Ellen Langer把正念定义为。"留心眼前的情景"。这样你就能留意到环境中的可能性。我给幽默下的定义和正念的定义。几乎一模一样。原因是 我认为。当你真的留心眼前的情景。当你以某种模式来看待世界时。你实际上就是在用一种方式分析世界。这种方式使得你能看到比环境本身更多的可能性。所以Maslow认为我们会有高峰体验。这时我们会暂时地披上了自我实现的披风。当我们披上这件自我实现的披风时。我们就能暂时地获得和看到。我们实际拥有的潜力。所以我认为幽默也是这样。我认为我们暂时地假装达成了自我实现。因为当我们假装有这个现实时。在这个现实里 行为后果。不会对我们产生负面影响。我们会看到人生的真实和幸福。即使是我们处于一个别人觉得枯燥悲伤的境地。平凡就变得非凡。这是Abraham Maslow所说的高峰体验的。一个特征。 

我觉得"办公室"就是一个很好的例子。我有两段"办公室"的一分钟视频。我想放给你们看。在这个片段里 他们可以把一些…。在办公室里的工作是很平凡的。他们能把平凡的变为非凡的。班上有多少同学看"办公室"?有多少同学不看?有多少同学是不幽默的?有些人没举手 我知道不止这么少。好的。有多少同学认为自己的幽默感。是在中上水平的?举起手来。有多少同学。认为自己的幽默感是中下水平的?刚才没举手的那些都是了。我不知道你们还想我举出什么水平来让你们选。有多少人认为自己正好处于中等水平的?幽默的同学 好的。我们来看这两段视频。来看看如何把平凡变为非凡。片段:办公室 第三季 15集。糟了 又丢了一个文件 又要重启了。Dwight 你想吃薄荷糖吗?你说呢?上学时 我们学过一个故事 说有个科学家。训练狗一听到铃声就流口水。方法就是每次响铃时都给东西它们吃。过去几周。我在做一个类似的实验。Dwight 想吃薄荷糖吗?好的。吃薄荷糖吗?好的。要薄荷糖吗 Dwight?Inbwit的吗?好的。你在干什么?我…。什么?我不知道 我突然嘴巴很臭。还有一个。片段:办公室 第二季 21集。Dwight 你说。"有人用蜡笔换走了我所有的钢笔和铅笔。我怀疑是Jim Halpert。所有人都叫了我一整天的Dwanye。我觉得是Jim Halpert给钱他们 叫他们这么做的"。是的 每个五美元 花得很值。"到下班时。我的办公桌离复印机近了两英尺"。是的。每次他上洗手间 我就移动一英寸。我那天一整天只干了这件事。"每次我输入我的名字 都会变成'尿布'"。用一个简单的宏就行了。"今天早上。我拿电话时 撞到了自己的头"。这个花了我不少时间。我要一次放一个硬币进他的电话听筒。等他习惯了那个重量以后。我就把全部硬币都拿出来。 

Jim用Dwight来做这些实验。这些实验也告诉了我们。幽默对我们有多大的积极影响。我们不但会看"办公室"。我们还会选择与幽默的人为伴。或者让自己变得幽默。为了弄明白这点 我要讲两个身体部位。写下来 只有两个。交感神经系统。和副交感神经系统。副字拼作para 副交感神经系统。交感神经系统是我们身体里。一个负责释放化学物质进我们身体里。让你准备作出或战或逃反应的系统。它能暂时地让你当一回超人。你变强了 你变快了。你忍耐力变强了 你更有活力了。但问题是 一直当这样的超人的话。是有分解代谢作用的。当你把这些化学物释放到你的身体时。虽然你的身体暂时变强壮了。但其实它在损害你身体里的所有器官。压力是有影响的 它会劳损所有组织。你体内的活组织。幸好 不是只有这么一个系统。因为我们不会变得越来越兴奋。我们的心跳不会一直飙升 直到心脏爆炸。因为我们中枢神经系统里还有另一部分。叫做副交感神经系统。副交感神经系统。负责让我们的身体冷静下来。它负责减慢我们的呼吸率 心跳率。真正让我们恢复过来的就是它。当我们能量水平没有飙升时。我们的能量就能维持更长时间。因为这个系统会不断地激活。修复 并注入更多的活力。很有帮助的图表。这个称之为…。有人称之为黑道家族效应。其实只有我称之为黑道家族效应。但我们还是会考。黑道家族效应。是指长期地激活交感神经系统。很多哈佛学生 包括我自己都会这么做。我们准备为考试复习时。或者处理生活中某些事情时。我听到很多人在考试前会说。"这次考试我一定不会及格了。我压根就没准备好。我一定会不及格了"。或者他们心里想。"我太忙了 我压力太大了"。当他们这么想时。他们其实是把事情都留到最后一刻才做。就像Tal说的拖延症。结果是。你的能量会短暂地飙升。因为你的交感神经系统被激活了。所以我们习惯了这么做。我们会上瘾地用。交感神经系统刺激我们的能量。我们认为。能量可能让我们完成工作。Tony Soprano不是现实人物 是HBO剧集里的。黑帮老大。这个电视剧的一个主旨就是。他要去接受心理治疗。对于一个恶贯满盈的黑帮老大 没必要这样做。他要去看心理医生 因为他惊恐症发作。他的身体停止运作 因为他长期激活。他的交感神经系统 所以就发作了。不管是他当黑帮老大时做种种犯法的事。黑帮老大。黑帮老大 前面加一连串的形容词。还是平常和他家人的生活。过一个正常的生活 他住在新泽西。住在新泽西的压力就更大了。因此 他的身体停止运作。所以原因之一就是。他长期激活他的交感神经系统。哈佛的学生也是这样。经常我们都会利用这种激活。我们会长期激活我们的交感神经系统。因此我们就会耗损我们的身体。我们的身体就会开始停止运作。幸好 我们还有救星。对抗黑道家族效应的最佳缓冲剂是幽默。原因是 幽默。就像正念和冥想。它能激活副交感神经系统。因此 非凡的事就发生了。很让人惊讶的是。我们觉得大笑是一件奢侈的事。我们有时间的时候才会大笑。我们没事可做时才会大笑。但它对我们的生活却有如此巨大的影响。首先 笑本身既是一种药也是一种运动。如果你喜欢Jack Candy和"周六夜现场"的"深层思考"小品。我引用几句话给大家听 其中有一句是。"我爸爸总是说 笑是最好的药。可能这就是为什么我们几个有家人死于肺结核"。其实不是很好笑 但我还是决定说这句。笑也是一种运动 每次你笑。如果你想在这节课上笑。或者你想下课后才笑。你笑时 你的身体就在做大量运动。我已经说过。你笑时 身体有多少块肌肉在伸缩。我们发现 只要笑10到15分钟。笑10到15分钟。所消耗的卡路里。就相等于一块中等大小的巧克力 非常棒。有些人的报告说 只要笑几分钟。就足够让他们的心率飙升到。比在划式练力机上做15分钟运动还要高。这就要看你划的是什么了。 

他们还发现它能导致…非常厉害的是。幽默还能够在化学层面上起作用。它能增加我们体内的T细胞。从而加强我们的免疫系统。增加我们的γ-干扰素。它是一种抵抗疾病的蛋白质。还会增加我们的B细胞。B细胞负责生成抗体。加强我们身体抵抗疾病的能力。所以幽默真的能帮我们更好地对抗现实。对吧。它真的能加强我们的身体 帮助我们。对抗外部世界。这就是幽默的作用之一。 

它能够影响黑道家族效应的原因之一。就是像Sultanoff这样的研究人员发现的那样。笑能降低我们体内的。血清皮质醇。当我们感受到压力时 就会释放皮质醇。应激反应…。如果我做了某些事 让你感到有压力了。你的皮质醇水平就会上升。如果你在这时候引人发笑。他们的皮质醇水平就会降低。也就是说在他们感受到压力时。大笑起来。在压力中笑起来 能够使他们。降低压力带来的消积影响。大家回忆一下 压力会耗损我的身体。除此以外 幽默还能给我们的心理和生理。累积大量的益处。首先 他们发现。即使是一点点的幽默 就能提高我们的免疫系统。他们的研究方法是。他们找来一些患有支气管哮喘病的人。让他们接触尘螨。这些病人对尘螨是极度过敏的。这样是为了让他们产生过敏反应。我很肯定这群研究人员就是。上次做那个吐根怀孕研究的那些人。我们在这门课上讲过那个研究 或者是那个。一次又一次地电击狗 习得无助感实验的那群研究员。在这个支气管哮喘实验中。他们让这些病人接触尘螨 有致命危险的。他们只让他们接触一点点。但他们至少要保证数量要足以。让所有病人都对尘螨产生过敏反应。然后他们找这些人回来做实验。看来这些人是真的学不精的。这些患有支气管哮喘的病人。回来做同一个实验。有一半人在接触尘螨之前。先听一个笑话。我不知道他们听了什么笑话。反正他们笑了。然后另一组是对照组。他们只是看了或听了一个很普通的故事。这些故事不会减轻。支气管哮喘患者对尘螨的过敏反应。他们发现。数据上表明 那些先听过一个笑话的人。哮喘病发作的数量。要少很多。 

我很难想象…。我想说一个好笑的笑话 好让你们讲给别人听。但我想不出来。你们想的笑话可能都比我想的这个要好笑。但我想…算了 我不说这个笑话。不说了 不好笑的。我本来想说一个关于哮喘病的"敲门"玩笑。但我发现不好笑。 

我们还发现 幽默还能提高忍痛度。有一群德国研究人员。除了德国研究人员还会有谁。他们找人来。找参与者来参加这个研究。把他们的手放在冰冷的冰水里。这是心理学以及医学界中。测试一个人的忍痛度的标准做法。因为这个测试不会对人造成永久伤害。只会加重他们的冻疮。他们发现。他们把手放进冰水桶里。在这期间。你的手放得越久 你的忍痛度就越高。你可以和朋友试试 这是一个很好的派对游戏。你可以。让一个人看好笑的视频。一个人看不悲不喜的视频。一个人看悲伤的视频 这时候他们的手都放在冰水里。他们发现。看幽默视频的那一组。在数据上来看。手放在冰桶里的时间要比其他组长很多。比其他组的人长很多。这意味着 如果你觉得幽默好笑。你可能没有别人聪明。但你能把手放在冰桶里。更久。但这个研究还有一个惊人发现。幽默的影响不仅现出在实验过程中。他们在这些人看完幽默视频后20分钟。再对他们进行测试。看完幽默视频20分钟后。还能有同样的效果 即使是看了幽默视频。20分钟后 幽默还能有一个后效应。提高你的忍痛度。这个结果。给了人们很多启示。那些日常生活中因为患有疾病。而有慢性疲劳 或者慢性痛的人。幽默可以减轻他们的疲劳或疼痛。他们还发现在你狂吃一顿后。如果你有II型糖尿病 而你狂吃了一顿。如果你吃完后。看了一些好笑的东西 而不是一些平平淡淡的东西。令人惊讶地 你的血糖会。比那些吃了同样多的人 低很多。这种效果太惊人了。最后 他们还发现幽默能减少压力。增加癌症的缓解率。例如 有一个关于睾丸癌的研究。研究人员发现 患者。如果能笑出来。能用笑作为他们的心理应付机制。那些用笑来作为心理应会机制的患者。比起那些没有这样做的患者 他们的缓解有所提高。甚至。他们的症状也减少了。他们的忍痛度提高了。伴随这些病痛而来的压力也减少了。因此。我们看到幽默真的有很多惊人的益处。我们看到它能改变我们的身体化学。改变我们的免疫系统。帮助我们对抗疾病。它改善我们的压力水平。改善我们身体对尘螨的反应。对食物的反应 对癌症的反应。幽默从方方面面。影响我们 效果非常惊人。 

我还想谈谈幽默的社会影响。如果你是第一天来这里上学的。我要跟你说几个数据。我只是想提醒大家几个数据。去年我们在哈佛学生中做了一个调查。我们发现。哈佛学生的谈恋爱的次数。少于一次。平均的性伴侣人数在0到0.5之间。用科学数据给二垒定了量。我们还发现你们当中有24%的同学不知道。自己是不是在谈恋爱。我想跟大家说这几个数据。是因为我觉得。幽默能给我们哈佛学生在恋情中带来很大帮助。 

我还在哈佛教一门叫人类性欲的课。那是一门很尴尬的课。有多少同学上了人类性欲?只有一个。你好 真尴尬。其中…。我说它尴尬的原因是。学生要写的第一篇论文 是一篇十页的。讲自己第一次性行为的论文。我要评阅这些论文。非常尴尬。我还要写评论。"不 不可能这样" 或者说。"写这篇论文的时间要比你做爱的时间长很多"。有很多尴尬的事。但是…。最尴尬的一次是。有一节课。他们想我给哈佛学生看看安全套是什么样的。因为数据表明 你们哈佛学生很可能。连安全套或者其他与性有关的东西都没见过。他们发现这个现象。他们想我在一节课上分发安全套。这样大家都能拿一个看看。我那天要教四个班。于是我把80个安全套塞进背包的前袋。当时我坐在星巴克里 星巴克是我的咖啡店兼办公室。我整天都坐在那里 一天只需交1.49美元的租金。我发现我坐在那里时。我不知道我为什么很喜欢说"发现"。我当时坐在那里 旁边坐了一个女孩。这时她转过身 跟我说。"你能借支笔给我吗?"。我说"当然能 当然能借支笔给你"。然后我完全忘了。我打开我的背包。80只安全套夺包而出 掉到了她身上。她的桌子上 她脚下的地板上满是安全套。我花了四分钟才捡完她脚下的安全套。我还递了一个给她。她不想要安全套 也不想要我的笔。 

研究这种第一次相遇的研究。得到非常惊人的发现。Fraley做了一些研究。他会找一些人来…有个知识点忘了讲。我们等一下再讲。Fraley发现 如果你把实验参与者带到一个房间。他们互不认识 你让他们做一个合作任务。你让他们做一个合作任务。有个助教很喜欢我标题中的"吸引与感情"这两个词。我觉得很好笑 我想不起要讲什么了。我讲另一个内容吧 镜像神经元。镜像神经元非常厉害 因为我刚才忘了讲这个。镜像神经元 我们以前不知道大脑里有这个。直到十年前才知道。或能还不到十年。我们那时才发现它们的存在。它们的作用就是 如果有人对你笑。通常你一定也会对他们笑。原因是 因为你大脑有个部位。在你笑时 亮起来了。就好像我站在讲台上 有个人走过来。狠狠地把我绊倒 大家都会惊呼。你们会觉得很愤怒。你们有大脑会亮起来 那些镜像神经元会亮起来。就好像你也被绊倒了一样。这也是为什么人们看橄榄球比赛时会发生痛苦的叫声。即使他们离体育场有2000英里远。我们天生会对别人有移情作用。因此 幽默非常具有传染性。因为当我们看到别人笑时。我们的镜像神经元就会激活。 

幽默之所以能帮助我们的社交 是因为。第一次相遇时 它能制造吸引力。他们发现 他们找来互不认识的参与者。让他们玩一个竞赛游戏。之后让他们评价对方的吸引力。以及他们对比赛对手产生了多少感情。通常他们的吸引力和感情都很低。中间那组。你找人来合作完成一个任务。跟Tal这个学期讲过的。华尔街和社区运动会的研究很相似。在这个实验里 如果是合作完成任务。参与者的吸引力 两个人的吸引力。和感情都上升了。但如果你让参与者做。一件很尴尬的事 例如玩扭扭乐。或者让人走进房间。实验参与者所在的房间。叫他们不小心地把水倒到其中一个参与者身上。告诉他们 他们笨手笨脚的 然后离开房间。让他们笑这件事 这些人…。因为这个过程中发生了幽默的事。所以给对方评的吸引力分数是最高的。而且感情也是最深的。所以 下次如果你坐在酒吧里。你不要试图去跟那个人比个高低。或者跟那个人合作。你可以请他们玩扭扭乐。或者把酒倒在他们身上 告诉他们 他们有多笨手笨脚。这个诡计。一定能增加你们的感情。 

他们发现在研究幽默这个课题时。约会是一个很有趣的研究对象。尤其是研究它的社会因素。我们知道人们是会。被某些身体特征吸引的。但为什么有些人会选择幽默的人。幽默应该不是一种进化优势 对吧?这班上有多少同学是不会。和赚不了多少钱的人约会的?举起手来 好的。很好 有几个诚实的人。从进化的角度来看 这个答案是正确的。但很少人举起手。那些发现约会对象没有幽默感以后。就不和他们约会的同学 请举起手来。几乎所有人都举了。也许是因为这个答案更为社会所接受。这样可能会把研究都搞砸了。事实是。我认为很多人都没有说谎 我们确实想要性感的…。我们确实想要有幽默感的伴侣 好的。我还会一次又一次地口误。我们发现在这些人身上。如果你在野外遇到危险 例如遇到狮子。你想打赢它 或者逃跑。你不会想说那个敲门笑话 对吧?幽默并不能增加你。对自然选择的适应性 对吧?但我们发现 人们似乎都很喜欢幽默。不同性别的幽默是不同的。科学家和心理学家认为。幽默是表示认知能力良好的一个信号。就像我们喜欢别人有良好的体格一样。我们也喜欢良好的认知能力。我们喜欢能用不同视角看世界的幽默之人。它不仅能极大地影响。他们看待世界的方式。还会影响他们帮助你应付现实的方式。我们刚才所说的这些。幽默对生理的益处 都会出现在。那些与笑的人为伴。或者与开玩笑的人为伴的人身上。如果你与幽默的人为伴。你就能得到所有这些益处。除此以外 他们还发现了几个有趣的差别。男人与女人的差别。这是由一个叫Bressler研究员发现的。我不知道这个人是男还是女的。Bressler发现男女的差别是。他们发现男女所寻求的幽默。是有差别的。他们观察在酒吧里调情的人。在这个实验里 他们看。那些在酒吧里调情的人 很怪。每次女人或男人笑时。他们就会打一个勾。然后 他们会走过去。给这两个陌生男女一份报告。让他们填完这份报告。并且在他们离开前 互换报告。他们发现女人笑的次数。与女人喜欢男人的的程度成正比。反过来也是。他们还发现 女人笑的次数。与男人喜欢她的程度成正比。他们还发现 男人的笑。对两个人的吸引力都没有任何影响。似乎这…。这只是从总体来看 不是每个人都会这样。大家从中看到。一个趋势 有些人似乎是说笑话的人。有些人似乎是听笑话的人。这给了我们一个很有趣的想法。在这两者当中 你会选择当哪一个。当你想用你的幽默风格来吸引对方时。我们还发现。在婚姻谈判中 幽默很重要的。基本上和上面那个一样。 

John Gottman发现…John Gottman就是那个。恋爱专家 他做了四骑士实验。我们在之前讲爱情的课上就已经讲过了。他说消除冲突的最佳方法。如果在谈判中出现冲突 或者婚姻出现问题时。最佳方法就是换一个角度 把幽默。当作应付机制。要回答幽默在…当你出现应激反应。或者说像我之前说的那个 皮质醇被释放出来。即使John Gottman也把幽默视为。达到效果的最重要方法。这个…这个很有趣。但和我现在讲的内容没什么关系。当我们叫一个大学男生想着爱时。他的大脑就是这样子的。大脑有些地方亮起来了。很漂亮。如果你叫他们想着性 剩下的地方都会亮起来。 

幽默作为一种治疗手段 我简单地讲一下这点。因为我想等一下讲幽默的最强大功效。Pennebaker的研究。我们之前讲写日记的时候讲过。当你开始写日记时。你的神经活动水平会出现峰值。其实你写日记的时候 感受到的压力会更大。但写完后 就会回落到一个低很多的水平。幽默也是一样。因为幽默过界了 它暂时增加了我们的压力。但能让我们回落到。一个更低的水平。因为它激活了我们的副交感神经系统。因此。在一个心理治疗中。幽默是我们可以利用的有效工具。印度有些大笑互助组就是用这个方法。他们会让人围成一圈。把头放到另一个人的腹部上。一个人开始叫"哈" 另一个人也叫"哈"。整个圈的人都这样做。直到所有人都自发地笑起来。这是一种很好的健身运动。做完以后 你会觉得自己。健康了很多。他们还有"抱抱兔子" 或者"抱抱组"。抱抱免子 这是从网上的梯子理论上找来的。大家想了解的话 就去搜索一下。抱抱组 是在纽约的。我没有时间讲了。这些小组能改变心理创伤和坏心情…。我想这当中很有趣的一点是。幽默真的可以让我们能够对我们以前认为是。消积的 不好的或者让人烦恼的事情改观。这就是我想说的 幽默最强大的功效。但在讲这点之前 我想说。可以增加我们幽默感的六种方法。首先 我们刚才已经讲了一个了。把只属于你的故事写进日记了。找出你日常生活中。好笑的故事 当你写日记时。要尽量写积极的经历。你要审视你的周围。找出日常生活中发生的积极事件。想培养幽默 也可以用这个方法 把它练成一种技能。就像把感恩练成一种培养乐观心态的技能。你要做的是 回忆起这一天。写下这一天发生的事情。然后拿它发挥 给它变形。改变它的方向。直到你能从中发现一些幽默。就像Jon Stewart的编剧。看报纸上的新闻一样。结果 你的大脑会养成一种模式。找到生活中好笑的事情的模式。 

第二个方法是观察幽默的人。当你观察幽默的人时。因为镜像神经元。你能从他们身上学到幽默的一些规律。网上还有这样的规律供你学习。如果你对此感兴趣的话。你可以学习幽默的规律 这就是为什么。培养模式和打破模式这么重要。有意识或无意识地学习这些规律。就是我们以前讲过的可视法练习。你在电视上看不擅长运动的人。如果你看他们。你也会变得不擅长于运动。所以如果你看擅长于运动的人。你会变得擅长于你看的这种运动。当你观察幽默时 会有同样的结果。 

还有一个方法是TQP 两问处理法。为什么我这么幽默?为什么别人没有发现我很幽默?不断地这么问自己。绝对能提升你的自尊。除此以外 还要允许自己当一个次等人类。我们和伯格森都认为。要说这些笑话。对着很创伤的事笑出来。我们要暂时地麻醉我们的心。暂时地麻醉我们的心。如果我们不允许自己当一会次等人类。就等于我们不允许自己有人类的缺点。就是不承认我们有本我冲动。多找点花样 打破模式。你打破模式的次数越多。你就越能看到一个情况里的可能性。有一个用老鼠来做的实验。把一只老鼠放进笼子里。让它和一只母老鼠性交。当你把它放进笼子两到三次后。它就会变得很累。它就会躲到角落里 不想和母老鼠拥抱。它会维持这样10分钟 这叫做潜伏��。这时候你生理上无法进行性行为��因为太累了。接下来 放一只���的母老鼠进笼子里���这时候。老鼠会走过去 继续做这个实验。不管有没有过潜伏期。这个实验说明了 我们的大脑。我们哺乳动物的大脑天生就要寻找多样性的。这不是谈恋爱的好方法。但这时我们会认识到。花样是生活的调味剂。你越是改变你现在的模式。你就越能看到环境中的可能性。我们之前讲过的俄罗斯方块效应。 

现在来讲幽默的最大功效。我认为这点很重要。我想讲一下 然后让大家看一段视频。因为我们只剩六分钟了。我想让大家看一段我觉得很有强大的视频。我不觉得上了一节幽默课后 看这段视频会坏了大家的心情。但我觉得它很重要 值得我们一看。这段视频是关于Jon Stewart在911悲剧后怎么做节目的。我觉得从心理研究的角度来看 它不好笑。但大家注意他的变化。观察他的脸部表情 看他怎样试图。在这种情况下把幽默带回节目的壮举。听听他是怎么形容幽默的。这段我们会看三分钟。然后看一段一分钟的"恶搞之家"。然后我会用一分钟进行总结。等等。 

视频:JON STEWART。 

2001年9月20日。晚上好 欢迎收看"今日秀"。我们又开播了。这是自从纽约发生悲剧以来 我们播出的第一集节目。要开始这集节目 真的没有别的方式。只有问各位家庭观众一个问题。这个问题我们已经问过今晚的现场观众。自911以来 我们已经问过我们认识的。每一个纽约人 这个问题是"你还好吗?"。我们祈祷大家都安好 大家的家人都安好。很抱歉要对你们做这种事。这又是一个娱乐节目。一个惊魂未定的主持 念着一段过分矫饰的开场白。电视节目除了多余 就真的什么都不是了。我为此向大家道歉。但很不幸。我们必须为自己这么做 这样我们才能抽干。心里的脓 继续我们的工作 逗大家笑。最近我们都很难笑得出来了。大家都来上班了 我知道我们晚了。我知道我们开播的时间刚刚好。赶在了"幸存者"剧组出来建议。我们在这种情况下应该怎么做之前。他们的建议就是 回去工作。对于一个抱成一团 躲在桌子下哭的男人来说。是很难找到工作的 找得到的话我会很乐意去做。所以我回来了。很显然今晚的节目不是一集平常的节目。我们翻遍了资料库。找了几段我们觉得能逗你们笑的视频。我觉得在现在这是非常有必要的。很多人问过我。"你上班后 怎么做这个节目?你会说什么?"。我是说 天啊 这种时候拍这样的节目多么不合适。我不觉得这是一个负担 我觉得这是一个荣幸。我觉得这是一个荣幸 这里的每个人都这么觉得。 

听到我了吗?我觉得这是一段很强有力的视频。你可以…。听到他内心正在痛苦地挣扎。你可以看到他试图说几句笑。然后又退缩回去了。你可以听到他讲他惯用的语言风格。他需要释放的感情。然后说他的挫败感。这集节目的最后。他带了一只小狗上台。他就让小狗留在台上 试图让我们。恢复心情 让我们看到。在眼前的悲剧以外 还是有很多可能性的。我觉得这是一段很优秀的视频 值得一看。过后 你可以看到人们。用不同的方法应付悲剧。我觉得从一个社会学角度来看。"恶搞之家"对911事件的回应也很有趣。他们经历的那段时间。我们都很害怕。用恐怖主义威胁美国的人 但他们却拍了这么一估。 

视频:恶搞之家 第四季 14集。(阿富汗某个地方)。所有美国异教徒听好了。准备在圣火的包围中死去吧。你们将会为你们的堕落受到惩罚。就在拉达曼的第一天。你…等等。我刚才说什么了拉达曼?啦啦啦 拉-达-曼。拉达曼 是什么?对 可能丹尼斯·拉达曼(应该是罗德曼)。会用他的怪头发惩罚你。不 你说啥?对 好的。好了 好了 我们再来一次。好的 所有美国人听好了。我现在拍不了 我要…。好了 不。不 我要…我要先笑个饱。好了 我要先笑个饱。好了 行了。今天 有时 你别再做鬼脸了。你在干什么?我不能…。他在做鬼脸 害得我要笑。好了 不如这样 你转过身 转过身。我不管你面朝哪里 朝那边吧 好的。好了 他们在那边大笑起来。好了 看看谁又在那边偷笑了。"我不能去做自杀式爆炸 因为我病了"先生。他…他还有医生签的请假条。 

他还有医生签的请假条。下面的不看了 因为越来越不好笑了。我只剩一分钟。我想告诉大家。我们今天讲的一些内容。我觉得这是一段很好的视频。大家看到 当我们从吓到我们的情况下走出来。并且拿这样的事来开玩笑。因为我们能看到当中的人性。我们能看到自己大笑。拍视频会议时犯错。不一定是拍他说的那种话。 

我们今天讲的内容是。幽默就像乐观主义。我们透过它来看世界。它需要我们用正念来找到可能性。它能加强我们的健康 改善我们的社交。和我们的身体。它能作为一种心理治疗方法。我们把幽默视为一个奢侈。但我认为在痛苦和冲突。悲剧 经济倒退 大萧条出现的时候…。我想没什么会比幽默更重要的。就像Jon Stewart说的 能给你们上课。是我的荣幸。谢谢。掌声。 

第21课-爱情和自尊 

嗨 大家好 我叫Megan。课后我们会进行一个非常有趣的研究。有关内观自省与知见。时间不会超过10到15分钟。我们需要至少60个人 所以如果你有兴趣。请课后到教室前面来。不会耽搁很久的 真的很有趣 谢谢。掌声。 

你们都听过格特鲁德·斯泰因的这个故事。她那时上William James的哲学课。就在哈佛拉德克利夫学院。要期末考试了 她上的是春季班。她来到考场 就跟今天一样是个晴朗的日子。考试的内容是形而上学及生命的意义。于是她打开试卷 写道。"多么美好的一天 不应该浪费在考试上"。然后走出了教室。而且传说。William James的课程她全A通过。本学期考试时。不要学她 或者拿她当借口。不过我真的很感谢各位今天出席。今天天气非常好。我想过要到户外上课的。不过…也许我们应该去的 是的。 

今天我们要讲完爱情。还差一点就讲完了。然后我们会开始讲本课程最后一个话题。也就是自尊。 

先回顾一下上回讲到的关于爱情的内容。我们讲到了人类要如何…考虑到人的本性。人类要如何获得 维持长久的爱情与激情。因为从心理学角度看。这似乎有违人的本性。当我们讲到研究那些最成功的恋情时。最成功的恋情有四个特点。根据David Schnarch和John Gottman的研究发现的。第一条是 经营爱情需要付出努力。人们往往误以为。寻找合适的爱情对象是最重要的。其实更重要的是如何经营你选择的爱情。就跟工作一样 如果我们找到了梦想中的工作。然后翘起腿 什么也不做。是不可能成功的。同样的 如果我们在一段恋情中抱有寻找心态。我们的恋情也不会成功。以为只要找到爱情就能幸福地生活下去。我们说过 电影结束时 正是爱情刚开始时。一段健康长久充满激情的爱情。第二个组成部分 跟第一个有关联。我们要被了解 而不是被认可。表达自己 而不是粉饰自己 坦开心扉。坦诚自己的弱点 优点 渴望。热情 恐惧与不安。这样的爱情。恋爱中的双方如果这样做了。会渐渐变得更加亲密 更加快乐。感情更好 激情不哀。这是第二个组成部分。健康爱情的第三个组成部分是。冲突是不可避免的。人们往往误以为。理想的爱情没有冲突。这是不可能的。除非双方都在刻意躲避严重问题。所以爱情中时有冲突发生。当然在一段恋情中 我们的挑战就是要。让积极的大于消极的。而且要学会如何应对分歧。应对冲突。最后。第四点是积极认知。要做优点感知者 不仅如此。还要创造优点。 

我快快回顾一遍。然后细讲一些上回没说的东西。关于爱情中的冲突。我觉得我读过的。关于爱情最重要的文章。是艾默生的《论友谊》。发表于1841年。艾默生在其中写了他理想的朋友。我来读一段。他说 "在朋友身上。我寻找的不是盲目的让步。对我千依百顺的人。我寻找的是一个美丽的敌人。能挑战我 敦促我。帮助我寻求真相"。美丽的敌人 多美的描述。没有冲突的爱情。就没有美丽的敌人。美丽的敌人是指因为爱我们。关心我们 所以要跟我们针锋相对的人。问问你自己 你想找的。是什么样的朋友或伴侣 是百依百顺的好好先生或太太吗?还是一个诚恳待人。直言规劝的诤友。理想的朋友 理想的伴侣 是什么样的?有趣的是。美丽敌人这一概念最早可以追溯到。西方世界。最有影响力的文本 即圣经。在创世纪中 上帝看到男人独居。于是为他造一个配偶帮助他 一个女人 helpmeet。helpmeet这个词是什么意思? 这是钦定版圣经的翻译。如果你看希伯来原文。"helpmeet"的原文是"ezer kenegdoor" 对立的帮助。也就是说 helpmeet中meet的意思。类似运动会的会 也就是竞争。不是指帮助并取得共识 而是对立的帮助。所以那时人们已经注意到 理想的爱情。不是一帆风顺 没有冲突的。而是有阻力的。这种阻力就是Gottman曾经谈到…。本世纪一直谈到的 对于健康的爱情关系。非常重要且关键的。一个美丽的敌人 对立的帮助。当然还要有积极的态度。 

我们继续讲积极认知。我想给大家放一段视频 是我最爱电影的片段。我认为这部影片是心理学领域。最成功的一部影片 《尽善尽美》。这段视频中海伦.亨特告诉杰克.尼克尔森。他最好称赞一下她。而且要称赞得很好听。否则她就马上离开。他之前刚说了非常冒犯她的话。于是他对他爱的女人这么说。 

(视频:电影《尽善尽美》片段)。 

好了 现在 我是有些称赞你的话 而且都是真心话。我担心你会说出很难听的话。悲观可不是你的风格。好了 我要说了:确实 我错了。我得了 怎么说…小毛病?我的医生 我常去看的心理医生…。说五到六成的病例…。服药就会有效。我憎恨吃药 药很危险 恨死了。我说药时用的是"恨死了" 恨死了。我想称赞你的是。那天晚上当你来我家 告诉我你决不会…。好吧 你当时在场 说过什么你知道。我要恭维你的是…。第二天早晨我开始服药。这算哪门子的称赞。你使我想成为更好的男人。这是我这辈子听过最好听的称赞。也许过头了。因为我只想着不让你走出去。 

这就是创造优点的本质。你使我想成为更好的男人 更好的女人。一个更好的人。你使我想自愿洗餐具做家务。 

那我们该怎么做?如何让爱情关系或伴侣。关注积极正面的东西。关注优点 创造优点?这就得回到第一堂课的内容。也就是提问的重要性。记得 提问是探求的开始。当我们问问题时。会留意到以往忽视的东西。我们问问题时。也会忽视一些现实情况。还记得几何图形吗。你们只看到了图形 没留意颜色。没看到公车上的孩子。大多数人没看到钟。在爱情关系中。我们在蜜月期之后常问的问题是。"出什么问题了 怎么会这样 怎么改进?"。重申。这些问题很重要 出发点是好的。但是我们回避或忽视了一些重要的现实。我们需要问的问题。应该是积极的。能让我们看到公车上的孩子的问题。 

我的伴侣有哪些优点 让我觉得感激。经常这么问问自己很重要。尤其是遇到矛盾时。因为总有些东西值得欣赏。如果我们不懂得欣赏 那优点就会贬值。这个道理对我们国家 组织。爱情及自己同样适用。 

第二个积极正面的问题是。我们的关系有什么美妙之处。我们怎么会走到一起的?我爱她 他 我们的哪一点?有哪些好的方面?我们问"哪些好的方面"时 我们就看到了好的方面。当我们看到 欣赏这些好的方面时 它就增值了。我们并不感到惊讶。上周看到的统计数据显示。大多数恋情都是苟延残喘。即使勉强在一起 也没有什么感情可言。这不是巧合。这是因为我们问的大多数问题。虽然意图是好的。当然了 没人谈恋爱。是为了分手 或原地踏步。但是我们问的大多数问题。我们受到的教育让我们问的问题。都集中在几何图形上。我们要做的是 换一个角度来探索这个问题。看到那些一直都在。但却被我们忽略了的东西。引用Robert M.Pirsig。在《万里任禅游》的一句话。我们常常追寻真相。当真相叩响我们的大门时。我们却说 "走开 我寻求的是真相"。事实上 真相往往就在我们眼前。我们要做的就是仔细看。而引导我们注意到。眼前的东西的一个方法。就是提出积极正面的问题。 

好了 目前我们讲了哪些。创造积极的爱情关系。做一个优点创造者。我们要关注潜能。还有一件事我们能做到。也就是多沟通积极正面事件。 

这是加利福尼亚大学的Shelly Gable研究。在我看来 她做了积极心理学领域里。与爱情关系课题有关的最重要的研究。有很多研究。是关于消极沟通的。什么意思呢 我们上节课讲过如何处理分歧。不要把分歧上升到认知层面。不要把分歧恶化为情感上的分裂。要针对实在的行为。而不是针对人和感情。我们知道。有很多关于组织行为学。及婚恋咨询的研究。但对情侣间积极沟通的研究。几乎为零。这和心理学领域的大多数研究一样。还记得21:1的比率吗。现在有所提高了。大约20:1 积极研究与消极研究的比例。爱情领域也一样。很不幸 没有例外。Shelly Gable认为。"我们还需要关注有用的东西"。于是她开始研究积极沟通。积极事件。注意 她的发现非常了不起。她发现。一对伴侣如何沟通积极事件。比起如何沟通消极事件。更能预测到这段感情能否天长地久。事情顺利时 比事情不顺时。更可能预测婚恋关系能否长久而甜蜜。所以我回家告诉我太太。"哦 我今天做了这事 太棒了"。或者"我看了这部电影 太激动人心了"。或者其他工作上的事。都是我生活中发生的点点滴滴。这是好事。而我太太如何回应。能预测到这段感情能否天长地久。Gable的意思是指。主动的有建设性的回应。 

什么是主动的有建设性的回应?她分类得出了二行二列的表格。我现在就给大家介绍下。这是直接抄自我三位同僚的著作中。来自澳大利亚的Jane Elsner Barbara Heilman和Amanda Horn。她们是这样进行的。把Shelly Gable的理论总结成了一张浅显易懂的2x2表。表的横行是有建设性的沟通。与破坏性的沟通。纵列是主动沟通与被动沟通。我们通过一个例子来说明四个格子。 

我太太下班回家 说。"我升职了。这个职位我等了好久。终于如愿以偿了"。我如何回应? 如果是被动的破坏性的。我会表现出兴趣缺乏 注意力分散。我会说"嗯嗯" 然后讲其他的。"你看到花园里种了新花了吗"。说些完全无关的话题 分散注意。 

这就是被动的破坏性的回应。升职后 她回家。我回应"哦 不。这样的话我们在一起的时间就更少了。孩子们怎么办?"。或者"度假的事怎么办?说好下个月去的 是不是去不了了?"。主动的破坏性的回应。 

然后是被动的建设性的回应。也是最常见的回应。"哦 太好的 太棒了 嗯"。 

然后是第四个框 主动且有建设性的。也就是"你升职了 太好了。告诉我 跟我说说 过程是怎么样的?是不是老板叫你去他办公室告诉你的?到底是怎么样的?"或者"我们得好好庆祝庆祝"。又或者"打电话叫朋友们出去庆祝。太棒了 干得好 你这么努力。我真为感到高兴"。主动且有建设性。 

结果显示 不同的回应。会带来不同的影响或后果。很可惜 主动且建设性的回应并不常见。在情侣之间并不常见。尤其是过了蜜月期后。所以一开始 生理上的新鲜感。刺激感强烈 此类回应很常见。但一段时间后 就消失了。Shelly Gable发现 这种回应一消失。这段关系就很难长久下去。 

主动且建设性的回应。适用于我们生活的方方面面。不仅是爱情关系。如何与室友沟通。如何与家人 父母 孩子沟通。作为心理医生 如何与病人沟通。因为心理学的初衷。就是为了医生坐在这儿。被动但有建设性地回应 微笑。说 "嗯 不错 然后呢"。而不是主动且有建设性地回应 这会引起不良后果。 

这并不是说。这在任何时候都是最正确的方法。因为这种回应方法是有些使用限制的。比如 必须是双赢事件。比如说我太太下班回家 说。"Tal 你不会相信的。我刚和同事疯狂做爱了。他太不可思议了 我从未有过这种感觉"。主动且有建设的回应:我要知道细节 然后呢。对大多数人来说 这种情况下。很难做到主动且有建设性。包括在座各位 很难 所以这个回应是为了双赢。至少我看来 刚才这个例子不是双赢 必须是双赢。换句话说 就是要我们双方。恋爱双方都能获益的 或至少。另一方不会因此受伤害。 

必须是真诚的。不能是假的 因为那样…。尤其是一旦伴侣发现。慢慢地就会出现问题。所以必须是真诚的。 

我们必须进入伴侣的角色中。产生认同 共鸣。设身处地 这样才能产生真诚的回应。如果我们能做出真诚的回应 如果这是一个双赢事件。就会带来几个良性循环。 

首先 记得吗。通常当积极事件发生时 会怎么样。会有一个峰值 然后情感预测。根据Gilbert的研究 会逐渐回落。有效的建设性回应。比如"好 再说说。当时是怎么样的?我们庆祝吧 告诉我细节。我想知道 我很有兴趣"。这样的回应会延长峰值。会使它持续更长的一段时间。而不是仅仅一个积极事件那样。很快回落到基本水平。因为无论我们获得什么升职。一开始感觉很好。无论什么情绪上涨 一开始感觉很棒。然后又会回到基本水平。而主动且有建设性的回应会延长。幸福感的提升与增强。所以这是其中一个良性循环 更好的感觉。另一种良性循环 是人际关系的。也就是说作出真诚主动有建设性回应的人。会获得相同的良性循环。也就是说。真诚回应的人会变得更快乐。因为他们也经历了这一事件。他们自己也从中获益 变得更快乐。所以又是一种双赢。最后一点 它和积极心理学的整体作用一样。累积积极正面情绪。记得吗 积极心理学。不仅让我们从0变得积极。还帮助我们更好地把消极变为0。作出主动且有建设性回应的伴侣。是在为艰难时刻累积积极正面情绪。事实上 很久以前。有一项有趣的实验。当生活顺利时 夫妻双方…。虽然当时的研究没有用这样的术语 但意思上是相同的。1920年代 夫妻双方能主动。且有建设性地回应对方 支持对方。大萧条发生时。他们的婚姻反而变得更坚固了。而没有做到这点的伴侣。他们的婚姻往往会瓦解 变糟。换句话说 他们累积正面情绪。应对困难时期。比如20年代 30年代。 

幸福的爱情 健康的爱情。没有捷径。如果你想要成功 就需要努力。跟生活的其他领域一样。没有捷径可走。但是 这不是说 努力就是痛苦的。它可以意义非凡且充满乐趣。当你的努力是充满乐趣 意义非凡时。渐渐地 就会使爱情更幸福。使你们更快乐 建立双赢的伴侣关系。 

结束这部分前 我还想说一点。你们很多人都问过我。当我们讲追求幸福时。我们应该讲恋情吗?大多数人的回答是 应该 但有所保留。说"是"的原因是 爱情是幸福的最重要来源。有所保留的原因是。追求幸福的本质是自私的。我的意思是 我思考我的幸福时。毫无疑问 我想的是自己。难道恋情不是与自私对立的吗。换句话说 恋情是利他主义的。关于这点我有几句话要说。就拿教书作为例子。想象一下 假如我不喜欢教书。我不喜欢教书。不喜欢每周二周四来这里。对我来说非常痛苦。但是 出于我受到的教育 我学的哲理。我有很强的责任感。我对别人有很强的责任感。我知道积极心理学。给教给别人一些很重要的知识。于是我来上课 不是因为我喜欢教书。而是因为我认为它能帮助到别人。我每周二周四早上醒来。痛苦地来到教室。一个半小时的课让我度日如年。我最不想做的事就是教书。但是我的责任感驱使我这样 因为我是利他主义者。我为伟大的事业牺牲自己。再想想另一种情况。想象与刚才那位老师相反的情况。就是我。我爱教书。我最喜欢的地方就是教室的讲台。我爱积极心理学。我等不及要打开电脑备课。等不及准备幻灯片。搜索Google图片。我等不及要与你们互动。我喜欢每周二周四站在这里。这是我热爱的 我的使命。和你们在一起上课 对我来说。意义非凡又充满乐趣。那么你们想要哪一位老师。是出于责任和利他主义动机。才来上课的老师。还是那个为了满足自己。喜欢上课这个爱好而来上课的老师?另一个问题。你觉得哪个老师会是更好的老师。引申开去 你们会想要怎样的伴侣。一个跟你在一起是因为他们觉得。"这人很需要我"。"这家伙真可怜"。"我不是真的喜欢他。我只是出于责任感 支持他"的人?还是跟你在一起是因为。你是他们生命中最重要的人。因为他们如此关心你 从"我"变成了"我们"?他们像关心自己一样 关心你。把你纳入了自己的一部分。你们认为哪种爱情会成功?这不是说。爱情中不会有牺牲。也不是说 如果我伴侣身体不适。或者需要我帮助 我不会全力以赴。即使那意味着需要放弃。我当时的梦想。当然了 在健康的恋情中。当我变成了我们 就会有牺牲。但这是健康的牺牲。这样的恋情。它的基础不是责任感。不是否定自我的利他主义。Nathaniel Branden说过 我们越独立。就越互相依赖。当我变成我们时 就会这样。这与我们所讲过的内容相反。不是很容易接受。但是这门课不仅是关于积极心理学。还有现实心理学。问问你自己 想成为哪种伴侣。想拥有哪种伴侣。想要哪种老师 举一反三。 

好了 我们来讲本课程最后一个话题。我说过很多次 我从不拖延。直接用了Google图片。 

当我还在哈佛这里读本科大四时。就开始非常认真地思考自尊这个话题。当时我注意到一个现象。那是我转专业到哲学和心理学后。我对自己的学业很满意。在壁球队里也很开心。哈佛生活对我来说一帆风顺。但后来我发现一些很费解。无法理解的事情。我自尊程度相当低。至少看起来是这样。我无法理解。因为我表现不错 得过奖。屡获表扬 支持。有来自室友的 朋友的 家人的。运动还是学术都屡获嘉奖。但我的自尊还是很低。这让我很不解。还有一件事更让我困惑。在某种程度上 我觉得我的成功在惩罚我。这是什么意思?每次我获得成功 或者被表扬。我的自尊就会增加 改进。但很快。又回到了基础水平 甚至更低。为了重新提升它。我就需要更多表扬。比以前更多的赞赏。有一个比喻不断出现在我脑海里。科林斯王推石头上山的故事。我就好像在推石头上山。挣扎奋斗直到获得成功。成功后我得到表扬。获得赞赏 奖品。我的自尊提升了。我处在了山顶。然后石头又滚了下来。我不得不跑下山重新再推。而且这次 山坡更陡峭了。我不得不更加努力 才能到达山顶。然后石头又滚下山。以前那些赞赏 表扬。已经不够了。我需要更多才能重新回到山顶。更陡峭 更困难。我想不通。到了大四 我选修了一门课。出于诸多原因 那成为了我最爱的课程之一。童话课 原因我现在不想细讲。我选修了童话课 我们读了卢梭的《爱弥儿》。Maria Tatar讲到了很多精彩的理论 我想通了。于是我的毕业论文。那门课的期末论文写的就是自尊。最终发展成了我的学位论文。我的博士论文写的就是这个话题。所以今天我想和大家分享。我在童话课上的感悟及其他思想家。对于自尊话题的思考。因为这非常重要。我以自尊的话题结束这门课程。不是毫无缘由的。因为它是基础。是我们之前讲过的很多内容的核心。无论是健康的婚恋关系。还是快乐。无论是寻找优点 还是创造优点。无论是被了解 还是被认可。自尊涉及方方面面的内容。 

我们要讲以下内容。首先 为什么自尊非常重要。对于自尊的种种误解。可以运用到生活中的。一些理论。然后我们会细讲我刚才提到的矛盾。有时候成功。会降低自尊。为什么会这样 我们到时会解释。然后我会介绍我关于自尊的学位论文。我把它分为三个清晰的部分。依赖性自尊。即由他人表扬和认同而产生的自尊。独立自尊 即内在产生的自尊。这种自尊不取决于。别人的评价 是自我生成的。最后 无条件的自尊 也可以称作一种自然状态。我们自然的存在感。我会讲到这三个层面的自尊。以及如何培养独立自尊。及无条件自尊。最后 下节课我们会讲到。如何提高自尊水平。很多人参加自尊讲座后。会进行自我评估 认为。"好了 我的自尊高吗?" 我说的"我"是说你们。或者"我的自尊比旁边那人高吗?比我伴侣或朋友高吗?"。回答是 你无法得知。因为我们无法客观地衡量自尊。我们还不知道它处在我们大脑哪个部位。也许10年20年后可以。但这个问题不重要 就像"我快乐吗"一样。也不是个重要问题。它其实是一个误导人的问题。记住 你要问的应该是"如何才能变得更快乐?"。自尊也是如此。你要问的不是"我的自尊是高是低"。而是"如何才能提高自尊"。因为我等一下会讲到 自尊越高越好。但有时候人们把自尊与自大。自负 自恋过多地联系在一起。这些不是自尊。恰恰相反 是缺乏自尊。所以强烈的自我感是件好事。问题是"如何提升它?"。今天 我的自尊比15年前。我刚考虑这个问题时提升了很多。希望5或15年后。会比今天还要高。这是个终身的过程。我们会谈这个终身过程是如何展开的。如何才能加快 加强这一过程。 

首先 自尊的定义。自尊有多种定义。我读几条重要思想家对此的认识。斯坦福大学教授Albert Bandura。解释了通向自尊的大门。说到人们是如何评估自我的 他是这样说的。"那些认为自己没有价值的人。会被视为自尊较低。那些表现出自傲的人。会被视为自尊极高"。这一领域另一位著名思想家。Germain认为自尊是。"对于自我的评判与感受"。非常简单 对于自我的评判与感受。Coopersmith 该领域最重要的思想家之一。"自尊是个体作出的。并经常保持的对自己的评价。表达了一种对自己赞许或不赞许的态度。标志了个体对自己能力 身份。成就及价值的信心"。能力 身份 成就 价值。"简而言之 自尊是对自我价值的评判。通过个人对于自我的态度表现出来"。所以自尊说的是我对自己的态度。是对自我概念的评估。我对自尊作出的定义 在我的论文里也用过这个定义。我认为这个定义最成功地道出了自尊的本质。是Nathaniel Branden的理论。Nathaniel Branden。被很多人称为美国乃至全世界的。自尊运动的先驱。身为心理医生及哲学家。他在这一领域研究了50年。他对自尊做出了如下定义。"一种觉得自己。能够应付生活中的基本挑战。值得享受快乐的感觉"。 

两个组成部分 能力感。价值感。两者都很重要。缺少任何一个 自尊就会很低。个人体会: 我觉得自己能胜任手头的工作。无论是在壁球场上。还是婚恋关系中。还是学术领域 但是我的自尊仍旧很低。因为我缺少自尊的。价值感这个组成部分。所以光有一个是不够的 要两者兼备。 

Nathaniel Branden继续说。"我们在生活中作出的所有判断。没有哪个比对自己作出的判断更重要了"。为什么 因为我们时刻有自我意识。它影响我们生活的各个领域 我们孤身一人时。与别人在一起时 工作时。独自工作时 与人合作时。歌德说过"降临于人最大的邪恶是。让他否认自己"。所以高自尊非常重要 或者说。自我感很重要。但是如果我们环顾四周。反省自身 会发现还有很多改进空间。有些人多 有些人少。但每个人都有一定的改进空间。 

以下是高自尊的一些益处。首先 心理健康。心理抵抗能力和应付困难的能力加强。从6岁孩童到96岁老人都是这样。纵观一生 自尊程度越高。心理抵抗能力越强。能更好地应对焦虑 抑郁。各种无法避免的困境。改善人际关系。这点非常适用于。婚恋关系 我们稍后会讲到。同样适用于友谊。和家庭关系。 

Branden说 自我概念就是命运。信念或自我实现预言。如果我相信自己 相信自己值得快乐。相信自己有能力。那么我成功的可能性更大。还记得John Carlton提出的。哈佛商学院研究生成功的两个因素吗。第一条是他们总是问问题。总是想学习更多。第二条是他们相信自己。这使他们在其他成功者中脱颖而出。自我概念即命运。Daniel Goleman普及的情商概念。最早是由耶鲁大学的Salovey。和哈佛大学的Gartner提出的。我们对他人的情感。我们的人际智能 这些合起来。使得自尊高的人。情商高很多。最后 快乐。无论是我的研究 还是其他人的研究。自尊与快乐的相关系数都超过0.6。这一数字相当高 虽然它不是快乐唯一的决定因素。但绝对是主要决定因素之一。另一方面 低自尊常常与焦虑并存。我说的不是自然的焦虑。如果我站在悬崖边。往下看时感到焦虑 这很自然。这是健康的。人的天性 经历这种焦虑很重要。这很自然。人人都会感受到 就像万有引力定律。是我们的生理天性之一。焦虑是人的天性之一。不健康的焦虑。是指毫无缘由的焦虑。Branden称之为自尊焦虑。半夜突然醒来。感到毫无缘由的焦虑。或者日常生活中经常感到一种焦虑感或恐惧感。却不明白是为了什么。这通常是低自尊的症状 抑郁。我们时刻有自我意识。我们时刻关注自己的脑袋。我时常在评估自己。如果评估显示不够格。如果评估很低 通常会导致抑郁。身心失调症状包括失眠 患病。因为我们的免疫系统变得脆弱。生理免疫系统变得脆弱。自我概念即命运。它是我们取得最高效率的原因。也是我们取得最差表现的原因。成绩差的学生通常…。自尊较低。 

Nathaniel Branden把自尊称为。"意识的免疫系统"。意识的免疫系统。我们自尊较高时。心理抵抗能力更强。记住 强健的免疫系统并不意味着不会生病。而是较少得病。如果生病了 能更迅速地恢复。意识的免疫系统。强健的意识疫系统 意味着我们常快乐少生病。意味着我们追求快乐。而不是躲避不快乐。这些是高自尊与低自尊的区别。 

心理抵抗能力。Bednar与Peterson。该领域两位研究者提出了一个问题。"大多数心理问题或精神病。根本病因是什么 或者说有没有根本病因?"。他们提出了一个构想。即自尊。自尊至关重要。首先 在理解层面。理解各种情绪及行为问题。其次 治疗大多数 并不是所有。情绪及行为问题 自尊都是核心。最后 它不仅适用于个人。还适用于社会层面。政府成立的加州健康专责小组。将自尊称为社会疫苗。因为他们发现自尊与…。低自尊与药物滥用。未成年怀孕 辍学 暴力及犯罪有关。而高自尊恰恰相反。能帮助克服这些社会顽疾。虽然不是万灵药 但在各个方面起到了积极作用。所以自尊是好的。非常重要。但在研究自尊这个领域里 不是所有人都意见一致的。有一些反对的声音。有些甚至是该领域重要的思想家提出的。其中之一是 佛罗里达州的Roy Baumeister提出的。他将自尊等同于自大自负。因为如果你进行关于自恋的问卷调查。会发现自恋程度高的人。自尊分也非常高。但问题在这里。你们阅读的论文。至少可以反驳这种批评 当然还有其他的。如果一个人自恋自大。自尊程度其实不高。这是常识。一个像高傲的孔雀一样走进教室。摆现炫耀。这人一定是高自尊吗 不一定。一个谦虚不摆现的人。往往自尊程度更高。正如Rolly May。人本主义心理学运动创始人说过。"软弱的人变成恶霸。自卑的人变成吹牛大王。炫耀武力 夸夸其谈 骄傲自大。厚颜无耻。都是个人或集体隐匿的焦虑症状。这与自尊恰恰相反。可惜 如今我们评估自尊的方法。都是通过问卷调查进行。如果你问一个自恋者 "你自尊高吗"。回答当然是 是。但是当今大多数。问卷调查还没有复杂到。足以区分真自尊。和Nathaniel Branden所说的伪自尊。自我效能与自尊心的伪装。不久的将来我们将能更客观的衡量它们。这是把自尊等同自恋的批评 但是还有其他批评。其中一个认为 自尊。高自尊会使人过分高估自己。最终对人造成伤害。我来读一段时代周刊的报道。这篇报道是讲全世界各国13岁孩子。进行的一次测验。去年 6个国家的13岁孩子。做了一次标准化数学测验。韩国孩子表现最好。美国孩子表现最差 排在西班牙 芬兰 加拿大之后。坏消息是 除了三角与等式外。试卷上还有一道判断题"我擅长数学"。美国孩子居然有68%选"是" 排在第一。美国孩子可能不懂数学。但显然他们对新式的自尊课程。掌握的非常好。这些课程教会孩子们自我感觉良好。这些课程教会孩子们自我感觉良好。我还要再加句"无论做什么"。因为很多老师。不分青红皂白地给予赞美 为什么?因为自尊是好的 是社会疫苗。有很多研究支持这一点。那如何提高自尊呢。要不停跟他们说"哦 你真厉害。你太棒了 你真行"。但结果显示 这对学生们并没有帮助。事实上 长期来看…短期内感觉良好。但长期来看反而降低了他们的学习动力。让他们变得不切实际。长期来看。反而减低了他们本来可能获得的快乐。产生了反效果。所以 意图是好的。因为老师们想提高学生的自尊。 

那如何拯救自尊呢?需要先理解它的真实性质。首先 自尊不是空洞的心理强化能够产生的。哦 你真棒 你好厉害。你真行 不管是不是都赞扬。长期来看还会损害自尊。自尊必须与伪自尊区分开。也就是脱离现实的。虚假的自我效能与自我尊重。那些不是自尊。那是自恋 脱离现实。我们最近反复提到。健康心理是要联系现实。所以什么是自尊呢 自尊存在于现实中。存在于真实的行动 真实的成功 与真实的实践中。它是努力的产物。好了 还记得2个月前我们讲到的成功的秘诀吗?乐观 相信自己 对所做之事充满热情。还有一点 努力奋斗。这就是秘诀。这些都是我们早就知道的。都是基本常识。引用伏尔泰的话"常识不一定人皆有之"。 

Nathaniel Branden提出了培养自尊的。6条重要实践。首先 正直。你们本周读书报告主题就是。如何提高正直意识。正直是指言行一致。小事大事。无论是"我5点过5分到"。那就5点过5分到。还是"我这周去健身"。那就要做到。因为我们沟通 说出每个词时。如果言行不一致。那我们本质是在对自己说。我的话无关紧要 不重要。反之 我保持高度正直。遵守诺言 当然没人是完美的。没人会是100%的正直 我们下次会讲到。但是我真诚的遵守诺言。言行一致。以低要求许诺 以高要求履诺。就是以行动对自己说。还记得 自我知觉理论吗 通过行动。我对自己说 "我的话很重要 我很重要"。所以要正直。要有自我察觉 了解你自己。也就是Branden说的自我意识。目的性 有目标 自我协调的目标。努力追求 有目标有使命地生活。担起责任 记住 没人能帮你。只有靠你自己创造生活。要自我接纳。每天都要允许自己有人的天性。最后 要有主见。该说不的时候说不。该说是的时候说是。支持我们信仰的东西。这些实践经年累月能培养我们的自尊。它们也是自尊的产物。我们下次会详细讲。这是个潜在的自我增强循环。这就是自尊 但还有一些问题需要解答。自尊领域还有很多问题。还有一项重要的批评。很多研究表明。高自尊的人更慷慨。对他人更博爱。总得来说他们都是好人。许多研究都表明这点 这很棒 很好。但是。还有很多研究结果恰恰相反。高自尊的人表现出反社会行为。无论是对不认识的人。尤其是对特别亲近的人。它与攻击性不合作的行为。有所关联。一些研究显示敌意程度。也与高自尊有关。这些证据自相矛盾 无法解释。自尊到底是怎么回事。这个问题直到最近才弄清楚。对自尊的另一项批评 之前我提到过了。自尊的矛盾。我在大四时 在哈佛求学期间。受到的表扬。基本上是基于事实的。我壁球打得很好 得到了赞扬。学术成就优异 得到了表扬。所以都是真实的 基于现实的。但是我还是觉得自尊很低。事实上 我获得的赞扬越多。越觉得石块越重。山坡越陡。这点传统的自尊研究无法解释。Bednar和Peterson谈到了这个矛盾。他们在书中举了例子 这是我第一次看到。有人谈论。我这么多年的经历。他们谈到了成功往往。与低自尊共存的矛盾。我读一段他们一位病人写的话。他俩都是心理医生 非常成功。他在日记中这样写道。"过去几年 我觉得生活毫无意义。这种状态非常讽刺。因为我事业有成。亲朋好友总是赞美。我的外表与智慧。父母双方家中 我都是第一个。大学毕业生。而且是优秀荣誉毕业生。我在大型会计事务所有一份好工作。不缺女伴。表面看上去 我的生活非常满足。但我仍旧感到痛苦 日益沮丧。6个月的心理疗法帮助我。瞥见了。我不快乐的潜在原因。似乎与我的低自尊。或自我否定有关。不知怎么 我似乎对自己评价很低。这点让我很困惑。因为我生活很成功"。Bednar与Peterson谈到了。自尊的矛盾。这是他们的叫法。自尊与成功没有关系。与社会地位 金钱没有关系。而是与其他因素有关。事实上 这一矛盾经常显示。越成功遇到的困难越多。我希望对这种困难进行研究。并且的确进行了研究。这就是我之前谈到的公式。第一层自尊 是依赖型自尊。因为我觉得自尊是一个构想。这还不够。还有各种矛盾。还有自尊无法解释矛盾。所以我把自尊分为了三个部分。根据Lovinger的理论。1960年代 她的研究是建立在Maslow。Carl Rogers以及Nathaniel Branden理论的基础上的。他们都谈到了这些问题。但是我们仍是小众。大多数人讲到自尊时。仍把它当做独立的概念。第二层自尊。不取决于其他因素的自尊。第三层自尊 本质上。甚至不必把它称为自尊。而是一种自我感。让我一个个来细说。我分析每层自尊时。会从自尊的两个组成部分去讲。所以本质上是建立了一个2x3列联表。三层自尊。每一层与价值感 或者说自我尊重的关系。与能力的关系。Nathaniel Branden理论里 自尊的两个组成部分。首先。从价值感方面来看。高依赖型自尊的人 价值由他人决定。他们喜欢 也需要别人的评估。如果我讲座表现不错。收到他人反馈让我感觉良好。如果没得到正面反馈 我会感觉很糟糕。如果我考试表现优异。得到助教或家长肯定 就感觉棒极了。如果有人说不喜欢我的某些言论。我会感觉糟透了。我的生活不断受到他人思想言论的影响。甚至是我以为他们怎么想 他们的意思。不断的评估 其他人是怎么看我的。仿佛他们是一面镜子。仿佛从他们身上能看到镜中的自己一样。把他们的评估当作自我感。高依赖型自尊的人主要。不是全部。但主要是由他人的想法言论驱动的。我会从事高声望。高社会地位的工作。能给我带来最多的赞赏与表扬。我选择伴侣。会选择受大多数人喜爱的人。他们的自我感 取决于他人。重要的决定。都是根据别人的赞同或不赞同而做出的。 

第二 能力感。根据Branden理论 自尊的另一个组成部分。来源于比较。跟其他人比 我表现怎么样。如果考试中 我比其他人考得好 我会感觉很好。如果其他人考得比我好。不管客观来说有多好或多差。重要的是我跟其他人比表现怎么样。William James抓住了这一点。他说 "我自以为掌握了心理学一切。如果有人比我更了解心理学 我会感到被羞辱"。与他人比较。事实上 大多数心理学家都把自尊。等同于依赖型自尊。Cooley在他的著作中谈到了镜子的比喻。1934年Mead的著作。谈到了依赖型自尊。我们根据他人言论或看法来评估自己。或者我们认为他们对我们的看法或言论。大多数思想家都谈到了这点。这点形象地在童话中展示出来了。我第一次上童话课时就想到了。白雪公主里的邪恶皇后生动的表现了这点。"魔镜魔镜告诉我 谁是世界上最美的女人"。只要魔镜说。"你 皇后陛下 你是世界上最美的女人" 她就觉得心满意足。一旦有人比她更漂亮。她就觉得蒙受奇耻大辱。说明了这种自尊的两个特点。首先 由他人决定。我问镜子谁漂亮 而不是由我自己决定。所以我的价值感来源于外部。第二 通过比较而获得能力感。有人比我更美吗 依赖型自尊。现在我似乎把依赖型自尊贬得一无是处。让我再说的清楚点。每个人都有依赖型自尊。我从未遇见或听说任何人。能完全不受他人想法言论。或社会比较的的干扰。这是人的天性的一部分。要支配天性 就必须服从天性。反抗天性 只会让其加剧。这学期我们已经看到过很多这方面的例子了。允许自己有人的缺点。依赖型自尊也是人性的一部分。问题在于程度。理解这个模型的一个关键是。这三个层面是一个渐成模型。渐成模型。我也不知道这是什么意思。渐成是指必须通过第一阶段。才能到达第二阶段。必须通过第二阶段。才能到达第三阶段。不能越过某一级 比如。从第一阶段直接到第三阶段 跳过了第二阶段。所以每个人都有依赖型自尊。但是 如果我们有健康的依赖型自尊。这很自然 我们小时候都经历过。一段时间后就会变成独立型自尊。如果我们能培养健康的独立型自尊。就能达到无条件自尊。达到互相依赖型 也就是最高层次。但是即使在最高层次 其他两种还是存在的。我们会一直拥有依赖型。及独立型自我感。研究一开始时。我想把独立型与依赖型对比。我当时还没有想到无条件型。想当然地认为独立型好 依赖型坏。我试图摆脱依赖型自我。结果它反而强化了。变得更强烈。我变得更加依赖。然而 一旦我开始接受它。记住吗 我说过 允许自己有人的天性。一旦我接受它为人性的一部分。对我的影响反而小多了。人人都有依赖型自尊 只是程度不同。第二个层面是独立型自尊。顾名思义。这种自尊不取决于他人。在价值感方面。我用自己的标准评估自己。自我决定。我自己评定写的书好不好。最终的公断人是我自己 但我会参考。和听取其他人的意见。但是最终还是由我自己说了算。在能力感方面 不与别人比较 而是与自己比较。我进步了吗 是不是比过去写得好了?我的数学是不是比。学期初进步了。我是不是比一开始能运用更多的。积极心理学知识了?不与室友比较。无论他懂的比我多还是少。我只关注自己。很多人都提到了这种自尊。Nathaniel Branden Abraham Maslow与Carl Rogers。还有其他很多人。有依赖型自尊的人。需要不断寻求他人肯定。害怕批评。经常与完美主义相连。而拥有独立型自我感的人。寻求批评。这样的人其实不断寻找"美丽的敌人"。那些挑战他。帮助他寻找真相的人。因为他们想要进步。他们的动力 主要动力。是"我热爱什么?" "我喜欢什么?"。"我真正想做的是什么"。他们想追求自我调节的目标。我的一个朋友Maltimore Devano。他是一位哲学家 写到自尊时曾说。他是这么说有依整型自尊的人。我们说 想寻找真相。真正的意思是 我们希望自己是正确的。我们说 想寻找真相。真正的意思是 我们希望自己是正确的。许多依赖型自尊的人说。"是的 我正在追求真理。追求高尚"。但是实际上我们的动机。是为了保护自己免受批评 免受负面评价。避免这种依赖他人的自尊变得更低。所以 不同于那些说想寻求真相。其实只希望自己是正确的的人。拥有独立型自尊的人。想寻找美丽的敌人。他们才是正真追寻真相。 

总结一下 最初的2x2表。依赖型自尊 价值感由他人决定。我做一切主要是为了赞赏 表演。能力感 与他人相比我做得怎么样。比他们好还是差。这决定了我的自尊。而独立型自尊的价值感和能力感是。拿自己跟自己比罗 我进步了吗。改进了吗 学到新东西了吗?我的价值感是由我自己的评估决定。 

第三个层次 无条件自尊。从多个角度看 这甚至不算是自尊。因为其中没有涉及自尊。第一部分 价值感。不取决于他人评价。也不取决于自我评价。我有充分的自信 不参与任何评价。其次是能力感 互相依赖。不把自己与他人比较。也不与自己比较。我处于某种状态。与他人互相依赖 同时又怡然自得。比如。一个高依赖型自尊的人写书。他的动力主要是将要获得的。赞赏和表扬。高依赖型自尊的人。还会把他的书与其他人的比较。高独立型人格的人写书。他的快乐与满足的来源是。通过改进写得比过去好了。还有对书的自我评价 这本书写得真好。而无条件型自尊的人只是写书。为什么? 因为一旦他们。有灵感 就会写出一本好书。他们从他人的好书中。也能获得同样的满足与乐趣。他们互相依靠。他们视自己为团体一员。所有人都有一定程度的依赖型自尊。一定程度的独立型自尊。以及一定程度的无条件自尊。问题在于程度 模型是渐成型的。通常需要一辈子时间。把无条件自我培养得越来越强。我和其他赞同无条件自尊的学者。受到的批评之一是。要达到这种状态 "我不在乎。不是很在乎我受到批评还是评价"。就会变得与他人脱离了 对什么都不在乎。但情况不是这样。瞧。无条件自尊与佛教的超然。有许多相似处。我给你们读一段。图丹却准法师的语录。"通常。我们受到批评或侮辱时 会变得非常不安。我们财产被偷时 会变得异常愤怒。如果期望已久的升职被别人抢去 我们会嫉妒。我们以自己的外表体能为荣。相比之下 当我们达到了超然的境界。我们的头脑变得更明晰。懂的享受事物的原本。我们懂得把握现在 欣赏事物的原本。不再对它们有诸多幻想。我们对别人的一举一动。不再那么敏感"。同时 超然或无条件自尊。我正比较两者的相似点。它们并不意味着冷漠。回避他人情感。她继续说"相反地 当我们超然后。我们与他人的关系变得更和谐。事实上 我们更关心他们了。变得更有同情心 更同情他人"。超然或无条件自尊只是指。我们让自己远离。羡慕嫉妒。高傲自卑。与他人对比的情绪。我们和他们融为一体。这就是研究结果。事实上 你可以把无条件自尊。比作看电影。你去电影院 然后呢。你与角色产生了共鸣。当他们经历艰难困苦时。你在心里也经历了一回。当他们获得成功时 你也觉得获得了成功。你和他们融为了一体。为什么? 因为你的自我不受威胁。他们的成功没有威胁到你。他们的美貌没有威胁到你。他们的成功对你来说不是威胁。为什么? 因为这只是电影。想象一下 如果我们能有这种情感。感受而不是对立。想象在真实生活中拥有这些情感。这样的人会有多强大?这是电影的经历。但对拥有高无条件自尊的人来说。会有多强大。培养自尊 无条件自尊需要时间。这是一生的过程。类似学走路。我们学走路是什么样的 一开始我们不会。一段时间后 我们会站起来 但需要帮助。我们依靠椅子 父母支撑我们。依靠其他人。下一阶段 又过了一段时间 我们独立了。能够自己走了。但走的时候仍旧要想着抬脚。而且觉得不安全。但我们是独立的。渐渐地 我们会自然的走了。不用再想着抬脚。自我感与之相似。我们刚出生时 没有自我感。一段时间后 自我感依赖于其他人。其他人的评价。然后 过了一段时间。我们变得独立 不受他人影响。试图坚持主见。但是我们不断的评估自己。比较自己。最后 又过了一段时间。我们培养出来强烈的独立感。自然存在了。这要顺其自然。而这需要时间。David Schnarch谈到普通人到了50岁。懂得被了解而不是被认可。这就是强烈的无条件自尊。Maslow认为人到了45或50岁。才能自我实现。这就是无条件型自尊。这需要时间 它是渐成的。但我们改进成长的这一过程。是可以很愉悦的。就像学走路一样充满乐趣。我们下周见。 

第22课-自尊与自我实现 

哈佛学习障碍协会。受到这门课的启发 我们将举行一个论坛。名叫学习缺陷。学习障碍和积极心理学。很多校内学生都有学习缺陷。有些甚至有学习障碍。包括学习困难和长期拖延症。而且 很多学习障碍。都与抑郁症高发率和低自尊有关。所以在这个论坛上 我们会问。"如何积极看待学习障碍?"。"我们如何用积极心理学。进一步探索这片研究领域?"。欢迎大家参加 不管是你有学习障碍。还是你的朋友。或者你最近遇到学习困难。或者你只是感兴趣。都请来参加。时间很方便。一周后 星期二11点半。地点是Adams宿舍音乐院。就在入口和Adams宿舍的食堂中间。记住在日程表上标出来了。这会是一次很愉快的讨论 希望到时能见到大家。非常谢谢。我叫Holly。我是校内一个协会的主席。这个协会叫哈佛大学领导才能协会。David和我想简单跟大家介绍一下我们的协会。我们的一位创立人就是你们的助教John Deutch。他几年前开始成立这个协会。简单来说 我们的协会组织各种活动。旨在提高校内学生的领导能力。我们有一个发展活动 在这个活动里 我们会请人。在研讨会和互动会上演讲。教会我们真正有用的技巧。例如主持会议 谈判 公众演讲。这些活动对所有本科生免费开放。我们还会举行午餐会和论坛 让学生主席。总编辑。校园队长进行合作。我们还有一个拓展计划训练哈佛本科生。当领导才能课程的讲师。我们有一个长达一年的活动 每周六举行。是和我们这里的Cambridge中学生一起参加的。最后 我们正在成立一份专注于领导才能的杂志。在将来不久就会发行。大家好 我叫David Tebaldi。我是领导才能协会发展领导才能活动的。首要负责人。我来为大家简单介绍一下。我们秋季会有什么活动。首先我们会举行主席论坛。这是一个聚集全校。学生领导的宴会。他们会分享他们。对领导才能以及相关挑战的经验。然后我们会出版我们第一份杂志 非常激动人心。里面会有很多文章关于以前。主持过活动的领导。以及他们在领导才能这方面的经验。我们接下来还会有很多活动 正如Holly刚才提到。我们会有关于谈判和公众演讲的活动。如果你们想了解更多。请搜索哈佛大学领导才能协会。大家就能找到我们的网页。谢谢你们。希望在下一年的活动上能看到大家。 

我们今天早上和班上的助教吃了早餐。这是我们最后一次的早餐聚餐。我有点伤心。希望上完这节课后我能开心一点。今天我们会讲完自尊。下节课是我们最后一次上课。至少是这个学期。我会总结一下我们讲过的内容。我们讲过的。和我们将要讲的内容。 

先回顾一下我们讲过的。就在幻灯片上。有很多研究。有很多观点都说了自尊有多重要。自尊确实很重要。有研究表明它与我们的健康。成功有关 低自尊通常与犯罪。药物滥用。心情低落 焦虑 抑郁有关。所以自尊很重要 这我们知道。但是在自尊这个课题里 也不是天下太平的。我们上节课讲过了。问题之一就是自相矛盾的证据。一方面。高自尊的人通常都仁慈 慷慨和富有同情心。但另一方面。高自尊的人又会表现出敌意。缺乏合作精神 有抵触情绪。另一个问题就是人们误解。自尊对表现的影响。对幸福的影响。以及当中的原因。自尊的矛盾。有时候我们觉得自己表现得很优秀。受到很多赞扬。但自尊上升后 很快又回落到基本水平。甚至更糟。因为我们要尽更大努力。才能让自尊回到基本水平。我们要得到更多的赞扬。取得更多的成就。做什么都不够了。所以有证明表明自尊的来源。以及它的后果。存在矛盾。解释这些矛盾的方法之一就是借鉴别人的研究。例如Maslow。Nathaniel Branden。或者Carl Rogers, Rollo May Lovinger。 

看看我们能怎么分解自尊的基本构成。这就是我的论文的内容。也是我上节课跟大家讲的内容。基本上就是讲了自尊这个领域里的研究人员的观点。那些思考这个问题的人。例如马路对面的教育学院的Robert Kegan。他研究自尊的角度更倾向于发展心理学。我给大家介绍了那个渐成模型 渐成。再说一次 它的意思就是必须实现了一层。或者至少部分实现了一层 才能去到下一层。这三层分别是依赖型自尊。独立型自尊。和无条件型自尊。依赖型自尊 它有两个构成部分。首先。有高度依赖性自尊的人时刻需要他人的赞扬。不管是选择职业 还是决定下午做什么这样的小事。都是根据他人的认同来做决定的。有依赖型自尊的人。在能力感方面。总是拿自己跟别人比较。我比他们好还是差? 比他们优秀还是不如他们?独立型自尊是一种取决于自我的自尊。这样的人在评价自己时。用的是自己的意见。他们的能力感取决于。自己认为自己进步了多少。改善了多少。觉得自己的潜力发挥了多少。这就是独立型自尊。并不取决于他人的言论或想法。无条件型自尊是最高层次的自尊。也就是Maslow所说的"自我实现"。David Schnarch所说的"分化良好型"。想被了解 而不是被认可。无条件型自尊是指我们的自尊高到。让我们对自己感觉很好。所以我们并不在乎别人怎么看自己。甚至不在乎自己怎么看自己。在比较方面。我们不会比较 我们是相互依存的。相互联系的。我上节课举的一个例子是写书。假设我出版一本书 我是一个高度依赖型自尊的人。首先 我写这本书。我出版这本书 是为了得到赞扬。我的首要动机是获得外界认同。我的首要动机是想出版一本。比别人的书都好的书。相比较而获得的能力感 以及取决于外界。又假设我写了一本书。我是一个高度独立型自尊的人。我写了书 我自己评价。"这是一本好书"或者"这本书不怎么样。需要修改"。在比较方面。我跟自己比较 "我比起刚开始写时进步了很多。这本书比我之前一本要好。我的书写得更真实了"。这是独立型自尊的评价。无条件型自尊。是我们所知的最高层次自尊。我写了一本书 我不在乎评价。不管它是好是坏 我当然想变得越来越好。但这并不影响我对自己的感觉。我就是我。我写了这本书 我处于心流状态。我体验这种经历。至于这本书是不是比别人的好。或者比我以前写的好。这都不重要。我高兴的是。我写成了一本书。如果别人也写了一本书。我也会同样替他感到高兴。别人写了一本更好的书 我会同样高兴 甚至更高兴。这本书更能帮助到人。这个模型有很重要的一点是需要明白的。我们从很小的年纪开始就有这三层自尊。并非只有达赖喇嘛或特蕾沙修女。曼德拉或者玛格丽特米德。这样的名人。才是自我实现的人。他们并不是不在乎别人的看法。但他们。大部分时候。都是自己决定的 做自己相信的事。和他人相互联系。想让世界变得更美好。但他们并非不在乎。并非他们评价自己时不会跟自己比较。也不会跟他人比较。这是人的天性。我犯过一个错误 当我开始这个研究时。我跟自己说。"好的 我想有独立型自尊"。当后来我明白无条件型自尊是什么时。我说"我想有无条件型自尊"。我把这两种自尊列为好的。把依赖型自尊列为坏的。结果呢?只加剧了我对他人的依赖。因为当我们有违天性时。天性就会跟我们作对 我们赢不了。就像第二节课一样。如果我跟自己说。"Tal 别紧张 别紧张 别心急"。结果呢? 我看到粉红色大象出来了。我变得更焦虑更紧张了。但当我接受这种天性时。放任它 不强求自己。它就顿时威力大减。因为我可以说。"好的 这个人不喜欢我的书 我无所谓。不喜欢我的课 无所谓"。我接受它 这是天性 我会伤心。我当然想他们都喜欢。但转念一想 我就想通了。"我要怎样做才能活出自我?我作为一个老师要怎样才能被了解。而不是整天想着得到别人的认同?"。积极接受 回顾一下。接受我的天性 然后决定。最合适的做法是什么。 

我的榜样之一是Warren Bennis。Warren Bennis在领导才能这个领域。做过很多研究 演讲和工作。他先在南加州大学教旁听课。后来才来哈佛和麻省理工。他来哈佛商学院教了三年。第一年 作为一个客座教授。第一年我上了他的班。一次非常特别的经历。第二年他教那个班时。我当了他的助教。我经常和他一起工作 互动。我经常为Warren折服。Warren当时80岁 他现在83岁了。他走进教室时 整间教室都亮了起来。他的微笑 姿势 直率。他丰富的表达。都是为了和同学发生互动。他的一举一动让其他人。对自己感觉更好了。他自信坚强。我把这样的人视为自我实现的人。他有很高的无条件自尊。我们走得很近 他在很多方面都给了我帮助。最大的帮助就是他活出自我 我观察他。学习他 吸收他给这个世界带来的东西。吸收他每天给数百万读他的书的人。听他演讲的人。和他有过接触的人。所带来的知识。几年后 有一天。我是他的助教。我们关系很亲近。我跟他说。"Warren 你是怎么变成现在这样的?"。你是怎么变成现在这样的?他的回应时 把手放在我肩膀。看着我 带着一个平静。接受 慷慨的微笑 说。"Tal 我不是一出生就这样的"。他就只说了这一句。这就是我需要听到的答案。因为在他的回答里。有很多重要的信息。而最重要的两个信息是。第一个 他是慢慢进步成这样的。需要时间 你不会在20岁时 或者40岁时。一夜之间 去到第三层自尊。变成一个自我实现的人。需要时间 需要下功夫。需要一次次跌倒和站起来。学会失败。然后从失败中学习。需要学习接受自己。需要敞开心扉 接受伤害 犯错。需要做一个彻彻底底的人。 

我不是一出生就这样的。他的回答中第二个信息是。Warren很真诚。他很真实 他没有跟我说。"Tal 别夸我。我也不是很厉害 过奖了"。他没有这样。他很真诚 他知道自己的价值。他自信。他没有虚伪的谦虚。这让我想起。当我开始思考自尊时。自尊的真正意义。Branden的著作中所写的意义。对了 他和Nathaniel Branden是好朋友。当我开始读Branden的著作时。当我明白真正的自尊是什么时。我明白有高自尊的人都是谦虚的。你不需要显摆。你们知道 自大。自恋是自尊的对立。目中无人 自以为是。跟真正意义上的自我感是完全对立的。所以当我明白了这点以后。我的一个重要目标就是变得。谦虚 最重要的目标。所以我越来越觉得。要让大家都知道我很谦虚。英国哲学家Francis Bacon说。谦虚不过是换了个法子在显摆而已。这句话用在我身上最合适不过了。但对别人来说不总是这样。我认为Warren Bennis是真正的谦虚。我认为曼德拉。当他谦虚时 他是真正的谦虚。但问题是 你怎样才能达到那种层次的谦虚?你要达到那种层次的谦虚 方法就是活出自我。活得真实。走完这个过程。这个过程就像我们上节课说的那样。就像学习走路。我们刚生出来时。我们连走路是什么都不知道。但慢慢地。我们能站起来。但我们需要别人扶着。或者靠着什么东西。例如 我们靠着桌子小心地走出第一步。慢慢地我们可以自己走了。但我们还是要想着每一步怎么迈。我们跌倒了就再爬起来。学会失败 或者一错再错。然后我们继续走。我们心里时刻想着怎么抬起这条腿。当你观察婴儿学走路时。会觉得很神奇。他们需要调动很多东西 才能完成。这么简单的一个动作。过了一段时间后。婴儿变得更自信了。她开始跑了。她不再想着怎么走或跑。因为她自然地。就会走了。自我感也是这样 一开始。我们没有自我感。区分不开"我"和妈妈。或者世界上其他物件。慢慢地我们发展出��我感。但这个自我感是完全依赖外界��。不管是身体上还是心理上。我��都依赖于父母。重要的��人。哥哥或姐姐。然后过了一段��间。通常是到我们进入青春期后。我们开始获得独立。这时我们开始渴望。听到自己的声音。想让别人听到自己的声音。青春期这段时间很难。对青少年来说难。对父母来说也难。但这是自然的。是进化的必经步骤。这时候我们需要我们的界限。但同时。我们有时候又需要打破界限。我们需要试探现实。确定自我。然后随着时间过去。如果我们这么做。随着时间过去 我们同时开始一个新过程。这个过程就是真正的分化。这时候独立的自我变得太强烈。我们能够和他们联系。再一次 引用回讲恋爱课的内容。Nathaniel Branden说过 我们越独立。我们就越互相依存。然后我们进步 这需要很长时间。如果我们没有依赖的一面 例如。如果我们年少时没有得到任何赞扬。或者用Carl Rogers的话来说无条件的。自我关注。我们就会有这个需求 回忆一下那个渐成模型。我们不能跳过一层。通常我们需要一个治疗师。帮助我们找到无条件的自我关注。这样我们就能到达下一层。或者我们需要一段关系。或几段关系帮助我们培养无条件的自我关注。这样我们就能到达下一层。如果…。如果我们在青少年时期。不停地有人叫我们安静。守规矩 阻止我们确立自我。活出自我。那么我们就不可能顺利地。到达下一层。通常在以后的生活里 我们需要做的事就是。首先要确立自己。就像青少年那样。这样就能到达下一层。但不管怎样 这个过程不会顺顺利利。为什么? 因为天下没有完美的父母。因为天下没有完美的成长环境。没有完美的老师帮助我们自然而顺利地。通过这些阶段。这是不现实的。在地球上不存在。我们在艰苦中随着时间进步。克服困难。犯错 跌倒。再爬起来。慢慢地独立的自我。无条件的自尊开始浮现。我们对自己更有自信。 

Maslow说他没见过。有45岁以下就自我实现的人。即使是那些自我实现的人。仍然有残留的一丝丝独立型自尊。和依赖型自尊。这种自尊是不会彻底消失的。但它这时变弱了。它不占主导地位 这时的人不再执着于。获得更多赞扬。当别人不喜欢他们时就伤心欲绝。当自己不是最好时就伤心欲绝。他们接受了 他们会说。"赢了固然是好 赢不了也无所谓"。然后他们会继续前进。他们会问。"我怎样才能让世界变得更美好?我怎么把积极情绪带给别人?"。Maslow没见过45岁以下自我实现的人。Schnarch说要到五六十岁时。人才分化。才到酝酿出一段关系中最高层次的感情。需要时间。我不是一生下来就这样的。正如Bennis指出 不是一生下来就这样的。非常重要的一个道理。 

很多人都问这个理论是不是放之四海皆准。因为你们知道 独立。或者说个人主义 是西方文化。或者说美国文化 西欧文化。如果你去非洲 这会很不同。群体的概念。比个体的概念更重要。如果大家去亚洲。"我们"的概念 集体的概念。群体的概念比个体的概念更重要。那在其他地方呢?这是一个专属西方的理论吗?答案是"不一定"。如果我们的目标是完全自我实现。或者用Rogers的话来说。在成为一个人的过程中。向着这个目标奋进。那么最后。互相依存这个最后阶段对所有文化来说都一样。所有文化都重视这一点。他们同样重视这一点。不同文化的人 这个过程可能看起来不同。日本人 美国人。埃及人 以色列人。津巴布韦人 都各不相同。这个过程看起来不同。但它有很多相同之处。我在第三节课时讲过达赖喇嘛。他说他发现西方心理的一个问题是。我们太着重文化差异了。达赖喇嘛是一个。文化触觉非常敏锐的人。他说差异确实存在 我们需要学习它们。但与此同时。我们更需要学习相同之处。正是这些相同之处把我们联合起来。团结起来。而且相同之处多不胜数。希望不久的将来会有越来越多人研究。这个话题 并把它应用于不同的文化中。看看它与西方文化以外的世界有什么联系。因为我等一下提到的大部分研究。都是在美国做的。有一些是在英国。如果能看看别的地方这个过程是如何的 那会很有趣。但它的核心是。相似多于差异。 

为什么独立很重要?为什么我们要奋力地达到独立阶段?为什么我们不应该遵从他人。依赖外界?我的理由 以及很多研究结果。还有很多历史事件都表明。培养独立是我们培养相互依存的。一个重要途径。首先。因为道德行为。如果大家想想历史上最可怕的暴行。历史上最可怕的暴行。正是那些遵从他人。服从权威的人犯下的。那些种族主义或民族主义的信念和行为导致的。现在来看看我刚才提到的三件事。服从权威 一个人。如果有很高的依赖性自尊。就越有可能服从。权威人物。为什么? 因为这样的人寻求赞扬。他们寻求认同。他们没有强烈的自我感。他们需要一个有魅力的领袖告诉他们。"你真棒 你真厉害 你太了不起了"。因为他们的自我感依赖于这些赞扬。如果我们需要不停地获得群体的赞扬。我们就更可能遵循群体的做法。大家想想Milgram的实验。Asch的实验。上过第15节和第1节的同学。大家想想种族主义 什么是种族主义?种族主义就是通过比较一些外在因素。来证明自己高人一等。换言之。我本身不够好。证明不了我有多好。我必须拿自己跟别人比较。我可以比较居住的地方。比较成长教育。比较肤色。它的中心是相对能力感。但是独立的人。没有这种比较的需求 或者说这种需求没那么强烈。遵循他人的需求也没那么强烈。他们坚持原则。他们自己的原则。但原则并不是万灵药。原则并不总是好的。因为纳粹分子。也很坚持自己的原则。但我等一下讲的研究。表明坚持原则的人更可能。听从自己的心。只要他们的心没有受伤。我说的受伤不是一个比喻。他们会追求做好事。他们会追求帮助他人。他们的移情会增加。这是移情的矛盾。当我培养移情时。或者说当我们培养…。实在找不到更好的词。我也不想用一个新词。当我们培养自爱时。当我们培养自尊时。我们就会更有可能变得。富有同情心 爱别人。大家想想黄金原则。要像爱自己一样爱你的邻居。在这个原则里。爱你的邻居的先决条件是什么?爱自己。你就是标准。我们自己是我们评价他人的标准。因为我们怎么对待他们经常…。不是通常。但经常都反映了我们怎么对待自己。所以那些培养独立自我感的人。那些有强烈自我认同的人 更能认同别人。除了道德行为以外。还会有更好的认知表现。有强烈依赖型自尊的人。或者说行动主要由他人言论决定的人。会遵循别人在遵循的规矩。他们很可能变得就像一台机器一样。因为他们会为了得到赞扬而努力。努力是好事。但他们不会打破陈规。而有强烈独立型自尊的人。就会打破陈规。选择别人没有走过的道路。但这并不代表他们不会选择。别人已经走过的路。只是这是他们真正。真正想要的生活就行了。所以如果我们说"别人说走东 我偏偏走西"。这同样也是依赖型的。因为我的行为还是取决于他人。所以这不是独立型自尊。而是依赖型自尊。 

如果我坐下说。"好的 我该怎么过我的人生?"。我很认真地想。把自己置身于一个没人知道我在做什么的世界。我会更可能找到我的爱好。我的核心 我的使命 也许是在投资银行工作。也许是在UniLu工作。这并不重要。问题是"我的动力是什么?"。 

有独立型自尊的人活到老学到老。回忆一下John Carlton的研究 在哈佛商学院的毕业生中。优等生与别的学生之间的区别就是。他们总是问问题。他们总是想学更多。他们不会。从大学或商学院毕业后。或者读完博士。诸如此类的学位后 说。"我懂得够多了"。"现在我是经理了 现在我是领导了。我会告诉人们应该做什么"。这是Warren Bennis的一个优点。他总是问问题。他对别人做的事感兴趣。他总是在学习。 

我等一下会给大家看一个研究。当我们有独立型自尊时。我们的幸福感会更高。我们会更平静 我们不会不停地。向别人证明自己。总是处于警惕状态是一件很劳心劳力的事。这个人会喜欢我吗? 我怎么得到他们的认同?当我们这样想时 我们会更平静:。"让我表达自己。没错 我也许会受伤害。如果他们不喜欢我 我会伤心。没问题 我承受得住。我很坚强 我可以接受"。存在感更平静了。当我们表达自己 而不是时刻想着表现自己。想象一下我们的生活会变得多轻松。如果 当我们在教室时。或者当我们走出教室时。我们感受到这种存在感。越来越多地…。 

重申一次 不是完全地。因为这有违人的天性。但我们可以越来越多地单纯地存在。感激我们的存在 作为一个整体存在。 

我给大家讲几个研究。这样的研究不是很多。我想给大家讲我做的一些研究。是我读博士学位时做的 后来还继续做了一点。第一个研究是关于。自尊的矛盾。也就是说 有些人得到很多赞扬。但他们的自尊下降了。而那些有很高自尊的人 有些有敌对情绪。但有些却对别人宽厚仁慈。第一个研究是Kernis做的。他在1995年提出自尊稳定性这个概念。他说 真正能够预测。一个人究竟是敌对还是宽厚。他们的自尊会随着时间增高。还是降低的 不是自尊本身。而是自尊的稳定性。他发现…他追踪实验对象。例如 一两周。他会测评他们的状态自尊而不是特质自尊。他们当时的状态自尊。有些人…。假设这里是自尊水平。这里是时间。在某个时候 有些人的自尊在这里。然后第二天。在这里 然后这里。然后又回到这里 这里。这里…。假设他测了十次。他想测出。这些数据分布的标准偏差。它有多接近? 有多稳定?假设这个人。平均来说。有很高的自尊。但有很多起伏。有些人则像这样。也有起起伏伏。有时是很大的起伏。但总体来说。很稳定。同样水平的自尊。特质自尊。但状态自尊的起伏少了很多。他发现 是自尊的稳定性。而不是自尊本身 决定了。一个人是敌对还是仁慈。自尊不稳定的人。更可能充满敌对情绪。而自尊更稳定的人。更可能宽厚仁慈。这就解释了自尊理论的。很多矛盾和误解。当我看到这个理论时 我顿悟了。亲身体验到的一次顿悟。我说。"就是它了 他找到答案了"。 

我在2000年写我的硕士论文时。我设计了一个问卷。一个很简单的问卷 测量依赖型自尊。和独立型自尊的对比。我测评学生。大部分是哈佛学生 我还在新加坡做了代表性抽样调查。我那时经常在新加坡 得出同样的结果。亚洲和美国的一样。至少新加坡和美国的一样。我发现 依赖型自尊。与不稳定性有关。而独立型自尊与稳定性有关。换言之。我只测量了一次 而不是十次。到底这种自尊会导致敌对情绪。焦虑 还是平静或宽厚。我只用了一个测量数据。我发现。依赖性自尊与自尊不稳定性。高度相关。两者成正比。而独立型自尊和自尊稳定性。也成正比。当时我的一些学生也就此做了一些研究。例如 Mellisa Christino。你们会在这门课上。看到她写的一些东西。她本科时写的毕业论文。就是表明独立型自尊和快乐。独立型自尊与心流的关系。那些有高度独立型自尊的人。最有可能在运动场上体验到心流。因为这就是她的研究对象 哈佛运动员。高度独立自尊的人也有更高的幸福感。我的论文是想准确地找出。自尊理论存在的问题。我说过 高度自尊意味着自恋。怎么区分这两者?我在论文里发现。自恋其实是与高度依赖型自尊有关。而自恋与独立型自尊没有什么关系。至于宽厚。这也许是研究中最令人意外的结果。有独立型自尊的人。其实更宽厚仁兹。不仅自我报告如此 对他人的行为也是如此。非常强烈的自我感。我说过 像爱自己一样爱你的邻居。己所不欲 勿施于人。当你对自我有一个高标准时。对他人也会有一个高标准。通常这是无意识地发生的。因为我们怎么对待他人就会怎么对待自己。还有敌对情绪。敌对情绪与自尊本身没有关系。因为有些。有高度自尊的人 有很高的敌对情绪。但有些有高度自尊的人 敌对情绪很低。而且更宽厚仁慈。但依赖型和独立型自尊能预测敌对情绪。那些有高度自尊的人 而且自尊主要是依赖型的。这样的人更可能…但也有例外 不是所有人都会这样。但平均来说 这样的人更可能表现出敌对。而那些有高度自尊 而且主要是独立型自尊的人。会更坦率 合作 宽厚。 

完美主义 那时我有两个学生。Tiffany Ignaczyk和Kate Richey。她是这个班的助教。做了一个很优秀的研究 表明完美主义。和自尊的关系 有高度独立型自尊的人。他们是完美主义者的可能性更低。为什么? 独立型自尊的人更注重表达自己。注重学习和成长。而学习和成长。就必定要跌倒再爬起。学习只有这个途径。而依赖型自尊更注重…。我上节课引用Mihnea Moldoveanu说的话。我们声称自己想要真相。但其实我们只是想自己是正确的。更依赖于他人的看法。更完美主义 更害怕失败。这同样适用于…我们之前讲过。但我想再说一次 适用于恋爱。这也是David Schnarch。就这个课题所做的研究 我们说过。我们怎么达到被了解的阶段。而不是被认可的阶段。这并不代表在七八十岁时。当我们完全进化了。我们并不喜欢 或者不需要我们伴侣的认可。我们需要。但同时。我们更关心的是表达自我。David Schnarch是这样说的。"分化是你维持自我感的能力。当你情绪上和/或身体上。与别人亲近。尤其是这些人变得对你越来越重要时。分化使你维持自己的道路。当爱人 朋友。家庭向你施压 让你同意和遵从时。分化良好的人能够同意 但又不会觉得。他们"失去自我"。能够不同意 但又不会觉得被排挤和难受"。这正是被了解的含义。做你自己 分化出来。这就是独立型自尊的核心。这是最高境界的相互依存的。真谛。说来容易做来难。但用它来启发我们是非常有用的。在我们潜意识里。当它发生时 能够注意到。因为在20岁这个年龄。你们也会经历相互依存的时刻。与朋友家人产生真正的联系。走在后院里 欣赏美景。这些珍贵的时刻 随着时间过去。我们会有越来越多。只要我们跟从这条自然的道路。自然的发展。继续用回走路那个比喻。我经常…我们知道。历史上有很多故事。有些人被迫穿上。太小的鞋子 或者被剥夺走路的自由。Ceausescu统治下的罗马尼亚就是这样。人们无法自然发展。当我们的自然发展被阻碍了。这是有可能的。我们就不能如我们所能的那样发展。所以在心理层面上也有这个问题。我们怎样才能如我们所能地发展? 你们在这个学期。所学到的一些方法。不管是认知的学习方法。允许自己有人的天性。还是你们具体地运用的方法。例如写日记。做运动。学会感恩。冥想 练瑜伽。这些方法都是。有助实现自我的方法。它们能加速这个过程。当我们加速这个过程。就意味着此时此刻我们就能越来越多地。体验到那种存在感。我们不用等到50 60或70岁。我们可以早早地就踏上那条健康的道路。 

那么我们该怎么做?我们说回。我们讲改变时讲过的那个模型。我们讲过改变的ABC。A是情感 B是行为 C是认知。我说过改变的最重要方法。改变的最有效方法是B 行为。我用一个模型讲过态度如何改变行为。反过来行为如何改变态度。大家回忆一下自我知觉理论。如果你想在某个生活领域得到改善。最好的方法就是让自己的行为变得像改善了一样。如果我想要高度自尊。我的行为就要像一个有高度自尊的人。为什么? 因为自尊就是一种态度。是我对自我的态度。是我对自己的评价。要提高这种态度。我就要…或者说最有效的办法。不是唯一的办法。最有效的办法 就是表现出一个。有高度自尊的人的行为。所以榜样的作用很重要。我说的榜样不止一个。有榜样很重要。我们可以学习这些人。说。"我想有这样的性格…。不 我想有Warren Bennis那样的。平静 包容和慷慨精神。我想有Marva Collins身上的。那种对教学的热爱。把榜样的所有优点综合起来。这样我们就能有他们那样的行为 并且慢慢地培养。我们追求的那种态度。 

我来讲一下提高自尊的几个方法。大部分都已经讲过了。所以这也算是很好地总结一下我们讲过的内容。首先有高度独立型自尊的人通常…。正如我们之前从研究中看到的那样。包括Melissa Christino和其他人的研究。他们通常更平静。所以独立型自尊是一种态度。平静是一种行为。也就是说 行为平静的人。他们的独立型自尊会随着时间增加。我们在和冥想相关的研究中看过 例如。正念冥想。当我们越来越多地练习这种方法。我们的自尊。我们独立型的自尊。因为它不需要赞扬。"我今天冥想得真好。大家给我鼓掌"。这不是这种自尊的重点。它的重点是带来平静。关注自己 活出自己。又例如运动。运动能带来平静。因为它在体内释放某种酶。某些化学物质 这样随着时间过去。它能直接增强我们的独立型自尊。因为行为。体验了平静 导致态度的转变。假设你中了一个无名咒语。从现在起在你的余生里。没有人会知道你做什么。除了你自己 没有人知道你有多好。你有多宽厚。你有多仁慈。除了你自己 没有人知道你有多富有多强大。你有多重要。只有你自己知道你的成就。只有你知道你有多好的操守。道德和性格。在这样的一个世界里 你中了无名咒语。你会怎么做?有谁觉得很熟悉吗? 你们会怎么做?正如我在课后作业里提过那样。这并不代表我们在匿名世界里怎么生活。在现实世界里就应该怎么生活。这个方法的意义是 在这样一个世界里。或者说做这样的练习。它能磨练我们。在抛开赞扬和声望的情况下。我们会怎么做。我经常练习这个方法 大家知道。在这个班上 我介绍给大家的方法 我都一一练过。我都练过。我也像大家一样边上课边学习。我会经常用这个方法。例如 我在这里读研究生的时候。我当时的答案是在那样的一个世界里。我可能连博士学位都拿不到。我不会拿到博士学位。因为读博对我来说不是一件很愉快的事 虽然很有意义。但没那么愉快。我读得不是很好。不觉得我自己有什么惊人能力。但我也跟自己说 在现实世界里。不是那个没人知道我在做什么的世界。而是在一个别人知道我在做什么的世界。我需要一个博士学位才能在大学任教。所以我继续学习 最终拿到了我的博士学位。我是在毕业前不久做这个练习的。我当时在考虑不同的出路。我想 在一个没有人知道我在做什么的世界里。我会做什么?我的答案是教书。我的答案是我不会做研究。这是很肯定的。所以那时候我决定放弃争取终身任期。即使当时外界 那些关心我的人。给了我很大的压力。他们说"再写两三篇文章。你就能拿到一个很好的职位 争取终身任期。你能做到的 你的博士论文可以发表的"。但我知道我不想过那样的生活。我要为此付出代价吗? 要。其中一个代价就是。这是我在哈佛的倒数第二节课了。但我想要什么样的世界。我想过什么样的生活。这个方法帮助我找到了。你热爱什么? 你真正。想过的是怎么样的人生?你希望10年 20年后。你会变成什么样?什么对你来说是非常重要的?什么对你来说是重要的。就算没有别人的认同 点头 赞扬和欢呼。反对或批评。你也会做的?你在这样一个世界里会做什么?这是一个很重要的问题。 

那些…。有高度独立型自尊的人会追求他们的兴趣。那些追求他们兴趣的人会有更高的自尊。那些有很多心流的人。他们的独立型自尊也会增加。问问你自己 "你什么情况下体验过心流?你什么时候忘我过? 是你读书的时候吗?是坐在。你队友旁边吃晚餐时吗?是你写作时吗?是你站在全班同学前面。介绍一个课后活动时吗?你什么时候体验过?你以前什么时候感受过心流? 问问你自己。因为它最能预测到你未来什么情况下会有心流。你想过什么样的人生?你真正想做的是什么?有高度独立型自尊的人会付诸行动。他们会应对 不停地加强自己。因为他们总是想学习成长。进步 发挥更大的作用。触及更多的人。那些应对的人…。大家回忆一下 自我知觉理论。那些应对的人 随着时间过去。会获得更高的独立型自尊。即使他们这样做的时候跌倒过。即使失败了四次。他们会重新站起来 变得更强大。他们的独立型自尊会随着时间增强。那个图并不是这样子的。起起伏伏 有高有低。真正的图是有起有伏。但健康 快乐和自尊的走势。是呈上升趋势的。 

谦虚 真诚的谦虚。我最喜欢的一本书。也是我的文化里最重要的一本书之一。Pirkei Avos 《先贤箴言》。里面有一句话。一句谚语 我读了一遍又一遍。但一直都参透不了。直到我听到一个著名的拉比。拉比Mitchell Wohlberg 他对句话的解释。这句话是这样说的。我翻译过来:那些追逐荣誉的人。荣誉会躲避他们。那些躲避荣誉的人。荣誉会追逐他们。再说一次:那些追…。追逐荣誉的人。荣誉会躲避他们。那些躲避荣誉的人。荣誉会追逐他们。我一直都参透不了 直到我听到拉比Wohlberg。关于自尊的一场演讲 并把它跟这句话结合起来理解。荣誉的本质是尊敬自己。相信自己。如果我不停地追逐荣誉。追逐下一个赞扬 下一个奖。下一个认同 下一个认可。荣誉就会躲避我。下一次我就要在更陡的山坡上。把石头推上去。我继续追逐荣誉 等我到达了山顶。石头又会滚下来。科林斯王就要继续推石头 而山坡变得。越来越陡峭。相反 如果我躲避荣誉 言行谦虚。随着时间过去 真正的荣誉。打心底发出的荣誉感就会追逐我。行为可以改变态度。这需要时间。重申 一开始。正如我之前说过。我想变得谦虚 我想大家都知道。我有多谦虚。我明白。那不是真诚的谦虚。但我仍然可以保持谦虚的言行。随着时间过去。它就会慢慢地融入我们。行为不会一夜之间就改变态度。但这并不意味着一个80岁。完成进化 实现自我的人 不会追寻荣誉。当他们被赞扬时 不会觉得高兴。那是很自然的 是人的天性。 

以下这个知识点很重要。我们在班上经常讲简化生活。那…。我经常想 为什么在一个。像美国这样的国家里。一个追崇个人主义 自为更生的国家。大家在高中时读了爱默生。那么多关于个人主义和自我的书。为什么你们还是那么随波逐流?为什么有那么多人还是走。以前很多人已经走过的路?这并不意味着走这条路的人都是随波逐流的。再一次 你可能很循规蹈矩。选择做一个顾问或者在流浪者之家。做自愿者。你可能是取决于他人的 也可能是取决于自我的。重点不在于选哪条路。重点是为什么。我们会看到这么多随波逐流的人?我想这是因为他们的成长教育。我想是因为美国人。或者普遍来说 西方世界。没有花足够的时间来反省。思考这样的问题:。"我真正想过的人生是怎样的?"。问这样的问题:。"在一个没有人知道我在做什么的世界里。我会做什么?"。用三问法来问自己 意义。"什么有意义?" "什么让我快乐?"。"我喜欢什么?" "我擅长什么?"。"我的强项是什么?"。问自己这些问题。问这些重要的问题。因为当我们退后一步 慢慢思考。这时我们才能变成自我。我们才能找到适合我们的道路。我想用一个阶段模型来给大家讲解。这是以Daniel Gilbert的研究为基础的。与他的认知晕眩研究有间接关系。他和其他人的研究发现。当人的认知忙乱时 当他们匆匆忙忙时。他们的行为会非常不同于。当他们慢慢花时间。专心思考一下他们所做的事时。也跟Tim Kasser的时间充裕研究有关。第一个段阶是我们没有时间的阶段。我们很赶时间 匆匆忙忙。我们的反应是无意识的。在这个阶段 我们会随波逐流。我们会遵从群体。就像Asch做的实验那样。在这个阶段 当那个穿白袍的人。告诉我们实验必须继续时 我们会遵从。我们会按下按钮 电击一个无辜的人。在这个阶段我们会随波逐流地服从。服从群体。当我们没有时间时。当我们匆忙时。当我们不认真思考。我们想做什么 可以做什么时。这是依赖的自我 第一阶段。通常这时我们会到达第二阶段。不用等很久。这时 如果我们慢慢来。因为如果我们慢慢来。这时我们就开始有意识思考。把思考与行为结合起来。这时候我们会想我们做的事。想我们想的事。我们慢慢来。这个阶段的自我是自主的。这时候独立型自尊出现了。当我们慢慢来时。我们随波逐流的可能性就会减低。这时我们会培养。表达我们独立的核心自我。 

想想接下来的一个实验。今天做不了 很可惜。其实也不可惜。今天做不了。这个实验其实是重复Milgram的实验 但有一个改动。大家想象一下 有些同学熟悉Milgram实验。我们在班上已经讲过了。你坐在椅子里 手放在按钮上。或者在你面前有一个控制台。你选择60伏电压。那个学习者没有什么反应。你上升到75伏。增加了15伏。你升高到120伏 那个被电击的人。你觉得他真的被电击了。他说"哎哟 很痛"。你抬起头看那个穿白袍的人。他说"实验必须继续"。你继续提高电压 那个人被电得尖叫。告诉你他要心脏病发了。你抬头看实验人员。他看起来就像那样。他告诉你…我之前一直没注意过他。很高兴我从没注意过他。他跟你说"实验必须继续"。你继续。那些人开始大哭起来。"我有心脏病 快放我出去"。你抬头看。实验人员说。"没有答案是错误的答案。提高电压 实验必须继续。这时63%的参与者会把电压提高到350伏。超过那个人所能承受的水平。他们继续电击到 直到XXX点。超过了危险水平。因为他告诉他们 实验必须继续。现在想象一下 同样的实验 但有一个小小的改动。你用75伏电压电他们 他们说。"哎哟 很痛"。然后实验人员告诉你。"实验必须继续。不如你先休息15分钟"。你出去15分钟。等你回来时。这时是90伏。假设你提高到120伏。120伏 那个人在尖叫。"放我出去 你们没权把我关在这里"。你抬头看实验人员 他告诉你。"实验必须继续。不如你先休息15分钟?"。你们认为这时会怎么样。如果这些人电击人之前有15分钟的时间休息?你认为会有63%的人会把电压提高到超过危险水平吗?很可能不会 因为在这15分钟里。我们不再处于热态。而是处于冷态。这是引用George Loewenstein的话。当他们处于冷态时 他们会想。"我疯了吗? 我在这里干什么?"。他们会去打电话报警。而不是15分钟后再走进去。"好的 现在我愿意杀了他们"。因为他们在这个实验里就是以为自己杀了这些人。如果他们没有那15分钟。而是"实验必须继续"。"实验必须继续"。他们就会继续。他们会为失去自我付出代价。我们的现代生活也是这样。当然后果没有Milgram的实验。那么可怕。而且在现代 即使我们服从权威。我们也没做什么不道德的事。但通常我们会伤害到自己。因为我们随波逐流 遵从主流。我们做别人认为我们该做的事。我们做那把声音叫我们做的事。我们没有做自己想做的事。你们知道我认为这门课最重要的部分是什么吗?是课堂和课后作业。我这么说不是因为谦虚。没错 我想你们觉得我谦虚。但这不是我。这么说的原因。而是因为这时候你们有时间思考。和你们的助教讨论。和你们的同学讨论。思考。你们被迫用额外时间来思考。这时你们坐下写课后作业。回答我提出的问题。这时你们真正的独立自我浮现了。这时你跟朋友谈这门课。或者在食堂跟队友分享。这时你有时间和你的父母分享。这是独立自我就出现了。最后 正直。Nathaniel Branden。Nathaniel Branden和他的很多病人这样做过 他是心理治疗师。他叫病人来 告诉他们。"好的 下周 暂停。不能说谎 要完全诚实坦白"。他发现这个方法 尤其是对于那些。会撒一点谎的病人来说。能够提高他们的自尊。我记得在1996年 96或97年。克林顿弹劾案正在审判。CNN请来DePaulo。她是维珍尼亚大学的一位教授。她专门研究说谎。他们采访了她 那是第一次。我接触到她的研究。她在CNN上说她的研究发现。基本上所有人都说谎 很多时候都是一些小谎。为了让别人觉得自己很厉害。很多时候是大谎。她发现。人平均每天说三个谎。当我听到的时候。我轻笑了一声。"也许其他人每天说三次谎。但我不说谎"。我解释一下为什么说我不撒谎。这是真的。因为我读大学时写的毕业论文。就是说诚实是会带来好处的。我当时既学心理也学哲学。我对道德 自尊 动机感兴趣。我根据研究。根据亚里士多德 亚当·史密斯的哲学著作。发现每次我们说谎 欺骗 不诚实时。要付出巨大的。心理和情绪代价。所以我跟自己说。" 

让我试试DePaulo的方法"。她的方法就是叫人讲完一段对话后。马上评估有哪些是真的 有哪些是说谎的。趁他们还清楚记得。自己说过什么。他们会写一本日记。完全匿名的。所以我想。"我会写 虽然我不说谎"。因为我知道说谎的后果。因为我的论文就是写这个的。我是诚实这个课题的专家。我说我会写日记的。因为我想实验一下。就在我写日记的期间。有一天晚上我出去约会。一个我没见过的人 不是Arianna。是我妈妈给我介绍的。我妈和她妈说我们在同一间学校。然后我听说了她的一些很棒的事迹。我很激动。我们见了面以后就更加激动了。我们坐下来 吃饭聊天 很愉快。她是一个经济学家。非常成功的经济学家。她问我是做什么的。我告诉她我是心理学家。我用我全部的心理知识在她面前表现自己。她说"我也对心理学感兴趣"。我说"这一定会很顺利"。我自己心里说的 不是跟她说。我学到了教训 刚开始约会时不要说这样的话。我们坐下来…。要学会失败。我们坐下来聊天。然后她说。"我刚读到一个心理学家的著作"。她说了这个人的名字。非常著名的一个心理学家。"你听说过他吧?"。我完全不知道。她说的是谁 完全不知道。我记得我呆了一下。微笑地看着她。"嗯嗯"。然后我们继续聊。我没有注意到自己说谎了。但后来我回到家。回想这次的对话。我说"慢着 我刚才跟她说谎了"。从那时起。我就经常这么做 我指的是说谎。不 我经常这么做。从那时起。我就坚持磨练自己的正直。用DePaulo说的方法。用Nathaniel Branden说的方法。非常留意我说的话。我这么说是为了表现自己。而说的假话吗?我是不是说了小谎?因为你们要明白 当我们说实话时。我们给自己传达了一个信息。我们给自己传达的信息是。我的话是有价值的。我的话很重要 我很重要。当我们不说实话时。当我们说谎时。或者我们想不停地表现自己时。我们是在说"这样的我不够好"。我需要成为另一个人 让别人来喜欢我。我需要知道这个心理学家。不然她就不会喜欢我。而不是做真实的自己。选择被认同而不是被了解。用这个方法坚持一周。直到行为改变了我们的态度。我开始更重视我的话。当我跟自己传达 我是谁。而不是想着让别人觉得自己很厉害时。当我肯定自己时。当我时不时地说"不"时。这样我就能简化我的生活。当我该说是的时候说是。即使这个"是"不是很受欢迎。但我应该这样说。还要从时间和空间上把这个方法扩展开来。首先从我自己开始。为什么写日记这么有用?为什么有移情心的治疗师。他们的治疗这么有效?因为我们活出了自我。我们没有想着在治疗师面前表现自己。一段时间后 我们打开心扉。我们没有想着在日记里表现自己。我们活出了自我。从让我们感觉良好的日记扩展到。亲近的关系和人。从一天一次对话中练习正直。到坚持一周 然后坚持一生。想象一下 过这样的生活。当你不用再不停地表达或表现自己时。我们会感到多自由。我们能够被了解。能够活出自我。这样会多轻松。多有力量。我们会觉得多正直和真实。我们会多快乐? 

我想在下课前读一段Melissa Christino的日记。是她在她的毕业论文中引用的。当然这是得到她允许的。Melissa Christino是哈佛本科毕业生 2002届。她是这样写的。"你真正的潜力。埋藏在你的灵魂深处。在你腹部最底处。在你的知识和紧张触及不到的地方。埋藏在你的恐惧和焦虑之下。虽然它隐藏起来了。但它还在那里。我知道 因为我以前感受过它的存在。我知道别人也有。因为我见过别人做过奇迹一般的事。我们体内都有一股无可匹敌的潜力 散发着微弱的光芒。当我们找到它时 它就会发出万丈光芒"。我们每个人都有这丝微光。如果我们助长它 培养它 它就会壮大。它就会壮大到超越自己。这时候 像Warren Bennis这样的人。像曼德拉这样的人。像Anita Roddick这样的人。像特蕾莎修女这样的人 他们的存在。他们的存在 就能为别人散发光芒。现在我想这样做。只剩四分钟。我想这样做。我会放一首歌给大家听。通常我会在上课前放的。但今天我想在最后才放。如果你们喜欢的话。可以闭上眼睛 听歌词。它们对你有什么意义。对你的生活有什么意义。所以如果你们喜欢。闭上眼睛听。听听以下的歌词。 

我就是我。我独一无二。看我一眼。对我嗤之以鼻。或为我鼓掌。这是我的世界。我想拥有一点骄傲。我的世界。在这里我不用躲藏。生命毫无价值。直到你能说。我就是我。 

睁开眼睛。来吧 上来。我就是我。好的 我们下去。和他们一起跳。 

第23课-收获交流 

大家好。我叫Pearl Hoddling。是来自Kirkland House的大四学生。我希望你们能在这周六上午10点。到下午2点参加Hendriana和Maquat的免费薄饼计划。九年前。我父亲被诊断出患有帕金森症。一种神经疾病。尽管他 我的家人和我经历了很多起伏坎坷。既是生理上的 也是情感上的。我最终找到一种方式来献绵薄之力。一个明确的选择。我想参与这个活动"帕金森薄饼"。已经办了好几年了。直到今年我才鼓起勇气参加。谢谢你 Tal。帮我克服我的拖延症。开始积极地。来寻找治疗帕金森的方法。所有捐赠直接捐给Michael J.Fox。帕金森研究基金。期待周六10点到2点你们的参与。出于善意来享用薄饼。谢谢。我叫Maureen Hilton。我也是这门课的一个学生。我在哈佛工作。我想邀请诸位。来参加我的积极心理学报告。是下周五。5月9号下午两点在哈佛礼堂。下周五的主要目的。是希望能聚集。哈佛所有对积极心理学有兴趣的人们。包括各层次的哈佛管理人员。这门课的教师 学生。不管是曾经上过幸福课的。或是正在上的。不管是本科还是继续教育。或者希望未来能拥有积极心态的学生。以及你们想邀请的任何人。希望这能成为你们一次分享故事。以及这门课如何影响你人生的机会。此外。为了协助这次聚会的联系和交流。哈佛Callback合唱团会进行演出。你们可能认识的大四学生 Gina。Skylar和Kedar会表演他们。可能在哈佛的最后一次独唱。现场会提供免费食物。诸位可以随意邀请想邀请的人。尤其是希望未来还能开设。幸福课这样一门课程的同学。无需回复。来就行了。因为我们给哈佛教务处留下的。最深印象。就是我们一起享受愉悦的感觉。并聚集了一屋子为了幸福课。能有所提升的人。希望你们下周五能来。5月9号。下午两点在哈佛礼堂。谢谢。我也不知该说什么。所以我找了别人来替我说。(录像:大青蛙布偶秀"道别")。挥手道别。扬帆启程。"再见"一词哽塞在喉。双手紧握。想问为何。又是别离时分。Kermit。再见。再见。挥手道别。满心伤悲。又回忆起过去的好时光。千言万语。欲言又止。又是别离时分。不舍离开。你我都懂。有时离开也许更好。而我知道。终会重逢。虽不知何时何地。你永在我心。重逢之前。又是别离时分。而我知道。终会重逢。虽不知何时何地。你永在我心 重逢之前。带着微笑。也噙着泪。与你道别 啦啦啦啦。啦啦啦啦。又是别离时分。啦啦啦啦。啦啦啦啦。啦啦啦啦。我需要你们的帮助。我十分紧张。我该怎么办。冥想? 谢谢。锻炼 现在?你是说跳舞吗。讲个笑话。你们给我讲吧。还有什么别的建议没。拥抱 好主意。至少要12个拥抱 对吧。还有什么 深呼吸 很好。紧张也没关系? 谢谢。我已经感觉好多了。这是我在哈佛的最后一节课了。还有谁也是一样。你的最后一节课。我今天想做的是。尽可能地总结一下我们一路来。所得的一些收获。我想和你们分享。这个总结的责任。你们也要一起分享。所以我想从你们思考和回忆课程内容开始。换言之我们今天这节课。就像个亲密互动环节。之后我会讲一遍我认为的。这学期的一些重点知识。我想先从你们开始。给你们机会回顾一下。我尤其希望你们能够回顾。以下主题 并把它们写下来。翻阅一下你们的笔记 搜索一下脑海。写下至少两样你觉得这学期最有意义。或你觉得最有趣的东西。不管是课上听到的内容。小组讨论时的话题。你阅读时读到的东西。课后你和朋友讨论过的。和这门课有所联系的问题。什么都可以 至少两样对你来说最有意义。或最有趣的东西。这是第一个内容。我想让你们写下的第二个内容。你们可能以前试过。可能以前写过。至少两样你所决心做的转变。两种行为转变或态度转变。两样你要付诸实施的改变。既然学期也都快结束了。再说一遍 至少两样有意义或有趣的事。以及至少两个转变。不管是建立习惯。态度转变 或行为转变。花一点时间把它们写下来。好。现在我希望你们做的 当然是完全。自愿的。就是分享一些你们学过的。或至少一样这学期学过的有意义。或很有趣的东西。以及如果你们愿意的话。那两样转变其中的一样。或两样一起分享给。你身边的那位同学。或你身边的两位同学。花五分钟来彼此分享一下。一样有趣或有意义的事。以及一到两种转变。好。我希望你们在今天或人生中。余下的时间还能继续分享学过的知识。正在学的知识。你如何成长。我现在想做的是。听听你们讲讲一样。你认为它。有趣或有意义的事。可以是课堂内容 也可以来自其他涉猎。但却是这学期对你来说有趣或有意义的东西。我们把话筒在整个教室传递。有没有人愿意做第一个自愿讲讲的。还记得吗 勇气并不是没有畏惧。而是有了畏惧还坚持向前。如果勇气是没有畏惧。我就不会来教这门课了。Elizabeth。我觉得对我来说最重要的东西就是。自我和谐的目标。不是浮躁或自私的。而是一种每个人都应该。积极追求的 可以帮助你身边的人。而非别人让你追求什么 你就追求什么。很好 消除自我和他人之间的分歧。还记得吗。达赖喇嘛 当他到了西方国家后有多惊讶。他听说我们的"同情"这个词。指的是对其他人的同情。而藏语中Sawai既指对他人的同情。也包括对自我的同情。它们是彼此交融的。所以要找到自我和谐的目标。很大程度上就是要帮助自我。并最终使世界变得更美好。非常好 谢谢你。我觉得Roger Bannister对我来说很有意义。尤其因为我现在明白了一切皆有可能。即使科学家说有些事是不可能的。你也还是可以做到的。对 人们经常会跟说 这是不可能的。你做不到。我们的脑海中总是回响着这样的声音。这些狭隘的想法。这种狭隘既包括对我们的感觉。也包括我们所能做到的事。而Roger Bannister的故事。能够帮我们打破一些这样的束缚。挖掘自身的潜能。谢谢你 请说。那天你提到了关于你祖母的故事。她是如何历经坎坷后 终于看到。世间种种美景的。这深深地打动了我。在这样的学术氛围中能如此感动。实在是一种特殊而难得的体验。真的很有力量。也使我重新审视自己。看待人生的方式以及。珍惜我所有的东西的方式。谢谢 谢谢。我很喜欢那个"我们忘记坏事情。比我们想象的快得多"的观点。我经常思考这一点。我们总是骗自己说有多成功。多么有潜能 多么快乐。但我想。朝着目标向前冲。就算没能达到也会最终释怀的想法。是非常重要的。没错。我们有所成就时。能回复平和 感觉就像从前一样。而当我们失败时。要是知道我们是能重新振作的。是会很快恢复的。是很能让人解脱的。谢谢。请说。在我看来 找到竞争主义者和享乐主义者之间的。平衡点很重要。以及短期益处和长期益处之间的平衡。是的。人生经常会遇到"非此则彼"的情况。要么我是个成功的竞争主义者。要么就是个放弃成功的享乐主义者。而我们其实通常都能将两者融合。就像咬一口蛋糕还保持其基本完整。长远益处和近期益处的结合。谢谢。被人所知而非被证明 以及你教课的方式。比如上课前会讲。你人生中经历的许多失败。我就觉得自己。我自己的失败 它们是不同的。但又是统一的。我觉得没那么孤单了。谢谢 你知道吗。我在第一节课就讲过的 Carl Rogers说过。"最个人的也是最大众的"。因为从本质来说 这些都是人性。当我们活在这世上 有被了解的欲望。表达自我。这是我们最能和他人触碰的时候。不管是爱情。友情。或是同学之间。谢谢。请说。我从这门课上学到的一点就是。要学会感恩。因为有时我感觉。人们总是把最爱的人当做理所当然。感恩 我们的周围和内心。都有那么多快乐和幸福。最重要的是睁开我们的双眼。去感受它们。请说。对我来说特别棒的就是。这门课对其他不是学生的人很开放。比如家人以及其他不学这门课的同学。同样也能这样改变他们的人生。他们并不会刻意感觉到这对他们有好处。而是。"对 也许我该记下所有。我生活中该感恩的东西" 以及其他的。是的。这就是传递。有时我们很害怕说出自己看重的东西。而当我们说出它时。当我们是真实的。当我们是真挚的。当我们表达自我时。这些东西是会传染的。我们会将其传递出去。它们再继续不断传递。谢谢 请说。在学这门课之前。我将自己的悲喜都归咎于周围的事物。比如我的环境。学这门课真的帮到我很多。意识到周围环境其实并不会掌控你的快乐。我应该更积极地了解自己的感受。是的。还记得我们讲过。至少我不认为什么事都能一帆风顺。有时我们会遇到坏情况。糟糕的情况。艰难的情况。最大的挑战其实是我们如何。能更好地利用发生的一切。我们确实比自己想象的更有力量来掌控。请说。这门课让我几乎每天都能很多次地。感受到我是个普通人。我允许自己为人。而不是自己臆想出来的。或者总想着我没做好的事。允许为人。我反复多次提到了它 并将继续提到。即使到了生命最后几分钟也会继续。因为从很多方面来说它确实是幸福的基础。不幸的是我们如今的文化。过于强调我所说的。"巨大的骗局"。我们认为每个人都很棒 就我不行。所以我要假装我也很棒。然后我们就成了"巨大骗局"的一部分。一定要反复强调。允许为人。请讲。关于完美主义的那些内容。我从没觉得自己是个完美主义者。我觉得那是种自我否定。通过课堂上的练习以及。课本的内容。我们做的各种阅读 真的使我。去探索 去接受它。并主动地作出改变来远离它。很好 完美主义同样是人们经常隐藏的。因为完美主义者从定义上说就不愿意显露出弱点。正如我在课上所说的。我认为不完美主义是在向着。幸福前进的道路上最重要的一点。接受不可避免的失败。接受人生不会是一条直线。而是有起有伏的螺旋。但主轴仍然在。也就是进步 成长 和允许为人。我觉得最有共鸣的一点就是。就是重新构建问题。在学期刚开始时。你讲到消极和积极的研究。所得到的21:1的比例。对于我这样马上要进行神经生物学研究的人来说。能够重建世界观是很有趣的。开始思考。也许。开始思考对于研究来说什么是最重要的。我会在自己的学术领域中用到它。问题的重要性 问题能创造我们的现实。还记得公车上的孩子们的问题吗。很多人都忽视了。年复一年我们在城市中对高危人群。进行研究。却总是找不到改善他们生活的方法。因为他们没有问出积极的问题。我很快会讲 这也适用于我们生活的各个方面。请说。此前我一直是个拖延的人。这是我学习中的大问题。现在我进步了很多。非常感谢你。拖延症。70%的大学生抱怨说有拖延症。这是个大问题。克服它的一种方式。克服它的最好方式就是去做。还记得五分钟休息吗。那非常重要。谢谢。请说。我对我们。阅读的Gilbert的作品印象非常深刻。讲的是关于生活中的好运。以及你觉得自己置身的美好状态。其实都是你自己创造的。通过改变你的外表和衣着。你可以每日逐渐改变自己的幸福感。很多人活着会想。"这个……"。要是我没遇到这个人就好了。要是这种事情没发生在我身上。我就会很幸福的。要明白这其中很大一部分。都是由我们的理解决定的是很重要的。请说。对我来说最有意义的。其实就是这门课的存在。有这么多人关心幸福。因为在哈佛这样的氛围中。我们不一定总有机会来思考这个。正如你所说。你只是提醒我们一些已经知道的事。也让我们知道 我们了解它是正常的。是的。我们并不孤单。我们需要有人提醒 其他人也需要。通过提醒其他人。我们也提醒了自己。谢谢。请说。我觉得是正直的概念。因为我认为我们都是诚实的人。但就是每天的各种小事情。让我们的信仰屈服。或者让我们言行不一。这会对我们的自尊造成影响。所以时刻意识到并保持我们的正直。能让我们对自我感觉更良好。当我们说真话时。其实正给自己传递了某种信息。当我们表达自我时。我们在给自己传递某种信息。这个信息就是我的话有分量 我有分量。当我们表达。"我很棒"。无需过分想给他人留下印象。而这个信息。这个自我知觉理论能提升。我们对自身的看法。更不用说对人际关系的提升了。也能使世界变得更美好。如果越来越多人能够正直。那么同样自我和他人之间的分歧就会消除。请说。你在课堂上和我们分享的。一句话深深震撼了我。我事实上已经将其作为了座右铭。"学会失败 或从失败中学习 没有别的法子"。经过这门课的学习 我意识到。我曾经是个怎样的完美主义者。以及我人生中有多少次。我的早期回忆一直在折磨自己。焦虑 以及对犯错的恐惧。重构这些 真正回头看那些不可避免的。失败时刻 能成为我前进的动力。我面前突然一片开阔。非常感谢你。谢谢。"学会失败 从失败中学习"的概念。其实也是我的个人座右铭。它能让人解脱。在人际关系中。在学业成绩中。在一个组织的各个层面中。重要的是要知道没有其他更好的学习方法。不管是画圆圈或学走路。这只是人生的一部分。如果我们接受它。我们就能更快乐。谢谢。Elizabeth。对我来说特别有意义的就是。真正要专心追求的东西。快乐是最终的目的。要试图消除外界的影响。认真思考对我来说有意义的到底是什么。在我所学的东西中什么能让我快乐。谢谢。最终目的。很多人说美国人。之所以抑郁水平不断攀升。之所以有那么多不快乐的人。其中一个原因就是美国人有过高的期望。这是错的。我不同意 研究也不能证明这一点。这不是高期望或低期望的问题。而是正确或错误的期望的问题。这适用于很多领域。但我们要明白的一点是。我们在人生中从何处获得满足。是来自下一句赞美。下一次升职 或人们说我们有多棒吗。还是来自我们想做的事。换言之我们最终的目的所决定的事。而最终的目的就是快乐。谢谢。请说。我觉得这整个学期 这一整年。我都一直在。要么上课 要么跑向实验室。寻找方法来填满我的每一天。似乎我永远在从一处跑向另一处。似乎我整个人生。都在试图填满每一分钟。我觉得特别重要的。我真的学到的就是你所说的。做得少其实是更多。我会很认真地在我接下来。在哈佛的三年中将这一点铭记于心。对 做得少其实是更多。数量会影响质量。我们经常好奇。"我为什么不快乐"。我人生中有这么多美好的东西。这可能就是问题的一部分。因为更多不一定意味着更好。要把事情简化很难 这是很重要的。非常重要。请说。复杂性另一面的简单性的概念。我觉得非常奇妙。当一个人成长时。就会加入各种各样的变量 变得越来越复杂。不能把问题简化是很让人沮丧的。但你说到了复杂的顶点后。你会发现那种美。发现简单是出现在复杂之后。而非复杂之前。而对复杂性的理解。那是个非常棒的概念。Oliver Wendell Holmes讲到了。复杂性另一面的简单性。这适用于不管是…… 这门课大部分内容。都是很简单的。没有什么特别复杂的概念。但它们很多都是复杂性另一面的简单。对个人来说也是一样。我们经常会遇到困难 生活很艰难。我们会经历危机 严酷的考验。而我们要做的就是明白。当我们经历了这些复杂 艰难。严峻的考验和坎坷起伏之后。另一面就是简单。通常我们要做的往往就是坚持。再等待一会儿。就会更清楚明了。更能成长。更能进步。最终更快乐。请说。在传递话筒时。我在想。"天啊 我不知该说什么。太多人了 我没法做到的"。然后你说勇气不是没有畏惧。而是有了畏惧还坚持向前。所以我一下就把顾虑都抛弃了。就像"给我话筒 我要说!"。而它现在在我手上 我在说话。所以非常感谢。就是这样。这就是行为转变。而不是理论 很好。请说。你前几节课讲了关于匿名的世界。如果一切行为是匿名的 我们会如何生活。我认为知道我们真正想要的并勇敢追求。我们想要的东西是很有挑战性的。甚至是让人畏惧的。能够真正了解自己想要的。并一心追求它。我觉得这是非常值得自豪的。而你确实实践了你所教给我们的东西。因此我对你有着崇高的敬意。谢谢。谢谢。你讲过很多东西。但有一样我确实学到的就是。专注于我们真正想做的事。你在课堂上说过很多次。教书就是你的使命。我能看得出来你有多热爱教学。这让我很有感触。我会记住这一点。并不断寻找我的使命。我真正想做的到底是什么。谢谢。请说。找到拿着话筒的人。把话筒给你。有人有话筒吗。好的。我以前是个总喜欢逃避问题的人。总是试图忽视问题。直到它们消失。而现在我不再这样了。因为我知道如果我不开心。唯一能够解决的人就是我自己。我要做出改变。这门课给我了很多这样做的帮助。谢谢。谢谢。我想谢谢你告诉我们承诺的力量。它帮助我找到了工作。一边是我想做的事。也就是记者 一边是我不太感兴趣的。公共关系。我想为此感谢你。因为你使我决定投身其中。抛弃一切顾虑。所以我才找到自己想要的工作 谢谢。事实上。粉丝们好。是的。我记得你说过当拉伸肌肉时。它会变强。由此引出"拉伸区"的概念。我以前没有意识到当我拉伸自己。去尝试做一些在我的拉伸区。和恐慌区里的事时 我有多强大。谢谢你。我觉得这门课对我们很多人来说。就像地图一样指引我们改变 变得快乐。谢谢。谢谢。各位下午好。我首先想说你太棒了。我是Milwaukee公立学校体系的一名老师。在座有将来想做老师的吗。"Milwaukee"。你之前讲过的。Albert Bandura和自我效能理论。是非常重要的。任何对帮助青年成长或教育有兴趣的人。对孩子们有高期望。是非常非常重要的。尤其是有色人种的孩子。作为教育工作者。我明白Bandura和Dweck想表达的意思。对这些孩子来说弱化了的现实。为他们创建了一个完全不同的世界。我想强调并着重突出这一点。很感谢你讲了这个。对孩子有高期望是很重要的。当然对自己也是很重要的。没错 自我实现预言经常会改变世界。我们的以及我们周围的世界。基于我们的信仰。我们再听几位。然后我再快速总结一下。请说。好的。请讲吧。我知道这已经被提到好几次了。但真的很容易将我们的不快乐归咎他人。或他人的行为。甚至我们自己的行为。但你说过很多次我们潜意识中。发生的事以及。潜意识的影响。可能才是阻止我们不快乐 错了。阻止我们快乐的原因。这让我回想起Marva Collins。如果学生做错了什么事。她会让他们看一遍字母表说。这些都是我好的方面。我们真的应该尝试去想。我想的是什么 对他人不做什么。或他人对我做什么。但要思考是什么阻止我变成一个更幸福的人。很好 谢谢。再听两位同学。那位同学。这门课上我印象最深的两样东西就是。你放的小宝宝学走路的录像。你想表达的是这个宝宝。并不知道他为何会摔倒。也不知道他该为此感到羞耻。这是种很有效的方法让我们知道。我们要学会失败 从失败中学习。这让我感触良多。以及你愿意说出你自己在做。心理学研究时经历的起起伏伏。以及你在剑桥和哈佛的经历。你愿意很开诚布公地和我们分享。这让我感受到。我要毕业了 要继续发展。并不清楚自己想做什么。会考虑各种选择。会感受到对失败的恐惧。尤其是来自像哈佛这样的地方。更是让人倍感压力。而你愿意和我们分享这些经历。我就会重新考虑我要做的决定。对此我非常感谢。谢谢。请说。每一天都是美好的。对此感恩能够让我们在做自己。不喜欢的事时能更轻松。以及常识其实并不那么人尽皆知。这些都是…… 学的很多东西都是我已经知道的。但我现在是在哈佛学到的。好的 很好。我们再听一位同学。我是大四的 马上要毕业了。这可能是我在哈佛上的最后一节课。如果我说的东西对于班里的低年级同学。太过难以接受 我将为此道歉。我印象非常深刻的一点是。这样一门涉及面广泛的大课。有如此多人想上这门课。班里的同学各种各样。但我想和大家分享的经历就是。我来到这个课堂。我很诚实 必须承认。我认为这会是对艰难的大四。春季学期一个很好的补充。我就想"这应该是门挺好玩的课"。在我大二时。我有点开始探索自我。压抑的生活让我用不同的方式来看待事物。我当时。我有了大二的这些想法。我就想"也许这门课就是多余的了"。结果后来我又在大四的春季学期。经历了艰难的时期。抱歉。我最想和在座的低年级同学分享的。就是如果你是大一的。大二的。甚至大三都不算晚。最重要的是搞清楚。在学校里对你最重要的是什么。我告诉你们不是论文。不是讨论的问题。不是当你的朋友需要你。空出一小时时 你却说要保证八小时睡眠。因为这里的人是最重要的。你以后会想念他们。如果你不珍惜现在和他们一起的时光。以后很难再有时间和他们共度。这非常重要。搞清楚真正对你有意义的是什么。和你在学校中真正想做的事。因为在你不经意间。它就要结束了。谢谢你让我铭记这一点。使我的这个学期变得有意义。谢谢 谢谢你们。现在我来总结一下。你们刚刚讲到的这些所有东西。这些不一定是最重要的。这些东西。我无须翻阅笔记就能想到。首先。你所问的问题。经常都会决定你所做的探索。如果我们只问消极的问题 比如。"为什么这么多人失败"。我们就没法看到潜藏在。每个人心中的伟大 如果我们只问。"我的人际关系该怎样改善"。我们就无法看见身边的人。所拥有的宝贵财富和奇迹。我们同样要问积极的问题。我的人际关系中有什么好的方面。我的同伴有哪些优点。我自己有哪些优点。什么对我最有意义。什么能使我愉快。我擅长什么。问题会带来探索。探索的内容取决于我们。所问的问题。我们常说信念创造现实。不仅是信念 更宽泛地说。我们如何理解现实才是最后所得的结果。Marva Collins相信她的学生。相信她的学生 然后奇迹发生了。那些朽木难雕的学生因为信念而成就卓越。还记得John Carlton关于商学院学生的研究吗。将卓越和平庸划分开的有两样东西。一个是他们总在问问题。总想学习到更多。心怀谦逊 对成长 幸福和自尊。尤为重要。其次。他们相信自我。他们有自信。他们有自我效能通往成功和进步。我们如何提升信念?通过拉伸自我 去尝试。挑战自我。通过具象化使我们明白自己可以做到。成功没有捷径。学会失败 从失败中学习。历史上很多成功的杰出人士。同样也是那些失败最惨重的人。这并非巧合。无论是在体育界 Babe Ruth。比同时代任何球员三振出局次数都多。但也打出过最多的全垒打。还有Michael Jordan。一次次地表现出失败。赛点时最后一个绝杀没有投进。或不断重复的机械训练。Thomas Edison 比任何科学家。获得专利都多的人。同样也是失败过最多次的人。这也并非巧合。还记得Dean Simonton对科学家和艺术家的研究吗。最成功的往往是失败得最多的。没有别的方法可以学习。而且如果我们接受这一点。如果我们接受这种成长的心态。而不是Carol Dweck所说的固定心态。我们不仅会更有进步。也会更加快乐幸福。因为这样就不会每次经历都是。对自我 自尊的一种威胁。如果我们接受失败是不可避免的。不管是在人际关系中。存在失望和争吵。还是在个人生活。或从国家来说。没有完美的国家 没有完美的人。没有完美的组织。如果我们明白生命是个上升的螺旋。而非一条直线。我们会幸福得多。冷静得多。并且也更加成功。允许为人 世界上有两种人。不会经历情感创伤。哪两种人。精神病患者和死人。我很高兴你们都不是精神病患者 也都活着。这让这门课好上很多。安全很多。你不可能学完这门课后。就能一直保持人生平顺。没人能一直保持人生平顺。你还会经历坎坷起伏。我希望你们唯一能。感受到的区别就是当经历感情创伤时。你会看着它说。"我只是普通人 我很难过。真希望事情不是这样。但我接受它。就像接受重力定律一样。因为重力定律是一种物理本质。就像感情创伤是一种人性本质。允许为人。在课堂上我们讲了很多种干预。大多数干预并不起作用 正如。Somerville的研究和其他研究显示的。而日记可以。在写东西时 我们在打开内心。不管是写 还是和我们关心的人倾诉。可以是心理咨询师。可以是好朋友 爱人。当我们分享 开放内心时。我们就能建立一种感觉。用Antonovsky的话来说就是"关联性的感觉"。来看待我们生命中的经历。得到更高层次的快乐。这是有用的。我们周围和内心有那么多快乐。而问题在于我们注意到它们了吗。还是把它们想成理所当然。回忆一下appreciate这个词。有两种意思。一种是表示谢谢 这是很好的。人们常说感恩是最高尚的美德。感激是一件很好的事。第二种意思也同样重要。意思是增值。当我们感激生活中好的方面。感激人际关系中好的方面。感激我们的伴侣。朋友 同学。感激我们的国家。这些好处就会增值。不幸的是相反的事也会发生。但我们不予以感恩时。好的方面就会贬值。世上没有完美的人际关系。没有完美的人 没有完美的国家。最重要的是要关注什么是有效的。感激它 它才能增值。很多人疑惑自己为什么总��不快乐。即使他们已经得到了想得��的一切。为什么 因为好东���太多有时也不是好事。还记得我最���欢的歌吗。Whitney Houston的《我将永远爱你》。10分满分。第二喜欢的 贝多芬第九交响乐 9.5分。在一起播放 就是噪音。不是19.5分 10分或5分。只是噪音。好东西太多也不是好事。要将一切简化说起来容易 做起来难。但记住我们说过比物质充裕。更能带来幸福的是时间充裕。简化 果断坚决。在适当的时候学会说"不"。弄清楚你究竟。真正想做的东西 然后去做。最能预示幸福的是 人际关系。是最能预示幸福的。你现在和未来所经营的一段。亲密关系比世界任何事都重要。对甚至比问问题更重要。比考试更重要。比我们有多成功 多被人景仰。更重要得多。对 成功会使我们的幸福感陡升。而很快又降到基础水平。就像获得长期聘用的大学教授。就像彩票中了大奖的人。那些以为自己。非常成功的人。很快会回到基准水平甚至更低。因为他们不知道什么会让他们更快乐。这就是一种。其中一种让我们更快乐的方法。经营一段感情就意味着为之投入。向其中投入时间。彼此分享和交流。就像我们会投入。如果我们找到理想的工作。我们会将自己投入其中。人际关系也是一样。电影总是在结尾时爱情才开始。在大屏幕暗下来后继续经营感情。在日落之后 此时我们的投入。我们的投入才有了回报。思想带给我们的前进是有限的。我很了解积极心理学。我实践了所有我教的知识。然而我真的觉得如果。我不知道精神肉体联系。并将精神肉体联系应用于我的生活。我一定没有今天这么快乐。为什么 因为我会屈服于上帝决定的基准水平。基因决定的基准幸福水平。因为当我们不锻炼时。就像打了镇静剂。我们必须和本性抗争。和本性抗争是很难的。提升我们幸福的水平是很难的。而同时要和本性抗争则是难以想象的困难。还记得灵药吗。每周锻炼四次 每次半小时。如果可以 如果你有兴趣。去做做瑜伽或冥想。打坐冥想。如果不行 那么至少。每天深呼吸几次。三次深呼吸就有显著效果。现在就可以试试。睡眠 可能的话八小时或接近八小时。以及触摸。抱 拥抱 这个总被忽视的感官却非常重要。被了解而非被证明。我们所有人都需要被证明。并不是说要完全去除我们依赖别人的自尊。但更首要的是被了解 去表达。而非给他人留下印象。人生会变得更轻松 更简单。更让人兴奋。如果我们能表达自我。被了解而非被证明。无论是和我们的伴侣。家人 朋友 同事 同学。这都是在长期人际关系中。亲密和激情增加的基础。说起来容易做起来难。这非常重要。我想列出前十。但我做不到 有一个不得不提。所以你们要谅解。最后一点 能将所有东西整合在一起的一点。就是关于转变的内容。终于讲到了 你们一整个学期就等着它呢。希望不会辜负你们的期待。有两种…… 事实上有很多种类型。但有两种类型的学生。上过这门课或其他。关于获得更高层次的幸福的课。一种是会感觉很愉快。或美好的回忆 然后回到基准水平。第二种同样感觉很愉快。虽然不可能持续整个人生。但仍然在未来持续增长。其区别 最基本的区别就在于。你是否发生了行为转变。还记得Peter Drucker说过。每当人们去参加他的周末研讨班。他周一时都会说。"别打电话跟我说这课有多棒。而是告诉我你开始变得不一样了"。在他和世界上最成功 最出色。最聪明的人。一起合作的六十五年间。他发现这才是能让人。一直保持愉悦平和的原因。不要告诉我这课有多棒。告诉我你周一做了什么改变。从现在开始进行行为转变。这就是改变的核心。你们在这门课上学到的只是开始。仅仅是开始。回忆下在Carlton的研究中。成功的人 最快乐的人。就最成功地实现了终极目标。从长远来看 他一生都在学习。永远在问问题 永远探索。有很多人都有很好的意图。很多人都有很好的意图。但很好的意图还不够。我们需要的还有基础。实用的 严格的。科学的基础来起作用。当我们了解这些东西后。再将它们传递出去。和他人分享。在很多人文及社会的学科中 包括心理学。"你在哈佛收获了什么 得到了什么"。就是这样一种基础。来使世界变得更美好。我教这门课不仅因为觉得它有趣。它确实很有趣 我觉得很愉快。我说过我总是等不及到周二和周四。来和你们分享我最关心的东西。是很有趣。但我教这门课不仅仅因为这个。更因为我认为一些伟大心理学家的想法。我引用过 也讨论过的。他们的想法能使世界变得更好。1954年时Maslow写道。"我不仅是对纯粹的冷漠真理。公正客观的探索者。我也很有兴趣关注人类的命运。其结局 目标以及未来。我想帮助人类进步 改善其前景。我希望教会他如何友爱他人。和平互助 勇敢坚毅 正直公平。我认为科学是达到这个目标最大的希望。而在所有学科中。我认为心理学是最有潜力的。事实上我有时觉得。世界要么被广义的。心理学家拯救。要么就无法被拯救"。广义的心理学家。这并不意味着。你现在就应该去读个心理学的博士。或潜心研究心理学。而是要时刻意识到。了解你大脑中发生的一切。这宇宙中最令人惊异的实体。了解它发生的一切。周围环境以及你的心。它们是如何联系在一起的。并作为现实理想主义者来分享它。你们可以有影响。你们可以有影响。每个人都有不可思议的力量。还记得吗 如果你让三个人微笑。他们会让另外三个人微笑。如此20次后 就能传到地球上每一个人。快乐是会传染的。我再展示最后一段关于幸福快乐如何传染的录像。爸爸要惹你们笑了 准备好了吗。他们会在怎样的世界中成长取决于你。取决于你是否首先。关注了自身。引用甘地的话。"欲变世界 先变其身"。如果想要更快乐的世界 你自己先快乐起来。所有你们去的地方。我想提醒你们。每次你们去一个地方 通常情况下。你都会坐飞机去 这样你每次去时。都会听到一些会引起你反思。照顾自己有多重要的东西。要照顾别人。你首先要照顾自己。当你坐飞机时。乘务员会说。以防紧急情况 氧气面罩会脱落。首先自己戴上 在去帮助需要帮忙的人。因为这就是。黄金法则的基础。是更美好的世界的基础。终极目标。每次你去某个美丽的地方。国外或国内。记住要简化问题。记住照顾自己。记住要锻炼。记住要果断坚决 在适当的时候说"不"。记住要经营你的感情。记住你生命中什么才是。真正最最重要的东西。这个经历。这个学期对我来说非常特别。它之所以这么特别首先是因为这些。出色的教职员工。上台来吧。就几分钟 不用太长。你们不知道这幕后有多少工作。你们不知道。这每一个教职员工。主管助教Shawn和Debb 以及其他助教。他们投入了多少时间来经营和你们的关系。他们花了多少小时。耗费了多少苦心和思考。我想感谢他们 也希望你们能感激。他们的存在和。所做的工作。谢谢你们 来吧 上来。做这些很麻烦。但我想对我们的助教一些小小的表示。来庆祝你们的辛勤付出。你想说些什么吗。轮到你了。轮到我了 谢谢。快 快 快。其实我只想说感谢所有的助教。辛勤的付出。感谢到场的所有同学 以及进修的同学。最重要的是。我想感谢我的妈……不 我要感谢Tal。首先我要感谢Tal的祖父母。生下了他的父母。然后生下了…… 是吧。你们懂的。我想感谢Tammy让Tal总是面带微笑。以及Sherill和David将他和我们大家分享。感谢给了我们接触人性本质的机会。让我们更人性化 谢谢。要感激你的不仅仅只有我们。你这学期感动了在场这么多人 其实一直都是。但这学期你的学生都写了感谢信。我们将它们集合成册送给你。不管去哪你都可以将它带着。谢谢。这是感恩书的第一部。以后还有更多的信。你可以不断增加。最后的最后。我们想给你一个表彰来记住这堂课。以及这次伟大的机会。希望你以后每次上这门课时。都会用到的东西。不要有压力。因为你是首席 所以这顶帽子。谢谢 谢谢。太感谢了 谢谢。还有一件事。最后一件事。谢谢 非常感谢。还有两点要说的。我知道已经拖堂了。十分抱歉 再占用一点时间。我还想感谢一些人。我想感谢坐在后面的Barry Reid。以及出色的摄制组。这些摄制人员将课堂带回你们家。带到你们的电脑上。最后。我想感谢在座各位。你们每一个人都是一整个世界。这个世界正在。正在使你我所处的这个世界。变得愈发美好。谢谢。保重。保重。谢谢。 
\end{document}